\documentclass[12pt]{amsart}

%Below are some necessary packages for your course.
\usepackage{amsfonts,latexsym,amsthm,amssymb,amsmath,amscd,euscript,mathrsfs}
\usepackage{framed}
\usepackage{fullpage}
\usepackage{hyperref}
    \hypersetup{colorlinks=true,citecolor=blue,urlcolor =black,linkbordercolor={1 0 0}}
\usepackage{mathtools}
\usepackage[table]{xcolor}

\newenvironment{statement}[1]{\smallskip\noindent\color[rgb]{.6627, .3529, .6314} {\bf #1.}}{}
\allowdisplaybreaks[1]

%Below are the theorem, definition, example, lemma, etc. body types.

\newtheorem{theorem}{Theorem}
\newtheorem*{proposition}{Proposition}
\newtheorem{lemma}[theorem]{Lemma}
\newtheorem{corollary}[theorem]{Corollary}
\newtheorem{conjecture}[theorem]{Conjecture}
\newtheorem{postulate}[theorem]{Postulate}
\theoremstyle{definition}
\newtheorem{defn}[theorem]{Definition}
\newtheorem{example}[theorem]{Example}

\theoremstyle{remark}
\newtheorem*{remark}{Remark}
\newtheorem*{notation}{Notation}
\newtheorem*{note}{Note}

% You can define new commands to make your life easier.
\newcommand{\BR}{\mathbb R}
\newcommand{\BC}{\mathbb C}
\newcommand{\BF}{\mathbb F}
\newcommand{\BQ}{\mathbb Q}
\newcommand{\BZ}{\mathbb Z}
\newcommand{\BN}{\mathbb N}

% We can even define a new command for \newcommand!
\newcommand{\nc}{\newcommand}

% If you want a new function, use operatorname to define that function (don't use \text)
\nc{\on}{\operatorname}
\nc{\Spec}{\on{Spec}}

\title{\emph{How to Prove It}: Chapter 3} % IMPORTANT: Change the problemset number as needed.
\date{\today}

\begin{document}

\maketitle

\vspace*{-0.25in}
\centerline{Kyle Stratton}

\begin{framed}
These are the exercises for Chapter 3 from the third edition of \emph{How to Prove It} by Daniel J. Velleman.
They are numbered (Chapter).(Section).(Exercise).
\end{framed}

\begin{statement}{3.1.1}
Consider the following theorem.
(This theorem was proven in the introduction.)
\begin{theorem}
	Suppose $n$ is an integer larger than 1 and $n$ is not prime.
	Then $2^n - 1$ is not prime.
\end{theorem}
\begin{enumerate}
	\item Identify the hypotheses and the conclusion of the theorem.
	Are the hypotheses true when $n = 6$?
	What does the theorem tell you in this instance?
	Is it right?
	
	\item What can you conclude from the theorem in the case $n = 15$?
	Check directly that this conclusion is correct.
	
	\item What can you conclude from the theorem in the case $n = 11$?
\end{enumerate}
\end{statement}

\begin{proof}
\hfill
\begin{enumerate}
	\item This theorem has three hypotheses: $n$ is an integer, $n > 1$, and $n$ is not prime.
	The conclusion of the theorem is that $2^n - 1$ is not prime.
	In the case when $n = 6 = 2 \times 3$, all of the hypotheses are satisfied,
	so the theorem tells us that $2^6 - 1$ is not prime.
	We can directly check that $2^6 - 1 = 63 = 3^2 \times 7$ is not prime.
	
	\item In the case when $n = 15 = 3 \times 5$, all of the hypotheses are satisfied.
	This means that the theorem tells us that $2^{15} - 1$ is not prime.
	As discussed in Part (1) of Exercise I.1, $2^{15} - 1 = 32767 = 31 \times 1057$.
	
	\item In the case when $n = 11$, not all of the hypotheses are satisfied.
	In particular, 11 is prime.
	Because not all of the hypotheses of the theorem are satisfied,
	we cannot draw any conclusions from it.
	In particular, the theorem does not tell us anything about the primality of $2^{11} - 1$.
\end{enumerate}
\end{proof}


\begin{statement}{3.1.2}
Consider the following theorem.
(The theorem is correct, but we will not ask you to prove it here.)
\begin{theorem}
	Suppose that $b^2 > 4ac$.
	Then the quadratic equation $ax^2 + bx + c = 0$ has exactly two real solutions.
\end{theorem}
\begin{enumerate}
	\item Identify the hypotheses and conclusion of the theorem.
	
	\item To give an instance of the theorem, you must specify values for $a$, $b$, and $c$, but not $x$.
	Why?
	
	\item What can you conclude from the theorem in the case $a = 2$, $b = -5$, $c = 3$?
	Check directly that this conclusion is correct.
	
	\item What can you conclude from the theorem in the case $a = 2$, $b = 4$, $c = 3$?
\end{enumerate}
\end{statement}

\begin{proof}
\hfill
\begin{enumerate}
	\item This theorem has one implicit hypothesis, that $a, b, c$ are all real numbers,
	and one explicit hypothesis, that $b^2 > 4ac$.
	The conclusion of the theorem is that the quadratic equation $ax^2 + bx + c = 0$
	has exactly two real solutions.
	
	\item To give an instance of the theorem, we only need to specify values for $a$, $b$, and $c$
	since those are the only variables listed in the hypothesis.
	The values of $x$ associated with a specific instance of the theorem are determined
	by the given values of $a, b, c$, since the $x$ values are the two real solutions to the
	quadratic equation $ax^2 + bx + c = 0$.
	In other words, $x$ is a dummy variable for the set 
	$S = \{ x \in \BR \mid ax^2 + bx + c = 0 \}$ for given values of $a, b, c$.
	When phrased this way, the conclusion of theorem states that $S$ contains
	exactly two distinct elements.
	
	\item In the case $a = 2, b = -5, c = 3$, we have $b^2 = (-5)^2 = 25$ 
	and $4ac = 4(2)(3) = 24$.
	Thus, the hypotheses $b^2 > 4ac$ is satisfied, and the theorem applies.
	We can then conclude that the quadratic equation $2x^2 - 5x + 3 = 0$ has two real solutions.
	Factoring the quadratic as $(2x - 3)(x - 1) = 0$, we can directly check that the two
	real solutions are $x = 3/2$ and $x = 1$.
	
	\item In the case $a = 2, b = 4, c = 3$, we have $b^2 = 4^2 = 16$
	and $4ac = 4(2)(3) = 24$.
	In other words, $b^2 \ngtr 4ac$.
	Since not all of the hypotheses of the theorem are satisfied,
	we cannot draw any conclusions from it.
	In particular, the theorem does not tell us anything about the solution set of
	the quadratic equation $2x^2 + 4x + 3 = 0$.
\end{enumerate}
\end{proof}


\begin{statement}{3.1.3}
Consider the following incorrect theorem.
\begin{theorem}
	Suppose $n$ is a natural number larger than 2, and $n$ is not a prime number.
	Then $2n + 13$ is not a prime number.
\end{theorem}
What are the hypotheses and conclusion of this theorem?
Show that the theorem is incorrect by finding a counterexample.
\end{statement}

\begin{proof}
The theorem has three hypotheses: $n$ is a natural number, $n > 2$, and $n$ is not prime.
The conclusion of the theorem is that $2n + 13$ is not a prime number.
To see that this theorem is incorrect, consider the case of $n = 8$.
This value of $n$ is a natural number greater than 2 that is not prime, 
so it satisfies all of the hypotheses of the theorem.
However, $2n + 13 = 2(8) + 13 = 29$ is a prime number.
Since we could find a instance of the theorem where all of the hypotheses are satisfied
but an incorrect conclusion is drawn, the theorem itself is incorrect.
\end{proof}


\begin{statement}{3.1.4}
Complete the following alternative proof of the theorem in Example 3.1.2.
\begin{theorem}
	Suppose $a$ and $b$ are real numbers.
	If $0 < a < b$ then $a^2 < b^2$.
\end{theorem}
\begin{proof}
	Suppose $0 < a < b$.
	Then $b - a > 0$.
	[\emph{Fill in a proof of $b^2 - a^2 > 0$ here.}]
	Since $b^2 - a^2 > 0$, it follows that $a^2 < b^2$.
	Therefore, if $0 < a < b$ then $a^2 < b^2$.
\end{proof}
\end{statement}

\begin{proof}
Suppose $0 < a < b$.
Then $b - a > 0$.
In addition, since $0 < a < b$, we also know that $b + a > 0$.
We can then multiply both sides of the inequality $b - a > 0$ by $b + a$
to get $(b - a)(b + a) = b^2 - a^2 > 0$.
Since $b^2 - a^2 > 0$, it follows that $a^2 < b^2$.
Therefore, if $0 < a < b$ then $a^2 < b^2$.
\end{proof}


\begin{statement}{3.1.5}
Suppose $a$ and $b$ are real numbers.
Prove that if $a < b < 0$ then $a^2 > b^2$.
\end{statement}

\begin{proof}
Suppose $a < b < 0$.
Then $a - b < 0$ and $a + b < 0$.
We can then multiply both sides of the inequality $a - b < 0$ by $a + b$
to get $(a - b)(a + b) = a^2 - b^2 > 0$.
Since $a^2 - b^2 > 0$, it follows that $a^2 > b^2$.
Therefore, if $a < b < 0$ then $a^2 > b^2$.

\emph{Note that an alternative strategy would be to mimic the proof of the theorem in Example 3.1.2 and instead multiply the given inequality by the negative numbers $a$ and $b$, respectively.
This would then result in the chain of inequalities $a^2 > ab > b^2 > 0$, which also gives the desired conclusion.}
\end{proof}


\begin{statement}{3.1.6}
Suppose $a$ and $b$ are real numbers.
Prove that if $0 < a < b$ then $1/b < 1/a$.
\end{statement}

\begin{proof}
Suppose $0 < a < b$.
Dividing both sides of the inequality $a < b$ by the positive number $a$ gives us $1 < b/a$.
We can then divide both sides of $1 < b/a$ by the positive number $b$ to conclude $1/b < 1/a$.
Therefore, if $0 < a < b$ then $1/b < 1/a$.
\end{proof}


\begin{statement}{3.1.7}
Suppose $a$ is a real number.
Prove that if $a^3 > a$ then $a^5 > a$.
(\emph{Hint: One approach is to start by completing the following equation:
$a^5 - a = (a^3 - a) \cdot \underline{?}$.})
\end{statement}

\begin{proof}
Suppose $a^3 > a$, which we can rewrite as $a^3 - a > 0$.
Following the hint, we note that we can factor $a^5 - a$ as $(a^3 - a)(a^2 + 1)$.
Since $a^2 \geq 0$ for all $a \in \BR$, we know that $a^2 + 1 > 0$.
Thus, multiplying the inequality $a^3 - a > 0$ by the positive number $a^2 + 1$
gives us $0 < (a^3 - a)(a^2 + 1) = a^5 - a$.
In other words, we conclude that $a^5 > a$.
Therefore, if $a^3 > a$ then $a^5 > a$.
\end{proof}


\begin{statement}{3.1.8}
Suppose $A \setminus B \subseteq C \cap D$ and $x \in A$.
Prove that if $x \notin D$ then $x \in B$.
\end{statement}

\begin{proof}
We prove the contrapositive statement: if $x \notin B$ then $x \in D$.
Suppose $x \notin B$.
Then, since we also know that $x \in A$, it follows that $x \in A \setminus B$.
As $A \setminus B \subseteq C \cap D$, it then follows that $x \in C \cap D$.
In other words, $x \in C$ and $x \in D$.
The last inclusion, that $x \in D$, is what we wanted to prove to complete the proof of the contrapositive statement.
Therefore, if $x \notin D$ then $x \in B$.
\end{proof}


\begin{statement}{3.1.9}
Suppose $A \cap B \subseteq C \setminus D$ .
Prove that if $x \in A$, then if $x \in D$ then $x \notin B$.
\end{statement}

\begin{proof}
Suppose $x \in A$.
We prove the contrapositve statement: if $x \in B$ then $x \notin D$.
Now suppose $x \in B$.
Then, since $x \in A$ as well, it follows that $x \in A \cap B$.
As $A \cap B \subseteq C \setminus D$, it then follows that $x \in C \setminus D$.
From this, we conclude that $x \notin D$, which completes our proof of the contrapositive statement.
Therefore, if $x \in A$, then if $x \in D$ then $x \notin B$.
\end{proof}


\begin{statement}{3.1.10}
Suppose $a$ and $b$ are real numbers.
Prove that if $a < b$ then $(a + b) / 2 < b$.
\end{statement}

\begin{proof}
Suppose $a < b$.
Adding $b$ to both sides of the inequality then gives us $a + b < 2b$.
Next we divide both sides by 2 to obtain the desired inequality: $(a + b)/2 < b$.
Therefore, if $a < b$ then $(a + b)/2 < b$.
\end{proof}


\begin{statement}{3.1.11}
Suppose $x$ is a real number and $x \neq 0$.
Prove that if $(\sqrt[3]{x} + 5) / (x^2 + 6) = 1/x$ then $x \neq 8$.
\end{statement}

\begin{proof}
We prove the contrapositive statement:
if $x = 8$ then $(\sqrt[3]{x} + 5) / (x^2 + 6) \neq 1/x$.
Suppose $x = 8$.
Then $1/x = 1/8$ and
$(\sqrt[3]{x} + 5) / (x^2 + 6) = (\sqrt[3]{8} + 5) / (8^2 + 6) = 1/10$.
Since $1/8 \neq 1/10$, this completes the proof of the contrapositive statement.
Therefore, we conclude that $(\sqrt[3]{x} + 5) / (x^2 + 6) = 1/x$ then $x \neq 8$.
\end{proof}


\begin{statement}{3.1.12}
Suppose $a$, $b$, $c$, and $d$ are real numbers, $0 < a < b$, and $d > 0$.
Prove that if $ac \geq bd$ then $c > d$.
\end{statement}

\begin{proof}
Suppose $ac \geq bd$.
Since $b > a$ and $d > 0$, we can multiply both sides of the first inequality by $d$ to obtain $bd > ad$.
Then, since $ac \geq bd$, we have $ac \geq bd > ad$.
It then follows that $ac > ad$.
Dividing both sides of the inequality by the positive number $a$, we see that $c > d$.
Thus, we conclude that if $ac \geq bd$ then $c > d$.

\emph{Note that we could also approach this proof by proving the contrapositive statement: if $c \leq d$ then $ac < bd$.}
\end{proof}


\begin{statement}{3.1.13}
Suppose $x$ and $y$ are real numbers, and that $3x + 2y \leq 5$.
Prove that if $x > 1$ then $y < 1$.
\end{statement}

\begin{proof}
Suppose $x > 1$.
Then, $3x + 2y > 3 + 2y$.
Combining this with the given inequality, $3x + 2y \leq 5$, we have $3 + 2y < 3x + 2y \leq 5$.
In particular, $3 + 2y < 5$.
After isolating $y$ in the inequality, we end up with the desired result: $y < 1$.
Therefore, we conclude that if $x > 1$ then $y < 1$.

\emph{Note that we could also approach this proof by proving the contrapositive statement: if $y \geq 1$ then $x \leq 1$.}
\end{proof}


\begin{statement}{3.1.14}
Suppose $x$ and $y$ are real numbers.
Prove that if $x^2 + y = -3$ and $2x - y = 2$ then $x = -1$.
\end{statement}

\begin{proof}
Suppose that $x^2 + y = -3$ and $2x - y = 2$.
We can add the two equations together to obtain $x^2 + 2x = -1$.
After rearranging, this becomes $x^2 + 2x + 1 = 0$, which we can factor as $(x + 1)^2 = 0$.
Therefore, it follows that $x = -1$.
Thus, we conclude that if $x^2 + y = -3$ and $2x - y = 2$ then $x = -1$.
\end{proof}

\begin{statement}{3.1.15}
Prove the first theorem in Example 3.1.1.
\begin{theorem}
	Suppose $x > 3$ and $y < 2$.
	Then $x^2 - 2y > 5$.
\end{theorem}
(\emph{Hint: You might find it useful to apply the theorem from Example 3.1.2,
which stated that if $a$ and $b$ are real numbers such that $0 < a < b$, then $a^2 < b^2$.})
\end{statement}

\begin{proof}
We follow the hint and apply the fact that if $a$ and $b$ are real numbers such that
$0 < a < b$ then $a^2 < b^2$ with $a = 3$ and $b = x$ to conclude that $x^2 > 9$.
Next, since $y < 2$, it follows that $-2y > -4$.
Adding the inequalities $x^2 > 9$ and $-2y > -4$ together gives us the desired inequality:
$x^2 - 2y > 5$.
\end{proof}


\begin{statement}{3.1.16}
Consider the following theorem.
\begin{theorem}
	Suppose $x$ is a real number and $x \neq 4$.
	If $(2x - 5) / (x - 4) = 3$ then $x = 7$.
\end{theorem}
\begin{enumerate}
	\item What is wrong with the following proof of the theorem?
	\begin{proof}
		Suppose $x = 7$.
		Then $(2x - 5) / (x - 4) = (2 \cdot 7 - 5) / (7 - 4) = 9/3 = 3$.
		Therefore if $(2x - 5) / (x - 4) = 3$ then $x = 7$.
	\end{proof}
	
	\item Give a correct proof of the theorem.
\end{enumerate}
\end{statement}

\begin{proof}
\hfill
\begin{enumerate}
	\item The problem with the incorrect proof of the theorem is that it starts
	by assuming that the conclusion, that $x = 7$, is true when that is the ultimate fact
	that we are trying to prove.
	In other words, it is instead a proof of the converse statement: if $x = 7$ then
	$(2x - 5) / (x - 4) = 3$, which is is a completely different statement.
	
	\item Suppose $(2x - 5) / (x - 4) = 3$.
	We can then clear the denominator of the left-hand side by multiplying the equation by $x - 4$.
	This gives us the equation $2x - 5 = 3(x - 4)$, or in other words $2x - 5 = 3x - 12$.
	We subtract $2x$ from both sides and then add 12 to both sides to see that $x = 7$.
	Therefore, we conclude that if $(2x - 5) / (x - 4) = 3$ then $x = 7$.
\end{enumerate}
\end{proof}


\begin{statement}{3.1.17}
Consider the following incorrect theorem.
\begin{theorem}
	Suppose that $x$ and $y$ are real numbers and $x \neq 3$.
	If $x^2y = 9y$ then $y = 0$.
\end{theorem}
\begin{enumerate}
	\item What's wrong with the following proof of the theorem?
	\begin{proof}
		Suppose that $x^2y = 9y$.
		Then $(x^2 - 9)y = 0$.
		Since $x \neq 3$, $x^2 \neq 9$, so $x^2 - 9 \neq 0$.
		Therefore we can divide both sides of the equation $(x^2 - 9)y = 0$ by $x^2 - 9$,
		which leads to the conclusion that $y = 0$.
		Thus, if $x^2y = 9y$ then $y = 0$.
	\end{proof}
	
	\item Show that the theorem is incorrect by finding a counterexample.
\end{enumerate}
\end{statement}

\begin{proof}
\hfill
\begin{enumerate}
	\item The attempted proof makes a mistake in the statement
	``Since $x \neq 3$, $x^2 \neq 9$, so $x^2 - 9 \neq 0$.''
	We only know that $x \neq 3$, but otherwise $x$ could be any other real number.
	In particular, we could have $x = -3$, which would then result in $x^2 = 9$.
	Since the rest of the proof hinges on the assumption that $x^2 \neq 9$, it is rendered invalid.
	
	\item Consider $x = -3$ and $y = 1$.
	In this case, $x^2y = (-3)^2(1) = 9$ and $9y = 9(1) = 9$, but $y \neq 0$.
	Therefore, this counter example shows that the theorem is correct,
	since we were able to find values of $x$ and $y$ which satisfy the hypotheses but
	then lead to an incorrect conclusion.
\end{enumerate}
\end{proof}


\begin{statement}{3.2.1}
This problem could be solved by using truth tables, but don't do it that way.
Instead, use the methods for writing proofs discussed so far in this chapter.
\begin{enumerate}
	\item Suppose $P \rightarrow Q$ and $Q \rightarrow R$ are both true.
	Prove that $P \rightarrow R$ is true.
	
	\item Suppose $\neg R \rightarrow (P \rightarrow \neg Q)$ is true.
	Prove that $P \rightarrow (Q \rightarrow R)$ is true.
\end{enumerate}
\end{statement}

\begin{proof}
\hfill
\begin{enumerate}
	\item Suppose $P$ is true.
	Then, since by the assumption $P \rightarrow Q$ is also true, we apply modus ponens
	to conclude that $Q$ is true.
	Now we apply modus ponens again, this time the fact that $Q$ is true and the assumption
	$Q \rightarrow R$ is true to conclude that $R$ is true.
	Therefore, $P \rightarrow R$ is true.
	
	\item Suppose $P$ is true.
	Now we want to show that $Q \rightarrow R$ is true.
	This is equivalent to proving the contrapositive statement, $\neg R \rightarrow \neg Q$.
	To prove $\neg R \rightarrow \neg Q$, suppose $\neg R$ is true.
	Then, we can combine this assumption with the given assumption that
	$\neg R \rightarrow (P \rightarrow \neg Q)$ is true to conclude that
	$P \rightarrow \neg Q$ is true by modus ponens.
	Now we apply modus ponens again with $P$ and $P \rightarrow \neg Q$ to conclude
	that $\neg Q$ is true.
	Therefore, $\neg R \rightarrow \neg Q$ is true, so we conclude that $Q \rightarrow R$ is true.
	Finally, we conclude that $P \rightarrow (Q \rightarrow R)$ is true.
\end{enumerate}
\end{proof}


\begin{statement}{3.2.2}
This problem could be solved using truth tables, but don't do it that way.
Instead, use the methods for writing proofs discussed so far in this chapter.
\begin{enumerate}
	\item Suppose $P \rightarrow Q$ and $R \rightarrow \neg Q$ are both true.
	Prove that $P \rightarrow \neg R$ is true.
	
	\item Suppose that $P$ is true.
	Prove that $Q \rightarrow \neg (Q \rightarrow \neg P)$ is true.
\end{enumerate}
\end{statement}

\begin{proof}
\hfill
\begin{enumerate}
	\item Suppose $P$ is true.
	Then, since $P$ and $P \rightarrow Q$ are both true, we conclude that $Q$ is true
	by modus ponens.
	Next, since $Q$ and $R \rightarrow \neg Q$ are both true, we conclude that $\neg R$
	is true by modus tollens.
	Therefore, we conclude that $P \rightarrow \neg R$ is true.
	
	\item We prove the equivalent contrapositive statement: 
	that $(Q \rightarrow \neg P) \rightarrow \neg Q$ is true.
	Suppose $Q \rightarrow \neg P$ is true.
	Then, since $P$ is true by assumption, we use modus tollens to conclude that $\neg Q$ is true.
	Thus, $(Q \rightarrow \neg P) \rightarrow \neg Q$ is true.
	In other words, we conclude that $Q \rightarrow \neg (Q \rightarrow \neg P)$ is true.
\end{enumerate}
\end{proof}


\begin{statement}{3.2.3}
Suppose $A \subseteq C$, and $B$ and $C$ are disjoint.
Prove that if $x \in A$ then $x \notin B$.
\end{statement}

\begin{proof}
Suppose $x \in A$.
Then, since $A \subseteq C$, we also know that $x \in C$.
Since $B$ and $C$ are disjoint, the fact that $x \in C$ means that $x \notin B$.
Therefore, we conclude that if $x \in A$ then $x \notin B$.
\end{proof}


\begin{statement}{3.2.4}
Suppose that $A \setminus B$ is disjoint from $C$ and $x \in A$.
Prove that if $x \in C$ then $x \in B$.
\end{statement}

\begin{proof}
% Fill in
\end{proof}


\begin{statement}{3.2.5}
Prove that it cannot be the case that $x \in A \setminus B$ and $x \in B \setminus C$.
\end{statement}

\begin{proof}
% Fill in
\end{proof}


\begin{statement}{3.2.6}
Use the method of proof by contradiction to prove the theorem in Example 3.2.1.
\begin{theorem}
	Suppose $A \cap C \subseteq B$ and $a \in C$.
	Prove that $a \notin A \setminus B$.
\end{theorem}
\end{statement}

\begin{proof}
% Fill in
\end{proof}


\begin{statement}{3.2.7}
Use the method of proof by contradiction to prove the theorem in Example 3.2.5.
\begin{theorem}
	Suppose $A \subseteq B$, $a  \in A$, and $a \notin B \setminus C$.
	Prove that $a \in C$.
\end{theorem}
\end{statement}

\begin{proof}
% Fill in
\end{proof}


\begin{statement}{3.2.8}
Suppose that $y + x = 2y - x$, and $x$ and $y$ are not both zero.
Prove that $y \neq 0$.
\end{statement}

\begin{proof}
% Fill in
\end{proof}


\begin{statement}{3.2.9}
Suppose that $a$ and $b$ are nonzero real numbers.
Prove that if $a < 1/a < b < 1/b$ then $a < -1$.
\end{statement}

\begin{proof}
% Fill in
\end{proof}


\begin{statement}{3.2.10}
Suppose that $x$ and $y$ are real numbers.
Prove that if $x^2y = 2x + y$, then if $y \neq 0$ then $x \neq 0$.
\end{statement}

\begin{proof}
% Fill in
\end{proof}


\begin{statement}{3.2.11}
Suppose that $x$ and $y$ are real numbers.
Prove that if $x \neq 0$, then if $y = (3x^2 + 2y) / (x^2 + 2)$ then $y = 3$.
\end{statement}

\begin{proof}
% Fill in
\end{proof}


\begin{statement}{3.2.12}
Consider the following incorrect theorem.
\begin{theorem}
	Suppose $x$ and $y$ are real numbers and $x + y = 10$.
	Then $x \neq 3$ and $y \neq 8$.
\end{theorem}
\begin{enumerate}
	\item What's wrong with the following proof of the theorem?
	\begin{proof}
		Suppose the conclusion of the theorem if false.
		Then $x = 3$ and $y = 8$.
		But then $x + y = 11$, which contradicts the given information that $x + y = 10$.
		Therefore, the conclusion must be true.
	\end{proof}
	
	\item Show that the theorem is incorrect by finding a counterexample.
\end{enumerate}
\end{statement}

\begin{proof}
% Fill in
\end{proof}


\begin{statement}{3.2.13}
Consider the following incorrect theorem.
\begin{theorem}
	Suppose that $A \subseteq C$, $B \subseteq C$, and $x \in A$.
	Then $x \in B$.
\end{theorem}
\begin{enumerate}
	\item What's wrong with the following proof of the theorem?
	\begin{proof}
		Suppose that $x \notin B$.
		Since $x \in A$ and $A \subseteq C$, $x \in C$.
		Since $x \notin B$ and $B \subseteq C$, $x \notin C$.
		But now we have proven both $x \in C$ and $x \notin C$,
		so we have reached a contradiction.
		Therefore $x \in B$.
	\end{proof}
	
	\item Show that the theorem is incorrect by finding a counterexample.
\end{enumerate}
\end{statement}

\begin{proof}
% Fill in
\end{proof}


\begin{statement}{3.2.14}
Use truth tables so show that modus tollens is a valid rule of inference.
\end{statement}

\begin{proof}
% Fill in
\end{proof}


\begin{statement}{3.2.15}
Use truth tables to check the correctness of the theorem in Example 3.2.4.
\begin{theorem}
	Suppose $P \rightarrow (Q \rightarrow R)$.
	Prove that $\neg R \rightarrow (P \rightarrow \neg Q)$.
\end{theorem}
\end{statement}

\begin{proof}
% Fill in
\end{proof}


\begin{statement}{3.2.16}
Use truth tables to check the correctness of the statements in Exercise 3.2.1.
\end{statement}

\begin{proof}
% Fill in
\end{proof}


\begin{statement}{3.2.17}
Use truth tables to check the correctness of the statements in Exercise 3.2.2.
\end{statement}

\begin{proof}
% Fill in
\end{proof}


\begin{statement}{3.2.18}
Can the proof in Example 3.2.2 be modified to prove that if $x^2 + y = 13$ and $x \neq 3$ then $y \neq 4$?
Explain.
Below is the theorem from Example 3.2.2 and its proof.
\begin{theorem}
	If $x^2 + y = 13$ and $y \neq 4$ then $x \neq 3$.
\end{theorem}
\begin{proof}
	Suppose $x^2 + y = 13$ and $y \neq 4$.
	Suppose $x = 3$.
	Substituting this into the equation $x^2 + y = 13$, we get $9 + y = 13$, so $y = 4$.
	But this contradicts the fact that $y \neq 4$.
	Therefore $x \neq 3$.
	Thus, if $x^2 + y = 13$ and $y \neq 4$ then $x \neq 3$.
\end{proof}
\end{statement}

\begin{proof}
% Fill in
\end{proof}


\begin{statement}{3.3.1}
% Fill in
\end{statement}

\begin{proof}
% Fill in
\end{proof}


\begin{statement}{3.3.2}
% Fill in
\end{statement}

\begin{proof}
% Fill in
\end{proof}


\begin{statement}{3.3.3}
% Fill in
\end{statement}

\begin{proof}
% Fill in
\end{proof}


\begin{statement}{3.3.4}
% Fill in
\end{statement}

\begin{proof}
% Fill in
\end{proof}


\begin{statement}{3.3.5}
% Fill in
\end{statement}

\begin{proof}
% Fill in
\end{proof}


\begin{statement}{3.3.6}
% Fill in
\end{statement}

\begin{proof}
% Fill in
\end{proof}


\begin{statement}{3.3.7}
% Fill in
\end{statement}

\begin{proof}
% Fill in
\end{proof}


\begin{statement}{3.3.8}
% Fill in
\end{statement}

\begin{proof}
% Fill in
\end{proof}


\begin{statement}{3.3.9}
% Fill in
\end{statement}

\begin{proof}
% Fill in
\end{proof}


\begin{statement}{3.3.10}
% Fill in
\end{statement}

\begin{proof}
% Fill in
\end{proof}


\begin{statement}{3.3.11}
% Fill in
\end{statement}

\begin{proof}
% Fill in
\end{proof}


\begin{statement}{3.3.12}
% Fill in
\end{statement}

\begin{proof}
% Fill in
\end{proof}


\begin{statement}{3.3.13}
% Fill in
\end{statement}

\begin{proof}
% Fill in
\end{proof}


\begin{statement}{3.3.14}
% Fill in
\end{statement}

\begin{proof}
% Fill in
\end{proof}


\begin{statement}{3.3.15}
% Fill in
\end{statement}

\begin{proof}
% Fill in
\end{proof}


\begin{statement}{3.3.16}
% Fill in
\end{statement}

\begin{proof}
% Fill in
\end{proof}


\begin{statement}{3.3.17}
% Fill in
\end{statement}

\begin{proof}
% Fill in
\end{proof}


\begin{statement}{3.3.18}
% Fill in
\end{statement}

\begin{proof}
% Fill in
\end{proof}


\begin{statement}{3.3.19}
% Fill in
\end{statement}

\begin{proof}
% Fill in
\end{proof}


\begin{statement}{3.3.20}
% Fill in
\end{statement}

\begin{proof}
% Fill in
\end{proof}


\begin{statement}{3.3.21}
% Fill in
\end{statement}

\begin{proof}
% Fill in
\end{proof}


\begin{statement}{3.3.22}
% Fill in
\end{statement}

\begin{proof}
% Fill in
\end{proof}


\begin{statement}{3.3.23}
% Fill in
\end{statement}

\begin{proof}
% Fill in
\end{proof}


\begin{statement}{3.3.24}
% Fill in
\end{statement}

\begin{proof}
% Fill in
\end{proof}


\begin{statement}{3.3.25}
% Fill in
\end{statement}

\begin{proof}
% Fill in
\end{proof}


\begin{statement}{3.3.26}
% Fill in
\end{statement}

\begin{proof}
% Fill in
\end{proof}


\begin{statement}{3.4.1}
% Fill in
\end{statement}

\begin{proof}
% Fill in
\end{proof}


\begin{statement}{3.4.2}
% Fill in
\end{statement}

\begin{proof}
% Fill in
\end{proof}


\begin{statement}{3.4.3}
% Fill in
\end{statement}

\begin{proof}
% Fill in
\end{proof}


\begin{statement}{3.4.4}
% Fill in
\end{statement}

\begin{proof}
% Fill in
\end{proof}


\begin{statement}{3.4.5}
% Fill in
\end{statement}

\begin{proof}
% Fill in
\end{proof}


\begin{statement}{3.4.6}
% Fill in
\end{statement}

\begin{proof}
% Fill in
\end{proof}


\begin{statement}{3.4.7}
% Fill in
\end{statement}

\begin{proof}
% Fill in
\end{proof}


\begin{statement}{3.4.8}
% Fill in
\end{statement}

\begin{proof}
% Fill in
\end{proof}


\begin{statement}{3.4.9}
% Fill in
\end{statement}

\begin{proof}
% Fill in
\end{proof}


\begin{statement}{3.4.10}
% Fill in
\end{statement}

\begin{proof}
% Fill in
\end{proof}


\begin{statement}{3.4.11}
% Fill in
\end{statement}

\begin{proof}
% Fill in
\end{proof}


\begin{statement}{3.4.12}
% Fill in
\end{statement}

\begin{proof}
% Fill in
\end{proof}


\begin{statement}{3.4.13}
% Fill in
\end{statement}

\begin{proof}
% Fill in
\end{proof}


\begin{statement}{3.4.14}
% Fill in
\end{statement}

\begin{proof}
% Fill in
\end{proof}


\begin{statement}{3.4.15}
% Fill in
\end{statement}

\begin{proof}
% Fill in
\end{proof}


\begin{statement}{3.4.16}
% Fill in
\end{statement}

\begin{proof}
% Fill in
\end{proof}


\begin{statement}{3.4.17}
% Fill in
\end{statement}

\begin{proof}
% Fill in
\end{proof}


\begin{statement}{3.4.18}
% Fill in
\end{statement}

\begin{proof}
% Fill in
\end{proof}


\begin{statement}{3.4.19}
% Fill in
\end{statement}

\begin{proof}
% Fill in
\end{proof}


\begin{statement}{3.4.20}
% Fill in
\end{statement}

\begin{proof}
% Fill in
\end{proof}


\begin{statement}{3.4.21}
% Fill in
\end{statement}

\begin{proof}
% Fill in
\end{proof}


\begin{statement}{3.4.22}
% Fill in
\end{statement}

\begin{proof}
% Fill in
\end{proof}


\begin{statement}{3.4.23}
% Fill in
\end{statement}

\begin{proof}
% Fill in
\end{proof}


\begin{statement}{3.4.24}
% Fill in
\end{statement}

\begin{proof}
% Fill in
\end{proof}


\begin{statement}{3.4.25}
% Fill in
\end{statement}

\begin{proof}
% Fill in
\end{proof}


\begin{statement}{3.4.26}
% Fill in
\end{statement}

\begin{proof}
% Fill in
\end{proof}


\begin{statement}{3.4.27}
% Fill in
\end{statement}

\begin{proof}
% Fill in
\end{proof}


\begin{statement}{3.5.1}
% Fill in
\end{statement}

\begin{proof}
% Fill in
\end{proof}


\begin{statement}{3.5.2}
% Fill in
\end{statement}

\begin{proof}
% Fill in
\end{proof}


\begin{statement}{3.5.3}
% Fill in
\end{statement}

\begin{proof}
% Fill in
\end{proof}


\begin{statement}{3.5.4}
% Fill in
\end{statement}

\begin{proof}
% Fill in
\end{proof}


\begin{statement}{3.5.5}
% Fill in
\end{statement}

\begin{proof}
% Fill in
\end{proof}


\begin{statement}{3.5.6}
% Fill in
\end{statement}

\begin{proof}
% Fill in
\end{proof}


\begin{statement}{3.5.7}
% Fill in
\end{statement}

\begin{proof}
% Fill in
\end{proof}


\begin{statement}{3.5.8}
% Fill in
\end{statement}

\begin{proof}
% Fill in
\end{proof}


\begin{statement}{3.5.9}
% Fill in
\end{statement}

\begin{proof}
% Fill in
\end{proof}


\begin{statement}{3.5.10}
% Fill in
\end{statement}

\begin{proof}
% Fill in
\end{proof}


\begin{statement}{3.5.11}
% Fill in
\end{statement}

\begin{proof}
% Fill in
\end{proof}


\begin{statement}{3.5.12}
% Fill in
\end{statement}

\begin{proof}
% Fill in
\end{proof}


\begin{statement}{3.5.13}
% Fill in
\end{statement}

\begin{proof}
% Fill in
\end{proof}


\begin{statement}{3.5.14}
% Fill in
\end{statement}

\begin{proof}
% Fill in
\end{proof}


\begin{statement}{3.5.15}
% Fill in
\end{statement}

\begin{proof}
% Fill in
\end{proof}


\begin{statement}{3.5.16}
% Fill in
\end{statement}

\begin{proof}
% Fill in
\end{proof}


\begin{statement}{3.5.17}
% Fill in
\end{statement}

\begin{proof}
% Fill in
\end{proof}


\begin{statement}{3.5.18}
% Fill in
\end{statement}

\begin{proof}
% Fill in
\end{proof}


\begin{statement}{3.5.19}
% Fill in
\end{statement}

\begin{proof}
% Fill in
\end{proof}


\begin{statement}{3.5.20}
% Fill in
\end{statement}

\begin{proof}
% Fill in
\end{proof}


\begin{statement}{3.5.21}
% Fill in
\end{statement}

\begin{proof}
% Fill in
\end{proof}


\begin{statement}{3.5.22}
% Fill in
\end{statement}

\begin{proof}
% Fill in
\end{proof}


\begin{statement}{3.5.23}
% Fill in
\end{statement}

\begin{proof}
% Fill in
\end{proof}


\begin{statement}{3.5.24}
% Fill in
\end{statement}

\begin{proof}
% Fill in
\end{proof}


\begin{statement}{3.5.25}
% Fill in
\end{statement}

\begin{proof}
% Fill in
\end{proof}


\begin{statement}{3.5.26}
% Fill in
\end{statement}

\begin{proof}
% Fill in
\end{proof}


\begin{statement}{3.5.27}
% Fill in
\end{statement}

\begin{proof}
% Fill in
\end{proof}


\begin{statement}{3.5.28}
% Fill in
\end{statement}

\begin{proof}
% Fill in
\end{proof}


\begin{statement}{3.5.29}
% Fill in
\end{statement}

\begin{proof}
% Fill in
\end{proof}


\begin{statement}{3.5.30}
% Fill in
\end{statement}

\begin{proof}
% Fill in
\end{proof}


\begin{statement}{3.5.31}
% Fill in
\end{statement}

\begin{proof}
% Fill in
\end{proof}


\begin{statement}{3.5.32}
% Fill in
\end{statement}

\begin{proof}
% Fill in
\end{proof}


\begin{statement}{3.5.33}
% Fill in
\end{statement}

\begin{proof}
% Fill in
\end{proof}


\begin{statement}{3.6.1}
% Fill in
\end{statement}

\begin{proof}
% Fill in
\end{proof}


\begin{statement}{3.6.2}
% Fill in
\end{statement}

\begin{proof}
% Fill in
\end{proof}


\begin{statement}{3.6.3}
% Fill in
\end{statement}

\begin{proof}
% Fill in
\end{proof}


\begin{statement}{3.6.4}
% Fill in
\end{statement}

\begin{proof}
% Fill in
\end{proof}


\begin{statement}{3.6.5}
% Fill in
\end{statement}

\begin{proof}
% Fill in
\end{proof}


\begin{statement}{3.6.6}
% Fill in
\end{statement}

\begin{proof}
% Fill in
\end{proof}


\begin{statement}{3.6.7}
% Fill in
\end{statement}

\begin{proof}
% Fill in
\end{proof}


\begin{statement}{3.6.8}
% Fill in
\end{statement}

\begin{proof}
% Fill in
\end{proof}


\begin{statement}{3.6.9}
% Fill in
\end{statement}

\begin{proof}
% Fill in
\end{proof}


\begin{statement}{3.6.10}
% Fill in
\end{statement}

\begin{proof}
% Fill in
\end{proof}


\begin{statement}{3.6.11}
% Fill in
\end{statement}

\begin{proof}
% Fill in
\end{proof}


\begin{statement}{3.6.12}
% Fill in
\end{statement}

\begin{proof}
% Fill in
\end{proof}


\begin{statement}{3.6.13}
% Fill in
\end{statement}

\begin{proof}
% Fill in
\end{proof}


\begin{statement}{3.7.1}
% Fill in
\end{statement}

\begin{proof}
% Fill in
\end{proof}


\begin{statement}{3.7.2}
% Fill in
\end{statement}

\begin{proof}
% Fill in
\end{proof}


\begin{statement}{3.7.3}
% Fill in
\end{statement}

\begin{proof}
% Fill in
\end{proof}


\begin{statement}{3.7.4}
% Fill in
\end{statement}

\begin{proof}
% Fill in
\end{proof}


\begin{statement}{3.7.5}
% Fill in
\end{statement}

\begin{proof}
% Fill in
\end{proof}


\begin{statement}{3.7.6}
% Fill in
\end{statement}

\begin{proof}
% Fill in
\end{proof}


\begin{statement}{3.7.7}
% Fill in
\end{statement}

\begin{proof}
% Fill in
\end{proof}


\begin{statement}{3.7.8}
% Fill in
\end{statement}

\begin{proof}
% Fill in
\end{proof}


\begin{statement}{3.7.9}
% Fill in
\end{statement}

\begin{proof}
% Fill in
\end{proof}


\begin{statement}{3.7.10}
% Fill in
\end{statement}

\begin{proof}
% Fill in
\end{proof}
\end{document}