\documentclass[12pt]{amsart}

%Below are some necessary packages for your course.
\usepackage{amsfonts,latexsym,amsthm,amssymb,amsmath,amscd,euscript,mathrsfs}
\usepackage{framed}
\usepackage{fullpage}
\usepackage{hyperref}
    \hypersetup{colorlinks=true,citecolor=blue,urlcolor =black,linkbordercolor={1 0 0}}
\usepackage{mathtools}
\usepackage[table]{xcolor}

\newenvironment{statement}[1]{\smallskip\noindent\color[rgb]{.6627, .3529, .6314} {\bf #1.}}{}
\allowdisplaybreaks[1]

%Below are the theorem, definition, example, lemma, etc. body types.

\newtheorem{theorem}{Theorem}
\newtheorem*{proposition}{Proposition}
\newtheorem{lemma}[theorem]{Lemma}
\newtheorem{corollary}[theorem]{Corollary}
\newtheorem{conjecture}[theorem]{Conjecture}
\newtheorem{postulate}[theorem]{Postulate}
\theoremstyle{definition}
\newtheorem{defn}[theorem]{Definition}
\newtheorem{example}[theorem]{Example}

\theoremstyle{remark}
\newtheorem*{remark}{Remark}
\newtheorem*{notation}{Notation}
\newtheorem*{note}{Note}

% You can define new commands to make your life easier.
\newcommand{\BR}{\mathbb R}
\newcommand{\BC}{\mathbb C}
\newcommand{\BF}{\mathbb F}
\newcommand{\BQ}{\mathbb Q}
\newcommand{\BZ}{\mathbb Z}
\newcommand{\BN}{\mathbb N}
\newcommand{\powerset}[1]{\mathscr{P} \left( #1 \right)}

% We can even define a new command for \newcommand!
\newcommand{\nc}{\newcommand}

% If you want a new function, use operatorname to define that function (don't use \text)
\nc{\on}{\operatorname}
\nc{\Spec}{\on{Spec}}

\title{\emph{How to Prove It}: Chapter 3} % IMPORTANT: Change the problemset number as needed.
\date{\today}

\begin{document}

\maketitle

\vspace*{-0.25in}
\centerline{Kyle Stratton}

\begin{framed}
These are the exercises for Chapter 3 from the third edition of \emph{How to Prove It} by Daniel J. Velleman.
They are numbered (Chapter).(Section).(Exercise).
\end{framed}

\begin{statement}{3.1.1}
Consider the following theorem.
(This theorem was proven in the introduction.)
\begin{theorem}
	Suppose $n$ is an integer larger than 1 and $n$ is not prime.
	Then $2^n - 1$ is not prime.
\end{theorem}
\begin{enumerate}
	\item Identify the hypotheses and the conclusion of the theorem.
	Are the hypotheses true when $n = 6$?
	What does the theorem tell you in this instance?
	Is it right?
	
	\item What can you conclude from the theorem in the case $n = 15$?
	Check directly that this conclusion is correct.
	
	\item What can you conclude from the theorem in the case $n = 11$?
\end{enumerate}
\end{statement}

\begin{proof}
\hfill
\begin{enumerate}
	\item This theorem has three hypotheses: $n$ is an integer, $n > 1$, and $n$ is not prime.
	The conclusion of the theorem is that $2^n - 1$ is not prime.
	In the case when $n = 6 = 2 \times 3$, all of the hypotheses are satisfied,
	so the theorem tells us that $2^6 - 1$ is not prime.
	We can directly check that $2^6 - 1 = 63 = 3^2 \times 7$ is not prime.
	
	\item In the case when $n = 15 = 3 \times 5$, all of the hypotheses are satisfied.
	This means that the theorem tells us that $2^{15} - 1$ is not prime.
	As discussed in Part (1) of Exercise I.1, $2^{15} - 1 = 32767 = 31 \times 1057$.
	
	\item In the case when $n = 11$, not all of the hypotheses are satisfied.
	In particular, 11 is prime.
	Because not all of the hypotheses of the theorem are satisfied,
	we cannot draw any conclusions from it.
	In particular, the theorem does not tell us anything about the primality of $2^{11} - 1$.
\end{enumerate}
\end{proof}


\begin{statement}{3.1.2}
Consider the following theorem.
(The theorem is correct, but we will not ask you to prove it here.)
\begin{theorem}
	Suppose that $b^2 > 4ac$.
	Then the quadratic equation $ax^2 + bx + c = 0$ has exactly two real solutions.
\end{theorem}
\begin{enumerate}
	\item Identify the hypotheses and conclusion of the theorem.
	
	\item To give an instance of the theorem, you must specify values for $a$, $b$, and $c$, but not $x$.
	Why?
	
	\item What can you conclude from the theorem in the case $a = 2$, $b = -5$, $c = 3$?
	Check directly that this conclusion is correct.
	
	\item What can you conclude from the theorem in the case $a = 2$, $b = 4$, $c = 3$?
\end{enumerate}
\end{statement}

\begin{proof}
\hfill
\begin{enumerate}
	\item This theorem has one implicit hypothesis, that $a, b, c$ are all real numbers,
	and one explicit hypothesis, that $b^2 > 4ac$.
	The conclusion of the theorem is that the quadratic equation $ax^2 + bx + c = 0$
	has exactly two real solutions.
	
	\item To give an instance of the theorem, we only need to specify values for $a$, $b$, and $c$
	since those are the only variables listed in the hypothesis.
	The values of $x$ associated with a specific instance of the theorem are determined
	by the given values of $a, b, c$, since the $x$ values are the two real solutions to the
	quadratic equation $ax^2 + bx + c = 0$.
	In other words, $x$ is a dummy variable for the set 
	$S = \{ x \in \BR \mid ax^2 + bx + c = 0 \}$ for given values of $a, b, c$.
	When phrased this way, the conclusion of theorem states that $S$ contains
	exactly two distinct elements.
	
	\item In the case $a = 2, b = -5, c = 3$, we have $b^2 = (-5)^2 = 25$ 
	and $4ac = 4(2)(3) = 24$.
	Thus, the hypotheses $b^2 > 4ac$ is satisfied, and the theorem applies.
	We can then conclude that the quadratic equation $2x^2 - 5x + 3 = 0$ has two real solutions.
	Factoring the quadratic as $(2x - 3)(x - 1) = 0$, we can directly check that the two
	real solutions are $x = 3/2$ and $x = 1$.
	
	\item In the case $a = 2, b = 4, c = 3$, we have $b^2 = 4^2 = 16$
	and $4ac = 4(2)(3) = 24$.
	In other words, $b^2 \ngtr 4ac$.
	Since not all of the hypotheses of the theorem are satisfied,
	we cannot draw any conclusions from it.
	In particular, the theorem does not tell us anything about the solution set of
	the quadratic equation $2x^2 + 4x + 3 = 0$.
\end{enumerate}
\end{proof}


\begin{statement}{3.1.3}
Consider the following incorrect theorem.
\begin{theorem}
	Suppose $n$ is a natural number larger than 2, and $n$ is not a prime number.
	Then $2n + 13$ is not a prime number.
\end{theorem}
What are the hypotheses and conclusion of this theorem?
Show that the theorem is incorrect by finding a counterexample.
\end{statement}

\begin{proof}
The theorem has three hypotheses: $n$ is a natural number, $n > 2$, and $n$ is not prime.
The conclusion of the theorem is that $2n + 13$ is not a prime number.
To see that this theorem is incorrect, consider the case of $n = 8$.
This value of $n$ is a natural number greater than 2 that is not prime, 
so it satisfies all of the hypotheses of the theorem.
However, $2n + 13 = 2(8) + 13 = 29$ is a prime number.
Since we could find a instance of the theorem where all of the hypotheses are satisfied
but an incorrect conclusion is drawn, the theorem itself is incorrect.
\end{proof}


\begin{statement}{3.1.4}
Complete the following alternative proof of the theorem in Example 3.1.2.
\begin{theorem}
	Suppose $a$ and $b$ are real numbers.
	If $0 < a < b$ then $a^2 < b^2$.
\end{theorem}
\begin{proof}
	Suppose $0 < a < b$.
	Then $b - a > 0$.
	[\emph{Fill in a proof of $b^2 - a^2 > 0$ here.}]
	Since $b^2 - a^2 > 0$, it follows that $a^2 < b^2$.
	Therefore, if $0 < a < b$ then $a^2 < b^2$.
\end{proof}
\end{statement}

\begin{proof}
Suppose $0 < a < b$.
Then $b - a > 0$.
In addition, since $0 < a < b$, we also know that $b + a > 0$.
We can then multiply both sides of the inequality $b - a > 0$ by $b + a$
to get $(b - a)(b + a) = b^2 - a^2 > 0$.
Since $b^2 - a^2 > 0$, it follows that $a^2 < b^2$.
Therefore, if $0 < a < b$ then $a^2 < b^2$.
\end{proof}


\begin{statement}{3.1.5}
Suppose $a$ and $b$ are real numbers.
Prove that if $a < b < 0$ then $a^2 > b^2$.
\end{statement}

\begin{proof}
Suppose $a < b < 0$.
Then $a - b < 0$ and $a + b < 0$.
We can then multiply both sides of the inequality $a - b < 0$ by $a + b$
to get $(a - b)(a + b) = a^2 - b^2 > 0$.
Since $a^2 - b^2 > 0$, it follows that $a^2 > b^2$.
Therefore, if $a < b < 0$ then $a^2 > b^2$.

\emph{Note that an alternative strategy would be to mimic the proof of the theorem in Example 3.1.2 and instead multiply the given inequality by the negative numbers $a$ and $b$, respectively.
This would then result in the chain of inequalities $a^2 > ab > b^2 > 0$, which also gives the desired conclusion.}
\end{proof}


\begin{statement}{3.1.6}
Suppose $a$ and $b$ are real numbers.
Prove that if $0 < a < b$ then $1/b < 1/a$.
\end{statement}

\begin{proof}
Suppose $0 < a < b$.
Dividing both sides of the inequality $a < b$ by the positive number $a$ gives us $1 < b/a$.
We can then divide both sides of $1 < b/a$ by the positive number $b$ to conclude $1/b < 1/a$.
Therefore, if $0 < a < b$ then $1/b < 1/a$.
\end{proof}


\begin{statement}{3.1.7}
Suppose $a$ is a real number.
Prove that if $a^3 > a$ then $a^5 > a$.
(\emph{Hint: One approach is to start by completing the following equation:
$a^5 - a = (a^3 - a) \cdot \underline{?}$.})
\end{statement}

\begin{proof}
Suppose $a^3 > a$, which we can rewrite as $a^3 - a > 0$.
Following the hint, we note that we can factor $a^5 - a$ as $(a^3 - a)(a^2 + 1)$.
Since $a^2 \geq 0$ for all $a \in \BR$, we know that $a^2 + 1 > 0$.
Thus, multiplying the inequality $a^3 - a > 0$ by the positive number $a^2 + 1$
gives us $0 < (a^3 - a)(a^2 + 1) = a^5 - a$.
In other words, we conclude that $a^5 > a$.
Therefore, if $a^3 > a$ then $a^5 > a$.
\end{proof}


\begin{statement}{3.1.8}
Suppose $A \setminus B \subseteq C \cap D$ and $x \in A$.
Prove that if $x \notin D$ then $x \in B$.
\end{statement}

\begin{proof}
We prove the contrapositive statement: if $x \notin B$ then $x \in D$.
Suppose $x \notin B$.
Then, since we also know that $x \in A$, it follows that $x \in A \setminus B$.
As $A \setminus B \subseteq C \cap D$, it then follows that $x \in C \cap D$.
In other words, $x \in C$ and $x \in D$.
The last inclusion, that $x \in D$, is what we wanted to prove to complete the proof of the contrapositive statement.
Therefore, if $x \notin D$ then $x \in B$.
\end{proof}


\begin{statement}{3.1.9}
Suppose $A \cap B \subseteq C \setminus D$ .
Prove that if $x \in A$, then if $x \in D$ then $x \notin B$.
\end{statement}

\begin{proof}
Suppose $x \in A$.
We prove the contrapositve statement: if $x \in B$ then $x \notin D$.
Now suppose $x \in B$.
Then, since $x \in A$ as well, it follows that $x \in A \cap B$.
As $A \cap B \subseteq C \setminus D$, it then follows that $x \in C \setminus D$.
From this, we conclude that $x \notin D$, which completes our proof of the contrapositive statement.
Therefore, if $x \in A$, then if $x \in D$ then $x \notin B$.
\end{proof}


\begin{statement}{3.1.10}
Suppose $a$ and $b$ are real numbers.
Prove that if $a < b$ then $(a + b) / 2 < b$.
\end{statement}

\begin{proof}
Suppose $a < b$.
Adding $b$ to both sides of the inequality then gives us $a + b < 2b$.
Next we divide both sides by 2 to obtain the desired inequality: $(a + b)/2 < b$.
Therefore, if $a < b$ then $(a + b)/2 < b$.
\end{proof}


\begin{statement}{3.1.11}
Suppose $x$ is a real number and $x \neq 0$.
Prove that if $(\sqrt[3]{x} + 5) / (x^2 + 6) = 1/x$ then $x \neq 8$.
\end{statement}

\begin{proof}
We prove the contrapositive statement:
if $x = 8$ then $(\sqrt[3]{x} + 5) / (x^2 + 6) \neq 1/x$.
Suppose $x = 8$.
Then $1/x = 1/8$ and
$(\sqrt[3]{x} + 5) / (x^2 + 6) = (\sqrt[3]{8} + 5) / (8^2 + 6) = 1/10$.
Since $1/8 \neq 1/10$, this completes the proof of the contrapositive statement.
Therefore, we conclude that $(\sqrt[3]{x} + 5) / (x^2 + 6) = 1/x$ then $x \neq 8$.
\end{proof}


\begin{statement}{3.1.12}
Suppose $a$, $b$, $c$, and $d$ are real numbers, $0 < a < b$, and $d > 0$.
Prove that if $ac \geq bd$ then $c > d$.
\end{statement}

\begin{proof}
Suppose $ac \geq bd$.
Since $b > a$ and $d > 0$, we can multiply both sides of the first inequality by $d$ to obtain $bd > ad$.
Then, since $ac \geq bd$, we have $ac \geq bd > ad$.
It then follows that $ac > ad$.
Dividing both sides of the inequality by the positive number $a$, we see that $c > d$.
Thus, we conclude that if $ac \geq bd$ then $c > d$.

\emph{Note that we could also approach this proof by proving the contrapositive statement: if $c \leq d$ then $ac < bd$.}
\end{proof}


\begin{statement}{3.1.13}
Suppose $x$ and $y$ are real numbers, and that $3x + 2y \leq 5$.
Prove that if $x > 1$ then $y < 1$.
\end{statement}

\begin{proof}
Suppose $x > 1$.
Then, $3x + 2y > 3 + 2y$.
Combining this with the given inequality, $3x + 2y \leq 5$, we have $3 + 2y < 3x + 2y \leq 5$.
In particular, $3 + 2y < 5$.
After isolating $y$ in the inequality, we end up with the desired result: $y < 1$.
Therefore, we conclude that if $x > 1$ then $y < 1$.

\emph{Note that we could also approach this proof by proving the contrapositive statement: if $y \geq 1$ then $x \leq 1$.}
\end{proof}


\begin{statement}{3.1.14}
Suppose $x$ and $y$ are real numbers.
Prove that if $x^2 + y = -3$ and $2x - y = 2$ then $x = -1$.
\end{statement}

\begin{proof}
Suppose that $x^2 + y = -3$ and $2x - y = 2$.
We can add the two equations together to obtain $x^2 + 2x = -1$.
After rearranging, this becomes $x^2 + 2x + 1 = 0$, which we can factor as $(x + 1)^2 = 0$.
Therefore, it follows that $x = -1$.
Thus, we conclude that if $x^2 + y = -3$ and $2x - y = 2$ then $x = -1$.
\end{proof}

\begin{statement}{3.1.15}
Prove the first theorem in Example 3.1.1.
\begin{theorem}
	Suppose $x > 3$ and $y < 2$.
	Then $x^2 - 2y > 5$.
\end{theorem}
(\emph{Hint: You might find it useful to apply the theorem from Example 3.1.2,
which stated that if $a$ and $b$ are real numbers such that $0 < a < b$, then $a^2 < b^2$.})
\end{statement}

\begin{proof}
We follow the hint and apply the fact that if $a$ and $b$ are real numbers such that
$0 < a < b$ then $a^2 < b^2$ with $a = 3$ and $b = x$ to conclude that $x^2 > 9$.
Next, since $y < 2$, it follows that $-2y > -4$.
Adding the inequalities $x^2 > 9$ and $-2y > -4$ together gives us the desired inequality:
$x^2 - 2y > 5$.
\end{proof}


\begin{statement}{3.1.16}
Consider the following theorem.
\begin{theorem}
	Suppose $x$ is a real number and $x \neq 4$.
	If $(2x - 5) / (x - 4) = 3$ then $x = 7$.
\end{theorem}
\begin{enumerate}
	\item What is wrong with the following proof of the theorem?
	\begin{proof}
		Suppose $x = 7$.
		Then $(2x - 5) / (x - 4) = (2 \cdot 7 - 5) / (7 - 4) = 9/3 = 3$.
		Therefore if $(2x - 5) / (x - 4) = 3$ then $x = 7$.
	\end{proof}
	
	\item Give a correct proof of the theorem.
\end{enumerate}
\end{statement}

\begin{proof}
\hfill
\begin{enumerate}
	\item The problem with the incorrect proof of the theorem is that it starts
	by assuming that the conclusion, that $x = 7$, is true when that is the ultimate fact
	that we are trying to prove.
	In other words, it is instead a proof of the converse statement: if $x = 7$ then
	$(2x - 5) / (x - 4) = 3$, which is is a completely different statement.
	
	\item Suppose $(2x - 5) / (x - 4) = 3$.
	We can then clear the denominator of the left-hand side by multiplying the equation by $x - 4$.
	This gives us the equation $2x - 5 = 3(x - 4)$, or in other words $2x - 5 = 3x - 12$.
	We subtract $2x$ from both sides and then add 12 to both sides to see that $x = 7$.
	Therefore, we conclude that if $(2x - 5) / (x - 4) = 3$ then $x = 7$.
\end{enumerate}
\end{proof}


\begin{statement}{3.1.17}
Consider the following incorrect theorem.
\begin{theorem}
	Suppose that $x$ and $y$ are real numbers and $x \neq 3$.
	If $x^2y = 9y$ then $y = 0$.
\end{theorem}
\begin{enumerate}
	\item What's wrong with the following proof of the theorem?
	\begin{proof}
		Suppose that $x^2y = 9y$.
		Then $(x^2 - 9)y = 0$.
		Since $x \neq 3$, $x^2 \neq 9$, so $x^2 - 9 \neq 0$.
		Therefore we can divide both sides of the equation $(x^2 - 9)y = 0$ by $x^2 - 9$,
		which leads to the conclusion that $y = 0$.
		Thus, if $x^2y = 9y$ then $y = 0$.
	\end{proof}
	
	\item Show that the theorem is incorrect by finding a counterexample.
\end{enumerate}
\end{statement}

\begin{proof}
\hfill
\begin{enumerate}
	\item The attempted proof makes a mistake in the statement
	``Since $x \neq 3$, $x^2 \neq 9$, so $x^2 - 9 \neq 0$.''
	We only know that $x \neq 3$, but otherwise $x$ could be any other real number.
	In particular, we could have $x = -3$, which would then result in $x^2 = 9$.
	Since the rest of the proof hinges on the assumption that $x^2 \neq 9$, it is rendered invalid.
	
	\item Consider $x = -3$ and $y = 1$.
	In this case, $x^2y = (-3)^2(1) = 9$ and $9y = 9(1) = 9$, but $y \neq 0$.
	Therefore, this counter example shows that the theorem is correct,
	since we were able to find values of $x$ and $y$ which satisfy the hypotheses but
	then lead to an incorrect conclusion.
\end{enumerate}
\end{proof}


\begin{statement}{3.2.1}
This problem could be solved by using truth tables, but don't do it that way.
Instead, use the methods for writing proofs discussed so far in this chapter.
\begin{enumerate}
	\item Suppose $P \rightarrow Q$ and $Q \rightarrow R$ are both true.
	Prove that $P \rightarrow R$ is true.
	
	\item Suppose $\neg R \rightarrow (P \rightarrow \neg Q)$ is true.
	Prove that $P \rightarrow (Q \rightarrow R)$ is true.
\end{enumerate}
\end{statement}

\begin{proof}
\hfill
\begin{enumerate}
	\item Suppose $P$ is true.
	Then, since by the assumption $P \rightarrow Q$ is also true, we apply modus ponens
	to conclude that $Q$ is true.
	Now we apply modus ponens again, this time the fact that $Q$ is true and the assumption
	$Q \rightarrow R$ is true to conclude that $R$ is true.
	Therefore, $P \rightarrow R$ is true.
	
	\item Suppose $P$ is true.
	Now we want to show that $Q \rightarrow R$ is true.
	This is equivalent to proving the contrapositive statement, $\neg R \rightarrow \neg Q$.
	To prove $\neg R \rightarrow \neg Q$, suppose $\neg R$ is true.
	Then, we can combine this assumption with the given assumption that
	$\neg R \rightarrow (P \rightarrow \neg Q)$ is true to conclude that
	$P \rightarrow \neg Q$ is true by modus ponens.
	Now we apply modus ponens again with $P$ and $P \rightarrow \neg Q$ to conclude
	that $\neg Q$ is true.
	Therefore, $\neg R \rightarrow \neg Q$ is true, so we conclude that $Q \rightarrow R$ is true.
	Finally, we conclude that $P \rightarrow (Q \rightarrow R)$ is true.
\end{enumerate}
\end{proof}


\begin{statement}{3.2.2}
This problem could be solved using truth tables, but don't do it that way.
Instead, use the methods for writing proofs discussed so far in this chapter.
\begin{enumerate}
	\item Suppose $P \rightarrow Q$ and $R \rightarrow \neg Q$ are both true.
	Prove that $P \rightarrow \neg R$ is true.
	
	\item Suppose that $P$ is true.
	Prove that $Q \rightarrow \neg (Q \rightarrow \neg P)$ is true.
\end{enumerate}
\end{statement}

\begin{proof}
\hfill
\begin{enumerate}
	\item Suppose $P$ is true.
	Then, since $P$ and $P \rightarrow Q$ are both true, we conclude that $Q$ is true
	by modus ponens.
	Next, since $Q$ and $R \rightarrow \neg Q$ are both true, we conclude that $\neg R$
	is true by modus tollens.
	Therefore, we conclude that $P \rightarrow \neg R$ is true.
	
	\item We prove the equivalent contrapositive statement: 
	that $(Q \rightarrow \neg P) \rightarrow \neg Q$ is true.
	Suppose $Q \rightarrow \neg P$ is true.
	Then, since $P$ is true by assumption, we use modus tollens to conclude that $\neg Q$ is true.
	Thus, $(Q \rightarrow \neg P) \rightarrow \neg Q$ is true.
	In other words, we conclude that $Q \rightarrow \neg (Q \rightarrow \neg P)$ is true.
\end{enumerate}
\end{proof}


\begin{statement}{3.2.3}
Suppose $A \subseteq C$, and $B$ and $C$ are disjoint.
Prove that if $x \in A$ then $x \notin B$.
\end{statement}

\begin{proof}
Suppose $x \in A$.
Then, since $A \subseteq C$, we also know that $x \in C$.
Since $B$ and $C$ are disjoint, the fact that $x \in C$ means that $x \notin B$.
Therefore, we conclude that if $x \in A$ then $x \notin B$.
\end{proof}


\begin{statement}{3.2.4}
Suppose that $A \setminus B$ is disjoint from $C$ and $x \in A$.
Prove that if $x \in C$ then $x \in B$.
\end{statement}

\begin{proof}
We prove the equivalent contrapositive statement: if $x \notin B$ then $x \notin C$.
Suppose $x \notin B$.
Then, since by assumption $x \in A$, $x \in A \setminus B$.
Since $A \setminus B$ is disjoint from $C$, it follows that $x \notin C$.
Thus, if $x \notin B$ then $x \notin C$.
In other words, we conclude that if $x \in C$ then $x \in B$.
\end{proof}


\begin{statement}{3.2.5}
Prove that it cannot be the case that $x \in A \setminus B$ and $x \in B \setminus C$.
\end{statement}

\begin{proof}
Suppose $x \in A \setminus B$ and $x \in B \setminus C$.
Since $x \in A \setminus B$, it follows that $x \in A$ and $x \notin B$.
Similarly, since $x \in B \setminus C$, it follows that $x \in B$ and $x \in C$.
However, having $x \notin B$ and $x \in B$ is a contradiction.
Therefore, we conclude that it cannot be the case that
$x \in A \setminus B$ and $x \in B \setminus C$.
\end{proof}


\begin{statement}{3.2.6}
Use the method of proof by contradiction to prove the theorem in Example 3.2.1.
\begin{theorem}
	Suppose $A \cap C \subseteq B$ and $a \in C$.
	Prove that $a \notin A \setminus B$.
\end{theorem}
\end{statement}

\begin{proof}
Suppose, towards a contradiction, that $a \in A \setminus B$.
In other words, $a \in A$ and $a \notin B$.
Since also $a \in C$, we have $a \in A \cap C \subseteq B$.
Therefore, $a \in B$.
But this contradicts that $a \notin B$.
Thus, we conclude that $a \notin A \setminus B$.
\end{proof}


\begin{statement}{3.2.7}
Use the method of proof by contradiction to prove the theorem in Example 3.2.5.
\begin{theorem}
	Suppose $A \subseteq B$, $a  \in A$, and $a \notin B \setminus C$.
	Prove that $a \in C$.
\end{theorem}
\end{statement}

\begin{proof}
Suppose, towards a contradiction, that $a \notin C$.
Since $a \in A$ and $A \subseteq B$, it follows that $a \in B$.
Then, since $a \in B$ and $a \notin C$, $a \in B \setminus C$.
This, however, contradicts the fact that $a \notin B \setminus C$.
Therefore, we conclude that $a \in C$.
\end{proof}


\begin{statement}{3.2.8}
Suppose that $y + x = 2y - x$, and $x$ and $y$ are not both zero.
Prove that $y \neq 0$.
\end{statement}

\begin{proof}
Suppose, towards a contradiction, that $y = 0$.
Substituting $y = 0$ into the equation $y + x = 2y - x$ gives us $x = -x$, which implies that $x = 0$.
However, having $y = 0$ and $x = 0$ contradicts the fact that $x$ and $y$ are not both zero.
Thus, we conclude that $y \neq 0$.
\end{proof}


\begin{statement}{3.2.9}
Suppose that $a$ and $b$ are nonzero real numbers.
Prove that if $a < 1/a < b < 1/b$ then $a < -1$.
\end{statement}

\begin{proof}
Suppose $a < 1/a < b < 1/b$.
We then break the proof into two steps: proving that $a < 0$ and then proving that $a < -1$.
	\begin{lemma}
		$a < 0$
	\end{lemma}
	\begin{proof}
	Suppose, towards a contradiction, that $a \geq 0$.
	Since $a$ is nonzero, we can assume, without loss of generality, that $a > 0$.
	Then, since $0 < a < b$, we can apply Exercise 3.1.6 to conclude that $1/b < 1/a$.
	This, however, contradicts the fact that $1/a < 1/b$.
	Thus, we conclude that $a < 0$.
	\end{proof}
Now that we have proved that $a < 0$ in the above lemma, we move on to showing that $a < -1$.
We once again use proof by contradiction.
Suppose, towards a contradiction, that $a \geq -1$.
In other words, $-1 \leq a < 0$.
Since $a$ is negative, dividing both sides of the inequality $a \geq -1$ gives us $1 \leq -1/a$.
Rearranging this inequality results in $-1 \geq 1/a$.
Thus, we have $a \geq -1 \geq 1/a$, or most importantly, $a \geq 1/a$.
This however, contradicts the fact that $a < 1/a$, so we must instead have $a < -1$, as desired.
Therefore, we conclude that if $a < 1/a < b < 1/b$ then $a < -1$.
\end{proof}


\begin{statement}{3.2.10}
Suppose that $x$ and $y$ are real numbers.
Prove that if $x^2y = 2x + y$, then if $y \neq 0$ then $x \neq 0$.
\end{statement}

\begin{proof}
Suppose $x^2y = 2x + y$.
Now our goal is to prove that if $y \neq 0$ then $x \neq 0$.
We do this by proving the contrapositive statement: if $x = 0$ then $y = 0$.
Now suppose that $x = 0$.
Plugging this value for $x$ into $x^2y = 2x + y$ gives us $0 = y$, which is exactly what we wanted to show.
Thus, if $x = 0$ then $y = 0$.
In other words, if $y \neq 0$ then $x \neq 0$.
We then conclude that if $x^2y = 2x + y$, then if $y \neq 0$ then $x \neq 0$.
\end{proof}


\begin{statement}{3.2.11}
Suppose that $x$ and $y$ are real numbers.
Prove that if $x \neq 0$, then if $y = (3x^2 + 2y) / (x^2 + 2)$ then $y = 3$.
\end{statement}

\begin{proof}
Suppose $x \neq 0$, and then suppose $y = (3x^2 + 2y) / (x^2 + 2)$.
Multiplying both sides of the equation by $x^2 + 2$ gives us $x^2y + 2y = 3x^2 + 2y$.
After subtracting $2y$ from both sides, we are left with $x^2y = 3x^2$.
Since $x \neq 0$, $x^2 \neq 0$ as well, so we can divide both sides by $x^2$ to conclude that $y = 3$.
Therefore, we conclude that if $x \neq 0$, then if $y = (3x^2 + 2y) / (x^2 + 2)$ then $y = 3$.
\end{proof}


\begin{statement}{3.2.12}
Consider the following incorrect theorem.
\begin{theorem}
	Suppose $x$ and $y$ are real numbers and $x + y = 10$.
	Then $x \neq 3$ and $y \neq 8$.
\end{theorem}
\begin{enumerate}
	\item What's wrong with the following proof of the theorem?
	\begin{proof}
		Suppose the conclusion of the theorem is false.
		Then $x = 3$ and $y = 8$.
		But then $x + y = 11$, which contradicts the given information that $x + y = 10$.
		Therefore, the conclusion must be true.
	\end{proof}
	
	\item Show that the theorem is incorrect by finding a counterexample.
\end{enumerate}
\end{statement}

\begin{proof}
\hfill
\begin{enumerate}
	\item The proof incorrectly negates the conclusion when attempting to begin the proof
	by contradiction.
	The conclusion, that $x \neq 3$ and $y \neq 8$, has the logical form 
	$(x \neq 3) \wedge (y \neq 8)$, so its negation is $(x = 3) \vee (y = 8)$, by DeMorgan's laws.
	Thus, the correct negation of the conclusion is that $x = 3$ \textbf{or} $y = 8$.
	
	\item One counterexample is the case of $x = 3$ and $y = 7$.
	These are two real numbers such that $x + y = 10$, but they do not satisfy the conclusion
	that $x \neq 3$ \textbf{and} $y \neq 8$, as only $y \neq 8$.
\end{enumerate}
\end{proof}


\begin{statement}{3.2.13}
Consider the following incorrect theorem.
\begin{theorem}
	Suppose that $A \subseteq C$, $B \subseteq C$, and $x \in A$.
	Then $x \in B$.
\end{theorem}
\begin{enumerate}
	\item What's wrong with the following proof of the theorem?
	\begin{proof}
		Suppose that $x \notin B$.
		Since $x \in A$ and $A \subseteq C$, $x \in C$.
		Since $x \notin B$ and $B \subseteq C$, $x \notin C$.
		But now we have proven both $x \in C$ and $x \notin C$,
		so we have reached a contradiction.
		Therefore $x \in B$.
	\end{proof}
	
	\item Show that the theorem is incorrect by finding a counterexample.
\end{enumerate}
\end{statement}

\begin{proof}
\hfill
\begin{enumerate}
	\item The proof incorrectly concludes that since $x \notin B$ and $B \subseteq C$, $x \notin C$.
	This is an incorrect conclusion because $B$ may be a proper subset of $C$,
	or in other words $C \setminus B$ may be nonempty.
	We can also confirm the incorrectness of this conclusion by examining the logical form of
	the statement $B \subseteq C$, which is $\forall x (x \in B \rightarrow x \in C)$.
	This is equivalent to $\forall x (x \notin B \vee x \in C)$, which is still true in the case
	when there are elements $x$ such that $x \notin B$ and $x \in C$.
	
	\item Consider the case where $A = \{ 1 \}$, $B = \{ 2 \}$, $C = \{ 1, 2 \}$, and $x = 1$.
	Clearly $x \in A$, $A \subseteq C$, and $B \subseteq C$, but $x \notin B$.
\end{enumerate}
\end{proof}


\begin{statement}{3.2.14}
Use truth tables so show that modus tollens is a valid rule of inference.
\end{statement}

\begin{proof}
Recall that modus tollens says that if $P \rightarrow Q$ is true and $Q$ is false, then you can conclude that $P$ must also be false.
We now construct a truth table to evaluate the validity of this form of argument.
\begin{equation*}
	\begin{array}{| c c | c c | c |}
		P & Q & P \rightarrow Q & \neg Q & \neg P \\
		\hline
		T & T & T & F & F \\
		T & F & F & T & F \\
		F & T & T & F & T \\
		\rowcolor[HTML]{85C0F9} F & F & T & T & T
	\end{array}
\end{equation*}
This form of argument is valid because whenever all of the premises, $P \rightarrow Q$ and $\neg Q$, are true, the conclusion, $\neg P$, is also true, as indicated by the blue highlighted row.
\end{proof}


\begin{statement}{3.2.15}
Use truth tables to check the correctness of the theorem in Example 3.2.4.
\begin{theorem}
	Suppose $P \rightarrow (Q \rightarrow R)$.
	Prove that $\neg R \rightarrow (P \rightarrow \neg Q)$.
\end{theorem}
\end{statement}

\begin{proof}
\begin{equation*}
	\begin{array}{| c c c | c | c |}
		P & Q & R & P \rightarrow (Q \rightarrow R) & \neg R \rightarrow (P \rightarrow \neg Q) \\
		\hline
		\rowcolor[HTML]{85C0F9} T & T & T & T & T \\
		T & T & F & F & F \\
		\rowcolor[HTML]{85C0F9} T & F & T & T & T \\
		\rowcolor[HTML]{85C0F9} T & F & F & T & T \\
		\rowcolor[HTML]{85C0F9} F & T & T & T & T \\
		\rowcolor[HTML]{85C0F9} F & T & F & T & T \\
		\rowcolor[HTML]{85C0F9} F & F & T & T & T \\
		\rowcolor[HTML]{85C0F9} F & F & F & T & T \\
	\end{array}
\end{equation*}
This theorem is correct because whenever the premise, $P \rightarrow (Q \rightarrow R)$ is true, the conclusion $\neg R \rightarrow (P \rightarrow \neg Q)$ is also true, as indicated by the blue highlighted rows.

We can also see that the theorem is correct by comparing the logical form of the premise with the logical form of the conclusion.
The premise, $P \rightarrow (Q \rightarrow R)$, is equivalent to $\neg P \vee (\neg Q \vee R)$, or simply $\neg P \vee \neg Q \vee R$ by associativity of disjunction.
Similarly, the conclusion, $\neg R \rightarrow (P \rightarrow \neg Q)$ is equivalent to $R \vee (\neg P \vee \neg Q)$, which is equivalent to $\neg P \vee \neg Q \vee R$ by associativity and commutativity of disjunction.
Since the premise and conclusion are equivalent logical statements, the conclusion is true whenever the premise is true.
\end{proof}


\begin{statement}{3.2.16}
Use truth tables to check the correctness of the statements in Exercise 3.2.1.
\end{statement}

\begin{proof}
\hfill
\begin{enumerate}
	\item Recall that the statement was 
	``Suppose $P \rightarrow Q$ and $Q \rightarrow R$ are both true.
	Prove that $P \rightarrow R$ is true.''
	\begin{equation*}
		\begin{array}{| c c c | c  c | c |}
		P & Q & R & P \rightarrow Q &  Q \rightarrow R & P \rightarrow R\\
		\hline
		\rowcolor[HTML]{85C0F9} T & T & T & T & T & T \\
		T & T & F & T & F & F \\
		T & F & T & F & T & T \\
		T & F & F & F & T & F \\
		\rowcolor[HTML]{85C0F9} F & T & T & T & T & T \\
		F & T & F & T & F & T \\
		\rowcolor[HTML]{85C0F9} F & F & T & T & T & T \\
		\rowcolor[HTML]{85C0F9} F & F & F & T & T & T \\
		\end{array}
	\end{equation*}
	This statement is correct because whenever the premises, $P \rightarrow Q$ and
	$Q \rightarrow R$ are both true, the conclusion $P \rightarrow R$ is also true,
	as indicated by the blue highlighted rows.
	
	\item Recall that the statement was 
	``Suppose $\neg R \rightarrow (P \rightarrow \neg Q)$ is true.
	Prove that $P \rightarrow (Q \rightarrow R)$ is true.''
	\begin{equation*}
		\begin{array}{| c c c | c | c |}
			P & Q & R & \neg R \rightarrow (P \rightarrow \neg Q)
				& P \rightarrow (Q \rightarrow R) \\
			\hline
			\rowcolor[HTML]{85C0F9} T & T & T & T & T \\
			T & T & F & F & F \\
			\rowcolor[HTML]{85C0F9} T & F & T & T & T \\
			\rowcolor[HTML]{85C0F9} T & F & F & T & T \\
			\rowcolor[HTML]{85C0F9} F & T & T & T & T \\
			\rowcolor[HTML]{85C0F9} F & T & F & T & T \\
			\rowcolor[HTML]{85C0F9} F & F & T & T & T \\
			\rowcolor[HTML]{85C0F9} F & F & F & T & T \\
		\end{array}
	\end{equation*}
	This statement is correct because whenever the premise, 
	$\neg R \rightarrow (P \rightarrow \neg Q)$ is true, the conclusion 
	$P \rightarrow (Q \rightarrow R)$ is also true, as indicated by the blue highlighted rows.
	
	Our analysis from Exercise 3.2.15 also provides another perspective for the correctness
	of the statement, because the premise, $\neg R \rightarrow (P \rightarrow \neg Q)$,
	and the conclusion, $P \rightarrow (Q \rightarrow R)$, have equivalent logical forms.
\end{enumerate}
\end{proof}


\begin{statement}{3.2.17}
Use truth tables to check the correctness of the statements in Exercise 3.2.2.
\end{statement}

\begin{proof}
\hfill
\begin{enumerate}
	\item Recall that the statement was
	``Suppose $P \rightarrow Q$ and $R \rightarrow \neg Q$ are both true.
	Prove that $P \rightarrow \neg R$ is true.''
	\begin{equation*}
		\begin{array}{| c c c | c  c | c |}
		P & Q & R & P \rightarrow Q &  R \rightarrow \neg Q & P \rightarrow \neg R\\
		\hline
		T & T & T & T & F & F \\
		\rowcolor[HTML]{85C0F9} T & T & F & T & T & T \\
		T & F & T & F & T & F \\
		T & F & F & F & T & T \\
		F & T & T & T & F & T \\
		\rowcolor[HTML]{85C0F9} F & T & F & T & T & T \\
		\rowcolor[HTML]{85C0F9} F & F & T & T & T & T \\
		\rowcolor[HTML]{85C0F9} F & F & F & T & T & T \\
		\end{array}
	\end{equation*}
	This statement is correct because whenever the premises, $P \rightarrow Q$ and
	$R \rightarrow \neg Q$ are both true, the conclusion $P \rightarrow \neg R$ is also true,
	as indicated by the blue highlighted rows.
	
	\item Recall that the statement was 
	``Suppose that $P$ is true.
	Prove that $Q \rightarrow \neg (Q \rightarrow \neg P)$ is true.''
	\begin{equation*}
		\begin{array}{| c c | c | c |}
			P & Q & P & Q \rightarrow \neg (Q \rightarrow \neg P)\\
			\hline
			\rowcolor[HTML]{85C0F9} T & T & T & T \\
			\rowcolor[HTML]{85C0F9} T & F & T & T \\
			F & T & F & F \\
			F & F & F & T
		\end{array}
	\end{equation*}
	The statement is correct because whenever the premise $P$ is true, the conclusion
	$Q \rightarrow \neg (Q \rightarrow \neg P)$ is also true, as indicated by the blue
	highlighted rows.
	
	Note that we can more easily fill out the truth table by observing that the conclusion
	is logically equivalent to the simpler statement $\neg Q \vee P$.
\end{enumerate}
\end{proof}


\begin{statement}{3.2.18}
Can the proof in Example 3.2.2 be modified to prove that if $x^2 + y = 13$ and $x \neq 3$ then $y \neq 4$?
Explain.
Below is the theorem from Example 3.2.2 and its proof.
\begin{theorem}
	If $x^2 + y = 13$ and $y \neq 4$ then $x \neq 3$.
\end{theorem}
\begin{proof}
	Suppose $x^2 + y = 13$ and $y \neq 4$.
	Suppose $x = 3$.
	Substituting this into the equation $x^2 + y = 13$, we get $9 + y = 13$, so $y = 4$.
	But this contradicts the fact that $y \neq 4$.
	Therefore $x \neq 3$.
	Thus, if $x^2 + y = 13$ and $y \neq 4$ then $x \neq 3$.
\end{proof}
\end{statement}

\begin{proof}
The proof in Example 3.2.2 cannot be modified to prove that if $x^2 + y = 13$ and $x \neq 3$ then $y \neq 4$?
This is because the statement ``If $x^2 + y = 13$ and $x \neq 3$ then $y \neq 4$," is false.
To see why, consider the case when $x = -3$, which satisfies the hypothesis that $x \neq 3$.
In this situation, when also $x^2 + y = 13$, we can plug in $x = -3$ to obtain the equation $9 + y = 13$.
From the equation $9 + y = 13$ it follows that $y = 4$, which does not satisfy the conclusion $y \neq 4$.
Since we have found a situation where the premises of the statement are true, but the conclusion is false, the statement itself is false.
\end{proof}


\begin{statement}{3.3.1}
In Exercise 2.2.7 you used logical equivalences to show that $\exists x (P(x) \rightarrow Q(x))$ is equivalent to $\forall x P(x) \rightarrow \exists x Q(x)$.
Now use the methods of this section to prove that if $\exists x (P(x) \rightarrow Q(x))$ is true, then $\forall x P(x) \rightarrow \exists x Q(x)$ is true.
(\emph{Note: The other direction of the equivalence is quite a bit harder to prove.
See Exercise 3.5.30.})
\end{statement}

\begin{proof}
Suppose that $\exists x (P(x) \rightarrow Q(x))$ is true.
Now our goal is to prove that the statement $\forall x P(x) \rightarrow \exists x Q(x)$ is true.
We then suppose that $\forall x P(x)$ is true and update our goal to proving that $\exists x Q(x)$ is true.
Since $\exists x (P(x) \rightarrow Q(x))$ is true, there exists $x = x_0$ such that $P(x_0) \rightarrow Q(x_0)$ is true.
Then, since $\forall x P(x)$ is true, the statement $P(x_0)$, in particular, is true.
Because $P(x_0) \rightarrow Q(x_0)$ is true and $P(x_0)$ is true, we conclude that $Q(x_0)$ is true as well.
In other words, we have found that there exists $x$ such that $Q(x)$ is true.
Thus, the statement $\forall x P(x) \rightarrow \exists x Q(x)$ is true, and we ultimately conclude that if $\exists x (P(x) \rightarrow Q(x))$ is true, then $\forall x P(x) \rightarrow \exists x Q(x)$ is true as well.
This completes the proof.
\end{proof}


\begin{statement}{3.3.2}
Prove that if $A$ and $B \setminus C$ are disjoint, then $A \cap B \subseteq C$.
\end{statement}

\begin{proof}
Suppose $A$ and $B \setminus C$ are disjoint.
Let $x \in A \cap B$ be arbitrary.
We now want to show that $x \in C$.
Suppose, towards a contradiction, that $x \notin C$.
Since $x \in A \cap B$, $x \in A$ and $x \in B$.
Then, since $x \in B$ and $x \notin C$, $x \in B \setminus C$.
It then follows that $x \in A \cap (B \setminus C)$, but this contradicts the assumption that $A$ and $B \setminus C$ are disjoint.
Thus, $x \in C$, so $A \cap B \subseteq C$.
Therefore, we conclude that if $A$ and $B \setminus C$ are disjoint, then $A \cap B \subseteq C$.
\end{proof}


\begin{statement}{3.3.3}
Prove that if $A \subseteq B \setminus C$ then $A$ and $C$ are disjoint.
\end{statement}

\begin{proof}
Suppose $A \subseteq B \setminus C$ and suppose, towards a contradiction that $A$ and $C$ are not disjoint.
In other words, there is $x \in A \cap C$.
Since $x \in A$ and $A \subseteq B \setminus C$, it follows that $x \in B \setminus C$.
In particular, $x \notin C$.
However, this contradicts the fact that $x \in A \cap C$ means that $x \in C$.
Thus, we conclude that $A$ and $C$ are disjoint.
Therefore, if $A \subseteq B \setminus C$ then $A$ and $C$ are disjoint.
\end{proof}


\begin{statement}{3.3.4}
Suppose $A \subseteq \powerset{A}$.
Prove that $\powerset{A} \subseteq \powerset{\powerset{A}}$.
\end{statement}

\begin{proof}
Let $X \in \powerset{A}$ be arbitrary.
By the definition of the power set of a set $A$, $X \subseteq A$.
Then, since $A \subseteq \powerset{A}$, it follows that $X \subseteq \powerset{A}$.
In other words, $X$ is an element of the power set of $\powerset{A}$.
Since $X$ was arbitrary, we conclude that $\powerset{A} \subseteq \powerset{\powerset{A}}$.
\end{proof}


\begin{statement}{3.3.5}
The hypothesis of the theorem in Exercise 3.3.4 is $A \subseteq \powerset{A}$.
\begin{enumerate}
	\item Can you think of a set $A$ for which this hypothesis is true?
	
	\item Can you think of another?
\end{enumerate}
\end{statement}

\begin{proof}
\hfill
\begin{enumerate}
	\item Since the empty set $\varnothing$ is a subset of any set $X$, it is a set for which the
	hypothesis $A \subseteq \powerset{A}$ is true.
	
	\item Once we observe that $A = \varnothing$ satisfies the hypothesis
	$A \subseteq \powerset{A}$, the theorem tells us that 
	$\powerset{A} \subseteq \powerset{\powerset{A}}$.
	In other words, $\powerset{A}$ is another set which satisfies the hypothesis.
	We can check this directly by noting that $\powerset{\varnothing} = \{ \varnothing \}$,
	$\powerset{\{ \varnothing \}} = \{ \varnothing, \{ \varnothing \} \}$,
	and $\{ \varnothing \} \subseteq \{ \varnothing, \{ \varnothing \} \}$.
\end{enumerate}
\end{proof}


\begin{statement}{3.3.6}
Suppose $x$ is a real number.
\begin{enumerate}
	\item Prove that if $x \neq 1$ then there is a real number $y$ such that $(y + 1)/(y - 2) = x$.
	
	\item Prove that if there is a real number $y$ such that $(y + 1)/(y - 2) = x$, then $x \neq 1$.
\end{enumerate}
\end{statement}

\begin{proof}
\hfill
\begin{enumerate}
	\item Suppose $x \neq 1$.
	Consider $y = (2x + 1)/(x - 1)$, which is a well-defined real number since $x \neq 1$.
	We then plug this value of $y$ into the expression $(y + 1)/(y - 2)$.
	\begin{equation*}
		\frac{y + 1}{y - 2}
		= \frac{\frac{2x + 1}{x - 1} + 1}{\frac{2x + 1}{x - 1} - 2}
		= \frac{2x + 1 + (x - 1)}{2x + 1 - 2(x - 1)}
		= \frac{3x}{3}
		= x
	\end{equation*}
	Thus, we have found a real number $y$ such that $(y + 1)/(y - 2) = x$.
	We conclude that that if $x \neq 1$ then there is a real number 
	$y$ such that $(y + 1)/(y - 2) = x$.
	
	\item Suppose there is a real number $y$ such that $(y + 1)/(y - 2) = x$.
	Since $y + 1 \neq y - 2$ for all real numbers $y$, it follows that $x = (y + 1)/(y - 2) \neq 1$.
	Therefore, we conclude that if there is a real number $y$ such that 
	$(y + 1)/(y - 2) = x$, then $x \neq 1$.
\end{enumerate}
\end{proof}


\begin{statement}{3.3.7}
Prove that for every real number $x$, if $x > 2$ then there is a real number $y$ such that $y + 1/y = x$.
\end{statement}

\begin{proof}
Let $x > 2$ be an arbitrary real number.
Consider $y = (x + \sqrt{x^2 - 4})/2$, which is a real number since $x > 2$.
We then plug this value of $y$ into the expression $y + 1/y$.
\begin{align*}
	y + \frac{1}{y}
	&= \frac{x + \sqrt{x^2 - 4}}{2} + \frac{2}{x + \sqrt{x^2 - 4}} \\
	&= \frac{x^2 + 2x\sqrt{x^2 - 4} + (x^2 - 4)}{2(x + \sqrt{x^2 - 4})}
		+ \frac{4}{2(x + \sqrt{x^2 - 4})} \\
	&= \frac{2x^2 + 2x\sqrt{x^2 - 4}}{2(x + \sqrt{x^2 - 4})} \\
	&= \frac{2x(x + \sqrt{x^2 - 4})}{2(x + \sqrt{x^2 - 4})} \\
	&= x
\end{align*}
Thus, we have found a real number $y$ such that $y + 1/y = x$.
Therefore, we conclude that for every real number $x$, if $x > 2$ then there is a real number $y$ such that $y + 1/y = x$.
\end{proof}


\begin{statement}{3.3.8}
Prove that if $\mathcal{F}$ is a family of sets and $A \in \mathcal{F}$, then $A \subseteq \bigcup \mathcal{F}$.
\end{statement}

\begin{proof}
Suppose $\mathcal{F}$ is a family of sets and $A \in \mathcal{F}$.
Let $a \in A$ be arbitrary.
We want to show that $a \in \bigcup \mathcal{F}$.
Since $\bigcup \mathcal{F} = \{ x \mid \exists X \in \mathcal{F} (x \in X) \}$, showing that $a \in \bigcup \mathcal{F}$ is equivalent to showing that there is $X \in \mathcal{F}$ such that $a \in X$.
Let $X = A$.
By assumption, $A \in \mathcal{F}$ and $a \in A$.
Thus, we have found $X \in \mathcal{F}$ such that $a \in X$, so $a \in \bigcup \mathcal{F}$.
Since $a \in A$ was arbitrary, we conclude that $A \subseteq \bigcup \mathcal{F}$.
Therefore, if $\mathcal{F}$ is a family of sets and $A \in \mathcal{F}$, then $A \subseteq \bigcup \mathcal{F}$.
\end{proof}


\begin{statement}{3.3.9}
Prove that if $\mathcal{F}$ is a family of sets and $A \in \mathcal{F}$, then $\bigcap \mathcal{F} \subseteq A$.
\end{statement}

\begin{proof}
Suppose $\mathcal{F}$ is a family of sets and $A \in \mathcal{F}$.
Let $x \in \bigcap \mathcal{F}$ be arbitrary.
Recall that $\bigcap \mathcal{F} = \{ x \mid \forall X \in \mathcal{F} (x \in X) \}$.
In particular, since $A \in \mathcal{F}$, it follows that $x \in A$.
Since $x \in \bigcap \mathcal{F}$ was arbitrary, we conclude that $\bigcap \mathcal{F} \subseteq A$.
Therefore, if $\mathcal{F}$ is a family of sets and $A \in \mathcal{F}$, then $\bigcap \mathcal{F} \subseteq A$.
\end{proof}


\begin{statement}{3.3.10}
Suppose that $\mathcal{F}$ is a nonempty family of sets, $B$ is a set, and for all $A \in \mathcal{F}$ that $B \subseteq A$.
Prove that $B \subseteq \bigcap \mathcal{F}$.
\end{statement}

\begin{proof}
Let $b \in B$ be arbitrary.
We want to show that $b \in \bigcap \mathcal{F}$.
Since $\bigcap \mathcal{F} = \{ x \mid \forall A \in \mathcal{F} (x \in A) \}$, showing that $b \in \bigcap \mathcal{F}$ is equivalent to showing that $b \in A$ for all $A \in \mathcal{F}$.
Because $B \subseteq A$ for all $A \in \mathcal{F}$ and $b \in B$, it follows that $b \in A$ for all $A \in \mathcal{F}$.
In other words, $b \in \bigcap \mathcal{F}$.
Since $b \in B$ was arbitrary, we conclude that $B \subseteq \bigcap \mathcal{F}$.
This completes the proof.
\end{proof}


\begin{statement}{3.3.11}
Suppose that $\mathcal{F}$ is a family of sets.
Prove that if $\varnothing \in \mathcal{F}$ then $\bigcap \mathcal{F} = \varnothing$.
\end{statement}

\begin{proof}
Suppose $\varnothing \in \mathcal{F}$, and suppose, towards a contradiction, that $\bigcap \mathcal{F} \neq \varnothing$.
In other words, there exists $x \in \bigcap \mathcal{F}$.
By the definition of $\bigcap \mathcal{F}$, this means that $x \in X$ for all $X \in \mathcal{F}$.
In particular, since $X = \varnothing \in \mathcal{F}$, this means that $x \in \varnothing$.
This, however, is a contradiction against the definition of the empty set.
We therefore conclude that $\bigcap \mathcal{F} = \varnothing$.
Thus, if $\varnothing \in \mathcal{F}$ then $\bigcap \mathcal{F} = \varnothing$.
\end{proof}


\begin{statement}{3.3.12}
Suppose $\mathcal{F}$ and $\mathcal{G}$ are families of sets.
Prove that if $\mathcal{F} \subseteq \mathcal{G}$ then $\bigcup \mathcal{F} \subseteq \bigcup \mathcal{G}$.
\emph{Note: In my digital copy of the book this exercise has a typo and asks for a proof of the incorrect statement that if $\mathcal{F} \subseteq \mathcal{G}$ then $\bigcup \mathcal{F} \subseteq \mathcal{G}$.}
\end{statement}

\begin{proof}
Suppose $\mathcal{F} \subseteq \mathcal{G}$.
Let $x \in \bigcup \mathcal{F}$ be arbitrary.
By the definition of $\bigcup \mathcal{F}$, this means that there is $X \in \mathcal{F}$ such that $x \in X$.
Then, since $\mathcal{F} \subseteq \mathcal{G}$, it follows that $X \in \mathcal{G}$ as well.
Since $x \in X$ and $X \in \mathcal{G}$, $x \in \bigcup \mathcal{G}$ by the definition of $\bigcup \mathcal{G}$.
Thus, since $x \in \bigcup \mathcal{F}$ was arbitrary, we conclude that $\bigcup \mathcal{F} \subseteq \bigcup \mathcal{G}$.
Therefore, if $\mathcal{F} \subseteq \mathcal{G}$ then $\bigcup \mathcal{F} \subseteq \bigcup \mathcal{G}$.
\end{proof}


\begin{statement}{3.3.13}
Suppose $\mathcal{F}$ and $\mathcal{G}$ are nonempty families of sets.
Prove that if $\mathcal{F} \subseteq \mathcal{G}$ then $\bigcap \mathcal{G} \subseteq \bigcap \mathcal{F}$.
\end{statement}

\begin{proof}
Suppose $\mathcal{F} \subseteq \mathcal{F}$.
Let $x \in \bigcap \mathcal{G}$ be arbitrary.
By the definition of $\bigcap \mathcal{G}$, this means that for all $X \in \mathcal{G}$, $x \in X$.
Note that the assumption that $\mathcal{G}$ is nonempty makes this a non-vacuous statement.
Now, let $Y \in \mathcal{F}$ be arbitrary.
Note that such a $Y$ exists by the assumption that $\mathcal{F}$ is nonempty.
Since $\mathcal{F} \subseteq \mathcal{G}$, $Y \in \mathcal{G}$.
Because $Y \in \mathcal{G}$ and for all $X \in \mathcal{G}$, $x \in X$, it follows that $x \in Y$.
Therefore, since $Y \in \mathcal{F}$ was arbitrary, $x \in Y$ for all $Y \in \mathcal{F}$.
In other words, $x \in \bigcap \mathcal{F}$ by the definition of $\bigcap \mathcal{F}$.
Since $x \in \bigcap \mathcal{G}$ was arbitrary, we conclude that $\bigcap \mathcal{G} \subseteq \bigcap \mathcal{F}$.
Thus, if $\mathcal{F} \subseteq \mathcal{G}$ then $\bigcap \mathcal{G} \subseteq \bigcap \mathcal{F}$.
\end{proof}


\begin{statement}{3.3.14}
Suppose that $\{ A_i \mid i \in I \}$ is an indexed family of sets.
Prove that $\bigcup_{i \in I} \powerset{A_i} \subseteq \powerset{\bigcup_{i \in I} A_i}$.
(\emph{Hint: First make sure you know what all the notation means!})
\end{statement}

\begin{proof}
Let $X \in \bigcup_{i \in I} \powerset{A_i}$ be arbitrary.
By the definition of $\bigcup_{i \in I} \powerset{A_i}$, this means that there is $i \in I$ such that $X \in \powerset{A_i}$.
Further unraveling the definition of being an element of the power set of a set $A$, this means that there is $i \in I$ such that $X \subseteq A_i$.
Since $X \subseteq A_i$ and $A_i \subseteq \bigcup_{i \in I} A_i$, it follows that $X \subseteq \bigcup_{i \in I} A_i$.
In other words, $X \in \powerset{\bigcup_{i \in I} A_i}$.
Since $X \in \bigcup_{i \in I} \powerset{A_i}$ was arbitrary, we conclude that $\bigcup_{i \in I} \powerset{A_i} \subseteq \powerset{\bigcup_{i \in I} A_i}$.
This completes the proof.
\end{proof}


\begin{statement}{3.3.15}
Suppose $\{ A_i \mid i \in I \}$ is an indexed family of sets and $I \neq \varnothing$.
Prove that $\bigcap_{i \in I} A_i \in \bigcap_{i \in I} \powerset{A_i}$.
\emph{Note: In my digital copy of the book this exercise has a typo and says that $I = \varnothing$.}
\end{statement}

\begin{proof}
We start by unraveling the definitions of what it means to be an element of $\bigcap_{i \in I} \powerset{A_i}$.
The first layer of definitions states that $X \in \bigcap_{i \in I} \powerset{A_i}$ if and only if $X \in \powerset{A_i}$ for all $i \in I$.
Unraveling one step further, this is equivalent to $X \subseteq A_i$ for all $i \in I$, by the definition of the power set of a set $A$.
In other words, proving that $\bigcap_{i \in I} A_i \in \bigcap_{i \in I} \powerset{A_i}$ is equivalent to proving that $\bigcap_{i \in I} A_i \subseteq A_i$ for all $i \in I$.
This follows directly from the definition of $\bigcap_{i \in I} A_i$.
Letting $x \in \bigcap_{i \in I} A_i$ be arbitrary, $x \in A_i$ for all $i \in I$.
Therefore, $\bigcap_{i \in I} A_i \subseteq A_i$ for all $i \in I$.
This completes the proof that $\bigcap_{i \in I} A_i \in \bigcap_{i \in I} \powerset{A_i}$.
\end{proof}


\begin{statement}{3.3.16}
Prove the converse of the statement proven in Example 3.3.5.
In other words, prove that if $\mathcal{F} \subseteq \powerset{B}$ then $\bigcup \mathcal{F} \subseteq B$.
\end{statement}

\begin{proof}
Suppose $\mathcal{F} \subseteq \powerset{B}$.
Let $x \in \bigcup \mathcal{F}$ be arbitrary.
By the definition of $\bigcup \mathcal{F}$, this means that there is $X \in \mathcal{F}$ such that $x \in X$.
Then, since $\mathcal{F} \subseteq \powerset{B}$, $X \in \powerset{B}$.
In other words, by the definition of the power set of $B$, $X \subseteq B$.
It then follows that $x \in B$.
Since $x \in \bigcup \mathcal{F}$ was arbitrary, we conclude that $\bigcup \mathcal{F} \subseteq B$.
Therefore, we conclude that if $\mathcal{F} \subseteq \powerset{B}$ then $\bigcup \mathcal{F} \subseteq B$.
\end{proof}


\begin{statement}{3.3.17}
Suppose $\mathcal{F}$ and $\mathcal{G}$ are nonempty families of sets, and every element of $\mathcal{F}$ is a subset of every element of $\mathcal{G}$.
Prove that $\bigcup \mathcal{F} \subseteq \bigcap \mathcal{G}$.
\end{statement}

\begin{proof}
Let $x \in \bigcup \mathcal{F}$ be arbitrary.
By the definition of $\bigcup \mathcal{F}$, this means that there is $X \in \mathcal{F}$ such that $x \in X$.
Then, by the assumption that every element of $\mathcal{F}$ is a subset of every element of $\mathcal{G}$, we know that for all $Y \in \mathcal{G}$, $X \subseteq Y$.
It follows that $x \in Y$ for all $Y \in \mathcal{G}$.
In other words, by the definition of $\bigcap \mathcal{G}$, $x \in \bigcap \mathcal{G}$.
Hence, since $x \in \bigcup \mathcal{F}$ was arbitrary, we conclude that $\bigcup \mathcal{F} \subseteq \bigcap \mathcal{G}$.
\end{proof}


\begin{statement}{3.3.18}
In this problem all variables range over $\BZ$, the set of all integers.
\begin{enumerate}
	\item Prove that if $a$ divides $b$ and $a$ divides $c$, then $a$ divides $b + c$.
	
	\item Prove that if $ac$ divides $bc$ and $c \neq 0$, then $a$ divides $b$.
\end{enumerate}
\end{statement}

\begin{proof}
\hfill
\begin{enumerate}
	\item Suppose $a$ divides $b$ and $a$ divides $c$.
	Since $a$ divides $b$, there is $m \in \BZ$ such that $ma = b$.
	Similarly, since $a$ divides $c$, there is $n \in \BZ$ such that $na = c$.
	It then follows that $b + c = ma + na = (m + n)a$.
	Since $m + n$ is an integer, we conclude that $a$ divides $b + c$.
	Therefore, if $a$ divides $b$ and $a$ divides $c$, then $a$ divides $b + c$.
	
	\item Suppose $ac$ divides $bc$ and $c \neq 0$.
	Since $ac$ divides $bc$, there is $m \in \BZ$ such that $mac = bc$.
	Then, since $c \neq 0$, we can divide both sides of the equation 
	by $c$ to conclude that $ma = b$.
	In other words, $a$ divides $b$.
	Therefore, we conclude that if $ac$ divides $bc$ and $c \neq 0$, then $a$ divides $b$.
\end{enumerate}
\end{proof}


\begin{statement}{3.3.19}
\begin{enumerate}
	\item Prove that for all real numbers $x$ and $y$ there is a real number $z$ such that
	$x + z = y - z$.
	
	\item Would the statement in Part (1) be correct if ``real number'' were changed to ``integer''?
	Justify your answer.
\end{enumerate}
\end{statement}

\begin{proof}
\hfill
\begin{enumerate}
	\item Let $x, y \in \BR$ be arbitrary. 
	Consider the real number $z = (y - x) / 2$.
	We plug this value for $z$ into the expressions $x + z$ and $y - z$ to check that they are equal.
	\begin{align*}
		x + z &= x + \frac{y - x}{2} = \frac{x + y}{2} \\
		y - z &= y - \frac{y - x}{2} = \frac{x + y}{2}
	\end{align*}
	As we can see, $x + z = (x + y) / 2 = y - z$.
	This completes the proof.
	
	\item The statement in Part (1) would be incorrect if ``real number'' were changed to ``integer''.
	This comes from the way in which $z$ is defined by 
	isolating $z$ in the equation $x + z = y - z$ to get $z = (y - x) / 2$.
	The difference of two real numbers is not guaranteed to be an integer,
	let alone an integer that is divisible by 2.
	For example, consider the case of $x = 0$ and $y = 1$.
	In this case, the proof gives us a value of $z = 1/2$, which is not an integer.
\end{enumerate}
\end{proof}


\begin{statement}{3.3.20}
Consider the following theorem:
\begin{theorem}
	For every real number $x$, $x^2 \geq 0$.
\end{theorem}
What's wrong with the following proof of the theorem?
\begin{proof}
	Suppose not.
	Then for every for every real number $x$, $x^2 < 0$.
	In particular, plugging in $x = 3$ we would get $9 < 0$, which is clearly false.
	This contradiction shows that for every number $x$, $x^2 \geq 0$.
\end{proof}
\end{statement}

\begin{proof}
The issue with the given attempted proof is in the negation of the statement ``For every real number $x$, $x^2 \geq 0$,'' to try and reach a contradiction.
Since the statement has the logical form $\forall x P(x)$, the correct negation has the logical form $\exists x \neg P(x)$.
In other words, the correct negation is ``There exists a real number $x$ such that $x^2 < 0$,'' instead of ``For every real number $x$, $x^2 < 0$.''
\end{proof}


\begin{statement}{3.3.21}
Consider the following incorrect theorem:
\begin{theorem}
	If $\forall x \in A (x \neq 0)$ and $A \subseteq B$ then $\forall x \in B (x \neq 0)$.
\end{theorem}
\begin{enumerate}
	\item What's wrong with the following proof of the theorem?
	\begin{proof}
		Suppose that $\forall x \in A (x \neq 0)$ and $A \subseteq B$.
		Let $x$ be an arbitrary element of $A$.
		Since $\forall x \in A (x \neq 0)$, we can conclude that $x \neq 0$.
		Also, since $A \subseteq B$, $x \in B$.
		Since $x \in B$, $x \neq 0$, and $x$ was arbitrary, 
		we can conclude that $\forall x \in B (x \neq 0)$.
	\end{proof}
	
	\item Find a counterexample to the theorem.
	In other words, find an example of sets $A$ and $B$ for which the hypotheses of the theorem
	are true but the conclusion is false.
\end{enumerate}
\end{statement}

\begin{proof}
\hfill
\begin{enumerate}
	\item The issue with the attempted proof is in the statement
	``Let $x$ be an arbitrary element of $A$''.
	The goal of the proof is to prove that for all $x \in B$, $x \neq 0$,
	so we must instead start with an arbitrary element of $B$.
	
	\item One counterexample to the theorem is when $A = \{ 1, 2, 3 \}$ and $B = \{ 0, 1, 2, 3 \}$.
	In this case, all elements of $A$ are nonzero and $A \subseteq B$, 
	but not all elements of $B$ are nonzero.
\end{enumerate}
\end{proof}


\begin{statement}{3.3.22}
Consider the following incorrect theorem:
\begin{theorem}
	$\exists x \in \BR \forall y \in \BR (xy^2 = y - x)$.
\end{theorem}
What's wrong with the following proof of the theorem?
\begin{proof}
	Let $x = y / (y^2 + 1)$. Then
	\begin{equation*}
		y - x
		= y - \frac{y}{y^2 + 1}
		= \frac{y^3}{y^2 + 1}
		= \frac{y}{y^2 + 1} \cdot y^2
		= xy^2.
	\end{equation*}
\end{proof}
\end{statement}

\begin{proof}
The issue in the proof is that this value of $x$ depends on the value of $y$.
This is different from what the theorem states: that there is a single value of $x$, \textbf{independent} of the value of $y$, such that $xy^2 = y - x$ for all values of $y$.
In terms of proof structure, we need to start by defining the real number $x$, which at this point cannot be defined in terms of $y$ since $y$ hasn't been introduced in the proof.
Only after $x$ is defined, would we take an arbitrary $y \in \BR$ and attempt to show that $xy^2 = y - x$.
\end{proof}


\begin{statement}{3.3.23}
Consider the following incorrect theorem:
\begin{theorem}
	Suppose $\mathcal{F}$ and $\mathcal{G}$ are families of sets.
	If $\bigcup \mathcal{F}$ and $\bigcup \mathcal{G}$ are disjoint,
	then so are $\mathcal{F}$ and $\mathcal{G}$.
\end{theorem}
\begin{enumerate}
	\item What's wrong with the following proof of the theorem?
	\begin{proof}
		Suppose $\bigcup \mathcal{F}$ and $\bigcup \mathcal{G}$ are disjoint.
		Suppose $\mathcal{F}$ and $\mathcal{G}$ are not disjoint.
		Then we can choose some set $A$ such that $A \in \mathcal{F}$ and $A \in \mathcal{G}$.
		Since $A \in \mathcal{F}$, by Exercise 3.3.8, $A \subseteq \bigcup \mathcal{F}$,
		so every element of $A$ is in $\bigcup \mathcal{F}$.
		Similarly, since $A \in \mathcal{G}$, every element of $A$ is in $\bigcup \mathcal{G}$.
		But then every element of $A$ is in both $\bigcup \mathcal{F}$ and $\bigcup \mathcal{G}$,
		and this is impossible since $\bigcup \mathcal{F}$ and $\bigcup \mathcal{G}$ are disjoint.
		Thus, we have reached a contradiction, 
		so $\mathcal{F}$ and $\mathcal{G}$ must be disjoint.
	\end{proof}
	% How do we know that A is nonempty?
	
	\item Find a counterexample to the theorem.
	% Choose F and G such that their intersection is the set containing the empty set.
\end{enumerate}
\end{statement}

\begin{proof}
\hfill
\begin{enumerate}
	\item The issue with the attempted proof lies in the statement
	``Then we can choose some set $A$ such that $A \in \mathcal{F}$ and $A \in \mathcal{G}$''.
	We do not know that $A$ is nonempty since there are no assumptions as to whether or not
	$\varnothing \in \mathcal{F} \cap \mathcal{G}$.
	If $A = \varnothing$, then there is no contradiction to having 
	$A \in \mathcal{F} \cap \mathcal{G}$ and having 
	$\bigcup \mathcal{F}$ and $\bigcup \mathcal{G}$ being disjoint,
	since $A$ contains no elements.
	
	\item One counterexample to the theorem is the case where 
	$\mathcal{F} = \{ \varnothing, \{ 1 \} \}$ and
	$\mathcal{G} = \{ \varnothing, \{ 2 \} \}$.
	In this case, $\bigcup \mathcal{F} = \{ 1 \}$ and $\bigcup \mathcal{G} = \{ 2 \}$,
	which are clearly disjoint, but $\mathcal{F} \cap \mathcal{G} = \{ \varnothing \}$.
	As discussed previously, the set containing the empty set is \textbf{not} empty.
\end{enumerate}
\end{proof}


\begin{statement}{3.3.24}
Consider the following putative theorem:
\begin{theorem}
	For all real numbers $x$ and $y$, $x^2 + xy - 2y^2 = 0$.
\end{theorem}
\begin{enumerate}
	\item What's wrong with the following proof of the theorem?
	\begin{proof}
		Let $x$ and $y$ be equal to some arbitrary real number $r$.
		Then
		\begin{equation*}
			x^2 + xy - 2y^2 = r^2 + r \cdot r - 2r^2 = 0.
		\end{equation*}
		Since $x$ and $y$ were both arbitrary, this shows that for all real number
		$x$ and $y$, $x^2 + xy - 2y^2 = 0$.
	\end{proof}
	% There is no assumption that x = y.
	
	\item Is the theorem correct?
	Justify your answer with either a proof or a counterexample.
	% Theorem is incorrect. Choose x = 1, y = 0 for a counterexample.
\end{enumerate}
\end{statement}

\begin{proof}
\hfill
\begin{enumerate}
	\item The issue with the attempted proof lies in the statement
	``Let $x$ and $y$ be equal to some arbitrary real number $r$'',
	since there is no assumption in the theorem statement that $x = y$.
	Instead, we must assume that $x$ is an arbitrary real number and that $y$ is an arbitrary real
	number, while not making any further assumptions about the relationship between $x$ and $y$.
	
	\item The theorem is incorrect.
	Consider the case where $x = 1$ and $y = 0$.
	In this case, $x^2 + xy - 2y^2 = 1^2 + (1)(0) - 2(0)^2 = 1 \neq 0$.
\end{enumerate}
\end{proof}


\begin{statement}{3.3.25}
Prove that for every real number $x$ there is a real number $y$ such that for every real number $z$, $yz = (x + z)^2 - (x^2 + z^2)$.
\end{statement}

\begin{proof}
Let $x \in \BR$ be arbitrary.
Let $y = 2x$.
Now let $z \in \BR$ be arbitrary.
We then have $yz = (2x)z = 2xz$.
We also have $(x + z)^2 - (x^2 + z^2) = x^2 + 2xz + z^2 - x^2 - z^2 = 2xz$.
Thus, $yz = 2xz = (x + z)^2 - (x^2 + z^2)$.
This completes the proof.

\emph{
Note that in order to properly prove the statement, we must be careful about the order in which we perform the setup.
We first have to let $x \in \BR$ be arbitrary.
Once we do that, we then define $y$.
When defining $y$, the only variable we can use is $x$, since that is the only other variable that has been introduced at this point.
Only after we have defined $y$ can we let $z \in \BR$ be arbitrary.
One way we can think about the order of introducing variables is that while the value of $y$ might depend on the choice of $x$, once the values of $x$ and $y$ have been set, then the rest of the statement has to hold true for any value of $z$.
}
\end{proof}


\begin{statement}{3.3.26}
\begin{enumerate}
	\item Comparing the various rules for dealing with quantifiers in proofs,
	you should see a similarity between the rules for goals of the form $\forall x P(x)$
	and givens of the form $\exists x P(x)$.
	What is this similarity?
	What about the rules for goals of the form $\exists x P(x)$ and givens of the form $\forall x P(x)$?
	
	\item Can you think of a reason why these similarities might be expected?
	(\emph{Hint: Think about how a proof by contradiction works when
	the goal starts with a quantifier.})
\end{enumerate}
\end{statement}

\begin{proof}
\hfill
\begin{enumerate}
	\item The similarity between the rules for goals of the form $\forall x P(x)$
	and givens of the form $\exists x P(x)$ is that both involve introduce a variable $x$.
	When working toward a goal of the form $\forall x P(x)$, we introduce an arbitrary $x$
	and try to prove $P(x)$, while when working with a given $\exists x P(x)$,
	we introduce an arbitrary $x$ that satisfies $P(x)$.
	For goals of the form $\exists x P(x)$ and givens of the form $\forall x P(x)$, we instead
	try to find or construct a particular value $a$ to plug in for $x$.
	When working toward a goal of the form $\exists x P(x)$, we construct a particular value $a$
	such that $P(a)$ is true, while when working with a given $\forall x P(x)$, we try to find
	a relevant value $a$ to then immediately plug into the statement $P$ and conclude
	that $P(a)$ is true.
	
	\item One reason why these similarities might be expected comes from the quantifier
	negation laws: $\neg \forall x P(x)$ is equivalent to $\exists x \neg P(x)$
	and $\neg \exists x P(x)$ is equivalent to $\forall x \neg P(x)$.
	If we are performing proof by contradiction when the goal starts with a quantifier,
	we use those negation laws when we add the negated goal as a given and then
	attempt to work toward a contradiction.
	In other words, a goal of the form $\forall x P(x)$ would turn into a given of the form
	$\exists x \neg P(x)$, while a goal of the form $\exists x P(x)$ would turn into a given
	of the form $\forall x \neg P(x)$.
\end{enumerate}
\end{proof}


\begin{statement}{3.4.1}
Use the methods of this chapter to prove that $\forall x (P(x) \wedge Q(x))$ is equivalent to $\forall x P(x) \wedge \forall x Q(x)$.
\end{statement}

\begin{proof}
We first prove the forward direction: $\forall x (P(x) \wedge Q(x)) \rightarrow \forall x P(x) \wedge \forall x Q(x)$.
Suppose $\forall x (P(x) \wedge Q(x))$ is true.
Now let $x_1$ and $x_2$ be arbitrary.
By the assumption, $P(x_1) \wedge Q(x_1)$ is true, so in particular $P(x_1)$ is true.
Since $x_1$ was arbitrary, we conclude that $\forall x P(x)$ is true.
Using the same argument with $x_2$, we also conclude that $\forall x Q(x)$ is true.
In other words, $\forall x P(x) \wedge \forall x Q(x)$ is true.
This completes the proof of the forward direction.

Now we prove the other direction: $\forall x P(x) \wedge \forall x Q(x) \rightarrow \forall x (P(x) \wedge Q(x))$.
Suppose $\forall x P(x) \wedge \forall x Q(x)$ is true.
Let $x_1$ be arbitrary.
By the assumption, $P(x_1)$ is true, and $Q(x_1)$ is true as well.
In other words, $P(x_1) \wedge Q(x_1)$ is true.
Since $x_1$ was arbitrary, we conclude that $\forall x (P(x) \wedge Q(x))$ is true. This completes the proof of the reverse direction.
\end{proof}


\begin{statement}{3.4.2}
Prove that if $A \subseteq B$ and $A \subseteq C$ then $A \subseteq B \cap C$.
\end{statement}

\begin{proof}
% Fill in
\end{proof}


\begin{statement}{3.4.3}
Suppose $A \subseteq B$.
Prove that for every set $C$, $C \setminus B \subseteq C \setminus A$.
\end{statement}

\begin{proof}
% Fill in
\end{proof}


\begin{statement}{3.4.4}
Prove that if $A \subseteq B$ and $A \nsubseteq C$ then $B \nsubseteq C$.
\end{statement}

\begin{proof}
% Fill in
\end{proof}


\begin{statement}{3.4.5}
Prove that if $A \subseteq B \setminus C$ and $A \neq \varnothing$ then $B \nsubseteq C$.
\end{statement}

\begin{proof}
% Fill in
\end{proof}


\begin{statement}{3.4.6}
Prove that for any sets $A$, $B$, and $C$, $A \setminus (B \cap C) = (A \setminus B) \cup (A \setminus C)$, by finding a string of equivalences starting with $x \in A \setminus (B \cap C)$ and ending with $x \in (A \setminus B) \cup (A \setminus C)$.
\end{statement}

\begin{proof}
% Fill in
\end{proof}


\begin{statement}{3.4.7}
Use the methods of this chapter to prove that for any sets $A$ and $B$, $\powerset{A \cap B} = \powerset{A} \cap \powerset{B}$.
\end{statement}

\begin{proof}
% Fill in
\end{proof}


\begin{statement}{3.4.8}
Prove that $A \subseteq B$ if and only if $\powerset{A} \subseteq \powerset{B}$.
\end{statement}

\begin{proof}
% Fill in
\end{proof}


\begin{statement}{3.4.9}
Prove that if $x$ and $y$ are odd integers, then $xy$ is odd.
\end{statement}

\begin{proof}
% Fill in
\end{proof}


\begin{statement}{3.4.10}
Prove that if $x$ and $y$ are odd integers, then $x - y$ is even.
\end{statement}

\begin{proof}
% Fill in
\end{proof}


\begin{statement}{3.4.11}
Prove that for every integer $n$, $n^3$ is even if and only if $n$ is even.
\end{statement}

\begin{proof}
% Fill in
\end{proof}


\begin{statement}{3.4.12}
Consider the following putative theorem:
\begin{theorem}
	Suppose $m$ is an even integer and $n$ is an odd integer.
	Then $n^2 - m^2 = n + m$.
\end{theorem}
\begin{enumerate}
	\item What is wrong with the following proof of the theorem?
	\begin{proof}
		Since $m$ is even, we can choose some integer $k$ such that $m = 2k$.
		Similarly, since $n$ is odd we have $n = 2k + 1$.
		Therefore
		\begin{align*}
			n^2 - m^2 &= (2k + 1)^2 - (2k)^2 \\
			&= 4k^2 + 4k + 1 - 4k^2 \\
			&= 4k + 1 \\
			&= (2k + 1) + (2k) \\
			&= n + m.
		\end{align*}
	\end{proof}
	
	\item Is the theorem correct?
	Justify your answer with either a proof or a counterexample.
\end{enumerate}
\end{statement}

\begin{proof}
% Fill in
\end{proof}


\begin{statement}{3.4.13}
Prove that for all $x \in \BR$, there exists $y \in \BR$ such that $x + y = xy$ if and only if $x = 1$.
\end{statement}

\begin{proof}
% Fill in
\end{proof}


\begin{statement}{3.4.14}
Prove that there exists $z \in \BR$, such that for all $x \in \BR^+$, there exists $y \in \BR$ such that $y - x = y/x$ if and only if $x = z$.
\end{statement}

\begin{proof}
% Fill in
\end{proof}


\begin{statement}{3.4.15}
Suppose $B$ is a set and $\mathcal{F}$ is a family of sets.
Prove that $\bigcup \{ A \setminus B \mid A \in \mathcal{F} \} \subseteq \bigcup (\mathcal{F} \setminus \powerset{B})$.
\end{statement}

\begin{proof}
% Fill in
\end{proof}


\begin{statement}{3.4.16}
Suppose $\mathcal{F}$ and $\mathcal{G}$ are nonempty families of sets and every element of $\mathcal{F}$ is disjoint from some element of $\mathcal{G}$.
Prove that $\bigcup \mathcal{F}$ and $\bigcap \mathcal{G}$ are disjoint.
\end{statement}

\begin{proof}
% Fill in
\end{proof}


\begin{statement}{3.4.17}
Prove that for any set $A$, $A = \bigcup \powerset{A}$.
\end{statement}

\begin{proof}
% Fill in
\end{proof}


\begin{statement}{3.4.18}
Suppose $\mathcal{F}$ and $\mathcal{G}$ are families of sets.
\begin{enumerate}
	\item Prove that $\bigcup (\mathcal{F} \cap \mathcal{G}) \subseteq
	\left( \bigcup \mathcal{F} \right) \cap \left( \bigcup \mathcal{G} \right)$.
	
	\item What is wrong with the following proof that 
	$\left( \bigcup \mathcal{F} \right) \cap \left( \bigcup \mathcal{G} \right) \subseteq
	 \bigcup (\mathcal{F} \cap \mathcal{G})$?
	 \begin{proof}
	 	Suppose $x \in \left( \bigcup \mathcal{F} \right) \cap \left( \bigcup \mathcal{G} \right)$.
	 	This means that $x \in \bigcup \mathcal{F}$ and $x \in \bigcup \mathcal{G}$,
	 	so there exists $A \in \mathcal{F}$ such that $x \in A$, 
	 	and there exists $A \in \mathcal{G}$ such that $x \in A$.
	 	Thus, we can choose a set $A$ such that 
	 	$A \in \mathcal{F}$, $A \in \mathcal{G}$, and $x \in A$.
	 	Since $A \in \mathcal{F}$ and $A \in \mathcal{G}$, $A \in \mathcal{F} \cap \mathcal{G}$.
	 	Therefore there exists $A \in \mathcal{F} \cap \mathcal{G}$ such that $x \in A$.,
	 	so $x \in \bigcup (\mathcal{F} \cap \mathcal{G})$.
	 	Since $x$ was arbitrary, we can conclude that
	 	$\left( \bigcup \mathcal{F} \right) \cap \left( \bigcup \mathcal{G} \right) \subseteq
	 	\bigcup (\mathcal{F} \cap \mathcal{G})$.
	 \end{proof}
	 % The existence statement doesn't guarantee that the A in F containing x and the A in G
	 % containing x are the same set. They could be different sets.
	 
	 \item Find an example of families of sets $\mathcal{F}$ and $\mathcal{G}$ for which
	 $\bigcup (\mathcal{F} \cap \mathcal{G}) \neq
	\left( \bigcup \mathcal{F} \right) \cap \left( \bigcup \mathcal{G} \right)$
\end{enumerate}
\end{statement}

\begin{proof}
% Fill in
\end{proof}


\begin{statement}{3.4.19}
Suppose $\mathcal{F}$ and $\mathcal{G}$ are families of sets.
Prove that $\left( \bigcup \mathcal{F} \right) \cap \left( \bigcup \mathcal{G} \right) \subseteq \bigcup (\mathcal{F} \cap \mathcal{G})$ if and only if for all $A \in \mathcal{F}$ and for all $B \in \mathcal{G}$, $A \cap B \subseteq \bigcup (\mathcal{F} \cap \mathcal{G})$ .
\end{statement}

\begin{proof}
% Fill in
\end{proof}


\begin{statement}{3.4.20}
Suppose $\mathcal{F}$ and $\mathcal{G}$ are families of sets.
Prove that $\bigcup \mathcal{F}$ and $\bigcup \mathcal{G}$ are disjoint if and only if for all $A \in \mathcal{F}$ and $B \in \mathcal{G}$, $A$ and $B$ are disjoint.
\end{statement}

\begin{proof}
% Fill in
\end{proof}


\begin{statement}{3.4.21}
Suppose $\mathcal{F}$ and $\mathcal{G}$ are families of sets.
\begin{enumerate}
	\item Prove that $\left( \bigcup \mathcal{F} \right) \setminus \left( \bigcup \mathcal{G} \right)
	\subseteq \bigcup (\mathcal{F} \setminus \mathcal{G})$.
	
	\item What's wrong with the following proof that $\bigcup (\mathcal{F} \setminus \mathcal{G})
	\subseteq \left( \bigcup \mathcal{F} \right) \setminus \left( \bigcup \mathcal{G} \right)$?
	\begin{proof}
		Suppose $x \in \bigcup (\mathcal{F} \setminus \mathcal{G})$.
		Then we can choose some $A \in \mathcal{F} \setminus \mathcal{G}$ such that $x \in A$.
		Since $A \in \mathcal{F} \setminus \mathcal{G}$, 
		$A \in \mathcal{F}$ and $A \notin \mathcal{G}$.
		Since $A \in \mathcal{F}$, $x \in \bigcup \mathcal{F}$.
		Since $x \in A$ and $A \notin \mathcal{G}$, $x \notin \bigcup \mathcal{G}$.
		Therefore $x \in \left( \bigcup \mathcal{F} \right) \setminus
		\left( \bigcup \mathcal{G} \right)$.
	\end{proof}
	% There might be another set B in G such that x is in B.
	
	\item Prove that $\bigcup (\mathcal{F} \setminus \mathcal{G})
	\subseteq \left( \bigcup \mathcal{F} \right) \setminus \left( \bigcup \mathcal{G} \right)$
	if and only if for all $A \in \mathcal{F} \setminus \mathcal{G}$ and for all $B \in \mathcal{G}$,
	$A \cap B = \varnothing$.
	
	\item Find an example of families of sets $\mathcal{F}$ and $\mathcal{G}$ for which
	$\bigcup (\mathcal{F} \setminus \mathcal{G})
	\neq \left( \bigcup \mathcal{F} \right) \setminus \left( \bigcup \mathcal{G} \right)$
\end{enumerate}
\end{statement}

\begin{proof}
% Fill in
\end{proof}


\begin{statement}{3.4.22}
Suppose $\mathcal{F}$ and $\mathcal{G}$ are families of sets.
Prove that if $\bigcup \mathcal{F} \subseteq \bigcup \mathcal{G}$, then there is some $A \in \mathcal{F}$ such that for all $B \in \mathcal{G}$, $A \subseteq B$.
\end{statement}

\begin{proof}
% Fill in
\end{proof}


\begin{statement}{3.4.23}
Suppose $B$ is a set, $\{ A_i \mid i \in I \}$ is an indexed family of sets, and $I \neq \varnothing$.
\begin{enumerate}
	\item What proof strategies are used in the following proof of the equation
	$B \cap \left( \bigcup_{i \in I} A_i \right) = \bigcup_{i \in I} (B \cap A_i)$?
	\begin{proof}
		Let $x$ be arbitrary.
		Suppose $x \in B \cap \left( \bigcup_{i \in I} A_i \right)$.
		Then $x \in B$ and $x \in \bigcup_{i \in I} A_i$,
		so we can choose some $i_0 \in I$ such that $x \in A_{i_0}$.
		Since $x \in B$ and $x \in A_{i_0}$, $x \in B \cap A_{i_0}$.
		Therefore, $x \in \bigcup_{i \in I} (B \cap A_i)$.
		
		Now suppose $x \in \bigcup_{i \in I} (B \cap A_i)$.
		Then we can choose some $i_0 \in I$ such that $x \in B \cap A_{i_0}$.
		Therefore $x \in B$ and $x \in A_{i_0}$.
		Since $x \in A_{i_0}$, $x \in \bigcup_{i \in I} A_i$.
		Since $x \in B$ and $x \in \bigcup_{i \in I} A_i$,
		$x \in B \cap \left( \bigcup_{i \in I} A_i \right)$.
		
		Since $x$ was arbitrary, we have shown that for all $x$,
		$x \in B \cap \left( \bigcup_{i \in I} A_i \right)$ if and only if
		$x \in \bigcup_{i \in I} (B \cap A_i)$, so
		$B \cap \left( \bigcup_{i \in I} A_i \right) = \bigcup_{i \in I} (B \cap A_i)$.
	\end{proof}
	
	\item Prove that $B \setminus \left( \bigcup_{i \in I} A_i \right) = 
	\bigcup_{i \in I} (B \setminus A_i)$.
	
	\item Can you discover and prove a similar theorem about
	$B \setminus \left( \bigcap_{i \in I} A_i \right)$?
	(\emph{Hint: Try to guess the theorem, and then try to prove it.
	If you can't finish the proof, it might be because your guess was wrong.
	Change your guess and try again.})
	
	\emph{Note: I think this exercise in my copy of the book has a typo, since it asks for
	us to discover and prove a theorem about $B \setminus \left( \bigcup_{i \in I} A_i \right)$,
	which we already did in Part (2).}
\end{enumerate}
\end{statement}

\begin{proof}
% Fill in
\end{proof}


\begin{statement}{3.4.24}
Suppose $\{ A_i \mid i \in I \}$ and $\{ B_i \mid i \in I \}$ are indexed families of sets and $I \neq \varnothing$.
\begin{enumerate}
	\item Prove that $\bigcup_{i \in I} (A_i \setminus B_i) \subseteq
	\left( \bigcup_{i \in I} A_i \right) \setminus \left( \bigcup_{i \in I} B_i \right)$.
	
	\item Find an example for which $\bigcup_{i \in I} (A_i \setminus B_i) \neq
	\left( \bigcup_{i \in I} A_i \right) \setminus \left( \bigcup_{i \in I} B_i \right)$.
\end{enumerate}
\end{statement}

\begin{proof}
% Fill in
\end{proof}


\begin{statement}{3.4.25}
Suppose $\{ A_i \mid i \in I \}$ and $\{ B_i \mid i \in I \}$ are indexed families of sets.
\begin{enumerate}
	\item Prove that $\bigcup_{i \in I} (A_i \cap B_i) \subseteq
	\left( \bigcup_{i \in I} A_i \right) \cap \left( \bigcup_{i \in I} B_i \right)$.
	
	\item Find an example for which $\bigcup_{i \in I} (A_i \cap B_i) \neq
	\left( \bigcup_{i \in I} A_i \right) \cap \left( \bigcup_{i \in I} B_i \right)$.
\end{enumerate}
\end{statement}

\begin{proof}
% Fill in
\end{proof}


\begin{statement}{3.4.26}
Prove that for all integers $a$ and $b$ there is an integer $c$ such that $a$ divides $c$ and $b$ divides $c$.
\end{statement}

\begin{proof}
% Fill in
\end{proof}


\begin{statement}{3.4.27}
\begin{enumerate}
	\item Prove that for ever integer $n$, $15$ divides $n$ if and only if
	$3$ divides $n$ and $5$ divides $n$.
	
	\item Prove that it is \emph{not} true that for every integer $n$,
	$60$ divides $n$ if and only if $6$ divides $n$ and $10$ divides $n$.
	% Consider n = 90
\end{enumerate}
\end{statement}

\begin{proof}
% Fill in
\end{proof}


\begin{statement}{3.5.1}
% Fill in
\end{statement}

\begin{proof}
% Fill in
\end{proof}


\begin{statement}{3.5.2}
% Fill in
\end{statement}

\begin{proof}
% Fill in
\end{proof}


\begin{statement}{3.5.3}
% Fill in
\end{statement}

\begin{proof}
% Fill in
\end{proof}


\begin{statement}{3.5.4}
% Fill in
\end{statement}

\begin{proof}
% Fill in
\end{proof}


\begin{statement}{3.5.5}
% Fill in
\end{statement}

\begin{proof}
% Fill in
\end{proof}


\begin{statement}{3.5.6}
% Fill in
\end{statement}

\begin{proof}
% Fill in
\end{proof}


\begin{statement}{3.5.7}
% Fill in
\end{statement}

\begin{proof}
% Fill in
\end{proof}


\begin{statement}{3.5.8}
% Fill in
\end{statement}

\begin{proof}
% Fill in
\end{proof}


\begin{statement}{3.5.9}
% Fill in
\end{statement}

\begin{proof}
% Fill in
\end{proof}


\begin{statement}{3.5.10}
% Fill in
\end{statement}

\begin{proof}
% Fill in
\end{proof}


\begin{statement}{3.5.11}
% Fill in
\end{statement}

\begin{proof}
% Fill in
\end{proof}


\begin{statement}{3.5.12}
% Fill in
\end{statement}

\begin{proof}
% Fill in
\end{proof}


\begin{statement}{3.5.13}
% Fill in
\end{statement}

\begin{proof}
% Fill in
\end{proof}


\begin{statement}{3.5.14}
% Fill in
\end{statement}

\begin{proof}
% Fill in
\end{proof}


\begin{statement}{3.5.15}
% Fill in
\end{statement}

\begin{proof}
% Fill in
\end{proof}


\begin{statement}{3.5.16}
% Fill in
\end{statement}

\begin{proof}
% Fill in
\end{proof}


\begin{statement}{3.5.17}
% Fill in
\end{statement}

\begin{proof}
% Fill in
\end{proof}


\begin{statement}{3.5.18}
% Fill in
\end{statement}

\begin{proof}
% Fill in
\end{proof}


\begin{statement}{3.5.19}
% Fill in
\end{statement}

\begin{proof}
% Fill in
\end{proof}


\begin{statement}{3.5.20}
% Fill in
\end{statement}

\begin{proof}
% Fill in
\end{proof}


\begin{statement}{3.5.21}
% Fill in
\end{statement}

\begin{proof}
% Fill in
\end{proof}


\begin{statement}{3.5.22}
% Fill in
\end{statement}

\begin{proof}
% Fill in
\end{proof}


\begin{statement}{3.5.23}
% Fill in
\end{statement}

\begin{proof}
% Fill in
\end{proof}


\begin{statement}{3.5.24}
% Fill in
\end{statement}

\begin{proof}
% Fill in
\end{proof}


\begin{statement}{3.5.25}
% Fill in
\end{statement}

\begin{proof}
% Fill in
\end{proof}


\begin{statement}{3.5.26}
% Fill in
\end{statement}

\begin{proof}
% Fill in
\end{proof}


\begin{statement}{3.5.27}
% Fill in
\end{statement}

\begin{proof}
% Fill in
\end{proof}


\begin{statement}{3.5.28}
% Fill in
\end{statement}

\begin{proof}
% Fill in
\end{proof}


\begin{statement}{3.5.29}
% Fill in
\end{statement}

\begin{proof}
% Fill in
\end{proof}


\begin{statement}{3.5.30}
% Fill in
\end{statement}

\begin{proof}
% Fill in
\end{proof}


\begin{statement}{3.5.31}
% Fill in
\end{statement}

\begin{proof}
% Fill in
\end{proof}


\begin{statement}{3.5.32}
% Fill in
\end{statement}

\begin{proof}
% Fill in
\end{proof}


\begin{statement}{3.5.33}
% Fill in
\end{statement}

\begin{proof}
% Fill in
\end{proof}


\begin{statement}{3.6.1}
% Fill in
\end{statement}

\begin{proof}
% Fill in
\end{proof}


\begin{statement}{3.6.2}
% Fill in
\end{statement}

\begin{proof}
% Fill in
\end{proof}


\begin{statement}{3.6.3}
% Fill in
\end{statement}

\begin{proof}
% Fill in
\end{proof}


\begin{statement}{3.6.4}
% Fill in
\end{statement}

\begin{proof}
% Fill in
\end{proof}


\begin{statement}{3.6.5}
% Fill in
\end{statement}

\begin{proof}
% Fill in
\end{proof}


\begin{statement}{3.6.6}
% Fill in
\end{statement}

\begin{proof}
% Fill in
\end{proof}


\begin{statement}{3.6.7}
% Fill in
\end{statement}

\begin{proof}
% Fill in
\end{proof}


\begin{statement}{3.6.8}
% Fill in
\end{statement}

\begin{proof}
% Fill in
\end{proof}


\begin{statement}{3.6.9}
% Fill in
\end{statement}

\begin{proof}
% Fill in
\end{proof}


\begin{statement}{3.6.10}
% Fill in
\end{statement}

\begin{proof}
% Fill in
\end{proof}


\begin{statement}{3.6.11}
% Fill in
\end{statement}

\begin{proof}
% Fill in
\end{proof}


\begin{statement}{3.6.12}
% Fill in
\end{statement}

\begin{proof}
% Fill in
\end{proof}


\begin{statement}{3.6.13}
% Fill in
\end{statement}

\begin{proof}
% Fill in
\end{proof}


\begin{statement}{3.7.1}
% Fill in
\end{statement}

\begin{proof}
% Fill in
\end{proof}


\begin{statement}{3.7.2}
% Fill in
\end{statement}

\begin{proof}
% Fill in
\end{proof}


\begin{statement}{3.7.3}
% Fill in
\end{statement}

\begin{proof}
% Fill in
\end{proof}


\begin{statement}{3.7.4}
% Fill in
\end{statement}

\begin{proof}
% Fill in
\end{proof}


\begin{statement}{3.7.5}
% Fill in
\end{statement}

\begin{proof}
% Fill in
\end{proof}


\begin{statement}{3.7.6}
% Fill in
\end{statement}

\begin{proof}
% Fill in
\end{proof}


\begin{statement}{3.7.7}
% Fill in
\end{statement}

\begin{proof}
% Fill in
\end{proof}


\begin{statement}{3.7.8}
% Fill in
\end{statement}

\begin{proof}
% Fill in
\end{proof}


\begin{statement}{3.7.9}
% Fill in
\end{statement}

\begin{proof}
% Fill in
\end{proof}


\begin{statement}{3.7.10}
% Fill in
\end{statement}

\begin{proof}
% Fill in
\end{proof}
\end{document}