\documentclass[12pt]{amsart}

%Below are some necessary packages for your course.
\usepackage{amsfonts,latexsym,amsthm,amssymb,amsmath,amscd,euscript,mathrsfs}
\usepackage{framed}
\usepackage{fullpage}
\usepackage{hyperref}
    \hypersetup{colorlinks=true,citecolor=blue,urlcolor =black,linkbordercolor={1 0 0}}
\usepackage{mathtools}
\usepackage[table]{xcolor}

\newenvironment{statement}[1]{\smallskip\noindent\color[rgb]{.6627, .3529, .6314} {\bf #1.}}{}
\allowdisplaybreaks[1]

%Below are the theorem, definition, example, lemma, etc. body types.

\newtheorem{theorem}{Theorem}
\newtheorem*{proposition}{Proposition}
\newtheorem{lemma}[theorem]{Lemma}
\newtheorem{corollary}[theorem]{Corollary}
\newtheorem{conjecture}[theorem]{Conjecture}
\newtheorem{postulate}[theorem]{Postulate}
\theoremstyle{definition}
\newtheorem{defn}[theorem]{Definition}
\newtheorem{example}[theorem]{Example}

\theoremstyle{remark}
\newtheorem*{remark}{Remark}
\newtheorem*{notation}{Notation}
\newtheorem*{note}{Note}

% You can define new commands to make your life easier.
\newcommand{\BR}{\mathbb R}
\newcommand{\BC}{\mathbb C}
\newcommand{\BF}{\mathbb F}
\newcommand{\BQ}{\mathbb Q}
\newcommand{\BZ}{\mathbb Z}
\newcommand{\BN}{\mathbb N}
\newcommand{\powerset}[1]{\mathscr{P} \left( #1 \right)}

%Commands for absolute value and norm
\DeclarePairedDelimiter\abs{\lvert}{\rvert}
\DeclarePairedDelimiter\norm{\lVert}{\rVert}

% Swap the definition of \abs* and \norm*, so that \abs
% and \norm resizes the size of the brackets, and the 
% starred version does not.
\makeatletter
	\let\oldabs\abs
	\def\abs{\@ifstar{\oldabs}{\oldabs*}}

	\let\oldnorm\norm
	\def\norm{\@ifstar{\oldnorm}{\oldnorm*}}
\makeatother

% We can even define a new command for \newcommand!
\newcommand{\nc}{\newcommand}

% If you want a new function, use operatorname to define that function (don't use \text)
\nc{\on}{\operatorname}
\nc{\Spec}{\on{Spec}}

\title{\emph{How to Prove It}: Chapter 3} % IMPORTANT: Change the problemset number as needed.
\date{\today}

\begin{document}

\maketitle

\vspace*{-0.25in}
\centerline{Kyle Stratton}

\begin{framed}
These are the exercises for Chapter 3 from the third edition of \emph{How to Prove It} by Daniel J. Velleman.
They are numbered (Chapter).(Section).(Exercise).
\end{framed}

\begin{statement}{3.1.1}
Consider the following theorem.
(This theorem was proven in the introduction.)
\begin{theorem}
	Suppose $n$ is an integer larger than 1 and $n$ is not prime.
	Then $2^n - 1$ is not prime.
\end{theorem}
\begin{enumerate}
	\item Identify the hypotheses and the conclusion of the theorem.
	Are the hypotheses true when $n = 6$?
	What does the theorem tell you in this instance?
	Is it right?
	
	\item What can you conclude from the theorem in the case $n = 15$?
	Check directly that this conclusion is correct.
	
	\item What can you conclude from the theorem in the case $n = 11$?
\end{enumerate}
\end{statement}

\begin{proof}
\hfill
\begin{enumerate}
	\item This theorem has three hypotheses: $n$ is an integer, $n > 1$, and $n$ is not prime.
	The conclusion of the theorem is that $2^n - 1$ is not prime.
	In the case when $n = 6 = 2 \times 3$, all of the hypotheses are satisfied,
	so the theorem tells us that $2^6 - 1$ is not prime.
	We can directly check that $2^6 - 1 = 63 = 3^2 \times 7$ is not prime.
	
	\item In the case when $n = 15 = 3 \times 5$, all of the hypotheses are satisfied.
	This means that the theorem tells us that $2^{15} - 1$ is not prime.
	As discussed in Part (1) of Exercise I.1, $2^{15} - 1 = 32767 = 31 \times 1057$.
	
	\item In the case when $n = 11$, not all of the hypotheses are satisfied.
	In particular, 11 is prime.
	Because not all of the hypotheses of the theorem are satisfied,
	we cannot draw any conclusions from it.
	In particular, the theorem does not tell us anything about the primality of $2^{11} - 1$.
\end{enumerate}
\end{proof}


\begin{statement}{3.1.2}
Consider the following theorem.
(The theorem is correct, but we will not ask you to prove it here.)
\begin{theorem}
	Suppose that $b^2 > 4ac$.
	Then the quadratic equation $ax^2 + bx + c = 0$ has exactly two real solutions.
\end{theorem}
\begin{enumerate}
	\item Identify the hypotheses and conclusion of the theorem.
	
	\item To give an instance of the theorem, you must specify values for $a$, $b$, and $c$, but not $x$.
	Why?
	
	\item What can you conclude from the theorem in the case $a = 2$, $b = -5$, $c = 3$?
	Check directly that this conclusion is correct.
	
	\item What can you conclude from the theorem in the case $a = 2$, $b = 4$, $c = 3$?
\end{enumerate}
\end{statement}

\begin{proof}
\hfill
\begin{enumerate}
	\item This theorem has one implicit hypothesis, that $a, b, c$ are all real numbers,
	and one explicit hypothesis, that $b^2 > 4ac$.
	The conclusion of the theorem is that the quadratic equation $ax^2 + bx + c = 0$
	has exactly two real solutions.
	
	\item To give an instance of the theorem, we only need to specify values for $a$, $b$, and $c$
	since those are the only variables listed in the hypothesis.
	The values of $x$ associated with a specific instance of the theorem are determined
	by the given values of $a, b, c$, since the $x$ values are the two real solutions to the
	quadratic equation $ax^2 + bx + c = 0$.
	In other words, $x$ is a dummy variable for the set 
	$S = \{ x \in \BR \mid ax^2 + bx + c = 0 \}$ for given values of $a, b, c$.
	When phrased this way, the conclusion of theorem states that $S$ contains
	exactly two distinct elements.
	
	\item In the case $a = 2, b = -5, c = 3$, we have $b^2 = (-5)^2 = 25$ 
	and $4ac = 4(2)(3) = 24$.
	Thus, the hypotheses $b^2 > 4ac$ is satisfied, and the theorem applies.
	We can then conclude that the quadratic equation $2x^2 - 5x + 3 = 0$ has two real solutions.
	Factoring the quadratic as $(2x - 3)(x - 1) = 0$, we can directly check that the two
	real solutions are $x = 3/2$ and $x = 1$.
	
	\item In the case $a = 2, b = 4, c = 3$, we have $b^2 = 4^2 = 16$
	and $4ac = 4(2)(3) = 24$.
	In other words, $b^2 \ngtr 4ac$.
	Since not all of the hypotheses of the theorem are satisfied,
	we cannot draw any conclusions from it.
	In particular, the theorem does not tell us anything about the solution set of
	the quadratic equation $2x^2 + 4x + 3 = 0$.
\end{enumerate}
\end{proof}


\begin{statement}{3.1.3}
Consider the following incorrect theorem.
\begin{theorem}
	Suppose $n$ is a natural number larger than 2, and $n$ is not a prime number.
	Then $2n + 13$ is not a prime number.
\end{theorem}
What are the hypotheses and conclusion of this theorem?
Show that the theorem is incorrect by finding a counterexample.
\end{statement}

\begin{proof}
The theorem has three hypotheses: $n$ is a natural number, $n > 2$, and $n$ is not prime.
The conclusion of the theorem is that $2n + 13$ is not a prime number.
To see that this theorem is incorrect, consider the case of $n = 8$.
This value of $n$ is a natural number greater than 2 that is not prime, 
so it satisfies all of the hypotheses of the theorem.
However, $2n + 13 = 2(8) + 13 = 29$ is a prime number.
Since we could find a instance of the theorem where all of the hypotheses are satisfied
but an incorrect conclusion is drawn, the theorem itself is incorrect.
\end{proof}


\begin{statement}{3.1.4}
Complete the following alternative proof of the theorem in Example 3.1.2.
\begin{theorem}
	Suppose $a$ and $b$ are real numbers.
	If $0 < a < b$ then $a^2 < b^2$.
\end{theorem}
\begin{proof}
	Suppose $0 < a < b$.
	Then $b - a > 0$.
	[\emph{Fill in a proof of $b^2 - a^2 > 0$ here.}]
	Since $b^2 - a^2 > 0$, it follows that $a^2 < b^2$.
	Therefore, if $0 < a < b$ then $a^2 < b^2$.
\end{proof}
\end{statement}

\begin{proof}
Suppose $0 < a < b$.
Then $b - a > 0$.
In addition, since $0 < a < b$, we also know that $b + a > 0$.
We can then multiply both sides of the inequality $b - a > 0$ by $b + a$
to get $(b - a)(b + a) = b^2 - a^2 > 0$.
Since $b^2 - a^2 > 0$, it follows that $a^2 < b^2$.
Therefore, if $0 < a < b$ then $a^2 < b^2$.
\end{proof}


\begin{statement}{3.1.5}
Suppose $a$ and $b$ are real numbers.
Prove that if $a < b < 0$ then $a^2 > b^2$.
\end{statement}

\begin{proof}
Suppose $a < b < 0$.
Then $a - b < 0$ and $a + b < 0$.
We can then multiply both sides of the inequality $a - b < 0$ by $a + b$
to get $(a - b)(a + b) = a^2 - b^2 > 0$.
Since $a^2 - b^2 > 0$, it follows that $a^2 > b^2$.
Therefore, if $a < b < 0$ then $a^2 > b^2$.

\emph{Note that an alternative strategy would be to mimic the proof of the theorem in Example 3.1.2 and instead multiply the given inequality by the negative numbers $a$ and $b$, respectively.
This would then result in the chain of inequalities $a^2 > ab > b^2 > 0$, which also gives the desired conclusion.}
\end{proof}


\begin{statement}{3.1.6}
Suppose $a$ and $b$ are real numbers.
Prove that if $0 < a < b$ then $1/b < 1/a$.
\end{statement}

\begin{proof}
Suppose $0 < a < b$.
Dividing both sides of the inequality $a < b$ by the positive number $a$ gives us $1 < b/a$.
We can then divide both sides of $1 < b/a$ by the positive number $b$ to conclude $1/b < 1/a$.
Therefore, if $0 < a < b$ then $1/b < 1/a$.
\end{proof}


\begin{statement}{3.1.7}
Suppose $a$ is a real number.
Prove that if $a^3 > a$ then $a^5 > a$.
(\emph{Hint: One approach is to start by completing the following equation:
$a^5 - a = (a^3 - a) \cdot \underline{?}$.})
\end{statement}

\begin{proof}
Suppose $a^3 > a$, which we can rewrite as $a^3 - a > 0$.
Following the hint, we note that we can factor $a^5 - a$ as $(a^3 - a)(a^2 + 1)$.
Since $a^2 \geq 0$ for all $a \in \BR$, we know that $a^2 + 1 > 0$.
Thus, multiplying the inequality $a^3 - a > 0$ by the positive number $a^2 + 1$
gives us $0 < (a^3 - a)(a^2 + 1) = a^5 - a$.
In other words, we conclude that $a^5 > a$.
Therefore, if $a^3 > a$ then $a^5 > a$.
\end{proof}


\begin{statement}{3.1.8}
Suppose $A \setminus B \subseteq C \cap D$ and $x \in A$.
Prove that if $x \notin D$ then $x \in B$.
\end{statement}

\begin{proof}
We prove the contrapositive statement: if $x \notin B$ then $x \in D$.
Suppose $x \notin B$.
Then, since we also know that $x \in A$, it follows that $x \in A \setminus B$.
As $A \setminus B \subseteq C \cap D$, it then follows that $x \in C \cap D$.
In other words, $x \in C$ and $x \in D$.
The last inclusion, that $x \in D$, is what we wanted to prove to complete the proof of the contrapositive statement.
Therefore, if $x \notin D$ then $x \in B$.
\end{proof}


\begin{statement}{3.1.9}
Suppose $A \cap B \subseteq C \setminus D$ .
Prove that if $x \in A$, then if $x \in D$ then $x \notin B$.
\end{statement}

\begin{proof}
Suppose $x \in A$.
We prove the contrapositve statement: if $x \in B$ then $x \notin D$.
Now suppose $x \in B$.
Then, since $x \in A$ as well, it follows that $x \in A \cap B$.
As $A \cap B \subseteq C \setminus D$, it then follows that $x \in C \setminus D$.
From this, we conclude that $x \notin D$, which completes our proof of the contrapositive statement.
Therefore, if $x \in A$, then if $x \in D$ then $x \notin B$.
\end{proof}


\begin{statement}{3.1.10}
Suppose $a$ and $b$ are real numbers.
Prove that if $a < b$ then $(a + b) / 2 < b$.
\end{statement}

\begin{proof}
Suppose $a < b$.
Adding $b$ to both sides of the inequality then gives us $a + b < 2b$.
Next we divide both sides by 2 to obtain the desired inequality: $(a + b)/2 < b$.
Therefore, if $a < b$ then $(a + b)/2 < b$.
\end{proof}


\begin{statement}{3.1.11}
Suppose $x$ is a real number and $x \neq 0$.
Prove that if $(\sqrt[3]{x} + 5) / (x^2 + 6) = 1/x$ then $x \neq 8$.
\end{statement}

\begin{proof}
We prove the contrapositive statement:
if $x = 8$ then $(\sqrt[3]{x} + 5) / (x^2 + 6) \neq 1/x$.
Suppose $x = 8$.
Then $1/x = 1/8$ and
$(\sqrt[3]{x} + 5) / (x^2 + 6) = (\sqrt[3]{8} + 5) / (8^2 + 6) = 1/10$.
Since $1/8 \neq 1/10$, this completes the proof of the contrapositive statement.
Therefore, we conclude that $(\sqrt[3]{x} + 5) / (x^2 + 6) = 1/x$ then $x \neq 8$.
\end{proof}


\begin{statement}{3.1.12}
Suppose $a$, $b$, $c$, and $d$ are real numbers, $0 < a < b$, and $d > 0$.
Prove that if $ac \geq bd$ then $c > d$.
\end{statement}

\begin{proof}
Suppose $ac \geq bd$.
Since $b > a$ and $d > 0$, we can multiply both sides of the first inequality by $d$ to obtain $bd > ad$.
Then, since $ac \geq bd$, we have $ac \geq bd > ad$.
It then follows that $ac > ad$.
Dividing both sides of the inequality by the positive number $a$, we see that $c > d$.
Thus, we conclude that if $ac \geq bd$ then $c > d$.

\emph{Note that we could also approach this proof by proving the contrapositive statement: if $c \leq d$ then $ac < bd$.}
\end{proof}


\begin{statement}{3.1.13}
Suppose $x$ and $y$ are real numbers, and that $3x + 2y \leq 5$.
Prove that if $x > 1$ then $y < 1$.
\end{statement}

\begin{proof}
Suppose $x > 1$.
Then, $3x + 2y > 3 + 2y$.
Combining this with the given inequality, $3x + 2y \leq 5$, we have $3 + 2y < 3x + 2y \leq 5$.
In particular, $3 + 2y < 5$.
After isolating $y$ in the inequality, we end up with the desired result: $y < 1$.
Therefore, we conclude that if $x > 1$ then $y < 1$.

\emph{Note that we could also approach this proof by proving the contrapositive statement: if $y \geq 1$ then $x \leq 1$.}
\end{proof}


\begin{statement}{3.1.14}
Suppose $x$ and $y$ are real numbers.
Prove that if $x^2 + y = -3$ and $2x - y = 2$ then $x = -1$.
\end{statement}

\begin{proof}
Suppose that $x^2 + y = -3$ and $2x - y = 2$.
We can add the two equations together to obtain $x^2 + 2x = -1$.
After rearranging, this becomes $x^2 + 2x + 1 = 0$, which we can factor as $(x + 1)^2 = 0$.
Therefore, it follows that $x = -1$.
Thus, we conclude that if $x^2 + y = -3$ and $2x - y = 2$ then $x = -1$.
\end{proof}

\begin{statement}{3.1.15}
Prove the first theorem in Example 3.1.1.
\begin{theorem}
	Suppose $x > 3$ and $y < 2$.
	Then $x^2 - 2y > 5$.
\end{theorem}
(\emph{Hint: You might find it useful to apply the theorem from Example 3.1.2,
which stated that if $a$ and $b$ are real numbers such that $0 < a < b$, then $a^2 < b^2$.})
\end{statement}

\begin{proof}
We follow the hint and apply the fact that if $a$ and $b$ are real numbers such that
$0 < a < b$ then $a^2 < b^2$ with $a = 3$ and $b = x$ to conclude that $x^2 > 9$.
Next, since $y < 2$, it follows that $-2y > -4$.
Adding the inequalities $x^2 > 9$ and $-2y > -4$ together gives us the desired inequality:
$x^2 - 2y > 5$.
\end{proof}


\begin{statement}{3.1.16}
Consider the following theorem.
\begin{theorem}
	Suppose $x$ is a real number and $x \neq 4$.
	If $(2x - 5) / (x - 4) = 3$ then $x = 7$.
\end{theorem}
\begin{enumerate}
	\item What is wrong with the following proof of the theorem?
	\begin{proof}
		Suppose $x = 7$.
		Then $(2x - 5) / (x - 4) = (2 \cdot 7 - 5) / (7 - 4) = 9/3 = 3$.
		Therefore if $(2x - 5) / (x - 4) = 3$ then $x = 7$.
	\end{proof}
	
	\item Give a correct proof of the theorem.
\end{enumerate}
\end{statement}

\begin{proof}
\hfill
\begin{enumerate}
	\item The problem with the incorrect proof of the theorem is that it starts
	by assuming that the conclusion, that $x = 7$, is true when that is the ultimate fact
	that we are trying to prove.
	In other words, it is instead a proof of the converse statement: if $x = 7$ then
	$(2x - 5) / (x - 4) = 3$, which is is a completely different statement.
	
	\item Suppose $(2x - 5) / (x - 4) = 3$.
	We can then clear the denominator of the left-hand side by multiplying the equation by $x - 4$.
	This gives us the equation $2x - 5 = 3(x - 4)$, or in other words $2x - 5 = 3x - 12$.
	We subtract $2x$ from both sides and then add 12 to both sides to see that $x = 7$.
	Therefore, we conclude that if $(2x - 5) / (x - 4) = 3$ then $x = 7$.
\end{enumerate}
\end{proof}


\begin{statement}{3.1.17}
Consider the following incorrect theorem.
\begin{theorem}
	Suppose that $x$ and $y$ are real numbers and $x \neq 3$.
	If $x^2y = 9y$ then $y = 0$.
\end{theorem}
\begin{enumerate}
	\item What's wrong with the following proof of the theorem?
	\begin{proof}
		Suppose that $x^2y = 9y$.
		Then $(x^2 - 9)y = 0$.
		Since $x \neq 3$, $x^2 \neq 9$, so $x^2 - 9 \neq 0$.
		Therefore we can divide both sides of the equation $(x^2 - 9)y = 0$ by $x^2 - 9$,
		which leads to the conclusion that $y = 0$.
		Thus, if $x^2y = 9y$ then $y = 0$.
	\end{proof}
	
	\item Show that the theorem is incorrect by finding a counterexample.
\end{enumerate}
\end{statement}

\begin{proof}
\hfill
\begin{enumerate}
	\item The attempted proof makes a mistake in the statement
	``Since $x \neq 3$, $x^2 \neq 9$, so $x^2 - 9 \neq 0$.''
	We only know that $x \neq 3$, but otherwise $x$ could be any other real number.
	In particular, we could have $x = -3$, which would then result in $x^2 = 9$.
	Since the rest of the proof hinges on the assumption that $x^2 \neq 9$, it is rendered invalid.
	
	\item Consider $x = -3$ and $y = 1$.
	In this case, $x^2y = (-3)^2(1) = 9$ and $9y = 9(1) = 9$, but $y \neq 0$.
	Therefore, this counter example shows that the theorem is correct,
	since we were able to find values of $x$ and $y$ which satisfy the hypotheses but
	then lead to an incorrect conclusion.
\end{enumerate}
\end{proof}


\begin{statement}{3.2.1}
This problem could be solved by using truth tables, but don't do it that way.
Instead, use the methods for writing proofs discussed so far in this chapter.
\begin{enumerate}
	\item Suppose $P \rightarrow Q$ and $Q \rightarrow R$ are both true.
	Prove that $P \rightarrow R$ is true.
	
	\item Suppose $\neg R \rightarrow (P \rightarrow \neg Q)$ is true.
	Prove that $P \rightarrow (Q \rightarrow R)$ is true.
\end{enumerate}
\end{statement}

\begin{proof}
\hfill
\begin{enumerate}
	\item Suppose $P$ is true.
	Then, since by the assumption $P \rightarrow Q$ is also true, we apply modus ponens
	to conclude that $Q$ is true.
	Now we apply modus ponens again, this time the fact that $Q$ is true and the assumption
	$Q \rightarrow R$ is true to conclude that $R$ is true.
	Therefore, $P \rightarrow R$ is true.
	
	\item Suppose $P$ is true.
	Now we want to show that $Q \rightarrow R$ is true.
	This is equivalent to proving the contrapositive statement, $\neg R \rightarrow \neg Q$.
	To prove $\neg R \rightarrow \neg Q$, suppose $\neg R$ is true.
	Then, we can combine this assumption with the given assumption that
	$\neg R \rightarrow (P \rightarrow \neg Q)$ is true to conclude that
	$P \rightarrow \neg Q$ is true by modus ponens.
	Now we apply modus ponens again with $P$ and $P \rightarrow \neg Q$ to conclude
	that $\neg Q$ is true.
	Therefore, $\neg R \rightarrow \neg Q$ is true, so we conclude that $Q \rightarrow R$ is true.
	Finally, we conclude that $P \rightarrow (Q \rightarrow R)$ is true.
\end{enumerate}
\end{proof}


\begin{statement}{3.2.2}
This problem could be solved using truth tables, but don't do it that way.
Instead, use the methods for writing proofs discussed so far in this chapter.
\begin{enumerate}
	\item Suppose $P \rightarrow Q$ and $R \rightarrow \neg Q$ are both true.
	Prove that $P \rightarrow \neg R$ is true.
	
	\item Suppose that $P$ is true.
	Prove that $Q \rightarrow \neg (Q \rightarrow \neg P)$ is true.
\end{enumerate}
\end{statement}

\begin{proof}
\hfill
\begin{enumerate}
	\item Suppose $P$ is true.
	Then, since $P$ and $P \rightarrow Q$ are both true, we conclude that $Q$ is true
	by modus ponens.
	Next, since $Q$ and $R \rightarrow \neg Q$ are both true, we conclude that $\neg R$
	is true by modus tollens.
	Therefore, we conclude that $P \rightarrow \neg R$ is true.
	
	\item We prove the equivalent contrapositive statement: 
	that $(Q \rightarrow \neg P) \rightarrow \neg Q$ is true.
	Suppose $Q \rightarrow \neg P$ is true.
	Then, since $P$ is true by assumption, we use modus tollens to conclude that $\neg Q$ is true.
	Thus, $(Q \rightarrow \neg P) \rightarrow \neg Q$ is true.
	In other words, we conclude that $Q \rightarrow \neg (Q \rightarrow \neg P)$ is true.
\end{enumerate}
\end{proof}


\begin{statement}{3.2.3}
Suppose $A \subseteq C$, and $B$ and $C$ are disjoint.
Prove that if $x \in A$ then $x \notin B$.
\end{statement}

\begin{proof}
Suppose $x \in A$.
Then, since $A \subseteq C$, we also know that $x \in C$.
Since $B$ and $C$ are disjoint, the fact that $x \in C$ means that $x \notin B$.
Therefore, we conclude that if $x \in A$ then $x \notin B$.
\end{proof}


\begin{statement}{3.2.4}
Suppose that $A \setminus B$ is disjoint from $C$ and $x \in A$.
Prove that if $x \in C$ then $x \in B$.
\end{statement}

\begin{proof}
We prove the equivalent contrapositive statement: if $x \notin B$ then $x \notin C$.
Suppose $x \notin B$.
Then, since by assumption $x \in A$, $x \in A \setminus B$.
Since $A \setminus B$ is disjoint from $C$, it follows that $x \notin C$.
Thus, if $x \notin B$ then $x \notin C$.
In other words, we conclude that if $x \in C$ then $x \in B$.
\end{proof}


\begin{statement}{3.2.5}
Prove that it cannot be the case that $x \in A \setminus B$ and $x \in B \setminus C$.
\end{statement}

\begin{proof}
Suppose $x \in A \setminus B$ and $x \in B \setminus C$.
Since $x \in A \setminus B$, it follows that $x \in A$ and $x \notin B$.
Similarly, since $x \in B \setminus C$, it follows that $x \in B$ and $x \in C$.
However, having $x \notin B$ and $x \in B$ is a contradiction.
Therefore, we conclude that it cannot be the case that
$x \in A \setminus B$ and $x \in B \setminus C$.
\end{proof}


\begin{statement}{3.2.6}
Use the method of proof by contradiction to prove the theorem in Example 3.2.1.
\begin{theorem}
	Suppose $A \cap C \subseteq B$ and $a \in C$.
	Prove that $a \notin A \setminus B$.
\end{theorem}
\end{statement}

\begin{proof}
Suppose, towards a contradiction, that $a \in A \setminus B$.
In other words, $a \in A$ and $a \notin B$.
Since also $a \in C$, we have $a \in A \cap C \subseteq B$.
Therefore, $a \in B$.
But this contradicts that $a \notin B$.
Thus, we conclude that $a \notin A \setminus B$.
\end{proof}


\begin{statement}{3.2.7}
Use the method of proof by contradiction to prove the theorem in Example 3.2.5.
\begin{theorem}
	Suppose $A \subseteq B$, $a  \in A$, and $a \notin B \setminus C$.
	Prove that $a \in C$.
\end{theorem}
\end{statement}

\begin{proof}
Suppose, towards a contradiction, that $a \notin C$.
Since $a \in A$ and $A \subseteq B$, it follows that $a \in B$.
Then, since $a \in B$ and $a \notin C$, $a \in B \setminus C$.
This, however, contradicts the fact that $a \notin B \setminus C$.
Therefore, we conclude that $a \in C$.
\end{proof}


\begin{statement}{3.2.8}
Suppose that $y + x = 2y - x$, and $x$ and $y$ are not both zero.
Prove that $y \neq 0$.
\end{statement}

\begin{proof}
Suppose, towards a contradiction, that $y = 0$.
Substituting $y = 0$ into the equation $y + x = 2y - x$ gives us $x = -x$, which implies that $x = 0$.
However, having $y = 0$ and $x = 0$ contradicts the fact that $x$ and $y$ are not both zero.
Thus, we conclude that $y \neq 0$.
\end{proof}


\begin{statement}{3.2.9}
Suppose that $a$ and $b$ are nonzero real numbers.
Prove that if $a < 1/a < b < 1/b$ then $a < -1$.
\end{statement}

\begin{proof}
Suppose $a < 1/a < b < 1/b$.
We then break the proof into two steps: proving that $a < 0$ and then proving that $a < -1$.
	\begin{lemma}
		$a < 0$
	\end{lemma}
	\begin{proof}
	Suppose, towards a contradiction, that $a \geq 0$.
	Since $a$ is nonzero, we can assume, without loss of generality, that $a > 0$.
	Then, since $0 < a < b$, we can apply Exercise 3.1.6 to conclude that $1/b < 1/a$.
	This, however, contradicts the fact that $1/a < 1/b$.
	Thus, we conclude that $a < 0$.
	\end{proof}
Now that we have proved that $a < 0$ in the above lemma, we move on to showing that $a < -1$.
We once again use proof by contradiction.
Suppose, towards a contradiction, that $a \geq -1$.
In other words, $-1 \leq a < 0$.
Since $a$ is negative, dividing both sides of the inequality $a \geq -1$ gives us $1 \leq -1/a$.
Rearranging this inequality results in $-1 \geq 1/a$.
Thus, we have $a \geq -1 \geq 1/a$, or most importantly, $a \geq 1/a$.
This however, contradicts the fact that $a < 1/a$, so we must instead have $a < -1$, as desired.
Therefore, we conclude that if $a < 1/a < b < 1/b$ then $a < -1$.
\end{proof}


\begin{statement}{3.2.10}
Suppose that $x$ and $y$ are real numbers.
Prove that if $x^2y = 2x + y$, then if $y \neq 0$ then $x \neq 0$.
\end{statement}

\begin{proof}
Suppose $x^2y = 2x + y$.
Now our goal is to prove that if $y \neq 0$ then $x \neq 0$.
We do this by proving the contrapositive statement: if $x = 0$ then $y = 0$.
Now suppose that $x = 0$.
Plugging this value for $x$ into $x^2y = 2x + y$ gives us $0 = y$, which is exactly what we wanted to show.
Thus, if $x = 0$ then $y = 0$.
In other words, if $y \neq 0$ then $x \neq 0$.
We then conclude that if $x^2y = 2x + y$, then if $y \neq 0$ then $x \neq 0$.
\end{proof}


\begin{statement}{3.2.11}
Suppose that $x$ and $y$ are real numbers.
Prove that if $x \neq 0$, then if $y = (3x^2 + 2y) / (x^2 + 2)$ then $y = 3$.
\end{statement}

\begin{proof}
Suppose $x \neq 0$, and then suppose $y = (3x^2 + 2y) / (x^2 + 2)$.
Multiplying both sides of the equation by $x^2 + 2$ gives us $x^2y + 2y = 3x^2 + 2y$.
After subtracting $2y$ from both sides, we are left with $x^2y = 3x^2$.
Since $x \neq 0$, $x^2 \neq 0$ as well, so we can divide both sides by $x^2$ to conclude that $y = 3$.
Therefore, we conclude that if $x \neq 0$, then if $y = (3x^2 + 2y) / (x^2 + 2)$ then $y = 3$.
\end{proof}


\begin{statement}{3.2.12}
Consider the following incorrect theorem.
\begin{theorem}
	Suppose $x$ and $y$ are real numbers and $x + y = 10$.
	Then $x \neq 3$ and $y \neq 8$.
\end{theorem}
\begin{enumerate}
	\item What's wrong with the following proof of the theorem?
	\begin{proof}
		Suppose the conclusion of the theorem is false.
		Then $x = 3$ and $y = 8$.
		But then $x + y = 11$, which contradicts the given information that $x + y = 10$.
		Therefore, the conclusion must be true.
	\end{proof}
	
	\item Show that the theorem is incorrect by finding a counterexample.
\end{enumerate}
\end{statement}

\begin{proof}
\hfill
\begin{enumerate}
	\item The proof incorrectly negates the conclusion when attempting to begin the proof
	by contradiction.
	The conclusion, that $x \neq 3$ and $y \neq 8$, has the logical form 
	$(x \neq 3) \wedge (y \neq 8)$, so its negation is $(x = 3) \vee (y = 8)$, by DeMorgan's laws.
	Thus, the correct negation of the conclusion is that $x = 3$ \textbf{or} $y = 8$.
	
	\item One counterexample is the case of $x = 3$ and $y = 7$.
	These are two real numbers such that $x + y = 10$, but they do not satisfy the conclusion
	that $x \neq 3$ \textbf{and} $y \neq 8$, as only $y \neq 8$.
\end{enumerate}
\end{proof}


\begin{statement}{3.2.13}
Consider the following incorrect theorem.
\begin{theorem}
	Suppose that $A \subseteq C$, $B \subseteq C$, and $x \in A$.
	Then $x \in B$.
\end{theorem}
\begin{enumerate}
	\item What's wrong with the following proof of the theorem?
	\begin{proof}
		Suppose that $x \notin B$.
		Since $x \in A$ and $A \subseteq C$, $x \in C$.
		Since $x \notin B$ and $B \subseteq C$, $x \notin C$.
		But now we have proven both $x \in C$ and $x \notin C$,
		so we have reached a contradiction.
		Therefore $x \in B$.
	\end{proof}
	
	\item Show that the theorem is incorrect by finding a counterexample.
\end{enumerate}
\end{statement}

\begin{proof}
\hfill
\begin{enumerate}
	\item The proof incorrectly concludes that since $x \notin B$ and $B \subseteq C$, $x \notin C$.
	This is an incorrect conclusion because $B$ may be a proper subset of $C$,
	or in other words $C \setminus B$ may be nonempty.
	We can also confirm the incorrectness of this conclusion by examining the logical form of
	the statement $B \subseteq C$, which is $\forall x (x \in B \rightarrow x \in C)$.
	This is equivalent to $\forall x (x \notin B \vee x \in C)$, which is still true in the case
	when there are elements $x$ such that $x \notin B$ and $x \in C$.
	
	\item Consider the case where $A = \{ 1 \}$, $B = \{ 2 \}$, $C = \{ 1, 2 \}$, and $x = 1$.
	Clearly $x \in A$, $A \subseteq C$, and $B \subseteq C$, but $x \notin B$.
\end{enumerate}
\end{proof}


\begin{statement}{3.2.14}
Use truth tables so show that modus tollens is a valid rule of inference.
\end{statement}

\begin{proof}
Recall that modus tollens says that if $P \rightarrow Q$ is true and $Q$ is false, then you can conclude that $P$ must also be false.
We now construct a truth table to evaluate the validity of this form of argument.
\begin{equation*}
	\begin{array}{| c c | c c | c |}
		P & Q & P \rightarrow Q & \neg Q & \neg P \\
		\hline
		T & T & T & F & F \\
		T & F & F & T & F \\
		F & T & T & F & T \\
		\rowcolor[HTML]{85C0F9} F & F & T & T & T
	\end{array}
\end{equation*}
This form of argument is valid because whenever all of the premises, $P \rightarrow Q$ and $\neg Q$, are true, the conclusion, $\neg P$, is also true, as indicated by the blue highlighted row.
\end{proof}


\begin{statement}{3.2.15}
Use truth tables to check the correctness of the theorem in Example 3.2.4.
\begin{theorem}
	Suppose $P \rightarrow (Q \rightarrow R)$.
	Prove that $\neg R \rightarrow (P \rightarrow \neg Q)$.
\end{theorem}
\end{statement}

\begin{proof}
\begin{equation*}
	\begin{array}{| c c c | c | c |}
		P & Q & R & P \rightarrow (Q \rightarrow R) & \neg R \rightarrow (P \rightarrow \neg Q) \\
		\hline
		\rowcolor[HTML]{85C0F9} T & T & T & T & T \\
		T & T & F & F & F \\
		\rowcolor[HTML]{85C0F9} T & F & T & T & T \\
		\rowcolor[HTML]{85C0F9} T & F & F & T & T \\
		\rowcolor[HTML]{85C0F9} F & T & T & T & T \\
		\rowcolor[HTML]{85C0F9} F & T & F & T & T \\
		\rowcolor[HTML]{85C0F9} F & F & T & T & T \\
		\rowcolor[HTML]{85C0F9} F & F & F & T & T \\
	\end{array}
\end{equation*}
This theorem is correct because whenever the premise, $P \rightarrow (Q \rightarrow R)$ is true, the conclusion $\neg R \rightarrow (P \rightarrow \neg Q)$ is also true, as indicated by the blue highlighted rows.

We can also see that the theorem is correct by comparing the logical form of the premise with the logical form of the conclusion.
The premise, $P \rightarrow (Q \rightarrow R)$, is equivalent to $\neg P \vee (\neg Q \vee R)$, or simply $\neg P \vee \neg Q \vee R$ by associativity of disjunction.
Similarly, the conclusion, $\neg R \rightarrow (P \rightarrow \neg Q)$ is equivalent to $R \vee (\neg P \vee \neg Q)$, which is equivalent to $\neg P \vee \neg Q \vee R$ by associativity and commutativity of disjunction.
Since the premise and conclusion are equivalent logical statements, the conclusion is true whenever the premise is true.
\end{proof}


\begin{statement}{3.2.16}
Use truth tables to check the correctness of the statements in Exercise 3.2.1.
\end{statement}

\begin{proof}
\hfill
\begin{enumerate}
	\item Recall that the statement was 
	``Suppose $P \rightarrow Q$ and $Q \rightarrow R$ are both true.
	Prove that $P \rightarrow R$ is true.''
	\begin{equation*}
		\begin{array}{| c c c | c  c | c |}
		P & Q & R & P \rightarrow Q &  Q \rightarrow R & P \rightarrow R\\
		\hline
		\rowcolor[HTML]{85C0F9} T & T & T & T & T & T \\
		T & T & F & T & F & F \\
		T & F & T & F & T & T \\
		T & F & F & F & T & F \\
		\rowcolor[HTML]{85C0F9} F & T & T & T & T & T \\
		F & T & F & T & F & T \\
		\rowcolor[HTML]{85C0F9} F & F & T & T & T & T \\
		\rowcolor[HTML]{85C0F9} F & F & F & T & T & T \\
		\end{array}
	\end{equation*}
	This statement is correct because whenever the premises, $P \rightarrow Q$ and
	$Q \rightarrow R$ are both true, the conclusion $P \rightarrow R$ is also true,
	as indicated by the blue highlighted rows.
	
	\item Recall that the statement was 
	``Suppose $\neg R \rightarrow (P \rightarrow \neg Q)$ is true.
	Prove that $P \rightarrow (Q \rightarrow R)$ is true.''
	\begin{equation*}
		\begin{array}{| c c c | c | c |}
			P & Q & R & \neg R \rightarrow (P \rightarrow \neg Q)
				& P \rightarrow (Q \rightarrow R) \\
			\hline
			\rowcolor[HTML]{85C0F9} T & T & T & T & T \\
			T & T & F & F & F \\
			\rowcolor[HTML]{85C0F9} T & F & T & T & T \\
			\rowcolor[HTML]{85C0F9} T & F & F & T & T \\
			\rowcolor[HTML]{85C0F9} F & T & T & T & T \\
			\rowcolor[HTML]{85C0F9} F & T & F & T & T \\
			\rowcolor[HTML]{85C0F9} F & F & T & T & T \\
			\rowcolor[HTML]{85C0F9} F & F & F & T & T \\
		\end{array}
	\end{equation*}
	This statement is correct because whenever the premise, 
	$\neg R \rightarrow (P \rightarrow \neg Q)$ is true, the conclusion 
	$P \rightarrow (Q \rightarrow R)$ is also true, as indicated by the blue highlighted rows.
	
	Our analysis from Exercise 3.2.15 also provides another perspective for the correctness
	of the statement, because the premise, $\neg R \rightarrow (P \rightarrow \neg Q)$,
	and the conclusion, $P \rightarrow (Q \rightarrow R)$, have equivalent logical forms.
\end{enumerate}
\end{proof}


\begin{statement}{3.2.17}
Use truth tables to check the correctness of the statements in Exercise 3.2.2.
\end{statement}

\begin{proof}
\hfill
\begin{enumerate}
	\item Recall that the statement was
	``Suppose $P \rightarrow Q$ and $R \rightarrow \neg Q$ are both true.
	Prove that $P \rightarrow \neg R$ is true.''
	\begin{equation*}
		\begin{array}{| c c c | c  c | c |}
		P & Q & R & P \rightarrow Q &  R \rightarrow \neg Q & P \rightarrow \neg R\\
		\hline
		T & T & T & T & F & F \\
		\rowcolor[HTML]{85C0F9} T & T & F & T & T & T \\
		T & F & T & F & T & F \\
		T & F & F & F & T & T \\
		F & T & T & T & F & T \\
		\rowcolor[HTML]{85C0F9} F & T & F & T & T & T \\
		\rowcolor[HTML]{85C0F9} F & F & T & T & T & T \\
		\rowcolor[HTML]{85C0F9} F & F & F & T & T & T \\
		\end{array}
	\end{equation*}
	This statement is correct because whenever the premises, $P \rightarrow Q$ and
	$R \rightarrow \neg Q$ are both true, the conclusion $P \rightarrow \neg R$ is also true,
	as indicated by the blue highlighted rows.
	
	\item Recall that the statement was 
	``Suppose that $P$ is true.
	Prove that $Q \rightarrow \neg (Q \rightarrow \neg P)$ is true.''
	\begin{equation*}
		\begin{array}{| c c | c | c |}
			P & Q & P & Q \rightarrow \neg (Q \rightarrow \neg P)\\
			\hline
			\rowcolor[HTML]{85C0F9} T & T & T & T \\
			\rowcolor[HTML]{85C0F9} T & F & T & T \\
			F & T & F & F \\
			F & F & F & T
		\end{array}
	\end{equation*}
	The statement is correct because whenever the premise $P$ is true, the conclusion
	$Q \rightarrow \neg (Q \rightarrow \neg P)$ is also true, as indicated by the blue
	highlighted rows.
	
	Note that we can more easily fill out the truth table by observing that the conclusion
	is logically equivalent to the simpler statement $\neg Q \vee P$.
\end{enumerate}
\end{proof}


\begin{statement}{3.2.18}
Can the proof in Example 3.2.2 be modified to prove that if $x^2 + y = 13$ and $x \neq 3$ then $y \neq 4$?
Explain.
Below is the theorem from Example 3.2.2 and its proof.
\begin{theorem}
	If $x^2 + y = 13$ and $y \neq 4$ then $x \neq 3$.
\end{theorem}
\begin{proof}
	Suppose $x^2 + y = 13$ and $y \neq 4$.
	Suppose $x = 3$.
	Substituting this into the equation $x^2 + y = 13$, we get $9 + y = 13$, so $y = 4$.
	But this contradicts the fact that $y \neq 4$.
	Therefore $x \neq 3$.
	Thus, if $x^2 + y = 13$ and $y \neq 4$ then $x \neq 3$.
\end{proof}
\end{statement}

\begin{proof}
The proof in Example 3.2.2 cannot be modified to prove that if $x^2 + y = 13$ and $x \neq 3$ then $y \neq 4$?
This is because the statement ``If $x^2 + y = 13$ and $x \neq 3$ then $y \neq 4$," is false.
To see why, consider the case when $x = -3$, which satisfies the hypothesis that $x \neq 3$.
In this situation, when also $x^2 + y = 13$, we can plug in $x = -3$ to obtain the equation $9 + y = 13$.
From the equation $9 + y = 13$ it follows that $y = 4$, which does not satisfy the conclusion $y \neq 4$.
Since we have found a situation where the premises of the statement are true, but the conclusion is false, the statement itself is false.
\end{proof}


\begin{statement}{3.3.1}
In Exercise 2.2.7 you used logical equivalences to show that $\exists x (P(x) \rightarrow Q(x))$ is equivalent to $\forall x P(x) \rightarrow \exists x Q(x)$.
Now use the methods of this section to prove that if $\exists x (P(x) \rightarrow Q(x))$ is true, then $\forall x P(x) \rightarrow \exists x Q(x)$ is true.
(\emph{Note: The other direction of the equivalence is quite a bit harder to prove.
See Exercise 3.5.30.})
\end{statement}

\begin{proof}
Suppose that $\exists x (P(x) \rightarrow Q(x))$ is true.
Now our goal is to prove that the statement $\forall x P(x) \rightarrow \exists x Q(x)$ is true.
We then suppose that $\forall x P(x)$ is true and update our goal to proving that $\exists x Q(x)$ is true.
Since $\exists x (P(x) \rightarrow Q(x))$ is true, there exists $x = x_0$ such that $P(x_0) \rightarrow Q(x_0)$ is true.
Then, since $\forall x P(x)$ is true, the statement $P(x_0)$, in particular, is true.
Because $P(x_0) \rightarrow Q(x_0)$ is true and $P(x_0)$ is true, we conclude that $Q(x_0)$ is true as well.
In other words, we have found that there exists $x$ such that $Q(x)$ is true.
Thus, the statement $\forall x P(x) \rightarrow \exists x Q(x)$ is true, and we ultimately conclude that if $\exists x (P(x) \rightarrow Q(x))$ is true, then $\forall x P(x) \rightarrow \exists x Q(x)$ is true as well.
This completes the proof.
\end{proof}


\begin{statement}{3.3.2}
Prove that if $A$ and $B \setminus C$ are disjoint, then $A \cap B \subseteq C$.
\end{statement}

\begin{proof}
Suppose $A$ and $B \setminus C$ are disjoint.
Let $x \in A \cap B$ be arbitrary.
We now want to show that $x \in C$.
Suppose, towards a contradiction, that $x \notin C$.
Since $x \in A \cap B$, $x \in A$ and $x \in B$.
Then, since $x \in B$ and $x \notin C$, $x \in B \setminus C$.
It then follows that $x \in A \cap (B \setminus C)$, but this contradicts the assumption that $A$ and $B \setminus C$ are disjoint.
Thus, $x \in C$, so $A \cap B \subseteq C$.
Therefore, we conclude that if $A$ and $B \setminus C$ are disjoint, then $A \cap B \subseteq C$.
\end{proof}


\begin{statement}{3.3.3}
Prove that if $A \subseteq B \setminus C$ then $A$ and $C$ are disjoint.
\end{statement}

\begin{proof}
Suppose $A \subseteq B \setminus C$ and suppose, towards a contradiction that $A$ and $C$ are not disjoint.
In other words, there is $x \in A \cap C$.
Since $x \in A$ and $A \subseteq B \setminus C$, it follows that $x \in B \setminus C$.
In particular, $x \notin C$.
However, this contradicts the fact that $x \in A \cap C$ means that $x \in C$.
Thus, we conclude that $A$ and $C$ are disjoint.
Therefore, if $A \subseteq B \setminus C$ then $A$ and $C$ are disjoint.
\end{proof}


\begin{statement}{3.3.4}
Suppose $A \subseteq \powerset{A}$.
Prove that $\powerset{A} \subseteq \powerset{\powerset{A}}$.
\end{statement}

\begin{proof}
Let $X \in \powerset{A}$ be arbitrary.
By the definition of the power set of a set $A$, $X \subseteq A$.
Then, since $A \subseteq \powerset{A}$, it follows that $X \subseteq \powerset{A}$.
In other words, $X$ is an element of the power set of $\powerset{A}$.
Since $X$ was arbitrary, we conclude that $\powerset{A} \subseteq \powerset{\powerset{A}}$.
\end{proof}


\begin{statement}{3.3.5}
The hypothesis of the theorem in Exercise 3.3.4 is $A \subseteq \powerset{A}$.
\begin{enumerate}
	\item Can you think of a set $A$ for which this hypothesis is true?
	
	\item Can you think of another?
\end{enumerate}
\end{statement}

\begin{proof}
\hfill
\begin{enumerate}
	\item Since the empty set $\varnothing$ is a subset of any set $X$, it is a set for which the
	hypothesis $A \subseteq \powerset{A}$ is true.
	
	\item Once we observe that $A = \varnothing$ satisfies the hypothesis
	$A \subseteq \powerset{A}$, the theorem tells us that 
	$\powerset{A} \subseteq \powerset{\powerset{A}}$.
	In other words, $\powerset{A}$ is another set which satisfies the hypothesis.
	We can check this directly by noting that $\powerset{\varnothing} = \{ \varnothing \}$,
	$\powerset{\{ \varnothing \}} = \{ \varnothing, \{ \varnothing \} \}$,
	and $\{ \varnothing \} \subseteq \{ \varnothing, \{ \varnothing \} \}$.
\end{enumerate}
\end{proof}


\begin{statement}{3.3.6}
Suppose $x$ is a real number.
\begin{enumerate}
	\item Prove that if $x \neq 1$ then there is a real number $y$ such that $(y + 1)/(y - 2) = x$.
	
	\item Prove that if there is a real number $y$ such that $(y + 1)/(y - 2) = x$, then $x \neq 1$.
\end{enumerate}
\end{statement}

\begin{proof}
\hfill
\begin{enumerate}
	\item Suppose $x \neq 1$.
	Consider $y = (2x + 1)/(x - 1)$, which is a well-defined real number since $x \neq 1$.
	We then plug this value of $y$ into the expression $(y + 1)/(y - 2)$.
	\begin{equation*}
		\frac{y + 1}{y - 2}
		= \frac{\frac{2x + 1}{x - 1} + 1}{\frac{2x + 1}{x - 1} - 2}
		= \frac{2x + 1 + (x - 1)}{2x + 1 - 2(x - 1)}
		= \frac{3x}{3}
		= x
	\end{equation*}
	Thus, we have found a real number $y$ such that $(y + 1)/(y - 2) = x$.
	We conclude that that if $x \neq 1$ then there is a real number 
	$y$ such that $(y + 1)/(y - 2) = x$.
	
	\item Suppose there is a real number $y$ such that $(y + 1)/(y - 2) = x$.
	Since $y + 1 \neq y - 2$ for all real numbers $y$, it follows that $x = (y + 1)/(y - 2) \neq 1$.
	Therefore, we conclude that if there is a real number $y$ such that 
	$(y + 1)/(y - 2) = x$, then $x \neq 1$.
\end{enumerate}
\end{proof}


\begin{statement}{3.3.7}
Prove that for every real number $x$, if $x > 2$ then there is a real number $y$ such that $y + 1/y = x$.
\end{statement}

\begin{proof}
Let $x > 2$ be an arbitrary real number.
Consider $y = (x + \sqrt{x^2 - 4})/2$, which is a real number since $x > 2$.
We then plug this value of $y$ into the expression $y + 1/y$.
\begin{align*}
	y + \frac{1}{y}
	&= \frac{x + \sqrt{x^2 - 4}}{2} + \frac{2}{x + \sqrt{x^2 - 4}} \\
	&= \frac{x^2 + 2x\sqrt{x^2 - 4} + (x^2 - 4)}{2(x + \sqrt{x^2 - 4})}
		+ \frac{4}{2(x + \sqrt{x^2 - 4})} \\
	&= \frac{2x^2 + 2x\sqrt{x^2 - 4}}{2(x + \sqrt{x^2 - 4})} \\
	&= \frac{2x(x + \sqrt{x^2 - 4})}{2(x + \sqrt{x^2 - 4})} \\
	&= x
\end{align*}
Thus, we have found a real number $y$ such that $y + 1/y = x$.
Therefore, we conclude that for every real number $x$, if $x > 2$ then there is a real number $y$ such that $y + 1/y = x$.
\end{proof}


\begin{statement}{3.3.8}
Prove that if $\mathcal{F}$ is a family of sets and $A \in \mathcal{F}$, then $A \subseteq \bigcup \mathcal{F}$.
\end{statement}

\begin{proof}
Suppose $\mathcal{F}$ is a family of sets and $A \in \mathcal{F}$.
Let $a \in A$ be arbitrary.
We want to show that $a \in \bigcup \mathcal{F}$.
Since $\bigcup \mathcal{F} = \{ x \mid \exists X \in \mathcal{F} (x \in X) \}$, showing that $a \in \bigcup \mathcal{F}$ is equivalent to showing that there is $X \in \mathcal{F}$ such that $a \in X$.
Let $X = A$.
By assumption, $A \in \mathcal{F}$ and $a \in A$.
Thus, we have found $X \in \mathcal{F}$ such that $a \in X$, so $a \in \bigcup \mathcal{F}$.
Since $a \in A$ was arbitrary, we conclude that $A \subseteq \bigcup \mathcal{F}$.
Therefore, if $\mathcal{F}$ is a family of sets and $A \in \mathcal{F}$, then $A \subseteq \bigcup \mathcal{F}$.
\end{proof}


\begin{statement}{3.3.9}
Prove that if $\mathcal{F}$ is a family of sets and $A \in \mathcal{F}$, then $\bigcap \mathcal{F} \subseteq A$.
\end{statement}

\begin{proof}
Suppose $\mathcal{F}$ is a family of sets and $A \in \mathcal{F}$.
Let $x \in \bigcap \mathcal{F}$ be arbitrary.
Recall that $\bigcap \mathcal{F} = \{ x \mid \forall X \in \mathcal{F} (x \in X) \}$.
In particular, since $A \in \mathcal{F}$, it follows that $x \in A$.
Since $x \in \bigcap \mathcal{F}$ was arbitrary, we conclude that $\bigcap \mathcal{F} \subseteq A$.
Therefore, if $\mathcal{F}$ is a family of sets and $A \in \mathcal{F}$, then $\bigcap \mathcal{F} \subseteq A$.
\end{proof}


\begin{statement}{3.3.10}
Suppose that $\mathcal{F}$ is a nonempty family of sets, $B$ is a set, and for all $A \in \mathcal{F}$ that $B \subseteq A$.
Prove that $B \subseteq \bigcap \mathcal{F}$.
\end{statement}

\begin{proof}
Let $b \in B$ be arbitrary.
We want to show that $b \in \bigcap \mathcal{F}$.
Since $\bigcap \mathcal{F} = \{ x \mid \forall A \in \mathcal{F} (x \in A) \}$, showing that $b \in \bigcap \mathcal{F}$ is equivalent to showing that $b \in A$ for all $A \in \mathcal{F}$.
Because $B \subseteq A$ for all $A \in \mathcal{F}$ and $b \in B$, it follows that $b \in A$ for all $A \in \mathcal{F}$.
In other words, $b \in \bigcap \mathcal{F}$.
Since $b \in B$ was arbitrary, we conclude that $B \subseteq \bigcap \mathcal{F}$.
This completes the proof.
\end{proof}


\begin{statement}{3.3.11}
Suppose that $\mathcal{F}$ is a family of sets.
Prove that if $\varnothing \in \mathcal{F}$ then $\bigcap \mathcal{F} = \varnothing$.
\end{statement}

\begin{proof}
Suppose $\varnothing \in \mathcal{F}$, and suppose, towards a contradiction, that $\bigcap \mathcal{F} \neq \varnothing$.
In other words, there exists $x \in \bigcap \mathcal{F}$.
By the definition of $\bigcap \mathcal{F}$, this means that $x \in X$ for all $X \in \mathcal{F}$.
In particular, since $X = \varnothing \in \mathcal{F}$, this means that $x \in \varnothing$.
This, however, is a contradiction against the definition of the empty set.
We therefore conclude that $\bigcap \mathcal{F} = \varnothing$.
Thus, if $\varnothing \in \mathcal{F}$ then $\bigcap \mathcal{F} = \varnothing$.
\end{proof}


\begin{statement}{3.3.12}
Suppose $\mathcal{F}$ and $\mathcal{G}$ are families of sets.
Prove that if $\mathcal{F} \subseteq \mathcal{G}$ then $\bigcup \mathcal{F} \subseteq \bigcup \mathcal{G}$.
\emph{Note: In my digital copy of the book this exercise has a typo and asks for a proof of the incorrect statement that if $\mathcal{F} \subseteq \mathcal{G}$ then $\bigcup \mathcal{F} \subseteq \mathcal{G}$.}
\end{statement}

\begin{proof}
Suppose $\mathcal{F} \subseteq \mathcal{G}$.
Let $x \in \bigcup \mathcal{F}$ be arbitrary.
By the definition of $\bigcup \mathcal{F}$, this means that there is $X \in \mathcal{F}$ such that $x \in X$.
Then, since $\mathcal{F} \subseteq \mathcal{G}$, it follows that $X \in \mathcal{G}$ as well.
Since $x \in X$ and $X \in \mathcal{G}$, $x \in \bigcup \mathcal{G}$ by the definition of $\bigcup \mathcal{G}$.
Thus, since $x \in \bigcup \mathcal{F}$ was arbitrary, we conclude that $\bigcup \mathcal{F} \subseteq \bigcup \mathcal{G}$.
Therefore, if $\mathcal{F} \subseteq \mathcal{G}$ then $\bigcup \mathcal{F} \subseteq \bigcup \mathcal{G}$.
\end{proof}


\begin{statement}{3.3.13}
Suppose $\mathcal{F}$ and $\mathcal{G}$ are nonempty families of sets.
Prove that if $\mathcal{F} \subseteq \mathcal{G}$ then $\bigcap \mathcal{G} \subseteq \bigcap \mathcal{F}$.
\end{statement}

\begin{proof}
Suppose $\mathcal{F} \subseteq \mathcal{F}$.
Let $x \in \bigcap \mathcal{G}$ be arbitrary.
By the definition of $\bigcap \mathcal{G}$, this means that for all $X \in \mathcal{G}$, $x \in X$.
Note that the assumption that $\mathcal{G}$ is nonempty makes this a non-vacuous statement.
Now, let $Y \in \mathcal{F}$ be arbitrary.
Note that such a $Y$ exists by the assumption that $\mathcal{F}$ is nonempty.
Since $\mathcal{F} \subseteq \mathcal{G}$, $Y \in \mathcal{G}$.
Because $Y \in \mathcal{G}$ and for all $X \in \mathcal{G}$, $x \in X$, it follows that $x \in Y$.
Therefore, since $Y \in \mathcal{F}$ was arbitrary, $x \in Y$ for all $Y \in \mathcal{F}$.
In other words, $x \in \bigcap \mathcal{F}$ by the definition of $\bigcap \mathcal{F}$.
Since $x \in \bigcap \mathcal{G}$ was arbitrary, we conclude that $\bigcap \mathcal{G} \subseteq \bigcap \mathcal{F}$.
Thus, if $\mathcal{F} \subseteq \mathcal{G}$ then $\bigcap \mathcal{G} \subseteq \bigcap \mathcal{F}$.
\end{proof}


\begin{statement}{3.3.14}
Suppose that $\{ A_i \mid i \in I \}$ is an indexed family of sets.
Prove that $\bigcup_{i \in I} \powerset{A_i} \subseteq \powerset{\bigcup_{i \in I} A_i}$.
(\emph{Hint: First make sure you know what all the notation means!})
\end{statement}

\begin{proof}
Let $X \in \bigcup_{i \in I} \powerset{A_i}$ be arbitrary.
By the definition of $\bigcup_{i \in I} \powerset{A_i}$, this means that there is $i \in I$ such that $X \in \powerset{A_i}$.
Further unraveling the definition of being an element of the power set of a set $A$, this means that there is $i \in I$ such that $X \subseteq A_i$.
Since $X \subseteq A_i$ and $A_i \subseteq \bigcup_{i \in I} A_i$, it follows that $X \subseteq \bigcup_{i \in I} A_i$.
In other words, $X \in \powerset{\bigcup_{i \in I} A_i}$.
Since $X \in \bigcup_{i \in I} \powerset{A_i}$ was arbitrary, we conclude that $\bigcup_{i \in I} \powerset{A_i} \subseteq \powerset{\bigcup_{i \in I} A_i}$.
This completes the proof.
\end{proof}


\begin{statement}{3.3.15}
Suppose $\{ A_i \mid i \in I \}$ is an indexed family of sets and $I \neq \varnothing$.
Prove that $\bigcap_{i \in I} A_i \in \bigcap_{i \in I} \powerset{A_i}$.
\emph{Note: In my digital copy of the book this exercise has a typo and says that $I = \varnothing$.}
\end{statement}

\begin{proof}
We start by unraveling the definitions of what it means to be an element of $\bigcap_{i \in I} \powerset{A_i}$.
The first layer of definitions states that $X \in \bigcap_{i \in I} \powerset{A_i}$ if and only if $X \in \powerset{A_i}$ for all $i \in I$.
Unraveling one step further, this is equivalent to $X \subseteq A_i$ for all $i \in I$, by the definition of the power set of a set $A$.
In other words, proving that $\bigcap_{i \in I} A_i \in \bigcap_{i \in I} \powerset{A_i}$ is equivalent to proving that $\bigcap_{i \in I} A_i \subseteq A_i$ for all $i \in I$.
This follows directly from the definition of $\bigcap_{i \in I} A_i$.
Letting $x \in \bigcap_{i \in I} A_i$ be arbitrary, $x \in A_i$ for all $i \in I$.
Therefore, $\bigcap_{i \in I} A_i \subseteq A_i$ for all $i \in I$.
This completes the proof that $\bigcap_{i \in I} A_i \in \bigcap_{i \in I} \powerset{A_i}$.
\end{proof}


\begin{statement}{3.3.16}
Prove the converse of the statement proven in Example 3.3.5.
In other words, prove that if $\mathcal{F} \subseteq \powerset{B}$ then $\bigcup \mathcal{F} \subseteq B$.
\end{statement}

\begin{proof}
Suppose $\mathcal{F} \subseteq \powerset{B}$.
Let $x \in \bigcup \mathcal{F}$ be arbitrary.
By the definition of $\bigcup \mathcal{F}$, this means that there is $X \in \mathcal{F}$ such that $x \in X$.
Then, since $\mathcal{F} \subseteq \powerset{B}$, $X \in \powerset{B}$.
In other words, by the definition of the power set of $B$, $X \subseteq B$.
It then follows that $x \in B$.
Since $x \in \bigcup \mathcal{F}$ was arbitrary, we conclude that $\bigcup \mathcal{F} \subseteq B$.
Therefore, we conclude that if $\mathcal{F} \subseteq \powerset{B}$ then $\bigcup \mathcal{F} \subseteq B$.
\end{proof}


\begin{statement}{3.3.17}
Suppose $\mathcal{F}$ and $\mathcal{G}$ are nonempty families of sets, and every element of $\mathcal{F}$ is a subset of every element of $\mathcal{G}$.
Prove that $\bigcup \mathcal{F} \subseteq \bigcap \mathcal{G}$.
\end{statement}

\begin{proof}
Let $x \in \bigcup \mathcal{F}$ be arbitrary.
By the definition of $\bigcup \mathcal{F}$, this means that there is $X \in \mathcal{F}$ such that $x \in X$.
Then, by the assumption that every element of $\mathcal{F}$ is a subset of every element of $\mathcal{G}$, we know that for all $Y \in \mathcal{G}$, $X \subseteq Y$.
It follows that $x \in Y$ for all $Y \in \mathcal{G}$.
In other words, by the definition of $\bigcap \mathcal{G}$, $x \in \bigcap \mathcal{G}$.
Hence, since $x \in \bigcup \mathcal{F}$ was arbitrary, we conclude that $\bigcup \mathcal{F} \subseteq \bigcap \mathcal{G}$.
\end{proof}


\begin{statement}{3.3.18}
In this problem all variables range over $\BZ$, the set of all integers.
\begin{enumerate}
	\item Prove that if $a$ divides $b$ and $a$ divides $c$, then $a$ divides $b + c$.
	
	\item Prove that if $ac$ divides $bc$ and $c \neq 0$, then $a$ divides $b$.
\end{enumerate}
\end{statement}

\begin{proof}
\hfill
\begin{enumerate}
	\item Suppose $a$ divides $b$ and $a$ divides $c$.
	Since $a$ divides $b$, there is $m \in \BZ$ such that $ma = b$.
	Similarly, since $a$ divides $c$, there is $n \in \BZ$ such that $na = c$.
	It then follows that $b + c = ma + na = (m + n)a$.
	Since $m + n$ is an integer, we conclude that $a$ divides $b + c$.
	Therefore, if $a$ divides $b$ and $a$ divides $c$, then $a$ divides $b + c$.
	
	\item Suppose $ac$ divides $bc$ and $c \neq 0$.
	Since $ac$ divides $bc$, there is $m \in \BZ$ such that $mac = bc$.
	Then, since $c \neq 0$, we can divide both sides of the equation 
	by $c$ to conclude that $ma = b$.
	In other words, $a$ divides $b$.
	Therefore, we conclude that if $ac$ divides $bc$ and $c \neq 0$, then $a$ divides $b$.
\end{enumerate}
\end{proof}


\begin{statement}{3.3.19}
\begin{enumerate}
	\item Prove that for all real numbers $x$ and $y$ there is a real number $z$ such that
	$x + z = y - z$.
	
	\item Would the statement in Part (1) be correct if ``real number'' were changed to ``integer''?
	Justify your answer.
\end{enumerate}
\end{statement}

\begin{proof}
\hfill
\begin{enumerate}
	\item Let $x, y \in \BR$ be arbitrary. 
	Consider the real number $z = (y - x) / 2$.
	We plug this value for $z$ into the expressions $x + z$ and $y - z$ to check that they are equal.
	\begin{align*}
		x + z &= x + \frac{y - x}{2} = \frac{x + y}{2} \\
		y - z &= y - \frac{y - x}{2} = \frac{x + y}{2}
	\end{align*}
	As we can see, $x + z = (x + y) / 2 = y - z$.
	This completes the proof.
	
	\item The statement in Part (1) would be incorrect if ``real number'' were changed to ``integer''.
	This comes from the way in which $z$ is defined by 
	isolating $z$ in the equation $x + z = y - z$ to get $z = (y - x) / 2$.
	The difference of two real numbers is not guaranteed to be an integer,
	let alone an integer that is divisible by 2.
	For example, consider the case of $x = 0$ and $y = 1$.
	In this case, the proof gives us a value of $z = 1/2$, which is not an integer.
\end{enumerate}
\end{proof}


\begin{statement}{3.3.20}
Consider the following theorem:
\begin{theorem}
	For every real number $x$, $x^2 \geq 0$.
\end{theorem}
What's wrong with the following proof of the theorem?
\begin{proof}
	Suppose not.
	Then for every for every real number $x$, $x^2 < 0$.
	In particular, plugging in $x = 3$ we would get $9 < 0$, which is clearly false.
	This contradiction shows that for every number $x$, $x^2 \geq 0$.
\end{proof}
\end{statement}

\begin{proof}
The issue with the given attempted proof is in the negation of the statement ``For every real number $x$, $x^2 \geq 0$,'' to try and reach a contradiction.
Since the statement has the logical form $\forall x P(x)$, the correct negation has the logical form $\exists x \neg P(x)$.
In other words, the correct negation is ``There exists a real number $x$ such that $x^2 < 0$,'' instead of ``For every real number $x$, $x^2 < 0$.''
\end{proof}


\begin{statement}{3.3.21}
Consider the following incorrect theorem:
\begin{theorem}
	If $\forall x \in A (x \neq 0)$ and $A \subseteq B$ then $\forall x \in B (x \neq 0)$.
\end{theorem}
\begin{enumerate}
	\item What's wrong with the following proof of the theorem?
	\begin{proof}
		Suppose that $\forall x \in A (x \neq 0)$ and $A \subseteq B$.
		Let $x$ be an arbitrary element of $A$.
		Since $\forall x \in A (x \neq 0)$, we can conclude that $x \neq 0$.
		Also, since $A \subseteq B$, $x \in B$.
		Since $x \in B$, $x \neq 0$, and $x$ was arbitrary, 
		we can conclude that $\forall x \in B (x \neq 0)$.
	\end{proof}
	
	\item Find a counterexample to the theorem.
	In other words, find an example of sets $A$ and $B$ for which the hypotheses of the theorem
	are true but the conclusion is false.
\end{enumerate}
\end{statement}

\begin{proof}
\hfill
\begin{enumerate}
	\item The issue with the attempted proof is in the statement
	``Let $x$ be an arbitrary element of $A$''.
	The goal of the proof is to prove that for all $x \in B$, $x \neq 0$,
	so we must instead start with an arbitrary element of $B$.
	
	\item One counterexample to the theorem is when $A = \{ 1, 2, 3 \}$ and $B = \{ 0, 1, 2, 3 \}$.
	In this case, all elements of $A$ are nonzero and $A \subseteq B$, 
	but not all elements of $B$ are nonzero.
\end{enumerate}
\end{proof}


\begin{statement}{3.3.22}
Consider the following incorrect theorem:
\begin{theorem}
	$\exists x \in \BR \forall y \in \BR (xy^2 = y - x)$.
\end{theorem}
What's wrong with the following proof of the theorem?
\begin{proof}
	Let $x = y / (y^2 + 1)$. Then
	\begin{equation*}
		y - x
		= y - \frac{y}{y^2 + 1}
		= \frac{y^3}{y^2 + 1}
		= \frac{y}{y^2 + 1} \cdot y^2
		= xy^2.
	\end{equation*}
\end{proof}
\end{statement}

\begin{proof}
The issue in the proof is that this value of $x$ depends on the value of $y$.
This is different from what the theorem states: that there is a single value of $x$, \textbf{independent} of the value of $y$, such that $xy^2 = y - x$ for all values of $y$.
In terms of proof structure, we need to start by defining the real number $x$, which at this point cannot be defined in terms of $y$ since $y$ hasn't been introduced in the proof.
Only after $x$ is defined, would we take an arbitrary $y \in \BR$ and attempt to show that $xy^2 = y - x$.
\end{proof}


\begin{statement}{3.3.23}
Consider the following incorrect theorem:
\begin{theorem}
	Suppose $\mathcal{F}$ and $\mathcal{G}$ are families of sets.
	If $\bigcup \mathcal{F}$ and $\bigcup \mathcal{G}$ are disjoint,
	then so are $\mathcal{F}$ and $\mathcal{G}$.
\end{theorem}
\begin{enumerate}
	\item What's wrong with the following proof of the theorem?
	\begin{proof}
		Suppose $\bigcup \mathcal{F}$ and $\bigcup \mathcal{G}$ are disjoint.
		Suppose $\mathcal{F}$ and $\mathcal{G}$ are not disjoint.
		Then we can choose some set $A$ such that $A \in \mathcal{F}$ and $A \in \mathcal{G}$.
		Since $A \in \mathcal{F}$, by Exercise 3.3.8, $A \subseteq \bigcup \mathcal{F}$,
		so every element of $A$ is in $\bigcup \mathcal{F}$.
		Similarly, since $A \in \mathcal{G}$, every element of $A$ is in $\bigcup \mathcal{G}$.
		But then every element of $A$ is in both $\bigcup \mathcal{F}$ and $\bigcup \mathcal{G}$,
		and this is impossible since $\bigcup \mathcal{F}$ and $\bigcup \mathcal{G}$ are disjoint.
		Thus, we have reached a contradiction, 
		so $\mathcal{F}$ and $\mathcal{G}$ must be disjoint.
	\end{proof}
	% How do we know that A is nonempty?
	
	\item Find a counterexample to the theorem.
	% Choose F and G such that their intersection is the set containing the empty set.
\end{enumerate}
\end{statement}

\begin{proof}
\hfill
\begin{enumerate}
	\item The issue with the attempted proof lies in the statement
	``Then we can choose some set $A$ such that $A \in \mathcal{F}$ and $A \in \mathcal{G}$''.
	We do not know that $A$ is nonempty since there are no assumptions as to whether or not
	$\varnothing \in \mathcal{F} \cap \mathcal{G}$.
	If $A = \varnothing$, then there is no contradiction to having 
	$A \in \mathcal{F} \cap \mathcal{G}$ and having 
	$\bigcup \mathcal{F}$ and $\bigcup \mathcal{G}$ being disjoint,
	since $A$ contains no elements.
	
	\item One counterexample to the theorem is the case where 
	$\mathcal{F} = \{ \varnothing, \{ 1 \} \}$ and
	$\mathcal{G} = \{ \varnothing, \{ 2 \} \}$.
	In this case, $\bigcup \mathcal{F} = \{ 1 \}$ and $\bigcup \mathcal{G} = \{ 2 \}$,
	which are clearly disjoint, but $\mathcal{F} \cap \mathcal{G} = \{ \varnothing \}$.
	As discussed previously, the set containing the empty set is \textbf{not} empty.
\end{enumerate}
\end{proof}


\begin{statement}{3.3.24}
Consider the following putative theorem:
\begin{theorem}
	For all real numbers $x$ and $y$, $x^2 + xy - 2y^2 = 0$.
\end{theorem}
\begin{enumerate}
	\item What's wrong with the following proof of the theorem?
	\begin{proof}
		Let $x$ and $y$ be equal to some arbitrary real number $r$.
		Then
		\begin{equation*}
			x^2 + xy - 2y^2 = r^2 + r \cdot r - 2r^2 = 0.
		\end{equation*}
		Since $x$ and $y$ were both arbitrary, this shows that for all real number
		$x$ and $y$, $x^2 + xy - 2y^2 = 0$.
	\end{proof}
	% There is no assumption that x = y.
	
	\item Is the theorem correct?
	Justify your answer with either a proof or a counterexample.
	% Theorem is incorrect. Choose x = 1, y = 0 for a counterexample.
\end{enumerate}
\end{statement}

\begin{proof}
\hfill
\begin{enumerate}
	\item The issue with the attempted proof lies in the statement
	``Let $x$ and $y$ be equal to some arbitrary real number $r$'',
	since there is no assumption in the theorem statement that $x = y$.
	Instead, we must assume that $x$ is an arbitrary real number and that $y$ is an arbitrary real
	number, while not making any further assumptions about the relationship between $x$ and $y$.
	
	\item The theorem is incorrect.
	Consider the case where $x = 1$ and $y = 0$.
	In this case, $x^2 + xy - 2y^2 = 1^2 + (1)(0) - 2(0)^2 = 1 \neq 0$.
\end{enumerate}
\end{proof}


\begin{statement}{3.3.25}
Prove that for every real number $x$ there is a real number $y$ such that for every real number $z$, $yz = (x + z)^2 - (x^2 + z^2)$.
\end{statement}

\begin{proof}
Let $x \in \BR$ be arbitrary.
Let $y = 2x$.
Now let $z \in \BR$ be arbitrary.
We then have $yz = (2x)z = 2xz$.
We also have $(x + z)^2 - (x^2 + z^2) = x^2 + 2xz + z^2 - x^2 - z^2 = 2xz$.
Thus, $yz = 2xz = (x + z)^2 - (x^2 + z^2)$.
This completes the proof.

\emph{
Note that in order to properly prove the statement, we must be careful about the order in which we perform the setup.
We first have to let $x \in \BR$ be arbitrary.
Once we do that, we then define $y$.
When defining $y$, the only variable we can use is $x$, since that is the only other variable that has been introduced at this point.
Only after we have defined $y$ can we let $z \in \BR$ be arbitrary.
One way we can think about the order of introducing variables is that while the value of $y$ might depend on the choice of $x$, once the values of $x$ and $y$ have been set, then the rest of the statement has to hold true for any value of $z$.
}
\end{proof}


\begin{statement}{3.3.26}
\begin{enumerate}
	\item Comparing the various rules for dealing with quantifiers in proofs,
	you should see a similarity between the rules for goals of the form $\forall x P(x)$
	and givens of the form $\exists x P(x)$.
	What is this similarity?
	What about the rules for goals of the form $\exists x P(x)$ and givens of the form $\forall x P(x)$?
	
	\item Can you think of a reason why these similarities might be expected?
	(\emph{Hint: Think about how a proof by contradiction works when
	the goal starts with a quantifier.})
\end{enumerate}
\end{statement}

\begin{proof}
\hfill
\begin{enumerate}
	\item The similarity between the rules for goals of the form $\forall x P(x)$
	and givens of the form $\exists x P(x)$ is that both involve introduce a variable $x$.
	When working toward a goal of the form $\forall x P(x)$, we introduce an arbitrary $x$
	and try to prove $P(x)$, while when working with a given $\exists x P(x)$,
	we introduce an arbitrary $x$ that satisfies $P(x)$.
	For goals of the form $\exists x P(x)$ and givens of the form $\forall x P(x)$, we instead
	try to find or construct a particular value $a$ to plug in for $x$.
	When working toward a goal of the form $\exists x P(x)$, we construct a particular value $a$
	such that $P(a)$ is true, while when working with a given $\forall x P(x)$, we try to find
	a relevant value $a$ to then immediately plug into the statement $P$ and conclude
	that $P(a)$ is true.
	
	\item One reason why these similarities might be expected comes from the quantifier
	negation laws: $\neg \forall x P(x)$ is equivalent to $\exists x \neg P(x)$
	and $\neg \exists x P(x)$ is equivalent to $\forall x \neg P(x)$.
	If we are performing proof by contradiction when the goal starts with a quantifier,
	we use those negation laws when we add the negated goal as a given and then
	attempt to work toward a contradiction.
	In other words, a goal of the form $\forall x P(x)$ would turn into a given of the form
	$\exists x \neg P(x)$, while a goal of the form $\exists x P(x)$ would turn into a given
	of the form $\forall x \neg P(x)$.
\end{enumerate}
\end{proof}


\begin{statement}{3.4.1}
Use the methods of this chapter to prove that $\forall x (P(x) \wedge Q(x))$ is equivalent to $\forall x P(x) \wedge \forall x Q(x)$.
\end{statement}

\begin{proof}
We first prove the forward direction: $\forall x (P(x) \wedge Q(x)) \rightarrow \forall x P(x) \wedge \forall x Q(x)$.
Suppose $\forall x (P(x) \wedge Q(x))$ is true.
Now let $x_1$ and $x_2$ be arbitrary.
By the assumption, $P(x_1) \wedge Q(x_1)$ is true, so in particular $P(x_1)$ is true.
Since $x_1$ was arbitrary, we conclude that $\forall x P(x)$ is true.
Using the same argument with $x_2$, we also conclude that $\forall x Q(x)$ is true.
In other words, $\forall x P(x) \wedge \forall x Q(x)$ is true.
This completes the proof of the forward direction.

Now we prove the other direction: $\forall x P(x) \wedge \forall x Q(x) \rightarrow \forall x (P(x) \wedge Q(x))$.
Suppose $\forall x P(x) \wedge \forall x Q(x)$ is true.
Let $x_1$ be arbitrary.
By the assumption, $P(x_1)$ is true, and $Q(x_1)$ is true as well.
In other words, $P(x_1) \wedge Q(x_1)$ is true.
Since $x_1$ was arbitrary, we conclude that $\forall x (P(x) \wedge Q(x))$ is true. This completes the proof of the reverse direction.
\end{proof}


\begin{statement}{3.4.2}
Prove that if $A \subseteq B$ and $A \subseteq C$ then $A \subseteq B \cap C$.
\end{statement}

\begin{proof}
Suppose $A \subseteq B$ and $A \subseteq C$.
Let $x \in A$ be arbitrary.
By the assumptions that $A \subseteq B$, it follows that $x \in B$.
Similarly, $x \in C$ as well.
In other words, $x \in B \cap C$.
Since our choice of $x \in A$ was arbitrary, we conclude that $A \subseteq B \cap C$.
Therefore, if $A \subseteq B$ and $A \subseteq C$ then $A \subseteq B \cap C$.
\end{proof}


\begin{statement}{3.4.3}
Suppose $A \subseteq B$.
Prove that for every set $C$, $C \setminus B \subseteq C \setminus A$.
\end{statement}

\begin{proof}
Let $C$ be an arbitrary set.
Now let $x \in C \setminus B$ be arbitrary.
By the definition of $C \setminus B$, $x \in C$ and $x \notin B$.
Since $x \notin B$ and $A \subseteq B$, it follows that $x \notin A$.
Then, since $x \in C$ and $x \notin A$, $x \in C \setminus A$.
Because $x \in C \setminus B$ was arbitrary, we conclude that $C \setminus B \subseteq C \setminus A$.
Finally, since $C$ was an arbitrary set, we conclude that for every set $C$, $C \setminus B \subseteq C \setminus A$.
\end{proof}


\begin{statement}{3.4.4}
Prove that if $A \subseteq B$ and $A \nsubseteq C$ then $B \nsubseteq C$.
\end{statement}

\begin{proof}
Suppose $A \subseteq B$ and $A \nsubseteq C$.
Since $A \nsubseteq C$, there is $x \in A$ such that $x \notin C$.
Then, since $A \subseteq B$, it follows that $x \in B$.
Therefore, since $x \in B$ and $x \notin C$, we conclude that $B \nsubseteq C$.
Thus, if $A \subseteq B$ and $A \nsubseteq C$ then $B \nsubseteq C$.
\end{proof}


\begin{statement}{3.4.5}
Prove that if $A \subseteq B \setminus C$ and $A \neq \varnothing$ then $B \nsubseteq C$.
\end{statement}

\begin{proof}
Suppose $A \subseteq B \setminus C$ and $A \neq \varnothing$.
Since $A \neq \varnothing$, there is $x \in A$.
Then, since $A \subseteq B \setminus C$, it follows that $x \in B \setminus C$.
By the definition of $B \setminus C$, this means that $x \in B$ and $x \notin C$.
In other words, $B \nsubseteq C$.
Thus, we conclude that if $A \subseteq B \setminus C$ and $A \neq \varnothing$ then $B \nsubseteq C$.
\end{proof}


\begin{statement}{3.4.6}
Prove that for any sets $A$, $B$, and $C$, $A \setminus (B \cap C) = (A \setminus B) \cup (A \setminus C)$, by finding a string of equivalences starting with $x \in A \setminus (B \cap C)$ and ending with $x \in (A \setminus B) \cup (A \setminus C)$.
\end{statement}

\begin{proof}
Let $A$, $B$, and $C$ be arbitrary sets.
First, $x \in A \setminus (B \cap C)$ is equivalent to $(x \in A) \wedge (x \notin B \cap C)$.
Since $x \notin B \cap C$ is equivalent to $(x \notin B) \vee (x \notin C)$, $(x \in A) \wedge (x \notin B \cap C)$ is equivalent to $(x \in A) \wedge [(x \notin B) \vee (x \notin C)]$.
This is then equivalent to $[(x \in A) \wedge (x \notin B)] \vee [(x \in A) \wedge (x \notin C)]$.
Finally, this is equivalent to $(x \in A \setminus B) \vee (x \in A \setminus C)$, or in other words, $x \in (A \setminus B) \cup (A \setminus C)$.
Since $A$, $B$, and $C$ were arbitrary sets, we conclude that for any sets $A$, $B$, and $C$, $A \setminus (B \cap C) = (A \setminus B) \cup (A \setminus C)$.
\end{proof}


\begin{statement}{3.4.7}
Use the methods of this chapter to prove that for any sets $A$ and $B$, $\powerset{A \cap B} = \powerset{A} \cap \powerset{B}$.
\end{statement}

\begin{proof}
Let $A$ and $B$ be arbitrary sets.
We start by showing $\powerset{A \cap B} \subseteq \powerset{A} \cap \powerset{B}$.
Let $X \in \powerset{A \cap B}$ be arbitrary.
By the definition of $\powerset{A \cap B}$, $X \subseteq A \cap B$.
Then, since $A \cap B \subseteq A$, $X \subseteq A$.
In other words, $X \in \powerset{A}$.
Similarly, since $A \cap B \subseteq B$, $X \in \powerset{B}$.
Thus, $X \in \powerset{A} \cap \powerset{B}$, and since $X \in \powerset{A \cap B}$ was arbitrary we conclude that $\powerset{A \cap B} \subseteq \powerset{A} \cap \powerset{B}$.

Next we show $\powerset{A} \cap \powerset{B} \subseteq \powerset{A \cap B}$.
Let $X \in \powerset{A} \cap \powerset{B}$ be arbitrary.
It follows that $X \subseteq A$ and $X \subseteq B$, so $X \subseteq A \cap B$.
In other words, $X \in \powerset{A \cap B}$.
Thus, since $X \in \powerset{A} \cap \powerset{B}$ was arbitrary, we conclude that $\powerset{A} \cap \powerset{B} \subseteq \powerset{A \cap B}$.
Now that we have checked both containments, we conclude that $\powerset{A \cap B} = \powerset{A} \cap \powerset{B}$.
Since $A$ and $B$ were arbitrary sets, this is true for any sets $A$ and $B$.
\end{proof}


\begin{statement}{3.4.8}
Prove that $A \subseteq B$ if and only if $\powerset{A} \subseteq \powerset{B}$.
\end{statement}

\begin{proof}
First, suppose $A \subseteq B$.
For this part of the proof, our goal is to show $\powerset{A} \subseteq \powerset{B}$.
Let $X \in \powerset{A}$ be arbitrary.
By the definition of $\powerset{A}$, $X \subseteq A$.
Then, since $A \subseteq B$, $X \subseteq B$.
In other words, $X \in \powerset{B}$.
Thus, $\powerset{A} \subseteq \powerset{B}$.

Now suppose $\powerset{A} \subseteq \powerset{B}$.
Our goal for this part of the proof is to show $A \subseteq B$.
There are two cases to consider: $A = \varnothing$ and $A \neq \varnothing$.
If $A = \varnothing$, then it is vacuously true that $A \subseteq B$.
Thus, we can assume, without loss of generality, that $A \neq \varnothing$.
Let $x \in A$ be arbitrary.
Then the singleton set $\{ x \}$ is a subset of $A$, and $\{ x \} \in \powerset{A}$.
Since $\powerset{A} \subseteq \powerset{B}$, $\{ x \} \in \powerset{B}$ as well.
In other words, $\{ x \} \subseteq B$, so $x \in B$.
Hence, $A \subseteq B$.
This completes the proof.
\end{proof}


\begin{statement}{3.4.9}
Prove that if $x$ and $y$ are odd integers, then $xy$ is odd.
\end{statement}

\begin{proof}
Suppose $x$ and $y$ are odd integers.
Since $x$ and $y$ are odd, there are $m, n \in \BZ$ such that $x = 2m + 1$ and $y = 2n + 1$.
Multiplying $x$ and $y$ together, we get 
\begin{equation*}
	xy = (2m + 1)(2n + 1) = 4mn + 2m + 2n + 1 = 2(2mn + m + n) + 1.
\end{equation*}
In other words, $xy = 2k + 1$ where $k = 2mn + m + n$ is an integer.
Thus, we conclude that $xy$ is also odd.
\end{proof}


\begin{statement}{3.4.10}
Prove that if $x$ and $y$ are odd integers, then $x - y$ is even.
\end{statement}

\begin{proof}
Suppose $x$ and $y$ are odd integers.
Since $x$ and $y$ are odd, there are $m, n \in \BZ$ such that $x = 2m + 1$ and $y = 2n + 1$.
Subtracting $y$ from $x$, we get 
\begin{equation*}
	x - y = (2m + 1) - (2n + 1) = 2m - 2n = 2(m - n).
\end{equation*}
In other words, $x - y = 2k$ where $k = m - n$ is an integer.
Thus we conclude that $x - y$ is even.
\end{proof}


\begin{statement}{3.4.11}
Prove that for every integer $n$, $n^3$ is even if and only if $n$ is even.
\end{statement}

\begin{proof}
Let $n \in \BZ$ be arbitrary.
To prove the forward direction, we instead check the contrapositive statement: if $n$ is odd then $n^3$ is odd.
Suppose $n$ is odd, so there is $k \in \BZ$ such that $n = 2k + 1$.
We now compute $n^3$.
\begin{equation*}
	n^3 = (2k + 1)^3 = 8k^3 + 12k^2 + 6k + 1 = 2(4k^3 + 6k^2 + 3k) + 1
\end{equation*}
In other words, $n^3 = 2j + 1$ where $j = 4k^3 + 6k^2 + 3k$ is an integer.
Thus we conclude that $n^3$ is even.
Hence, if $n^3$ is even, then $n$ is even as well.

Now we check the backward direction: if $n$ is even then $n^3$ is even.
Suppose $n$ is even, so there is $k \in \BZ$ such that $n = 2k$.
Computing $n^3$ gives us $n^3 = (2k)^3 = 8k^3 = 2(4k^3)$.
In other words, $n^3 = 2j$ where $j = 4k^3$ is an integer.
Thus we conclude that $n^3$ is even.
Hence, if $n$ is even, then $n^3$ is even as well.
This completes the proof that for every integer $n$, $n^3$ is even if and only if $n$ is even.
\end{proof}


\begin{statement}{3.4.12}
Consider the following putative theorem:
\begin{theorem}
	Suppose $m$ is an even integer and $n$ is an odd integer.
	Then $n^2 - m^2 = n + m$.
\end{theorem}
\begin{enumerate}
	\item What is wrong with the following proof of the theorem?
	\begin{proof}
		Since $m$ is even, we can choose some integer $k$ such that $m = 2k$.
		Similarly, since $n$ is odd we have $n = 2k + 1$.
		Therefore
		\begin{align*}
			n^2 - m^2 &= (2k + 1)^2 - (2k)^2 \\
			&= 4k^2 + 4k + 1 - 4k^2 \\
			&= 4k + 1 \\
			&= (2k + 1) + (2k) \\
			&= n + m.
		\end{align*}
	\end{proof}
	
	\item Is the theorem correct?
	Justify your answer with either a proof or a counterexample.
\end{enumerate}
\end{statement}

\begin{proof}
\hfill
\begin{enumerate}
	\item The issue with the attempted proof is in unraveling the definitions of
	$m$ being even and $n$ being odd.
	The proof incorrectly assumes that there is a single integer $k$ 
	such that $m = 2k$ and $n = 2k + 1$, but that is not necessarily the case.
	Instead, the proof should have said that there is some integer $k$ such that $m = 2k$
	and some integer $l$ such that $n = 2l$.
	
	\item The theorem is incorrect.
	Consider the case where $m = 2$ and $n = 1$.
	In this case, $n^2 - m^2 = 1^2 - 2^2 = -3$, while $n + m = 1 + 2 = 3$.
	Thus, we have found values for $m$ and $n$ that satisfy the hypotheses of the theorem,
	but do not satisfy the conclusion.
\end{enumerate}
\end{proof}


\begin{statement}{3.4.13}
Prove that $\forall x \in \BR [ \exists y \in \BR (x + y = xy) \leftrightarrow x \neq 1]$.
\emph{Note: This exercise in my copy of the book has a typo, where instead of $x \neq 1$ it says $x = 1$.}
\end{statement}

\begin{proof}
Translating the statement from logical symbols to mathematical English, we wish to prove that for all $x \in \BR$, there exists $y \in \BR$ such that $x + y = xy$ if and only if $x \neq 1$.
First, let $x \in \BR$ be arbitrary.
Now our goal is to prove that there exists $y \in \BR$ such that $x + y = xy$ if and only if $x \neq 1$.

We start by proving the forward direction.
Suppose there exists $y \in \BR$ such that $x + y = xy$. 
Next suppose, towards a contradiction, that $x = 1$.
Plugging this into our equation gives $1 + y = y$, which is a contradiction.
Thus, we conclude that $x \neq 1$.

Next we prove the backward direction.
Suppose $x \neq 1$.
Consider $y = x / (x - 1)$, which is a real number since $x \neq 1$.
Plugging in this value for $y$ into the expression $x + y$ gives us
\begin{equation*}
	x + \frac{x}{x - 1} = \frac{x^2 - x}{x - 1} + \frac{x - 1}{x - 1}
	= \frac{x^2}{x - 1} = xy.
\end{equation*}
Thus, we have found a value of $y$ such that $x + y = xy$.
Since $x \in \BR$ was arbitrary and we have proved both directions, this completes the proof.
\end{proof}


\begin{statement}{3.4.14}
Prove that $\exists z \in \BR \forall x \in \BR^+ [\exists y \in \BR (y - x = y / x) \leftrightarrow x \neq z]$.
\emph{Note: This exercise in my copy of the book has a typo, where instead of $x \neq z$ it says $x = z$.}
\end{statement}

\begin{proof}
Translating the statement from logical symbols to mathematical English, we wish to prove that there exists $z \in \BR$, such that for all $x \in \BR^+$, there exists $y \in \BR$ such that $y - x = y/x$ if and only if $x \neq z$.
To start, let $z = 1$ and let $x \in \BR^+$ be arbitrary.
Now our goal is to prove that there exists $y \in \BR$ such that $y - x = y / x$ if and only if $x \neq 1$.

We start by proving the forward direction.
Suppose there exists $y \in \BR$ such that $y - x = y / x$.
Next suppose, towards a contradiction, that $x = 1$.
Plugging this into our equation gives us $y - 1 = y$, which is a contradiction.
Thus, we conclude that $x \neq 1$.

Next we prove the backward direction.
Suppose $x \neq 1$.
Consider $y = x^2 / (x - 1)$, which is a real number since $x \neq 1$.
Plugging this value for $y$ into the expression $y - x$ gives us
\begin{equation*}
	\frac{x^2}{x - 1} - x = \frac{x^2}{x - 1} - \frac{x^2 - x}{x - 1}
	= \frac{x}{x - 1} = \frac{y}{x}.
\end{equation*}
Thus, we have found a value of $y$ such that $y - x = y / x$.

Since $x \in \BR^+$ was arbitrary and we have proved both directions, this completes the proof.
\emph{Note that the proof actually holds for all nonzero $x \in \BR$.}
\end{proof}


\begin{statement}{3.4.15}
Suppose $B$ is a set and $\mathcal{F}$ is a family of sets.
Prove that $\bigcup \{ A \setminus B \mid A \in \mathcal{F} \} \subseteq \bigcup (\mathcal{F} \setminus \powerset{B})$.
\end{statement}

\begin{proof}
% Can also revise wording to make proof a little less clunky and better utilize the fact that it is sufficient to prove the implication (x in A) --> (x in B) to show that A is a subset of B.
Since the empty set is always a subset of any set $S$, we may assume, without loss of generality, that $\bigcup \{ A \setminus B \mid A \in \mathcal{F} \}$ is nonempty.
Let $x \in \bigcup \{ A \setminus B \mid A \in \mathcal{F} \}$ be arbitrary.
Our goal is to show that there exists a set $A \in \mathcal{F} \setminus \powerset{B}$ such that $x \in A$.
By the definition of $\bigcup \{ A \setminus B \mid A \in \mathcal{F} \}$, there is $A \in \mathcal{F}$ such that $x \in A \setminus B$.
In particular, this means that $A \nsubseteq B$, or in other words, $A \notin \powerset{B}$.
Since $A \notin \powerset{B}$, we have $A \in \mathcal{F} \setminus \powerset{B}$, as desired.
Thus, $x \in \bigcup (\mathcal{F} \setminus \powerset{B})$.
Since $x \in \bigcup \{ A \setminus B \mid A \in \mathcal{F} \}$ was arbitrary, this completes the proof that $\bigcup \{ A \setminus B \mid A \in \mathcal{F} \} \subseteq \bigcup (\mathcal{F} \setminus \powerset{B})$.
\end{proof}


\begin{statement}{3.4.16}
Suppose $\mathcal{F}$ and $\mathcal{G}$ are nonempty families of sets and every element of $\mathcal{F}$ is disjoint from some element of $\mathcal{G}$.
Prove that $\bigcup \mathcal{F}$ and $\bigcap \mathcal{G}$ are disjoint.
\end{statement}

\begin{proof}
Suppose, towards a contradiction, that $\bigcup \mathcal{F}$ and $\bigcap \mathcal{G}$ are not disjoint.
Then there is $x \in \left( \bigcup \mathcal{F} \right) \cap \left( \bigcap \mathcal{G} \right)$.
Since $x \in \bigcup \mathcal{F}$, there is $A \in \mathcal{F}$ such that $x \in A$.
Then, by the assumption that every element of $\mathcal{F}$ is disjoint from some element of $\mathcal{G}$, there is a set $B_A \in \mathcal{G}$ such that $A \cap B_A = \varnothing$.
However, since $x \in \bigcap \mathcal{G}$, $x \in B$ for all $B \in \mathcal{G}$.
In particular, this means $x \in B_A$, so $x \in A \cap B_A$.
This, however, is a contradiction to the prior assumption that $A \cap B_A = \varnothing$.
Thus, we conclude that $\bigcup \mathcal{F}$ and $\bigcap \mathcal{G}$ are disjoint.
\end{proof}


\begin{statement}{3.4.17}
Prove that for any set $A$, $A = \bigcup \powerset{A}$.
\end{statement}

\begin{proof}
% Can also revise wording to make proof a little less clunky and better utilize the fact that it is sufficient to prove the implication (x in A) --> (x in B) to show that A is a subset of B.
Let $A$ be an arbitrary set.
First note that if $A$ is empty, then $\bigcup \powerset{A} = \bigcup \{ \varnothing \} = \varnothing$, so $A = \bigcup \powerset{A}$ in this case.
Now we assume that $A$ is nonempty.
We start by proving that $A \subseteq \bigcup \powerset{A}$.
Let $x \in A$ be arbitrary.
Since $x \in A$, the singleton set $S_x = \{ x \}$ is a subset of $A$.
In other words, $S_x \in \powerset{A}$ and $x \in S_x$, so $x \in \bigcup \powerset{A}$.
Since $x \in A$ was arbitrary, we conclude that $A \subseteq \bigcup \powerset{A}$.

Next we prove that $\bigcup \powerset{A} \subseteq A$.
Let $x \in \bigcup \powerset{A}$ be arbitrary.
By the definition of $\bigcup \powerset{A}$, there is $S \in \powerset{A}$ such that $x \in S$.
Then, by the definition of $\powerset{A}$, $S \subseteq A$, so it immediately follows that $x \in A$.
Since $x \in \bigcup \powerset{A}$ was arbitrary, we conclude that $\bigcup \powerset{A} \subseteq A$.
Hence, we conclude that $A = \bigcup \powerset{A}$.
Since $A$ was an arbitrary set, this holds for any set $A$.
\end{proof}


\begin{statement}{3.4.18}
Suppose $\mathcal{F}$ and $\mathcal{G}$ are families of sets.
\begin{enumerate}
	\item Prove that $\bigcup (\mathcal{F} \cap \mathcal{G}) \subseteq
	\left( \bigcup \mathcal{F} \right) \cap \left( \bigcup \mathcal{G} \right)$.
	
	\item What is wrong with the following proof that 
	$\left( \bigcup \mathcal{F} \right) \cap \left( \bigcup \mathcal{G} \right) \subseteq
	 \bigcup (\mathcal{F} \cap \mathcal{G})$?
	 \begin{proof}
	 	Suppose $x \in \left( \bigcup \mathcal{F} \right) \cap \left( \bigcup \mathcal{G} \right)$.
	 	This means that $x \in \bigcup \mathcal{F}$ and $x \in \bigcup \mathcal{G}$,
	 	so there exists $A \in \mathcal{F}$ such that $x \in A$, 
	 	and there exists $A \in \mathcal{G}$ such that $x \in A$.
	 	Thus, we can choose a set $A$ such that 
	 	$A \in \mathcal{F}$, $A \in \mathcal{G}$, and $x \in A$.
	 	Since $A \in \mathcal{F}$ and $A \in \mathcal{G}$, $A \in \mathcal{F} \cap \mathcal{G}$.
	 	Therefore there exists $A \in \mathcal{F} \cap \mathcal{G}$ such that $x \in A$.,
	 	so $x \in \bigcup (\mathcal{F} \cap \mathcal{G})$.
	 	Since $x$ was arbitrary, we can conclude that
	 	$\left( \bigcup \mathcal{F} \right) \cap \left( \bigcup \mathcal{G} \right) \subseteq
	 	\bigcup (\mathcal{F} \cap \mathcal{G})$.
	 \end{proof}
	 % The existence statement doesn't guarantee that the A in F containing x and the A in G
	 % containing x are the same set. They could be different sets.
	 
	 \item Find an example of families of sets $\mathcal{F}$ and $\mathcal{G}$ for which
	 $\bigcup (\mathcal{F} \cap \mathcal{G}) \neq
	\left( \bigcup \mathcal{F} \right) \cap \left( \bigcup \mathcal{G} \right)$.
\end{enumerate}
\end{statement}

\begin{proof}
\hfill
\begin{enumerate}
	% Can also revise wording to make proof a little less clunky and better utilize the fact that it is
	% sufficient to prove the implication (x in A) --> (x in B) to show that A is a subset of B.
	\item Let $x \in \bigcup (\mathcal{F} \cap \mathcal{G})$ be arbitrary.
	Then, by the definition of $\bigcup (\mathcal{F} \cap \mathcal{G})$, there is
	$X \in \mathcal{F} \cap \mathcal{G}$ such that $x \in X$.
	Since $X \in \mathcal{F}$, it follows that $x \in \bigcup \mathcal{F}$.
	Similarly, since $X \in \mathcal{G}$, it follows that $x \in \bigcup \mathcal{G}$.
	In other words, $x \in \left( \bigcup \mathcal{F} \right) \cap \left( \bigcup \mathcal{G} \right)$.
	Since $x \in \bigcup (\mathcal{F} \cap \mathcal{G})$ was arbitrary, we conclude
	that $\bigcup (\mathcal{F} \cap \mathcal{G}) \subseteq
	\left( \bigcup \mathcal{F} \right) \cap \left( \bigcup \mathcal{G} \right)$.
	
	\item The error in the proof occurs when the existence statement in the definitions of 
	$\bigcup \mathcal{F}$ and $\bigcup \mathcal{G}$ is incorrectly interpreted to mean
	the set $A \in \mathcal{F}$ containing $x$ and the set $A \in \mathcal{G}$ containing $x$
	are the same set.
	They could be different sets, and a correct proof would recognize this by using the definition
	to assert the existence of a set $A \in \mathcal{F}$ containing $x$ and a set $B \in \mathcal{G}$
	containing $x$.
	Here giving the two sets different variable names emphasizes the fact that they need not
	be equal.
	
	\item Consider $\mathcal{F} = \{ \{1, 2\}, \{3\} \}$ and $\mathcal{G} = \{ \{1\}, \{2, 3\} \}$.
	In this case, $\bigcup \mathcal{F} = \bigcup \mathcal{G} = \{ 1, 2, 3 \}$,
	so $\left( \bigcup \mathcal{F} \right) \cap \left( \bigcup \mathcal{G} \right) = \{ 1, 2, 3 \}$.
	However, $\mathcal{F} \cap \mathcal{G} = \varnothing$, so
	$\bigcup (\mathcal{F} \cap \mathcal{G}) = \varnothing$.
\end{enumerate}
\end{proof}


\begin{statement}{3.4.19}
Suppose $\mathcal{F}$ and $\mathcal{G}$ are families of sets.
Prove that $\left( \bigcup \mathcal{F} \right) \cap \left( \bigcup \mathcal{G} \right) \subseteq \bigcup (\mathcal{F} \cap \mathcal{G})$ if and only if for all $A \in \mathcal{F}$ and for all $B \in \mathcal{G}$, $A \cap B \subseteq \bigcup (\mathcal{F} \cap \mathcal{G})$.
\end{statement}

\begin{proof}
% Can also revise wording to make proof a little less clunky and better utilize the fact that it is sufficient to prove the implication (x in A) --> (x in B) to show that A is a subset of B.
We start by proving the forward direction.
Suppose $\left( \bigcup \mathcal{F} \right) \cap \left( \bigcup \mathcal{G} \right) \subseteq \bigcup (\mathcal{F} \cap \mathcal{G})$.
Let $A \in \mathcal{F}$ and $B \in \mathcal{G}$ be arbitrary.
Without loss of generality, we may assume that $A \cap B \neq \varnothing$, since otherwise there is nothing to check, as the empty set is a subset of any set.
Since $A$ and $B$ are not disjoint, there is $x \in A \cap B$.
Next, since $x \in A$ and $A \in \mathcal{F}$, it follows that $x \in \bigcup \mathcal{F}$.
Similarly, $x \in \bigcup \mathcal{G}$ as well.
In other words, $x \in \left( \bigcup \mathcal{F} \right) \cap \left( \bigcup \mathcal{G} \right)$, so it immediately follows that $x \in \bigcup (\mathcal{F} \cap \mathcal{G})$ from the assumption that $\left( \bigcup \mathcal{F} \right) \cap \left( \bigcup \mathcal{G} \right) \subseteq \bigcup (\mathcal{F} \cap \mathcal{G})$.
This allows us to conclude that $A \cap B \subseteq \bigcup (\mathcal{F} \cap \mathcal{G})$, and since $A$ and $B$ were arbitrary, this holds for all $A \in \mathcal{F}$ and all $B \in \mathcal{G}$.
This completes the proof of the forward direction.

Now suppose for all $A \in \mathcal{F}$ and for all $B \in \mathcal{G}$, $A \cap B \subseteq \bigcup (\mathcal{F} \cap \mathcal{G})$.
First, without loss of generality, we may assume that $\left( \bigcup \mathcal{F} \right) \cap \left( \bigcup \mathcal{G} \right)$ is nonempty, since once again there would be nothing to prove otherwise.
Let $x \in \left( \bigcup \mathcal{F} \right) \cap \left( \bigcup \mathcal{G} \right)$ be arbitrary.
Unraveling the definitions of $\left( \bigcup \mathcal{F} \right) \cap \left( \bigcup \mathcal{G} \right)$, this means that there is $A \in \mathcal{F}$ such that $x \in A$, and there is $B \in \mathcal{G}$ such that $x \in B$.
By the assumption, $A \cap B \subseteq \bigcup (\mathcal{F} \cap \mathcal{G})$, so since $x \in A \cap B$, it follows that $x \in \bigcup (\mathcal{F} \cap \mathcal{G})$.
Since $x$ was arbitrary, this completes the proof that $\left( \bigcup \mathcal{F} \right) \cap \left( \bigcup \mathcal{G} \right) \subseteq \bigcup (\mathcal{F} \cap \mathcal{G})$.
\end{proof}


\begin{statement}{3.4.20}
Suppose $\mathcal{F}$ and $\mathcal{G}$ are families of sets.
Prove that $\bigcup \mathcal{F}$ and $\bigcup \mathcal{G}$ are disjoint if and only if for all $A \in \mathcal{F}$ and $B \in \mathcal{G}$, $A$ and $B$ are disjoint.
\end{statement}

\begin{proof}
We instead prove the equivalent statement that $\bigcup \mathcal{F}$ and $\bigcup \mathcal{G}$ are not disjoint if and only if there exist $A \in \mathcal{F}$ and $B \in \mathcal{G}$ such that $A$ and $B$ are not disjoint.
Note that this equivalent statement comes from breaking the original biconditional statement into the individual directions and then taking the contrapositive of each direction.
First, suppose $\bigcup \mathcal{F}$ and $\bigcup \mathcal{G}$ are not disjoint.
This means that there is $x$ in the intersection of these two sets.
In other words, there exists $A \in \mathcal{F}$ such that $x \in A$ and also there exists $B \in \mathcal{G}$ such that $x \in B$.
Since $x \in A \cap B$, we have found two sets $A \in \mathcal{F}$ and $B \in \mathcal{G}$ such that $A$ and $B$ are not disjoint.

Next, suppose there exist $A \in \mathcal{F}$ and $B \in \mathcal{G}$ such that $A$ and $B$ are not disjoint.
Since $A$ and $B$ are not disjoint, there exists $x \in A \cap B$.
Then, since $x \in A$ and $A \in \mathcal{F}$, we know that $x \in \bigcup \mathcal{F}$.
Similarly, $x \in \bigcup \mathcal{G}$ as well, so we conclude that $\bigcup \mathcal{F}$ and $\bigcup \mathcal{G}$ are not disjoint.
This completes the proof.
\end{proof}


\begin{statement}{3.4.21}
Suppose $\mathcal{F}$ and $\mathcal{G}$ are families of sets.
\begin{enumerate}
	\item Prove that $\left( \bigcup \mathcal{F} \right) \setminus \left( \bigcup \mathcal{G} \right)
	\subseteq \bigcup (\mathcal{F} \setminus \mathcal{G})$.
	
	\item What's wrong with the following proof that $\bigcup (\mathcal{F} \setminus \mathcal{G})
	\subseteq \left( \bigcup \mathcal{F} \right) \setminus \left( \bigcup \mathcal{G} \right)$?
	\begin{proof}
		Suppose $x \in \bigcup (\mathcal{F} \setminus \mathcal{G})$.
		Then we can choose some $A \in \mathcal{F} \setminus \mathcal{G}$ such that $x \in A$.
		Since $A \in \mathcal{F} \setminus \mathcal{G}$, 
		$A \in \mathcal{F}$ and $A \notin \mathcal{G}$.
		Since $A \in \mathcal{F}$, $x \in \bigcup \mathcal{F}$.
		Since $x \in A$ and $A \notin \mathcal{G}$, $x \notin \bigcup \mathcal{G}$.
		Therefore $x \in \left( \bigcup \mathcal{F} \right) \setminus
		\left( \bigcup \mathcal{G} \right)$.
	\end{proof}
	% There might be another set B in G such that x is in B.
	
	\item Prove that $\bigcup (\mathcal{F} \setminus \mathcal{G})
	\subseteq \left( \bigcup \mathcal{F} \right) \setminus \left( \bigcup \mathcal{G} \right)$
	if and only if for all $A \in \mathcal{F} \setminus \mathcal{G}$ and for all $B \in \mathcal{G}$,
	$A \cap B = \varnothing$.
	
	\item Find an example of families of sets $\mathcal{F}$ and $\mathcal{G}$ for which
	$\bigcup (\mathcal{F} \setminus \mathcal{G})
	\neq \left( \bigcup \mathcal{F} \right) \setminus \left( \bigcup \mathcal{G} \right)$
\end{enumerate}
\end{statement}

\begin{proof}
\hfill
\begin{enumerate}
	% Can also revise wording to make proof a little less clunky and better utilize the fact that it is
	% sufficient to prove the implication (x in A) --> (x in B) to show that A is a subset of B.
	\item Let $x \in \left( \bigcup \mathcal{F} \right) \setminus \left( \bigcup \mathcal{G} \right)$
	be arbitrary.
	In other words, there is $A \in \mathcal{F}$ such that $x \in A$,
	and there does not exist $B \in \mathcal{G}$ such that $x \in B$, by the definitions of
	$x  \in \bigcup \mathcal{F}$ and $x \notin \bigcup \mathcal{G}$, respectively.
	To show that $x \in \bigcup (\mathcal{F} \setminus \mathcal{G})$, we check that
	$A \in \mathcal{F} \setminus \mathcal{G}$.
	In particular, we show that $A \notin \mathcal{G}$.
	If $A \in \mathcal{G}$, then we would have a set in $\mathcal{G}$ containing $x$,
	which would contradict the prior assumption that no such set exists in $\mathcal{G}$.
	Thus, we conclude that $A \notin \mathcal{G}$, so it then follows that as
	$A \in \mathcal{F} \setminus \mathcal{G}$, $x \in \bigcup (\mathcal{F} \setminus \mathcal{G})$.
	Since $x$ was arbitrary, we conclude that
	$\left( \bigcup \mathcal{F} \right) \setminus \left( \bigcup \mathcal{G} \right)
	\subseteq \bigcup (\mathcal{F} \setminus \mathcal{G})$.
	
	\item The issue with the proof lies in the fact that there might be another set $B \in \mathcal{G}$
	such that $x \in B$.
	In order for $x$ to be an element of 
	$\left( \bigcup \mathcal{F} \right) \setminus \left( \bigcup \mathcal{G} \right)$,
	there must not be \emph{any} set $B \in \mathcal{G}$ containing $x$.
	
	\item First, suppose that for all $A \in \mathcal{F} \setminus \mathcal{G}$
	and for all $B \in \mathcal{G}$, $A \cap B = \varnothing$.
	Let $x \in \bigcup \left( \mathcal{F} \setminus \mathcal{G} \right)$ be arbitrary.
	Then there is some $A \in \mathcal{F} \setminus \mathcal{G}$ such that $x \in A$.
	Since for all $A \in \mathcal{F} \setminus \mathcal{G}$ and for all $B \in \mathcal{G}$,
	$A \cap B = \varnothing$, there is no $B \in \mathcal{G}$ such that $x \in B$.
	Otherwise, if such a $B$ existed, then $x \in A \cap B$ would be a contradiction.
	Since there is no $B \in \mathcal{G}$ such that $x \in B$, it follows that
	$x \notin \bigcup \mathcal{G}$.
	As $x \in A$ and $A \in \mathcal{F}$ means $x \in \bigcup \mathcal{F}$,
	we conclude that 
	$x \in \left( \bigcup \mathcal{F} \right) \setminus \left( \bigcup \mathcal{G} \right)$.
	Hence, $\bigcup (\mathcal{F} \setminus \mathcal{G})
	\subseteq \left( \bigcup \mathcal{F} \right) \setminus \left( \bigcup \mathcal{G} \right)$.
	
	To prove the other direction, we work with the contrapositive statement:
	if there exist $A \in \mathcal{F} \setminus \mathcal{G}$ and $B \in \mathcal{G}$ such that
	$A \cap B \neq \varnothing$, then
	$\bigcup (\mathcal{F} \setminus \mathcal{G})
	\nsubseteq \left( \bigcup \mathcal{F} \right) \setminus \left( \bigcup \mathcal{G} \right)$.
	Suppose there is $A \in \mathcal{F} \setminus \mathcal{G}$ and $B \in \mathcal{G}$ such that
	$A \cap B \neq \varnothing$.
	Then, since $A \cap B \neq \varnothing$, there exists $x \in A \cap B$.
	Since $x \in A$ and $A \in \mathcal{F} \setminus \mathcal{G}$,
	$x \in \bigcup (\mathcal{F} \setminus \mathcal{G})$.
	In addition, since $A$ is in particular an element of $\mathcal{F}$, $x \in \bigcup \mathcal{F}$.
	On the other hand, since $x \in B$ and $B \in \mathcal{G}$, $x \in \bigcup \mathcal{G}$ as well.
	This means that $x \notin \left( \bigcup \mathcal{F} \right) 
	\setminus \left( \bigcup \mathcal{G} \right)$.
	In other words, this proves that $\bigcup (\mathcal{F} \setminus \mathcal{G})
	\nsubseteq \left( \bigcup \mathcal{F} \right) \setminus \left( \bigcup \mathcal{G} \right)$.
	This completes the proof.
	
	\item Consider $\mathcal{F} \{ \{1\}, \{1, 2\} \}$ and $\mathcal{G} = \{ \{1\}, \{2\} \}$.
	In this case, $\mathcal{F} \setminus \mathcal{G} = \{ \{1, 2\} \}$,
	so $\bigcup (\mathcal{F} \setminus \mathcal{G}) = \{ 1, 2 \}$.
	On the other hand, $\bigcup \mathcal{F} = \bigcup \mathcal{G} = \{ 1, 2 \}$,
	so $\left( \bigcup \mathcal{F} \right) \setminus \left( \bigcup \mathcal{G} \right) = \varnothing$.
\end{enumerate}
\end{proof}


\begin{statement}{3.4.22}
Suppose $\mathcal{F}$ and $\mathcal{G}$ are families of sets.
Prove that if $\bigcup \mathcal{F} \nsubseteq \bigcup \mathcal{G}$, then there is some $A \in \mathcal{F}$ such that for all $B \in \mathcal{G}$, $A \nsubseteq B$.
\end{statement}

\begin{proof}
Suppose $\bigcup \mathcal{F} \nsubseteq \bigcup \mathcal{G}$.
This means that there is $x \in \bigcup \mathcal{F}$ such that $x \notin \bigcup \mathcal{G}$.
Since $x \in \bigcup \mathcal{F}$, there is $A \in \mathcal{F}$ such that $x \in A$.
Similarly, since $x \notin \bigcup \mathcal{G}$, we know that for all $B \in \mathcal{G}$, $x \notin B$.
In particular, this means that $A \nsubseteq B$ for all $B \in \mathcal{G}$.
Thus, we have found a set $A \in \mathcal{F}$ such that for all $B \in \mathcal{G}$, $A \nsubseteq B$.
This completes the proof.
\end{proof}


\begin{statement}{3.4.23}
Suppose $B$ is a set, $\{ A_i \mid i \in I \}$ is an indexed family of sets, and $I \neq \varnothing$.
\begin{enumerate}
	\item What proof strategies are used in the following proof of the equation
	$B \cap \left( \bigcup_{i \in I} A_i \right) = \bigcup_{i \in I} (B \cap A_i)$?
	\begin{proof}
		Let $x$ be arbitrary.
		Suppose $x \in B \cap \left( \bigcup_{i \in I} A_i \right)$.
		Then $x \in B$ and $x \in \bigcup_{i \in I} A_i$,
		so we can choose some $i_0 \in I$ such that $x \in A_{i_0}$.
		Since $x \in B$ and $x \in A_{i_0}$, $x \in B \cap A_{i_0}$.
		Therefore, $x \in \bigcup_{i \in I} (B \cap A_i)$.
		
		Now suppose $x \in \bigcup_{i \in I} (B \cap A_i)$.
		Then we can choose some $i_0 \in I$ such that $x \in B \cap A_{i_0}$.
		Therefore $x \in B$ and $x \in A_{i_0}$.
		Since $x \in A_{i_0}$, $x \in \bigcup_{i \in I} A_i$.
		Since $x \in B$ and $x \in \bigcup_{i \in I} A_i$,
		$x \in B \cap \left( \bigcup_{i \in I} A_i \right)$.
		
		Since $x$ was arbitrary, we have shown that for all $x$,
		$x \in B \cap \left( \bigcup_{i \in I} A_i \right)$ if and only if
		$x \in \bigcup_{i \in I} (B \cap A_i)$, so
		$B \cap \left( \bigcup_{i \in I} A_i \right) = \bigcup_{i \in I} (B \cap A_i)$.
	\end{proof}
	
	\item Prove that $B \setminus \left( \bigcup_{i \in I} A_i \right) = 
	\bigcap_{i \in I} (B \setminus A_i)$.
	
	\emph{
	Note: This exercise in my copy of the book has a typo, where it asks for us to prove that
	$B \setminus \left( \bigcup_{i \in I} A_i \right) = \bigcup_{i \in I} (B \setminus A_i)$.
	This statement is incorrect.
	To see why, consider the example of 
	$I = \{ 1, 2 \}$, $A_i = \{ i, i + 1 \}$, and $B = \{ 1, 2, 3 \}$.
	In this case, since $\bigcup_{i \in I} A_i = \{ 1, 2, 3 \}$,
	$B \setminus \left( \bigcup_{i \in I} A_i \right) = \varnothing$,
	while $B \setminus A_1 = \{ 3 \}$ and $B \setminus A_2 = \{ 1 \}$ means that
	$\bigcup_{i \in I} (B \setminus A_i) = \{ 1, 3 \}$.
	}
	
	\item Can you discover and prove a similar theorem about
	$B \setminus \left( \bigcap_{i \in I} A_i \right)$?
	(\emph{Hint: Try to guess the theorem, and then try to prove it.
	If you can't finish the proof, it might be because your guess was wrong.
	Change your guess and try again.})
	
	\emph{Note: I think this exercise in my copy of the book has a typo, since it asks for
	us to discover and prove a theorem about $B \setminus \left( \bigcup_{i \in I} A_i \right)$,
	which we already did in Part (2).}
\end{enumerate}
\end{statement}

\begin{proof}
\hfill
\begin{enumerate}
	\item The proof is broken up into two parts: one part to check that
	$x \in B \cap \left( \bigcup_{i \in I} A_i \right) \rightarrow x \in  \bigcup_{i \in I} (B \cap A_i)$
	and one part to check that 
	$x \in \bigcup_{i \in I} (B \cap A_i) B \rightarrow x \in \cap \left( \bigcup_{i \in I} A_i \right)$.
	Both parts start by taking an arbitrary $x$ and then unraveling the definitions as to what
	it means to be an element of the starting set for each part.
	For the first part of the proof, this involves noting that $x \in B$ and using existential instantiation
	to choose some $i_0 \in I$ such that $x \in A_{i_0}$.
	For the second part of the proof, existential instantiation is used to choose some $i_0 \in I$
	such that $x \in B \cap A_{i_0}$.
	In both parts of the proof, we then check that this $i_0$ satisfies the desired property
	to conclude that $x \in \bigcup_{i \in I} (B \cap A_i)$ in the first part and that
	$x \in B \cap \left( \bigcup_{i \in I} A_i \right)$ in the second part.
	
	\item Suppose $x \in B \setminus \left( \bigcup_{i \in I} A_i \right)$.
	Then $x \in B$ and $x \notin \bigcup_{i \in I} A_i$.
	Since $x \notin \bigcup_{i \in I} A_i$, $x \notin A_i$ for all $i \in I$.
	In other words, $x \in B \setminus A_i$ for all $i \in I$.
	This means that $x \in \bigcap_{i \in I} (B \setminus A_i)$, so 
	$B \setminus \left( \bigcup_{i \in I} A_i \right) \subseteq \bigcap_{i \in I} (B \setminus A_i)$.
	
	Now suppose $x \in \bigcap_{i \in I} (B \setminus A_i)$.
	This means that $x \in B \setminus A_i$ for all $i \in I$, or in other words, $x \in B$
	and $x \notin A_i$ for all $i \in I$.
	Since $x \notin A_i$ for all $i \in I$, $x \notin \bigcup_{i \in I} A_i$.
	This means that $x \in B \setminus \left( \bigcup_{i \in I} A_i \right)$.
	In other words, 
	$\bigcap_{i \in I} (B \setminus A_i) \subseteq B \setminus \left( \bigcup_{i \in I} A_i \right)$.
	This completes the proof that $B \setminus \left( \bigcup_{i \in I} A_i \right) = 
	\bigcap_{i \in I} (B \setminus A_i)$.
	
	\item We will prove that the following theorem.
	\begin{theorem}
		$B \setminus \left( \bigcap_{i \in I} A_i \right) = 
		\bigcup_{i \in I} (B \setminus A_i)$.
	\end{theorem}
	
	Suppose $x \in B \setminus \left( \bigcap_{i \in I} A_i \right)$.
	Then $x \in B$ and $x \notin \bigcap_{i \in I} A_i$.
	Since $x \notin \bigcap_{i \in I} A_i$, there exists $i \in I$ such that $x \notin A_i$.
	In other words, $x \in B \setminus A_i$ for that value of $i$.
	This means that $x \in \bigcup_{i \in I} (B \setminus A_i)$, so 
	$B \setminus \left( \bigcap_{i \in I} A_i \right) \subseteq \bigcup_{i \in I} (B \setminus A_i)$.
	
	Now suppose $x \in \bigcup_{i \in I} (B \setminus A_i)$.
	This means that there exists $i \in I$ such that $x \in B \setminus A_i$, or in other words
	$x \in B$ and $x \notin A_i$ for that value of $i$.
	Since there exists $i \in I$ such that $x \notin A_i$, $x \notin \bigcap_{i \in I} A_i$.
	This means that $x \in B \setminus \left( \bigcap_{i \in I} A_i \right)$.
	In other words, 
	$\bigcup_{i \in I} (B \setminus A_i) \subseteq B \setminus \left( \bigcap_{i \in I} A_i \right)$.
	This completes the proof that $B \setminus \left( \bigcap_{i \in I} A_i \right) = 
	\bigcup_{i \in I} (B \setminus A_i)$.
\end{enumerate}
\end{proof}


\begin{statement}{3.4.24}
Suppose $\{ A_i \mid i \in I \}$ and $\{ B_i \mid i \in I \}$ are indexed families of sets and $I \neq \varnothing$.
\begin{enumerate}
	\item Prove that $\bigcup_{i \in I} (A_i \setminus B_i) \subseteq
	\left( \bigcup_{i \in I} A_i \right) \setminus \left( \bigcap_{i \in I} B_i \right)$.
	
	\emph{
	Note: This exercise in my copy of the book has a typo, where it asks for us to prove that
	$\bigcup_{i \in I} (A_i \setminus B_i) \subseteq
	\left( \bigcup_{i \in I} A_i \right) \setminus \left( \bigcup_{i \in I} B_i \right)$.
	This statement is incorrect.
	To see why, consider the example of 
	$I = \{ 1, 2 \}$, $A_i = \{ 1, 2, 3 \}$, and $B_i = \{ i, i + 1 \}$.
	In this case, since $\bigcup_{i \in I} B_i = \{ 1, 2, 3 \}$,
	$\left( \bigcup_{i \in I} A_i \right) \setminus \left( \bigcup_{i \in I} B_i \right) = \varnothing$,
	while $A_1 \setminus B_1 = \{ 3 \}$ and $A_2 \setminus B_2 = \{ 1 \}$ means that
	$\bigcup_{i \in I} (A_i \setminus B_i) = \{ 1, 3 \}$.
	}
	
	\item Find an example for which $\bigcup_{i \in I} (A_i \setminus B_i) \neq
	\left( \bigcup_{i \in I} A_i \right) \setminus \left( \bigcap_{i \in I} B_i \right)$.
	
	\emph{
	Note: Since Part (1) had the typo of considering 
	$\left( \bigcup_{i \in I} A_i \right) \setminus \left( \bigcup_{i \in I} B_i \right)$ instead of
	$\left( \bigcup_{i \in I} A_i \right) \setminus \left( \bigcap_{i \in I} B_i \right)$,
	I assume that Part (2) also had the same typo and have adjusted accordingly.
	}
\end{enumerate}
\end{statement}

\begin{proof}
\hfill
\begin{enumerate}
	\item Suppose $x \in \bigcup_{i \in I} (A_i \setminus B_i)$.
	This means that there exists $i \in I$ such that $x \in A_i \setminus B_i$, or in other words
	there exists $i \in I$ such that $x \in A_i$ and $x \notin B_i$.
	Since $x \in A_i$, $x \in \bigcup_{i \in I} A_i$.
	Since $x \notin B_i$, $x \notin \bigcap_{i \in I} B_i$.
	Hence, $x \in \left( \bigcup_{i \in I} A_i \right) \setminus \left( \bigcap_{i \in I} B_i \right)$,
	so we conclude that 
	$\bigcup_{i \in I} (A_i \setminus B_i) \subseteq
	\left( \bigcup_{i \in I} A_i \right) \setminus \left( \bigcap_{i \in I} B_i \right)$.
	
	\item Consider $I = \{ 1, 2 \}$, $A_i = \{ i, i + 1 \}$, and $B_i = \{ i \}$.
	In this case, since $A_i \setminus B_i = \{ i + 1 \}$, we have
	$\bigcup_{i \in I} (A_i \setminus B_i) = \{ 2, 3 \}$.
	On the other hand, since $\bigcup_{i \in I} A_i = \{ 1, 2, 3 \}$ and 
	$\bigcap_{i \in I} B_i = \varnothing$, 
	$\left( \bigcup_{i \in I} A_i \right) \setminus \left( \bigcap_{i \in I} B_i \right) = \{ 1, 2, 3 \}$.
\end{enumerate}
\end{proof}


\begin{statement}{3.4.25}
Suppose $\{ A_i \mid i \in I \}$ and $\{ B_i \mid i \in I \}$ are indexed families of sets.
\begin{enumerate}
	\item Prove that $\bigcup_{i \in I} (A_i \cap B_i) \subseteq
	\left( \bigcup_{i \in I} A_i \right) \cap \left( \bigcup_{i \in I} B_i \right)$.
	
	\item Find an example for which $\bigcup_{i \in I} (A_i \cap B_i) \neq
	\left( \bigcup_{i \in I} A_i \right) \cap \left( \bigcup_{i \in I} B_i \right)$.
\end{enumerate}
\end{statement}

\begin{proof}
\hfill
\begin{enumerate}
	\item Suppose $x \in \bigcup_{i \in I} (A_i \cap B_i)$.
	This means there is $i \in I$ such that $x \in A_i \cap B_i$.
	In other words, there is $i \in I$ such that $x \in A_i$ and $x \in B_i$.
	Since $x \in A_i$, it follows that $x \in \bigcup_{i \in I} A_i$.
	Similarly, $x \in B_i$ means that $x \in \bigcup_{i \in I} B_i$.
	Thus, $x \in \left( \bigcup_{i \in I} A_i \right) \cap \left( \bigcup_{i \in I} B_i \right)$.
	We conclude that $\bigcup_{i \in I} (A_i \cap B_i) \subseteq
	\left( \bigcup_{i \in I} A_i \right) \cap \left( \bigcup_{i \in I} B_i \right)$.
	
	\item Consider $A_1 = B_2 = \{ 1 \}$ and $A_2 = B_1 = \{ 2 \}$.
	In this case, $A_1 \cap B_1 = A_2 \cap B_2 = \varnothing$,
	so $\bigcup_{i \in I} (A_i \cap B_i) = \varnothing$.
	On the other hand, $\bigcup_{i \in I} A_i = \bigcup_{i \in I} B_i = \{ 1, 2 \}$, so
	$\left( \bigcup_{i \in I} A_i \right) \cap \left( \bigcup_{i \in I} B_i \right) = \{ 1, 2 \}$.
\end{enumerate}
\end{proof}


\begin{statement}{3.4.26}
Prove that for all integers $a$ and $b$ there is an integer $c$ such that $a$ divides $c$ and $b$ divides $c$.
\end{statement}

\begin{proof}
Let $a$ and $b$ be arbitrary integers.
Consider the integer $c = ab$.
Since $b$ is an integer and $c = ab$, we know that $a$ divides $c$.
Similarly, since $a$ is an integer and $c = ab$, we know that $b$ divides $c$ as well. 
Thus, we have found an integer $c$ such that $a$ divides $c$ and $b$ divides $c$.
Since $a$ and $b$ were arbitrary, this holds for all integers $a$ and $b$.
\end{proof}


\begin{statement}{3.4.27}
\begin{enumerate}
	\item Prove that for every integer $n$, 15 divides $n$ if and only if
	3 divides $n$ and 5 divides $n$.
	
	\item Prove that it is \emph{not} true that for every integer $n$,
	60 divides $n$ if and only if 6 divides $n$ and 10 divides $n$.
	% Consider n = 90
\end{enumerate}
\end{statement}

\begin{proof}
\hfill
\begin{enumerate}
	\item Let $n$ be an arbitrary integer.
	First, suppose 15 divides $n$.
	In other words, there is an integer $k$ such that $n = 15k$.
	Since $15 = 3 \times 5$, we have $n = 3(5k) = 5(3k)$.
	As $l = 5k$ and $m = 3k$ are both integers, we can write $n = 3l = 5m$,
	which means 3 and 5 both divide $n$.
	
	Now suppose 3 and 5 each divide $n$.
	In other words, there is an integer $k$ such that $n = 3k$
	and there is an integer $l$ such that $n = 5l$.
	To show that 15 divides $n$, we need to find an integer $m$ such that $n = 15m$.
	Consider $m = 2l - k$.
	Plugging this value of $m$ into the expression $15m$, we have
	\begin{equation*}
		15m = 15(2l - k) = 30l - 15k = 6(5l) - 5(3k) = 6n - 5n = n.
	\end{equation*}
	Thus, since $n = 15m$, we conclude that 15 divides $n$.
	Since $n$ was arbitrary, this completes the proof that for every integer $n$,
	15 divides $n$ if and only if 3 divides $n$ and 5 divides $n$.
	
	\item Consider the case where $n = 90$.
	Since $90 = 15 \times 6 = 9 \times 10$, both 6 and 10 divide $n$, but 60 does not divide $n$.
	The key point that allowed us to conclude that $n$ being divisible by 3 and 5 implies that $n$
	is also divisible by 15 in Part (1) is the fact that 3 and 5 are relatively prime.
\end{enumerate}
\end{proof}


\begin{statement}{3.5.1}
Suppose $A$, $B$, and $C$ are sets.
Prove that $A \cap (B \cup C) \subseteq (A \cap B) \cup C$.
\end{statement}

\begin{proof}
Suppose $x \in A \cap (B \cup C)$.
Then $x \in A$ and $x \in B \cup C$.
Since $x \in B \cup C$, there are two cases two consider: $x \in B$ and $x \in C$.
If $x \in C$, then we are done, since this automatically means $x \in (A \cap B) \cup C$.
If $x \in B$, then $x \in A \cap B$, and once again this means $x \in (A \cap B) \cup C$.
Thus, since in either case $x \in (A \cap B) \cup C$, we conclude that $A \cap (B \cup C) \subseteq (A \cap B) \cup C$.
\end{proof}


\begin{statement}{3.5.2}
Suppose $A$, $B$, and $C$ are sets.
Prove that $(A \cup B) \setminus C \subseteq A \cup (B \setminus C)$.
\end{statement}

\begin{proof}
Suppose $x \in (A \cup B) \setminus C$.
Then $x \in A \cup B$ and $x \notin C$.
Since $x \in A \cup B$, there are two cases to consider: $x \in A$ and $x \in B$.
If $x \in A$, then we are done, since this automatically means $x \in A \cup (B \setminus C)$.
If $x \in B$, then $x \in B \setminus C$, and once again this means $x \in A \cup (B \setminus C)$.
Thus, since in either case $x \in A \cup (B \setminus C)$, we conclude that $(A \cup B) \setminus C \subseteq A \cup (B \setminus C)$.
\end{proof}


\begin{statement}{3.5.3}
Suppose $A$ and $B$ are sets.
Prove that $A \setminus (A \setminus B) = A \cap B$.
\end{statement}

\begin{proof}
We first check that $A \cap B \subseteq A \setminus (A \setminus B)$.
Suppose $x \in A \cap B$.
This means $x \in A$ and $x \in B$.
We also note that this means $x \notin A \setminus B$.
Hence, as $x \in A$ and $x \notin A \setminus B$, we conclude that $x \in A \setminus (A \setminus B)$.
Thus, $A \cap B \subseteq A \setminus (A \setminus B)$.

Now we check that $A \setminus (A \setminus B) \subseteq A \cap B$.
Suppose $x \in A \setminus (A \setminus B)$.
This means $x \in A$ and $x \notin A \setminus B$.
Since $x \in A \setminus B$ means that $x \in A$ and $x \notin B$, $x \notin A \setminus B$ means that $x \notin A$ or $x \in B$.
We already know that $x \in A$, so we conclude that $x \in B$.
In other words, since $x \in A$ and $x \in B$, $x \in A \cap B$.
Thus, we conclude that $A \setminus (A \setminus B) \subseteq A \cap B$.
This completes the proof that $A \setminus (A \setminus B) = A \cap B$.
\end{proof}


\begin{statement}{3.5.4}
Suppose $A$, $B$, and $C$ are sets.
Prove that $A \setminus (B \setminus C) = (A \setminus B) \cup (A \cap C)$.
\end{statement}

\begin{proof}
We first check that $A \setminus (B \setminus C) \subseteq (A \setminus B) \cup (A \cap C)$.
Suppose $x \in A \setminus (B \setminus C)$.
This means $x \in A$ and $x \notin B \setminus C$.
Since $x \notin B \setminus C$, we have two cases to consider: $x \notin B$ and $x \in C$.
If $x \notin B$, then that $x \in A \setminus B$.
On the other hand, if $x \in C$, then $x \in A \cap C$.
In either case $x \in (A \setminus B) \cup (A \cap C)$, so we conclude that $A \setminus (B \setminus C) \subseteq (A \setminus B) \cup (A \cap C)$.

Now we check that $(A \setminus B) \cup (A \cap C) \subseteq A \setminus (B \setminus C)$.
Suppose $x \in (A \setminus B) \cup (A \cap C)$.
We have two cases to consider: $x \in A \setminus B$ and $x \in A \cap C$.
If $x \in A \setminus B$, then $x \in A$ and $x \notin B$.
Since $x \notin B$, then $x \notin B \setminus C$ as well, so $x \in A \setminus (B \setminus C)$.
In the other case, if $x \in A \cap C$, then $x \in A$ and $x \in C$.
Since $x \in C$, then $x \notin B \setminus C$ as well, so once again $x \in A \setminus (B \setminus C)$.
Since in either case $x \in A \setminus (B \setminus C)$, we conclude that $(A \setminus B) \cup (A \cap C) \subseteq A \setminus (B \setminus C)$.
This completes the proof that $A \setminus (B \setminus C) = (A \setminus B) \cup (A \cap C)$.
\end{proof}


\begin{statement}{3.5.5}
Suppose $A \cap C \subseteq B \cap C$ and $A \cup C \subseteq B \cup C$.
Prove that $A \subseteq B$.
\end{statement}

\begin{proof}
Suppose $x \in A$.
Since $x \in A$, $x \in A \cup C$ as well.
Then, since $x \in A \cup C$ and $A \cup C \subseteq B \cup C$, $x \in B \cup C$.
This gives us two cases to consider: $x \in B$ and $x \in C$.
If $x \in B$, then we have nothing left to check, so now suppose $x \in C$.
Since $x \in A$ and $x \in C$, $x \in A \cap C$, and since $A \cap C \subseteq B \cap C$, $x \in B \cap C$.
In particular, $x \in B$.
Since in either case $x \in B$, we conclude that $A \subseteq B$.
\end{proof}


\begin{statement}{3.5.6}
Recall from Section 1.4 that the symmetric difference of two sets $A$ and $B$ is the set $A \bigtriangleup B = (A \setminus B) \cup (B \setminus A) = (A \cup B) \setminus (A \cap B)$.
Prove that if $A \bigtriangleup B \subseteq A$ then $B \subseteq A$.
\end{statement}

\begin{proof}
Suppose $A \bigtriangleup B \subseteq A$.
Now suppose $x \in B$ is arbitrary.
There are two cases to consider: $x \in A$ and $x \notin A$.
If $x \in A$, then there is nothing left to check and we are done, so we focus on the case $x \notin A$.
If $x \notin A$, then since $x \in B$, $x \in B \setminus A$.
This means that $x \in A \bigtriangleup B = (A \setminus B) \cup (B \setminus A)$.
Since $A \bigtriangleup B \subseteq A$ by assumption, this means $x \in A$.
But this contradicts the prior assumption that $x \notin A$.
Thus, we conclude that we can only have $x \in A$.
Since $x$ was arbitrary, we conclude that $B \subseteq A$.
This completes the proof that if $A \bigtriangleup B \subseteq A$ then $B \subseteq A$.
\end{proof}


\begin{statement}{3.5.7}
Suppose $A$, $B$, and $C$ are arbitrary sets.
Prove that $A \cup C \subseteq B \cup C$ if and only if $A \setminus C \subseteq B \setminus C$.
\end{statement}

\begin{proof}
First, suppose $A \cup C \subseteq B \cup C$.
Suppose $x \in A \setminus C$ is arbitrary.
Since $x \in A \setminus C$, we know $x \in A$ and $x \notin C$.
To show that $x \in B \setminus C$, all we need to check is that $x \in B$.
Since $x \in A$, $x \in A \cup C \subseteq B \cup C$.
Then, since $x \in B \cup C$ and $x \notin C$, it follows that $x \in B$.
Thus, we conclude that $x \in B \setminus C$.
Since $x$ was arbitrary, this completes the proof that $A \setminus C \subseteq B \setminus C$.

Now suppose $A \setminus C \subseteq B \setminus C$.
Suppose $x \in A \cup C$ is arbitrary.
If $x \in C$, then automatically $x \in B \cup C$ and we are done.
If $x \notin C$, then since $x \in A \cup C$ it follows that $x \in A$.
Since $x \in A$ and $x \notin C$, $x \in A \setminus C \subseteq B \setminus C$, so $x \in B \setminus C$.
In particular, $x \in B$, so $x \in B \cup C$.
Thus, since $x$ was arbitrary, this completes the proof that $A \cup C \subseteq B \cup C$.
\end{proof}


\begin{statement}{3.5.8}
Prove that for any sets $A$ and $B$, $\powerset{A} \cup \powerset{B} \subseteq \powerset{A \cup B}$.
\end{statement}

\begin{proof}
Let $A$ and $B$ be arbitrary sets.
Suppose $X \in \powerset{A} \cup \powerset{B}$ is arbitrary.
If $X \in \powerset{A}$, this means $X \subseteq A$, and since $A \subseteq A \cup B$, it follows that $X \subseteq A \cup B$.
In other words, $X \in \powerset{A \cup B}$.
Similarly, if $X \in \powerset{B}$, $X \subseteq B$ and it follows that $X \in \powerset{A \cup B}$.
Since $X \in \powerset{A} \cup \powerset{B}$ was arbitrary and in either case $X \in \powerset{A \cup B}$, we conclude that $\powerset{A} \cup \powerset{B} \subseteq \powerset{A \cup B}$.
Finally, since $A$ and $B$ were arbitrary sets, this holds for any sets $A$ and $B$.
This completes the proof.
\end{proof}


\begin{statement}{3.5.9}
Prove that for any sets $A$ and $B$, if $\powerset{A} \cup \powerset{B} = \powerset{A \cup B}$ then either $A \subseteq B$ or $B \subseteq A$.
\end{statement}

\begin{proof}
Let $A$ and $B$ be arbitrary sets.
We will prove the equivalent contrapositive statement: if $A \nsubseteq B$ and $B \nsubseteq A$ then $\powerset{A} \cup \powerset{B} \neq \powerset{A \cup B}$.
Since we showed that  for any sets $A$ and $B$, $\powerset{A} \cup \powerset{B} \subseteq \powerset{A \cup B}$ in Exercise 3.5.8 above, we only need to show that $\powerset{A \cup B} \nsubseteq \powerset{A} \cup \powerset{B}$.
In other words, we need to find $X \in \powerset{A \cup B}$ such that $X \notin \powerset{A} \cup \powerset{B}$.

Suppose $A \nsubseteq B$ and $B \nsubseteq A$.
Since $A \nsubseteq B$, there is $a \in A \setminus B$.
Similarly, since $B \nsubseteq A$, there is $b \in B \setminus A$.
Consider $X = \{ a, b \}$, which is an element of $\powerset{A \cup B}$.
Since $b \notin A$, $X \nsubseteq A$, and since $a \notin B$, $X \nsubseteq B$ as well.
This means $X \notin \powerset{A}$ and $X \notin \powerset{B}$.
In other words, $X \notin \powerset{A} \cup \powerset{B}$.
Thus, we have found $X \in \powerset{A \cup B}$ such that $X \notin \powerset{A} \cup \powerset{B}$.
Since $A$ and $B$ were arbitrary sets, this completes the proof that  for any sets $A$ and $B$, if $\powerset{A} \cup \powerset{B} = \powerset{A \cup B}$ then either $A \subseteq B$ or $B \subseteq A$.
\end{proof}


\begin{statement}{3.5.10}
Suppose $x$ and $y$ are real numbers and $x \neq 0$.
Prove that $y + 1/x = 1 + y/x$ if and only if $x = 1$ or $y = 1$.
\emph{Note: My copy of the book has a typo where it says $x = 0$, which is incorrect.}
\end{statement}

\begin{proof}
We start with the equation $y + 1/x = 1 + y/x$.
Multiplying both sides by $x$, we have $xy + 1 = x + y$, which we can rearrange to get the equation $xy - x - y + 1 = 0$.
Factoring the left hand side of the equation gives us $(x - 1)(y - 1) = 0$.
In other words, $y + 1/x = 1 + y/x$ if and only if $(x - 1)(y - 1) = 0$.
Since the product $ab$ of two real numbers $a$ and $b$ is zero if and only if $a = 0$ or $b = 0$, 
we conclude that $x - 1 = 0$ or $y - 1 = 0$.
In other words, $(x - 1)(y - 1) = 0$ if and only if $x = 1$ or $y = 1$.
Thus, this chain of equivalences completes the proof that $y + 1/x = 1 + y/x$ if and only if $x = 1$ or $y = 1$.
\end{proof}


\begin{statement}{3.5.11}
Prove that for every real number $x$, if $\abs{x - 3} > 3$ then $x^2 > 6x$.

\emph{
Hint: According to the definition of $\abs{x - 3}$, if $x - 3 \geq 0$ then $\abs{x - 3} = x - 3$, and if $x - 3 < 0$ then $\abs{x - 3} = 3 - x$.
The easiest way to use this fact is to break your proof into cases.
Assume that $x - 3 \geq 0$ in the first case, and $x - 3 < 0$ in the second case.
}
\end{statement}

\begin{proof}
Let $x \in \BR$ be arbitrary.
Suppose $\abs{x - 3} > 3$.
Following the hint, we break the proof into two cases: $x - 3 \geq 0$ and $x - 3 < 0$.
First we consider the case when $x - 3 \geq 0$.
In this case, $\abs{x - 3} = x - 3$, so the inequality $\abs{x - 3} > 3$ becomes $x - 3 > 3$.
Rearranging this inequality gives us $x > 6$, so multiplying both sides by $x$ results in the inequality $x^2 > 6x$, as desired.
Next we consider the case when $x - 3 < 0$.
In this case, $\abs{x - 3} = 3 - x$, so the inequality $\abs{x - 3} > 3$ becomes $3 - x > 3$.
Rearranging this inequality gives us $x < 0$.
While $x^2 > 0$ for all real numbers, since $x < 0$ when know that $6x < 0$ as well.
Hence, $x^2 > 6x$ once again in this case.
Now that we have checked both cases, we conclude that $x^2 > 6x$.
Since $x \in \BR$ was arbitrary, this holds for all real numbers.
\end{proof}


\begin{statement}{3.5.12}
Prove that for every real number $x$, $\abs{2x - 6} > x$ if and only if $\abs{x - 4} > 2$.

\emph{Hint: Read the hint for Exercise 3.5.11.}
\end{statement}

\begin{proof}
Let $x \in \BR$ be arbitrary.

Suppose $\abs{2x - 6} > x$.
There are two cases to consider: $2x - 6 \geq 0$ and $2x - 6 < 0$.
First we consider the case when $2x - 6 \geq 0$, so $\abs{2x - 6} = 2x - 6$.
This means the inequality $\abs{2x - 6} > x$ can be rewritten as $2x - 6 > x$.
Isolating $x$ on one side of the inequality, we have $x > 6$.
Since $x > 6$, $x - 4 > 2$, and in particular, $x - 4$ is positive.
Thus, $\abs{x - 4} = x - 4 > 2$, as desired.

Next we consider the case when $2x - 6 < 0$, so $\abs{2x - 6} = 6 - 2x$.
This means the inequality $\abs{2x - 6} > x$ can be rewritten as $6 - 2x > x$.
Isolating $x$ on one side of the inequality, we have $x < 2$.
Since $x < 2$, $x - 4 < -2$, or in other words $x - 4$ is negative and $4 - x > 2$.
Thus, $\abs{x - 4} = 4 - x > 2$, as desired.
As we can see, in both cases we obtained the desired inequality $\abs{x - 4} > 2$.

Now suppose $\abs{x - 4} > 2$.
Once again there are two cases to consider: $x - 4 \geq 0$ and $x - 4 < 0$.
First we consider the case when $x - 4 \geq 0$, so $\abs{x - 4} = x - 4$.
This means the inequality $\abs{x - 4} > 2$ can be rewritten as $x - 4 > 2$.
Isolating $x$ on one side of the inequality, we have $x > 6$.
We can then add $x$ to both sides of the inequality to get $2x > 6 + x$, which we can rewrite as $2x - 6 > x$.
Since $x > 6$, it follows that $2x - 6 > 0$, so then $\abs{2x - 6} = 2x - 6 > x$, as desired.

Next we consider the case when $x - 4 < 0$, so $\abs{x - 4} = 4 - x$.
This means the inequality $\abs{x - 4} > 2$ can be rewritten as $4 - x > 2$.
Isolating $x$ on one side of the inequality, we have $x < 2$.
We can then multiply both sides by 3 to get $3x < 6$.
Subtracting $2x$ from both sides then gives us $6 - 2x > x$.
Since $x < 2$, it follows that $2x - 6 < -2$, and in particular $2x - 6 < 0$, so then $\abs{2x - 6} = 6 - 2x > x$, as desired.
As we can see, in both cases we obtained the desired inequality $\abs{2x - 6} > x$.

Since $x \in \BR$ was arbitrary, we conclude that for every real number $x$, $\abs{2x - 6} > x$ if and only if $\abs{x - 4} > 2$.
\end{proof}


\begin{statement}{3.5.13}
\begin{enumerate}
	\item Prove that for all real numbers $a$ and $b$, $\abs{a} \leq b$ if and only if 
	$-b \leq a \leq b$.
	
	\item Prove that for any real number $x$, $-\abs{x} \leq x \leq \abs{x}$.
	\emph{Hint: Use Part (1).}
	
	\item Prove that for all real numbers $x$ and $y$, $\abs{x + y} \leq \abs{x} + \abs{y}$.
	\emph{This is called the \emph{triangle inequality}.
	One way to prove this is to combine Parts (1) and (2), but you can also do it by
	considering a number of cases.}
	
	\item Prove that for all real numbers $x$ and $y$, $\abs{x + y} \geq \abs{x} - \abs{y}$.
	\emph{Hint: Start with the equation $\abs{x} = \abs{(x + y) + (-y)}$ and then
	apply the triangle inequality to the right-hand side.}
\end{enumerate}
\end{statement}

\begin{proof}
\hfill
\begin{enumerate}
	\item Let $a$ and $b$ be arbitrary real numbers.
	Suppose $\abs{a} \leq b$.
	First note that this means $b \geq 0$, since $\abs{a} \geq 0$ by the definition of absolute value.
	There are two cases to consider: $a \geq 0$ and $a < 0$.
	In the case when $a \geq 0$, $\abs{a} = a$, so $a = \abs{a} \leq b$.
	In addition, since $b \geq 0$, $-b \leq 0$, and it follows that $a \geq -b$ as well.
	In other words, $-b \leq a \leq b$, as desired.
	Now we consider the case when $a < 0$.
	In this case, it automatically follows that $a \leq b$, since $b \geq 0$.
	Moreover, $a < 0$ means $\abs{a} = -a$, so $-a = \abs{a} \leq b$,
	which means $a \geq - b$.
	Thus, we once again conclude that $-b \leq a \leq b$.
	Hence, we conclude that if $\abs{a} \leq b$, then $-b \leq a \leq b$.
	
	Now suppose $-b \leq a \leq b$.
	Once again this means $b \geq 0$, since $-b \leq b$.
	Again, we consider the cases $a \geq 0$ and $a < 0$.
	In the case when $a \geq 0$, $\abs{a} = a$, so $\abs{a} = a \leq b$ and we are done.
	In the case when $a < 0$, $\abs{a} = -a$, so since $a \geq -b$ it follows that
	$\abs{a} = -a \leq b$, and once again we are done.
	Thus, we conclude that $\abs{a} \leq b$.
	
	Since $a$ and $b$ were arbitrary, this completes the proof that for all real numbers $a$ and $b$,
	$\abs{a} \leq b$ if and only if $-b \leq a \leq b$.
	
	\item Let $x$ be an arbitrary real number.
	Following the hint to use Part (1), to show that $-\abs{x} \leq x \leq \abs{x}$
	it suffices to check that $\abs{a} \leq b$, where $a = x$ and $b = \abs{x}$.
	In other words, it suffices to check that $\abs{x} \leq \abs{x}$, which is automatically
	true as any real number is equal to itself.
	Thus, by Part (1), since $\abs{x} \leq \abs{x}$, we conclude that $-\abs{x} \leq x \leq \abs{x}$.
	Since $x$ was an arbitrary real number, this compound inequality holds for all real numbers $x$.
	
	\item Let $x$ and $y$ be arbitrary real numbers.
	To show that $\abs{x + y} \leq \abs{x} + \abs{y}$, we will show that
	$-(\abs{x} + \abs{y}) \leq x + y \leq \abs{x} + \abs{y}$ and apply Part (1).
	By Part (2), we know that $-\abs{x} \leq x \leq \abs{x}$ and $-\abs{y} \leq y \leq \abs{y}$.
	Adding these two inequalities together then gives us the desired inequality:
	$-(\abs{x} + \abs{y}) \leq x + y \leq \abs{x} + \abs{y}$.
	This allows us to apply Part (1) with $a = x + y$ and $b = \abs{x} + \abs{y}$ to conclude that
	$\abs{x + y} \leq \abs{x} + \abs{y}$.
	Since $x$ and $y$ were arbitrary real numbers, this inequality holds for all real numbers.
	This completes the proof.
	
	\item Let $x$ and $y$ be arbitrary real numbers.
	Following the hint, we start with the equation $\abs{x} = \abs{(x + y) + (-y)}$.
	Applying the triangle inequality to the right-hand side gives us
	$\abs{x} = \abs{(x + y) + (-y)} \leq \abs{x + y} + \abs{-y}$.
	In other words, $\abs{x} \leq \abs{x + y} + \abs{y}$.
	Rearranging gives us the desired inequality $\abs{x + y} \geq \abs{x} - \abs{y}$.
	Since $x$ and $y$ were arbitrary, this inequality holds for all real numbers.
\end{enumerate}
\end{proof}


\begin{statement}{3.5.14}
Prove for every integer $x$, $x^2 + x$ is even.
\end{statement}

\begin{proof}
Let $x$ be an arbitrary integer.
There are two cases to consider: $x$ is even and $x$ is odd.
First we consider the case when $x$ is even, so $x = 2n$ for some integer $n$.
In this case, $x^2 + x = 4n^2 + 2n = 2(2n^2 + n)$.
Since $k = 2n^2 + n$ is an integer, we conclude that $x^2 + x = 2k$, or in other words $x^2 + x$ is even.
Next we consider the case when $x$ is odd, so $x = 2n + 1$ for some integer $n$.
In this case, $x^2 + x = (4n^2 + 4n + 1) + (2n + 1) = 4n^2 + 6n + 2 = 2(2n^2 + 3n + 1)$.
Since $k = 2n^2 + 3n + 1$ is an integer, we conclude that $x^2 + x = 2k$, or in other words $x^2 + x$ is once again even.
Since $x$ was an arbitrary integer and we saw that $x^2 + x$ was even in both cases, we conclude that $x^2 + x$ is even for all integers $x$.
\end{proof}


\begin{statement}{3.5.15}
Prove that for every $x \in \BZ$, the remainder when $x^4$ is divided by 8 is either 0 or 1.
\end{statement}

\begin{proof}
Let $x$ be an arbitrary integer.
There are two cases to consider: $x$ is even and $x$ is odd.
First we consider the case when $x$ is even, so $x = 2n$ for some integer $n$.
In this case, $x^4 = (2n)^4 = 8(2n^4)$, and since $2n^4$ is an integer, the remainder when $x^4 = 8(2n^4)$ is divided by 8 is 0.
Next we consider the case when $x$ is odd, so $x = 2n + 1$ for some integer $n$.
Computing $x^4 = (2n + 1)^4$, we have
\begin{align*}
	x^4 &= (2n + 1)^4 \\
	&= (2n)^4 + 4(2n^3) + 6(2n)^2 + 4(2n) + 1 \\
	&= 8(2n^4) + 8(4n^3) + 8(3n^2) + 8n + 1 \\
	&= 8(2n^4 + 4n^3 + 3n^2 + n) + 1.
\end{align*}
Since $2n^4 + 4n^3 + 3n^2 + n$ is an integer, the remainder then $x^4 = 8(2n^4 + 4n^3 + 3n^2 + n) + 1$ is divided by 8 is 1.
Since $x$ was an arbitrary integer, we conclude that for every integer $x$, the remainder when $x^4$ is divided by 8 is either 0 (when $x$ is even) or 1 (when $x$ is odd).
\end{proof}


\begin{statement}{3.5.16}
Suppose $\mathcal{F}$ and $\mathcal{G}$ are nonempty families of sets.
\begin{enumerate}
	\item Prove that $\bigcup (\mathcal{F} \cup \mathcal{G}) =
	\left( \bigcup \mathcal{F} \right) \cup \left( \bigcup \mathcal{G} \right)$.
	
	\item Prove that $B \cup \left( \bigcup \mathcal{F} \right) = 
	\bigcup_{A \in \mathcal{F}} (B \cup A)$.
	
	\item Can you discover and prove a similar theorem about
	$\bigcap (\mathcal{F} \cup \mathcal{G})$?
\end{enumerate}
\end{statement}

\begin{proof}
\hfill
\begin{enumerate}
	\item First, suppose $x \in \bigcup (\mathcal{F} \cup \mathcal{G})$ is arbitrary.
	This means there is a set $X \in \mathcal{F} \cup \mathcal{G}$ such that $x \in X$.
	If $X \in \mathcal{F}$, then this means $x \in \bigcup \mathcal{F}$.
	Similarly, if $X \in \mathcal{G}$, then this means $x \in \bigcup \mathcal{G}$.
	Since this covers the two cases resulting from the fact that $X \in \mathcal{F} \cup \mathcal{G}$,
	we conclude that $x \in \left( \bigcup \mathcal{F} \right) \cup \left( \bigcup \mathcal{G} \right)$.
	In other words, since $x$ was arbitrary, we conclude that
	$\bigcup (\mathcal{F} \cup \mathcal{G}) \subseteq
	\left( \bigcup \mathcal{F} \right) \cup \left( \bigcup \mathcal{G} \right)$.
	
	Next, suppose $x \in \left( \bigcup \mathcal{F} \right) \cup \left( \bigcup \mathcal{G} \right)$
	is arbitrary.
	In this case, either $x \in \bigcup \mathcal{F}$ or $x \in \bigcup \mathcal{G}$.
	If $x \in \bigcup \mathcal{F}$, then there is a set $X \in \mathcal{F}$ such that $x \in X$.
	Since $X \in \mathcal{F}$ means $X \in \mathcal{F} \cup \mathcal{G}$ as well,
	this means $x \in \bigcup (\mathcal{F} \cup \mathcal{G})$.
	Similarly, if $x \in \bigcup \mathcal{G}$, 
	then there is a set $X \in \mathcal{G}$ such that $x \in X$.
	Once again, as $X \in \mathcal{F} \cup \mathcal{G}$ as well, this also means
	$x \in \bigcup (\mathcal{F} \cup \mathcal{G})$.
	Thus, since $x$ was arbitrary, we conclude that
	$\left( \bigcup \mathcal{F} \right) \cup \left( \bigcup \mathcal{G} \right) \subseteq
	\bigcup (\mathcal{F} \cup \mathcal{G})$.
	
	As we have now checked both containments, we finally conclude that
	$\bigcup (\mathcal{F} \cup \mathcal{G}) =
	\left( \bigcup \mathcal{F} \right) \cup \left( \bigcup \mathcal{G} \right)$.
	
	\item First, suppose $x \in B \cup \left( \bigcup \mathcal{F} \right)$ is arbitrary.
	There are two cases to consider: $x \in B$ and $x \in \bigcup \mathcal{F}$.
	If $x \in B$, then $x \in B \cup A$ for all $A \in \mathcal{F}$, so it immediately follows
	that $x \in \bigcup_{A \in \mathcal{F}} (B \cup A)$.
	If $x \in \bigcup \mathcal{F}$, then there is a set $A \in \mathcal{F}$ such that $x \in A$.
	Then, $x \in B \cup A$ for this set $A$, so $x \in \bigcup_{A \in \mathcal{F}} (B \cup A)$.
	Since $x$ was arbitrary and in both cases $x \in \bigcup_{A \in \mathcal{F}} (B \cup A)$,
	we conclude that $B \cup \left( \bigcup \mathcal{F} \right) \subseteq 
	\bigcup_{A \in \mathcal{F}} (B \cup A)$.
	
	Next, suppose $x \in \bigcup_{A \in \mathcal{F}} (B \cup A)$ is arbitrary.
	This means there is some $A \in \mathcal{F}$ such that $x \in B \cup A$.
	If $x \in B$, then it immediately follows that $x \in B \cup \left( \bigcup \mathcal{F} \right)$.
	If $x \in A$, then $x \in \bigcup \mathcal{F}$, so $x \in B \cup \left( \bigcup \mathcal{F} \right)$
	as well.
	Since $x$ was arbitrary and in both cases $x \in B \cup \left( \bigcup \mathcal{F} \right)$,
	we conclude that $\bigcup_{A \in \mathcal{F}} (B \cup A) \subseteq
	B \cup \left( \bigcup \mathcal{F} \right)$.
	
	As we have now checked both containments, we finally conclude that
	$B \cup \left( \bigcup \mathcal{F} \right) = 
	\bigcup_{A \in \mathcal{F}} (B \cup A)$.
	
	\item We'll prove that $\bigcap (\mathcal{F} \cup \mathcal{G}) =
	\left( \bigcap \mathcal{F} \right) \cap \left( \bigcap \mathcal{G} \right)$.
	
	First, suppose $x \in \bigcap (\mathcal{F} \cup \mathcal{G})$ is arbitrary.
	This means that for all $X \in \mathcal{F} \cup \mathcal{G}$, $x \in X$.
	Let $A \in \mathcal{F}$ be arbitrary.
	Since $A \in \mathcal{F} \cup \mathcal{G}$, we know that $x \in A$.
	Thus, $x \in A$ for all $A \in \mathcal{F}$ and it follows that $x \in \bigcap \mathcal{F}$.
	The same argument for an arbitrary $B \in \mathcal{G}$ shows that $x \in \bigcap \mathcal{G}$.
	We therefore conclude that 
	$x \in \left( \bigcap \mathcal{F} \right) \cap \left( \bigcap \mathcal{G} \right)$.
	Since $x$ was arbitrary, it follows that $\bigcap (\mathcal{F} \cup \mathcal{G}) \subseteq
	\left( \bigcap \mathcal{F} \right) \cap \left( \bigcap \mathcal{G} \right)$.
	
	Next, suppose $x \in \left( \bigcap \mathcal{F} \right) \cap \left( \bigcap \mathcal{G} \right)$
	is arbitrary.
	This means that $x \in \bigcap \mathcal{F}$ and $x \in \bigcap \mathcal{G}$.
	In other words, for all $A \in \mathcal{F}$, $x \in A$, and for all $B \in \mathcal{G}$, $x \in B$.
	Let $X \in \mathcal{F} \cup \mathcal{G}$ be arbitrary.
	If $X \in \mathcal{F}$, then it follows that $x \in X$ since $x \in \bigcap \mathcal{F}$.
	Similarly, if $X \in \mathcal{G}$ it follows that $x \in X$ since $x \in \bigcap \mathcal{G}$.
	Since in both cases $x \in X$, and $X$ was arbitrary, we know that $x \in X$ for all
	$X \in \mathcal{F} \cup \mathcal{G}$.
	In other words, $x \in \bigcap (\mathcal{F} \cup \mathcal{G})$.
	Since $x$ was arbitrary, it follows that
	$\left( \bigcap \mathcal{F} \right) \cap \left( \bigcap \mathcal{G} \right) \subseteq
	\bigcap (\mathcal{F} \cup \mathcal{G})$.
	
	As we have now checked both containments, we finally conclude that
	$\bigcap (\mathcal{F} \cup \mathcal{G}) =
	\left( \bigcap \mathcal{F} \right) \cap \left( \bigcap \mathcal{G} \right)$.
\end{enumerate}
\end{proof}


\begin{statement}{3.5.17}
Suppose $\mathcal{F}$ is a nonempty family of sets and $B$ is a set.
\begin{enumerate}
	\item Prove that $B \cup \left( \bigcup \mathcal{F} \right) = 
	\bigcup (\mathcal{F} \cup \{ B \})$.
	
	\item Prove that $B \cup \left( \bigcap \mathcal{F} \right) = 
	\bigcap_{A \in \mathcal{F}} (B \cup A)$.
	
	\item Can you discover and prove similar theorems about 
	$B \cap \left( \bigcup \mathcal{F} \right)$ and
	$B \cap \left( \bigcap \mathcal{F} \right)$?
\end{enumerate}
\end{statement}

\begin{proof}
Before tackling the exercise, we first prove the following lemma.
\begin{lemma}
	For any set $B$, $B = \bigcup \{ B \} = \bigcap \{ B \}$.
\end{lemma}
\begin{proof}
	Let $B$ be an arbitrary set.
	First, suppose $x \in B$ is arbitrary.
	Then clearly $x \in \bigcup \{ B \}$ since $B \in \{ B \}$, so $B \subseteq \bigcup \{ B \}$.
	In addition, as $B$ is the only element of $\{ B \}$, $x \in \bigcap \{ B \}$, which means
	$B \subseteq \bigcap \{ B \}$ as well.
	Next, suppose $x \in \bigcup \{ B \}$ be arbitrary.
	Since $B$ is the only element of $\{ B \}$, it follows that $x \in B$, which means
	$\bigcup \{ B \} \subseteq B$.
	Hence, $B = \bigcup \{ B \}$.
	Finally, suppose $x \in \bigcap \{ B \}$ be arbitrary.
	Once again, since $B$ is the only element of $\{ B \}$, it follows that $x \in B$.
	Thus, $\bigcap \{ B \} \subseteq B$, and we conclude that $B = \bigcap \{ B \}$.
	This completes the proof that $B = \bigcup \{ B \} = \bigcap \{ B \}$.
\end{proof}
Now we turn our attention to the exercise.
\begin{enumerate}
	\item By the above lemma, we know that $B = \bigcup \{ B \}$.
	This allows us to apply the result from Part (1) of Exercise 3.5.16, which stated that
	$\bigcup (\mathcal{F} \cup \mathcal{G}) =
	\left( \bigcup \mathcal{F} \right) \cup \left( \bigcup \mathcal{G} \right)$, with 
	$\mathcal{G} = \{ B \}$, to conclude that $B \cup \left( \bigcup \mathcal{F} \right) = 
	\bigcup (\mathcal{F} \cup \{ B \})$.
	
	\item First, suppose $x \in B \cup \left( \bigcap \mathcal{F} \right)$ is arbitrary.
	Then there are two cases to consider: $x \in B$ and $x \in \bigcap \mathcal{F}$.
	If $x \in B$, then it immediately follows $x \in B \cup A$ for all $A \in \mathcal{F}$,
	so we know that $x \in \bigcap_{A \in \mathcal{F}} (B \cup A)$.
	If $x \in \bigcap \mathcal{F}$, then $x \in A$ for all $A \in \mathcal{F}$.
	It then follows that $x \in B \cup A$ for all $A \in \mathcal{F}$ as well,
	so $x \in \bigcap_{A \in \mathcal{F}} (B \cup A)$ once again.
	Since $x$ was arbitrary and $x \in \bigcap_{A \in \mathcal{F}} (B \cup A)$, we conclude that
	$B \cup \left( \bigcap \mathcal{F} \right) \subseteq \bigcap_{A \in \mathcal{F}} (B \cup A)$.
	
	Next, suppose $x \in \bigcap_{A \in \mathcal{F}} (B \cup A)$ is arbitrary.
	This means $x \in B \cup A$ for all $A \in \mathcal{F}$.
	If $x \in B$, then clearly $x \in B \cup \left( \bigcap \mathcal{F} \right)$, so suppose $x \notin B$.
	In this case, since $x \in B \cup A$ for all $A \in \mathcal{F}$,
	it follows that $x \in A$ for all $A \in \mathcal{F}$.
	In other words, $x \in \bigcap \mathcal{F}$, so once again 
	$x \in B \cup \left( \bigcap \mathcal{F} \right)$.
	Since $x$ was arbitrary and $x \in B \cup \left( \bigcap \mathcal{F} \right)$, we conclude that
	$\bigcap_{A \in \mathcal{F}} (B \cup A) \subseteq B \cup \left( \bigcap \mathcal{F} \right)$.
	
	As we have checked both containments, we finally conclude that
	$B \cup \left( \bigcap \mathcal{F} \right) = \bigcap_{A \in \mathcal{F}} (B \cup A)$.
	
	\item Our first theorem is about $B \cap \left( \bigcap \mathcal{F} \right)$.
	\begin{theorem}
		$B \cap \left( \bigcap \mathcal{F} \right) = \bigcap (\mathcal{F} \cup \{ B \})$.
	\end{theorem}
	\begin{proof}
		By the lemma we proved at the start of the exercise, we know that
		$B = \bigcap \{ B \}$.
		This allows us to apply the result from Part (3) of Exercise 3.5.16, which stated that
		$\bigcap (\mathcal{F} \cup \mathcal{G}) =
		\left( \bigcap \mathcal{F} \right) \cap \left( \bigcap \mathcal{G} \right)$,
		with $\mathcal{G} = \{ B \}$, to conclude that $B \cap \left( \bigcap \mathcal{F} \right) = 
		\bigcap (\mathcal{F} \cup \{ B \})$.
	\end{proof}
	Our second theorem is about $B \cap \left( \bigcup \mathcal{F} \right)$.
	\begin{theorem}
		$B \cap \left( \bigcup \mathcal{F} \right) = \bigcup_{A \in \mathcal{F}} (B \cap A)$.
	\end{theorem}
	\begin{proof}
		First, suppose $x \in B \cap \left( \bigcup \mathcal{F} \right)$ is arbitrary.
		This means $x \in B$ and $x \in \bigcup \mathcal{F}$,
		so there is $A \in \mathcal{F}$ such that $x \in A$.
		In other words, for this $A \in \mathcal{F}$, $x \in B \cap A$.
		Since there is $A \in \mathcal{F}$ such that $x \in B \cap A$, it follows that
		$x \in \bigcup_{A \in \mathcal{F}} (B \cap A)$.
		Since $x$ was arbitrary, we conclude that
		$B \cap \left( \bigcup \mathcal{F} \right) \subseteq \bigcup_{A \in \mathcal{F}} (B \cap A)$.
		
		Next, suppose $x \in \bigcup_{A \in \mathcal{F}} (B \cap A)$ is arbitrary.
		This means there is $A \in \mathcal{F}$ such that $x \in B \cap A$.
		In particular $x \in B$ and, since $x \in A$ and $A \in \mathcal{F}$, 
		$x \in \bigcup \mathcal{F}$.
		In other words, $x \in B \cap \left( \bigcup \mathcal{F} \right)$.
		Since $x$ was arbitrary, we conclude that
		$\bigcup_{A \in \mathcal{F}} (B \cap A) \subseteq B \cap \left( \bigcup \mathcal{F} \right)$.
		
		As we have checked both containments, we finally conclude that
		$B \cap \left( \bigcup \mathcal{F} \right) = \bigcup_{A \in \mathcal{F}} (B \cap A)$.
	\end{proof}
\end{enumerate}
\end{proof}


\begin{statement}{3.5.18}
Suppose $\mathcal{F}$, $\mathcal{G}$, and $\mathcal{H}$ are nonempty families of sets and for every $A \in \mathcal{F}$ and every $B \in \mathcal{G}$, $A \cup B \in \mathcal{H}$.
Prove that $\bigcap \mathcal{H} \subseteq \left( \bigcap \mathcal{F} \right) \cup \left( \bigcap \mathcal{G} \right)$.
\end{statement}

\begin{proof}
Suppose $x \in \bigcap \mathcal{H}$ is arbitrary.
This means $x \in C$ for every $C \in \mathcal{H}$.
In particular, since $A \cup B \in \mathcal{H}$ for every $A \in \mathcal{F}$ and every $B \in \mathcal{G}$, we know that $x \in A \cup B$ for every $A \in \mathcal{F}$ and every $B \in \mathcal{G}$.
We now consider two cases: $x \in \bigcap \mathcal{G}$ and $x \notin \bigcap \mathcal{G}$.
In the first case, it automatically follows that $x \in \left( \bigcap \mathcal{F} \right) \cup \left( \bigcap \mathcal{G} \right)$.
In the second case, this means there is some $B_0 \in \mathcal{G}$ such that $x \notin B_0$.
Now we check that $x \in \bigcap \mathcal{F}$.
Since $x \in A \cup B$ for every $A \in \mathcal{F}$ and every $B \in \mathcal{G}$, we know that in particular $x \in A \cup B_0$ for every $A \in \mathcal{F}$.
Then, since $x \notin B_0$, this means $x \in A$ for every $A \in \mathcal{F}$.
In other words, $x \in \bigcap \mathcal{F}$.
Thus, since $x$ was arbitrary and we showed that $x \in \bigcap \mathcal{F}$, we conclude that $\bigcap \mathcal{H} \subseteq \left( \bigcap \mathcal{F} \right) \cup \left( \bigcap \mathcal{G} \right)$.
\end{proof}


\begin{statement}{3.5.19}
Suppose $A$ and $B$ are sets.
Prove that $\forall x (x \in A \bigtriangleup B \leftrightarrow (x \in A \leftrightarrow x \notin B))$.
\end{statement}

\begin{proof}
Let $x$ be arbitrary.
We start by proving the forward direction: $x \in A \bigtriangleup B \rightarrow (x \in A \leftrightarrow x \notin B)$.
Translating this into mathematical English, we want to show that if $x \in A \bigtriangleup B$, then $x \in A$ if and only if $x \notin B$.
Suppose $x \in A \bigtriangleup B$.
Recall that $A \bigtriangleup B = (A \setminus B) \cup (B \setminus A)$, so $x \in A \setminus B$ or $x \in B \setminus A$.
We first check that if $x \in A$ then $x \notin B$, so suppose $x \in A$.
Since $x \in A$, it follows that $x \notin B \setminus A$.
Then, since $x \in A \bigtriangleup B$ and $x \notin B \setminus A$, we must have $x \in A \setminus B$.
In other words, $x \notin B$, as desired.
Now we check that if $x \notin B$ then $x \in A$, so now suppose $x \notin B$.
In this case, since $x \notin B$, we know that $x \notin B \setminus A$.
This fact combined with the fact that $x \in A \bigtriangleup B$ means that $x \in A \setminus B$.
Thus, $x \in A$, as desired.
Since we have now checked that $x \in A$ if and only if $x \notin B$, this completes the proof of the forward direction.

Now we check the backward direction: $(x \in A \leftrightarrow x \notin B) \rightarrow x \in A \bigtriangleup B$.
Suppose we know that $x \in A$ if and only if $x \notin B$.
Our goal now is to prove that $x \in A \bigtriangleup B$.
We consider two possibilities: $x \in B \setminus A$ and $x \notin B \setminus A$.
In the first case, it immediately follows that $x \in A \bigtriangleup B = (A \setminus B) \cup (B \setminus A)$.
In the second case, either $x \notin B$ or $x \in A$.
If $x \notin B$, then the assumption that $x \in A$ if and only if $x \notin B$ lets us immediately conclude that $x \in A$.
Similarly, the assumption that $x \in A$ if and only if $x \notin B$ lets us immediately conclude that $x \notin B$ if $x \in A$.
In both situations, the ultimate conclusion is that $x \in A \setminus B$, which allows us to conclude that $x \in A \bigtriangleup B$.
This completes the proof of the backward direction.

Since $x$ was arbitrary, we conclude that $\forall x (x \in A \bigtriangleup B \leftrightarrow (x \in A \leftrightarrow x \notin B))$.
\end{proof}


\begin{statement}{3.5.20}
Suppose $A$, $B$, and $C$ are sets.
Prove that $A \bigtriangleup B$ and $C$ are disjoint if and only if $A \cap C = B \cap C$.
\end{statement}

\begin{proof}
We start by proving the forward direction.
Suppose $A \bigtriangleup B$ and $C$ are disjoint.
In other words, for all $x$, $x \notin (A \bigtriangleup B) \cap C$.
First, suppose $x \in A \cap C$ is arbitrary.
Since $x \in C$ and $x \notin (A \bigtriangleup B) \cap C$, we know that $x \notin A \bigtriangleup B = (A \cup B) \setminus (A \cap B)$.
Then, since $x \in A$, it follows that $x \in A \cup B$, so we must have $x \in A \cap B$ as well in order to have $x \notin A \bigtriangleup B$.
Finally, since $x \in A \cup B$, and in particular $x \in B$, we know that $x \in B \cap C$.
Hence, we conclude that $A \cap C \subseteq B \cap C$.
Repeating the same argument for arbitrary $x \in B \cap C$ shows that $B \cap C \subseteq A \cap C$ as well.
Therefore, $A \cap C = B \cap C$, which is what we wanted to prove.

Now we prove the backward direction.
Suppose $A \cap C = B \cap C$.
Next, suppose $x \in (A \bigtriangleup B) \cap C$, so $x \in A \bigtriangleup B$ and $x \in C$.
Since $x \in A \bigtriangleup B$, $x \in A \cup B$ and $x \notin A \cap B$.
There are two cases to consider: $x \in A$ and $x \in B$.
If $x \in A$, then it follows that $x \notin B$.
However, since $x \in A \cap C = B \cap C$, it also follows that $x \in B$.
This is a contradiction.
We reach a similar condition in the case where $x \in B$.
Therefore, we conclude that there does not exists $x \in (A \bigtriangleup B) \cap C$.
In other words, $A \bigtriangleup B$ and $C$ are disjoint.
This completes the proof.
\end{proof}


\begin{statement}{3.5.21}
Suppose $A$, $B$, and $C$ are sets.
Prove that $A \bigtriangleup B$ is a subset of $C$ if and only if $A \cup C = B \cup C$.
\end{statement}

\begin{proof}
We start by proving the forward direction.
Suppose $A \bigtriangleup B \subseteq C$.
Suppose $x \in A \cup C$ is arbitrary.
There are two cases to consider: $x \in C$ and $x \notin C$.
In the first case, it immediately follows that $x \in B \cup C$ and there is nothing else to check, so we focus our attention on the second case, where $x \notin C$.
Since $x \in A \cup C$ and $x \notin C$, we know that $x \in A$.
Now suppose, towards a contradiction, that $x \notin B$.
Since $x \in A$ and $x \notin B$, $x \in A \setminus B$, which means $x \in A \bigtriangleup B$.
Then, since $A \bigtriangleup B \subseteq C$, it follows that $x \in C$, which is a contradiction.
Therefore, $x \in B$ and subsequently $x \in B \cup C$.
Thus, since $x$ was arbitrary, $A \cup C \subseteq B \cup C$.
An analogous argument starting from $x \in B \cup C$ shows us that $B \cup C \subseteq A \cup C$.
We therefore conclude that $A \cup C = B \cup C$.

Now we prove the backward direction.
Suppose $A \cup C = B \cup C$.
Suppose $x \in A \bigtriangleup B = (A \setminus B) \cup (B \setminus A)$ is arbitrary.
Next suppose, towards a contradiction, that $x \notin C$.
There are two cases to consider: $x \in A \setminus B$ and $x \in B \setminus A$.
If $x \in A \setminus B$, then $x \in A$ and $x \notin B$.
Since $x \in A$ it follows that $x \in A \cup C = B \cup C$.
Then, since $x \notin C$, we conclude $x \in B$, but this contradicts the prior fact that $x \notin B$.
We reach a similar contradiction when starting from $x \in B \setminus A$.
Since we reach a contradiction in both cases, we conclude that $x \in C$.
In other words, since $x$ was arbitrary, $A \bigtriangleup B \subseteq C$.
This completes the proof.
\end{proof}


\begin{statement}{3.5.22}
Suppose $A$, $B$, and $C$ are sets.
Prove that $C \subseteq A \bigtriangleup B$ if and only if $C \subseteq A \cup B$ and $A \cap B \cap C = \varnothing$.
\end{statement}

\begin{proof}
We start by proving the forward direction.
Suppose $C \subseteq A \bigtriangleup B$.
First suppose $x \in C$ is arbitrary.
Since $x \in C$ and $C \subseteq A \bigtriangleup B = (A \cup B) \setminus (A \cap B)$, it follows that $x \in A \cup B$.
Hence, $C \subseteq A \cup B$.
Next, check that $A \cap B \cap C = \varnothing$ by unraveling the definition of $A \cap B \cap C$.
If $x \in A \cap B \cap C$, then in particular $x \in A \cap B$ and $x \in C$.
However, since $C \subseteq A \bigtriangleup B$, any element of $C$ is not an element of $A \cap B$.
Therefore, there does not exist $x \in A \cap B \cap C$, or in other words, $A \cap B \cap C = \varnothing$.

Now we prove the backward direction.
Suppose $C \subseteq A \cup B$ and $A \cap B \cap C = \varnothing$.
Next, suppose $x \in C$ is arbitrary.
Since $C \subseteq A \cup B$, we know that $x \in A \cup B$.
In addition, since $A \cap B \cap C = \varnothing$, we know $x \notin A \cap B \cap C$, which means $x \notin A \cap B$.
Thus, $x \in (A \cup B) \setminus (A \cap B) = A \bigtriangleup B$, so we conclude that $C \subseteq A \bigtriangleup B$.
This completes the proof.
\end{proof}


\begin{statement}{3.5.23}
Suppose $A$, $B$, and $C$ are sets.
\begin{enumerate}
	\item Prove that $A \setminus C \subseteq (A \setminus B) \cup (B \setminus C)$.
	
	\item Prove that $A \bigtriangleup C \subseteq (A \bigtriangleup B) \cup (B \bigtriangleup C)$.
\end{enumerate}
\end{statement}

\begin{proof}
\hfill
\begin{enumerate}
	\item Suppose $x \in A \setminus C$ is arbitrary.
	Then $x \in A$ and $x \notin C$.
	We consider two cases: $x \in B$ and $x \notin B$.
	If $x \in B$, then $x \in B \setminus C$ and hence $x \in (A \setminus B) \cup (B \setminus C)$.
	On the other hand, if $x \notin B$, then $x \in A \setminus B$ and once again
	$x \in (A \setminus B) \cup (B \setminus C)$.
	Since in both cases $x \in (A \setminus B) \cup (B \setminus C)$, we conclude that
	$A \setminus C \subseteq (A \setminus B) \cup (B \setminus C)$.
	
	\item Suppose $x \in A \bigtriangleup C = (A \setminus C) \cup (C \setminus A)$ is arbitrary.
	There are two cases to consider: $x \in A \setminus C$ and $x \in C \setminus A$.
	We first note that $(A \bigtriangleup B) \cup (B \bigtriangleup C) = 
	(A \setminus B) \cup (B \setminus A) \cup (B \setminus C) \cup (C \setminus B)$.
	If $x \in A \setminus C$, then by Part (1) above we know that 
	$x \in (A \setminus B) \cup (B \setminus C)$, so 
	$x \in (A \bigtriangleup B) \cup (B \bigtriangleup C)$.
	Similarly, by swapping the roles of $A$ and $C$ in Part (A) we know that
	$C \setminus A \subseteq (C \setminus B) \cup (B \setminus A)$, so if $x \in C \setminus A$, then
	$x \in (C \setminus B) \cup (B \setminus A)$.
	Once again, this means $x \in (A \bigtriangleup B) \cup (B \bigtriangleup C)$.
	Thus, we conclude that $A \bigtriangleup C \subseteq 
	(A \bigtriangleup B) \cup (B \bigtriangleup C)$.
\end{enumerate}
\end{proof}


\begin{statement}{3.5.24}
Suppose $A$, $B$, and $C$ are sets.
\begin{enumerate}
	\item Prove that $(A \cup B) \bigtriangleup C \subseteq
	(A \bigtriangleup C) \cup (B \bigtriangleup C)$.
	
	\item Find an example of sets $A$, $B$, and $C$ such that $(A \cup B) \bigtriangleup C \neq
	(A \bigtriangleup C) \cup (B \bigtriangleup C)$.
\end{enumerate}
\end{statement}

\begin{proof}
\hfill
\begin{enumerate}
	\item Suppose $x \in (A \cup B) \bigtriangleup C$ is arbitrary.
	This means $x \in A \cup B$ and $x \notin C$, or $x \notin A \cup B$ and $x \in C$.
	In the first case, when $x \in A \cup B$ and $x \notin C$, there are two subcases to consider:
	$x \in A$ and $x \in B$.
	If $x \in A$ then $x \in A \bigtriangleup C$, and if $x \in B$ then $x \in B \bigtriangleup C$.
	In either of these subcases, $x \in (A \bigtriangleup C) \cup (B \bigtriangleup C)$.
	Then the second case, when $x \notin A \cup B$ and $x \in C$, we know that
	$x \notin A$ and $x \notin B$.
	This means $x \in C \setminus A$ and $x \in C \setminus B$.
	Hence, we once again conclude that $x \in (A \bigtriangleup C) \cup (B \bigtriangleup C)$.
	Since $x$ was arbitrary, we conclude that $(A \cup B) \bigtriangleup C \subseteq
	(A \bigtriangleup C) \cup (B \bigtriangleup C)$.
	
	\item Consider $A = \{ 1, 2 \}$, $B = \{ 3, 4 \}$, and $C = \{ 1, 2, 3, 4, 5 \}$.
	In this case, $(A \cup B) \bigtriangleup C = \{ 5 \}$, while
	$(A \bigtriangleup C) \cup (B \bigtriangleup C) = \{ 1, 2, 3, 4, 5 \}$.
\end{enumerate}
\end{proof}


\begin{statement}{3.5.25}
Suppose $A$, $B$, and $C$ are sets.
\begin{enumerate}
	\item Prove that $(A \bigtriangleup C) \cap (B \bigtriangleup C) \subseteq
	(A \cap B) \bigtriangleup C$.
	
	\item Is it always true that $(A \cap B) \bigtriangleup C \subseteq
	(A \bigtriangleup C) \cap (B \bigtriangleup C)$?
	Give either a proof or a counterexample.
\end{enumerate}
\end{statement}

\begin{proof}
\hfill
\begin{enumerate}
	\item Suppose $x \in (A \bigtriangleup C) \cap (B \bigtriangleup C)$ is arbitrary.
	This means that $x \in A \bigtriangleup C$ and $x \in B \bigtriangleup C$.
	Since $x \in A \bigtriangleup C$, $x \in A \setminus C$ or $x \in C \setminus A$.
	Similarly, since $x \in B \bigtriangleup C$, $x \in B \setminus C$ or $x \in C \setminus B$.
	First, suppose $x \in C$.
	In this case, we must have $x \in C \setminus A$ and $x \in C \setminus B$.
	Then, since $(C \setminus A) \cap (C \setminus B) = C \setminus (A \cup B) \subseteq
	C \setminus (A \cap B)$, it follows that $x \in (A \cap B) \bigtriangleup C$.
	Next, suppose $x \notin C$.
	In this case, we must have $x \in A \setminus C$ and $x \in B \setminus C$.
	Then, since $(A \setminus C) \cap (B \setminus C) = (A \cap B) \setminus C$, it once again
	follows that $x \in (A \cap B) \bigtriangleup C$.
	Hence, since $x$ was arbitrary, we conclude that 
	$(A \bigtriangleup C) \cap (B \bigtriangleup C) \subseteq (A \cap B) \bigtriangleup C$.
	
	\item This is not always true.
	Consider $A = \{ 1, 2 \}$, $B = \{ 2, 3 \}$ and $C = \{ 3, 4 \}$.
	In this case, $(A \bigtriangleup C) \cap (B \bigtriangleup C) = \{ 2, 4 \}$,
	while $(A \cap B) \bigtriangleup C = \{ 2, 3, 4 \}$.
\end{enumerate}
\end{proof}


\begin{statement}{3.5.26}
Suppose $A$, $B$, and $C$ are sets.
Consider the sets $(A \setminus B) \bigtriangleup C$ and $(A \bigtriangleup C) \setminus (B \bigtriangleup C)$.
Can you prove that either is a subset of the other?
Justify your conclusions with either proofs or counterexamples.
\end{statement}

\begin{proof}
We will prove the following theorem.
\begin{theorem}
	\hfill
	\begin{enumerate}
		\item $(A \bigtriangleup C) \setminus (B \bigtriangleup C) \subseteq
		(A \setminus B) \bigtriangleup C$.
	
		\item $(A \setminus B) \bigtriangleup C \nsubseteq
		(A \bigtriangleup C) \setminus (B \bigtriangleup C)$.
	\end{enumerate}
\end{theorem}
We start by proving the first statement.
Suppose $x \in (A \bigtriangleup) \setminus (B \bigtriangleup C)$ is arbitrary.
Unraveling the definitions, this means $x \in A \setminus C$ or $x \in C \setminus A$, and in addition $x \notin B \setminus C$ and $b \notin C \setminus B$.
First suppose $x \in A \setminus C$, so then $x \in A$ and $x \notin C$.
Since $x \notin C$ and $x \notin B \setminus C$, we know that $x \notin B$.
Then, since $x \in A$, $x \notin B$, and $x \notin C$, we know that $x \in (A \setminus B) \setminus C$.
It follows that $x \in (A \setminus B) \bigtriangleup C$.
Next suppose $x \in C \setminus A$, so $x \in C$ and $x \notin A$.
In this case, since $x \in C$ and $x \notin C \setminus B$, we know that $x \in B$.
Since $x \in C$, $x \notin A$, and $x \in B$, we know that $x \in C \setminus (A \setminus B)$.
It follows that once again $x \in (A \setminus B) \bigtriangleup C$.
Since in both cases $x \in (A \setminus B) \bigtriangleup C$, we conclude that $(A \bigtriangleup C) \setminus (B \bigtriangleup C) \subseteq (A \setminus B) \bigtriangleup C$.

To prove the second statement, consider the following counterexample.
Let $A = \{ 1, 2 \}$, $B = \{ 2, 3 \}$, and $C = \{ 3, 4 \}$.
We then have
\begin{equation*}
	A \setminus B = \{ 1 \}, \quad
	A \bigtriangleup C = \{ 1, 2, 3, 4 \}, \quad
	B \bigtriangleup C = \{ 2, 4 \},
\end{equation*}
so $(A \setminus B) \bigtriangleup C = \{ 1, 3, 4 \}$ and $(A \bigtriangleup C) \setminus (B \bigtriangleup C) = \{ 1, 3 \}$.
\end{proof}


\begin{statement}{3.5.27}
Consider the following putative theorem.
\begin{theorem}
	For every real number $x$, if $\abs{x - 3} < 3$ then $0 < x < 6$.
\end{theorem}
Is the following proof correct?
If so, what proof strategies does it use?
If not, can it be fixed?
Is the theorem correct?
\begin{proof}
	Let $x$ be an arbitrary real number, and suppose $\abs{x - 3} < 3$.
	We consider two cases:
	
	Case 1. $x - 3 \geq 0$.
	Then $\abs{x - 3} = x - 3$.
	Plugging this into the assumption that $\abs{x - 3} < 3$, we get $x - 3 < 3$,
	so clearly $x < 6$.
	
	Case 2. $x - 3 < 0$.
	Then $\abs{x - 3} = 3 - x$, so the assumption $\abs{x - 3} < 3$ means that $3 - x < 3$.
	Therefore $3 < 3 + x$, so $0 < x$.
	
	Since we have proven both $0 < x$ and $x < 6$, we can conclude that $0 < x < 6$.
\end{proof}
\end{statement}

\begin{proof}
While the theorem is correct, the proof is not.
As written, the proof only shows that if $\abs{x - 3} < 3$ then $x < 6$ \textbf{or} $x > 0$.
This is different from what we need to show, that $x < 6$ \textbf{and} $x > 0$.
We can, however, fix the proof.
In the part for Case 1, we need to add that the assumption $x - 3 \geq 0$ implies that $x \geq 3$, so $x > 0$.
Similarly, in the part for Case 2, we need to add that the assumption $x - 3 < 0$ implies that $x < 3$, so $x < 6$.
These additions make it so in each case we conclude that $0 < x < 6$, which then allows us to draw the ultimate conclusion we desire.
\end{proof}


\begin{statement}{3.5.28}
Consider the following putative theorem.
\begin{theorem}
	For any sets $A$, $B$, and $C$, if $A \setminus B \subseteq C$ and $A \nsubseteq C$
	then $A \cap B \neq \varnothing$.
\end{theorem}
Is the following proof correct?
If so, what proof strategies does it use?
If not, can it be fixed?
Is the theorem correct?
\begin{proof}
	Suppose $A \setminus C \subseteq C$ and $A \nsubseteq C$.
	Since $A \nsubseteq C$, we can choose some $x$ such that $x \in A$ and $x \notin C$.
	Since $x \notin C$ and $A \setminus B \subseteq C$, $x \notin A \setminus B$.
	Therefore either $x \notin A$ or $x \in B$.
	But we already know that $x \in A$, so it follows that $x \in B$.
	Since $x \in A$ and $x \in B$, $x \in A \cap B$.
	Therefore $A \cap B \neq \varnothing$.
\end{proof}
\end{statement}

\begin{proof}
The theorem and the given proof are both correct.
Since the goal is of the form $P \rightarrow Q$, the proof assumes $P$ is true and then works to prove that $Q$ is true.
Existential instantiation is used with the statement $A \nsubseteq C$ to obtain $x$ such that $x \in A$ and $x \notin C$.
Then modus tollens is used with the statements $x \notin C$ and $A \setminus B \subseteq C$ to conclude that $x \notin A \setminus B$.
After we know that $x \notin A \setminus B$, the proof is broken into the two possible cases: $x \notin A$ and $x \in B$.
The first case is eliminated by the earlier fact that $x \in A$, to derive the conclusion $x \in B$.
Then, after proving the two goals $x \in A$ and $x \in B$ separately, the conclusion $x \in A \cap B$, so $A \cap B \neq \varnothing$ is drawn to complete the proof.
\end{proof}


\begin{statement}{3.5.29}
Consider the following putative theorem.
\begin{theorem}
	$\forall x \in \BR \exists y \in \BR (xy^2 \neq y - x)$.
\end{theorem}
Is the following proof correct?
If so, what proof strategies does it use?
If not, can it be fixed?
Is the theorem correct?
\begin{proof}
	Let $x$ be an arbitrary real number.
	
	Case 1. $x = 0$.
	Let $y = 1$.
	Then $xy^2 = 0$ and $y - x = 1 - 0 = 1$, so $xy^2 \neq y - x$.
	
	Case 2. $x \neq 0$.
	Let $y = 0$.
	Then $xy^2 = 0$ and $y - x = - x \neq 0$, so $xy^2 \neq y - x$.
	
	Since these cases are exhaustive, we have shown that $\exists y \in \BR (xy^2 \neq y - x)$.
	Since $x$ was arbitrary, this shows that $\forall x \in \BR \exists y \in \BR (xy^2 \neq y - x)$.
\end{proof}
\end{statement}

\begin{proof}
The theorem and the given proof are both correct.
Since the proof involves proving that a statement holds for all $x \in \BR$, it takes an arbitrary $x \in \BR$ and proves the statement for that $x$.
After taking an arbitrary $x \in \BR$, the proof is broken into an exhaustive set of cases: $x = 0$ and $x \neq 0$.
In each case, the existence of a suitable real number $y$ is demonstrated, which allows the conclusion that such a $y$ exists for this arbitrary $x$.
Finally, since the statement was proved for an arbitrary $x \in \BR$, it holds for all $x \in \BR$.
\end{proof}


\begin{statement}{3.5.30}
Prove that if $\forall x P(x) \rightarrow \exists x Q(x)$ then $\exists x (P(x) \rightarrow Q(x))$.
(\emph{Hint: Remember that $P \rightarrow Q$ is equivalent to $\neg P \vee Q$.})
\end{statement}

\begin{proof}
Suppose the statement $\forall x P(x) \rightarrow \exists x Q(x)$ is true.
Following the hint, this statement is equivalent to $(\neg \forall x P(x)) \vee (\exists x Q(x))$, which is equivalent to $(\exists x \neg P(x)) \vee (\exists x Q(x))$.
There are two cases to consider: $\exists x \neg P(x)$ and $\exists x Q(x)$.

First suppose $\exists x \neg P(x)$.
In this case, let $x_0$ be such that $\neg P(x_0)$ is true, or in other words such that $P(x_0)$ is false.
Then, by the definition of the logical form of a conditional statement, $P(x_0) \rightarrow Q(x_0)$ is (vacuously) true.

Next suppose $\exists x Q(x)$.
In this case, let $x_0$ be such that $Q(x_0)$ is true.
Then, either $P(x_0)$ is true or $P(x_0)$ is false.
In either case, since $Q(x_0)$ is true, we know that $P(x_0) \rightarrow Q(x_0)$ is true as well.

Since in both cases we found $x_0$ such that $P(x_0) \rightarrow Q(x_0)$ is true, this completes the proof.
\end{proof}


\begin{statement}{3.5.31}
Consider the following putative theorem.
\begin{theorem}
	Suppose $A$, $B$, and $C$ are sets and $A \subseteq B \cup C$.
	Then either $A \subseteq B$ or $A \subseteq C$.
\end{theorem}
Is the following proof correct?
If so, what proof strategies does it use?
If not, can it be fixed?
Is the theorem correct?
\begin{proof}
	Let $x$ be an arbitrary element of $A$.
	Since $A \subseteq B \cup C$, it follows that either $x \in B$ or $x \in C$.
	
	Case 1. $x \in B$.
	Since $x$ was an arbitrary element of $A$, it follows that $\forall x \in A (x \in B)$,
	which means that $A \subseteq B$.
	
	Case 2. $x \in C$.
	Similarly, since $x$ was an arbitrary element of $A$, we can conclude that $A \subseteq C$.
	
	Thus, either $A \subseteq B$ or $A \subseteq C$.
\end{proof}
\end{statement}

\begin{proof}
The theorem and the given proof are both incorrect.
The conclusions being drawn in each case from the fact that $x$ was an arbitrary element of $A$ are being drawn too early.
For a counterexample to the putative theorem, consider $A = \{ 1, 2 \}$, $B = \{ 1 \}$, and $C = \{ 2 \}$.
Here, $A \subseteq B \cup C = \{ 1, 2 \}$, but $A \nsubseteq B$ and $A \nsubseteq C$.
\end{proof}


\begin{statement}{3.5.32}
Suppose $A$, $B$, and $C$ are sets and $A \subseteq B \cup C$.
Prove that either $A \subseteq B$ or $A \cap C \neq \varnothing$.
\end{statement}

\begin{proof}
We consider two cases: $A \subseteq B$ and $A \nsubseteq B$.
In the first case, there is nothing to check, so it suffices to check that in the second case, when $A \nsubseteq B$, $A \cap C \neq \varnothing$.
Since $A \nsubseteq B$, there exists $x \in A$ such that $x \notin B$.
Since $x \in A$ and $A \subseteq B \cup C$, we know that $x \in B \cup C$.
Then, since $x \notin B$, it follows that $x \in C$.
In other words, $x \in A \cap C$, so we conclude that $A \cap C \neq \varnothing$.
In both cases, the we have proven that either $A \subseteq B$ or $A \cap C \neq \varnothing$.
This completes the proof.
\end{proof}


\begin{statement}{3.5.33}
Prove $\exists x (P(x) \rightarrow \forall y P(y))$.
(\emph{Note: Assume the universe of discourse is not the empty set.})
\end{statement}

\begin{proof}
Note that the statement we wish to prove is equivalent to $\exists x (\neg P(x) \vee \forall y P(y))$.
For any statement $P(x)$, there are two cases to consider: there exists $x$ in the universe of discourse such that $P(x)$ is false, or no such $x$ exists and $P(x)$ is true for all $x$ in the universe of discourse.
Since in the first case there exists $x$ such that $P(x)$ is false, we can plug in this value of $x$ to conclude that $\neg P(x) \vee \forall y P(y)$ is true.
In the second case, since $P(x)$ is true for all $x$, we can plug in any value of $x$ we wish to once again conclude that $\neg P(x) \vee \forall y P(y)$ is true.
In both cases, we see that there exists $x$ such that $\neg P(x) \vee \forall y P(y)$ is true.
This completes the proof.
\end{proof}


\begin{statement}{3.6.1}
Prove that for every real number $x$ there is a unique real number $y$ such that $x^2y = x - y$.
\end{statement}

\begin{proof}
Let $x$ be an arbitrary real number.
We first check uniqueness.
Suppose $y$ is a real number such that $x^2y = x - y$.
Adding $y$ to both sides gives us $y(x^2 + 1) = x$.
Since $x^2 + 1 \neq 0$, we can then divide both sides by $x^2 + 1$ to obtain $y = x / (x^2 + 1)$.
Thus, any real number $y$ which satisfies the equation $x^2y = x - y$ must be equal to $x / (x^2 + 1)$.
Now we check that $y = x / (x^2 + 1)$ actually satisfies the equation $x^2y = x - y$.
Plugging into the left-hand side, we obtain
\begin{equation*}
	x^2y = x \left( \frac{x}{x^2 + 1} \right) = \frac{x^3}{x^2 + 1}.
\end{equation*}
Plugging into the right-hand side gives us
\begin{equation*}
	x - y = x - \frac{x}{x^2 + 1} = \frac{x^3 + x}{x^2 + 1} - \frac{x}{x^2 + 1} 
	= \frac{x^3}{x^2 + 1}.
\end{equation*}
Thus, $y = x / (x^2 + 1)$ is the unique real number satisfying the equation $x^2y = x - y$ for our arbitrary value of $x$.
Since $x$ was arbitrary, we conclude that for every real number $x$ there is a unique real number $y$ such that $x^2y = x - y$.
\end{proof}


\begin{statement}{3.6.2}
Prove that there is a unique real number $x$ such that for every real number $y$, $xy + x - 4 = 4y$.
\end{statement}

\begin{proof}
We first demonstrate existence.
Let $x = 4$ and let $y$ be an arbitrary real number.
Then $xy + x - 4 = 4y + 4 - 4 = 4y$, as desired.
Since $y$ was arbitrary, we know that $xy + x - 4 = 4y$ for all real numbers $y$.
To check uniqueness, suppose $z$ is a real number such that for every real number $y$, $zy + z  - 4 = 4y$.
In particular, this equality holds for $y = 0$, which gives us $z - 4 = 0$.
In other words, $z = 4$.
This completes the proof.
\end{proof}


\begin{statement}{3.6.3}
Prove that for every real number $x$, if $x \neq 0$ and $x \neq 1$ then there is a unique real number $y$ such that $y/x = y - x$.
\end{statement}

\begin{proof}
Let $x$ be an arbitrary real number and suppose $x \neq 0$ and $x \neq 1$.
We first check uniqueness.
Suppose $y$ is a real number such that $y/x = y - x$.
We can rearrange this equation to get $x = y - y/x = y (1 - 1/x) = y((x - 1)/x)$.
Since $x  \neq 0$ and $x \neq 1$, we can divide both sides by $(x - 1)/x$ to obtain $y = x^2/(x - 1)$.
Thus, any real number $y$ which satisfies the equation $y/x = y - x$ must be equal to $x^2/(x - 1)$.
Now we check that $y = x^2/(x - 1)$ actually satisfies the equation.
Plugging into the left-hand side, we obtain
\begin{equation*}
	\frac{y}{x} = \frac{x^2}{x(x - 1)} = \frac{x}{x - 1}.
\end{equation*}
Plugging into the right-hand side, we obtain
\begin{equation*}
	y - x = \frac{x^2}{x - 1} - \frac{x^2 - x}{x - 1} = \frac{x}{x - 1}.
\end{equation*}
Thus, $y = x^2/(x - 1)$ is the unique real number satisfying the equation $y/x = y - x$ for our arbitrary value of $x$.
Since $x$ was arbitrary, this completes the proof.
\end{proof}


\begin{statement}{3.6.4}
Prove that for every real number $x$, if $x \neq 0$ then there is a unique real number $y$ such that for every real number $z$, $zy = z/x$.
\end{statement}

\begin{proof}
Let $x$ be an arbitrary real number, and suppose $x \neq 0$.
Let $y = 1/x$, which is a real number since $x \neq 0$.
Then, let $z$ be an arbitrary real number.
We have $zy = z(1/x) = z/x$, as desired.
This proves the existence of a real number $y$ such that for every real number $z$, $zy = z/x$.
Now we check uniqueness.
Suppose $y_1$ and $y_2$ are real numbers such that for every real number $z$, $zy_1 = z/x = zy_2$.
In particular, this equality holds for $z = 1$, so it follows that $y_1 = y_2$.
This proves uniqueness.
Lastly, since $x$ was arbitrary, we conclude that for every real number $x$, if $x \neq 0$ then there is a unique real number $y$ such that for every real number $z$, $zy = z/x$.
\end{proof}


\begin{statement}{3.6.5}
Recall that if $\mathcal{F}$ is a family of sets, then $\bigcup \mathcal{F} = \{ x \mid \exists A (A \in \mathcal{F} \wedge x \in A) \}$.
Suppose we define a new set $\bigcup ! \mathcal{F}$ by the formula $\bigcup ! \mathcal{F} = \{ x \mid \exists ! A (A \in \mathcal{F} \wedge x \in A) \}$.
\begin{enumerate}
	\item Prove that for any family of sets $\mathcal{F}$, $\bigcup ! \mathcal{F} \subseteq
	\bigcup \mathcal{F}$.
	
	\item A family of sets $\mathcal{F}$ is said to be \emph{pairwise disjoint} if every pair of distinct
	elements of $\mathcal{F}$ are disjoint; that is $\forall A \in \mathcal{F} \, \forall B \in \mathcal{F}
	(A \neq B \rightarrow A \cap B = \varnothing)$.
	Prove that for any family of sets $\mathcal{F}$, $\bigcup ! \mathcal{F} = \bigcup \mathcal{F}$
	if and only if $\mathcal{F}$ is pairwise disjoint.
\end{enumerate}
\end{statement}

\begin{proof}
\hfill
\begin{enumerate}
	\item Let $\mathcal{F}$ be an arbitrary family of sets.
	Suppose $x \in \bigcup ! \mathcal{F}$ is arbitrary.
	Then there is a unique set $A \in \mathcal{F}$ such that $x \in A$.
	In particular, the set $A$ is a set in $\mathcal{F}$ such that $x \in A$, so it follows that
	$x \in \bigcup \mathcal{F}$.
	Thus, since $x$ was arbitrary, we conclude that 
	$\bigcup ! \mathcal{F} \subseteq \bigcup \mathcal{F}$.
	
	\item Let $\mathcal{F}$ be an arbitrary family of sets.
	We first prove the forward direction.
	Suppose $\bigcup ! \mathcal{F} = \bigcup \mathcal{F}$.
	Now let $A, B \in \mathcal{F}$ be arbitrary and suppose $A \cap B \neq \varnothing$.
	Since $A$ and $B$ are not disjoint, there exists $x \in A \cap B$.
	Then, since $x \in A$, we know that $x \in \bigcup \mathcal{F} = \bigcup ! \mathcal{F}$.
	Hence, $A$ is the unique set in $\mathcal{F}$ such that $x \in A$.
	On the other hand, the same argument with the fact that $x \in B$ shows us that $B$ is the
	unique set in $\mathcal{F}$ such that $x \in B$.
	Therefore, we conclude that $A = B$.
	Hence, since we have showed that for any sets $A, B \in \mathcal{F}$, if $A \cap B \neq \varnothing$
	then $A = B$, we conclude that $\mathcal{F}$ is pairwise disjoint.
	
	Now we prove the reverse direction.
	Suppose $\mathcal{F}$ is pairwise disjoint.
	Since in Part (1) above we showed that $\bigcup ! \mathcal{F} \subseteq \bigcup \mathcal{F}$,
	it suffices to check that $\bigcup \mathcal{F} \subseteq \bigcup ! \mathcal{F}$.
	Suppose $x \in \bigcup \mathcal{F}$ is arbitrary, so there is $A \in \mathcal{F}$ such that $x \in A$.
	We now check that this set $A$ is unique.
	Suppose $B \in \mathcal{F}$ is another set such that $x \in B$.
	Thus, $x \in A \cap B$, so $A \cap B \neq \varnothing$.
	Since $\mathcal{F}$ is pairwise disjoint, we must have $A = B$.
	In other words, $A$ is the unique set in $\mathcal{F}$ such that $x \in A$.
	Thus, $\bigcup \mathcal{F} \subseteq \bigcup ! \mathcal{F}$, and we conclude that the
	two sets are equal.
\end{enumerate}
\end{proof}


\begin{statement}{3.6.6}
Let $U$ be any set.
\begin{enumerate}
	\item Prove that there is a unique $A \in \powerset{U}$ such that for every $B \in \powerset{U}$,
	$A \cup B = B$.
	
	\item Prove that there is a unique $A \in \powerset{U}$ such that for every $B \in \powerset{U}$,
	$A \cup B = A$.
\end{enumerate}
\end{statement}

\begin{proof}
\hfill
\begin{enumerate}
	\item We start by proving existence.
	Let $A = \varnothing$, which is an element of $\powerset{U}$.
	Then, since $B \cup \varnothing = B$ for all sets $B$, so in particular all sets $B \in \powerset{U}$,
	we know that $A = \varnothing$ satisfies the desired property.
	
	Next we check uniqueness.
	Suppose $A_1, A_2 \in \powerset{U}$ are two sets such that for all sets $B \in \powerset{U}$,
	$A_1 \cup B = B = A_2 \cup B$.
	In particular, letting $A = A_1$ and $B = A_2$ gives us $A_1 \cup A_2 = A_2$,
	and letting $A = A_2$ and $B = A_1$ gives us $A_2 \cup A_1 = A_1$.
	In other words, we have $A_1 = A_1 \cup A_2 = A_2$, so $A_1 = A_2$.
	This proves uniqueness.
	
	\item We start by proving existence.
	Let $A = U$, which is an element of $\powerset{U}$.
	Then, since $B \subseteq U$ for all $B \in \powerset{U}$, we know that $B \cup U = U$ for all
	sets $B \in \powerset{U}$.
	Thus, $A = U$ satisfies the desired property.
	
	Next we check uniqueness.
	Suppose $A \in \powerset{U}$ is a set such that for every $B \in \powerset{U}$, $A \cup B = A$.
	Then in particular $A \cup B = A$ for $B = U$, which gives $A = A \cup U = U$.
	This proves uniqueness.
\end{enumerate}
\end{proof}


\begin{statement}{3.6.7}
Let $U$ be any set.
\begin{enumerate}
	\item Prove that there is a unique $A \in \powerset{U}$ such that for every $B \in \powerset{U}$,
	$A \cap B = B$.
	
	\item Prove that there is a unique $A \in \powerset{U}$ such that for every $B \in \powerset{U}$,
	$A \cap B = A$.
\end{enumerate}
\end{statement}

\begin{proof}
\hfill
\begin{enumerate}
	\item We start by proving existence.
	Let $A = U$, which is an element of $\powerset{U}$.
	Then, since $B \subseteq U$ for all $B \in \powerset{U}$, we know that $B \cap U = B$ for all
	sets $B \in \powerset{U}$.
	Thus, $A = U$ satisfies the desired property.
	
	Next we check uniqueness.
	Suppose $A_1, A_2 \in \powerset{U}$ are two sets such that for all sets $B \in \powerset{U}$,
	$A_1 \cap B = B = A_2 \cap B$.
	In particular, letting $A = A_1$ and $B = A_2$ gives us $A_1 \cap A_2 = A_2$,
	and letting $A = A_2$ and $B = A_1$ gives us $A_2 \cap A_1 = A_1$.
	In other words, we have $A_1 = A_1 \cap A_2 = A_2$, so $A_1 = A_2$.
	This proves uniqueness.
	
	\item We start by proving existence.
	Let $A = \varnothing$, which is an element of $\powerset{U}$.
	Then, since $B \cap \varnothing = \varnothing$ for all sets $B$, 
	so in particular all sets $B \in \powerset{U}$, 
	we know that $A = \varnothing$ satisfies the desired property.
	
	Next we check uniqueness.
	Suppose $A \in \powerset{U}$ is a set such that for every $B \in \powerset{U}$, $A \cap B = A$.
	Then in particular $A \cap B = A$ for $B = \varnothing$, 
	which gives $A = A \cap \varnothing = \varnothing$.
	This proves uniqueness.
\end{enumerate}
\end{proof}


\begin{statement}{3.6.8}
Let $U$ be any set.
\begin{enumerate}
	\item Prove that for every $A \in \powerset{U}$ there is a unique $B \in \powerset{U}$ such that
	for every $C \in \powerset{U}$, $C \setminus A = C \cap B$.
	
	\item Prove that for every $A \in \powerset{U}$ there is a unique $B \in \powerset{U}$ such that
	for every $C \in \powerset{U}$, $C \cap A = C \setminus B$.
\end{enumerate}
\end{statement}

\begin{proof}
\hfill
\begin{enumerate}
	\item Let $A \in \powerset{U}$ be arbitrary.
	To check existence, we will show that $B = U \setminus A$ satisfies the desired property.
	Let $C \in \powerset{U}$ be arbitrary.
	Then $C \cap B = C \cap (U \setminus A) = (C \cap U) \setminus A = C \setminus A$, as desired.
	Thus, since $C$ was arbitrary, $C \cap B = C \setminus A$ for all $C \in \powerset{U}$.
	
	To check uniqueness, suppose we have $B' \in \powerset{U}$ such that for every 
	$C \in \powerset{U}$, $C \setminus A = C \cap B'$.
	Then in particular this is the case when $C = U$, giving us $B' = U \cap B' = U \setminus A = B$.
	Since $A$ was arbitrary, this completes the proof.
	
	\item Let $A \in \powerset{U}$ be arbitrary.
	To check existence, we will show that $B = U \setminus A$ satisfies the desired property.
	Let $C \in \powerset{U}$ be arbitrary.
	Then $C \setminus (U \setminus A) = (C \setminus U) \cup (C \cap A) = C \cap A$, where the
	first equality holds by Exercise 1.4.9 and we note that $C \setminus U = \varnothing$ since
	$C \subseteq U$.
	Thus, since $C$ was arbitrary, $C \setminus B = C \cap A$ for all $C \in \powerset{U}$.
	
	To check uniqueness, suppose we have $B' \in \powerset{U}$ such that for every
	$C \in \powerset{U}$, $C \cap A = C \setminus B'$.
	Then in particular this is the case when $C = U \setminus A$.
	In this case, since $\varnothing = (U \setminus A) \cap A = (U \setminus A) \cap B'$,
	we know that $U \setminus A \subseteq B'$.
	In addition, with $C = B'$, we know that $B' \cap A = B' \setminus B' = \varnothing$,
	so $B' \subseteq U \setminus A$ because $B'$ and $A$ are disjoint.
	Hence, $B' = U \setminus A = B$.
	Since $A$ was arbitrary, this completes the proof.
\end{enumerate}
\end{proof}


\begin{statement}{3.6.9}
Recall that you showed in Exercise 1.4.14 that symmetric difference is associative; in other words, for all sets $A$, $B$, and $C$, $A \bigtriangleup (B \bigtriangleup C) = (A \bigtriangleup B) \bigtriangleup C$.
You may also find it useful in this problem to note that symmetric difference is clearly commutative; in other words, for all sets $A$ and $B$, $A \bigtriangleup B = B \bigtriangleup A$.
\begin{enumerate}
	\item Prove that there is a unique identity element for symmetric difference.
	In other words, there is a unique set $X$ such that for every set $A$, $A \bigtriangleup X = A$.
	
	\item Prove that every set has a unique inverse for the operation of symmetric difference.
	In other words, for every set $A$ there is a unique set $B$ such that $A \bigtriangleup B = X$,
	where $X$ is the identity element from Part (1).
	
	\item Prove that for any sets $A$ and $B$ there is a unique set $C$ such that 
	$A \bigtriangleup C = B$.
	
	\item Prove that for every set $A$ there is a unique set $B \subseteq A$ such that for every set
	$C \subseteq A$, $B \bigtriangleup C = A \setminus C$.
\end{enumerate}
\end{statement}

\begin{proof}
\hfill
\begin{enumerate}
	\item We first check uniqueness.
	Suppose we have two identity elements $X$ and $Y$ for symmetric difference.
	Then, we have $Y = X \bigtriangleup Y = X$, where the first equality comes from the fact
	that $X$ is an identity element and the second equality comes from the fact that $Y$
	is an identity element.
	Thus, since $X = Y$, we conclude that if an identity element exists for symmetric difference,
	then it is unique.
	
	Now we check existence by showing $X = \varnothing$ is an identity element.
	Let $A$ be an arbitrary set.
	Then $A \bigtriangleup \varnothing = (A \setminus \varnothing) \cup (\varnothing \setminus A) =
	A \cup \varnothing = \varnothing$, as desired.
	
	\item Let $A$ be an arbitrary set.
	We first check uniqueness.
	Suppose we have two inverses for $A$ with respect to symmetric difference, $X$ and $Y$.
	We then note that $A \bigtriangleup X = A \bigtriangleup Y = \varnothing$, so then
	$X = X \bigtriangleup \varnothing = X \bigtriangleup (A \bigtriangleup Y)
	= (X \bigtriangleup A) \bigtriangleup Y = \varnothing \bigtriangleup Y = Y$.
	Note that in this chain of equalities we are implicitly using the facts that symmetric difference is 
	associative and commutative.
	Thus, since $X = Y$, we conclude that if an inverse for $A$ with respect to symmetric difference
	exists, then it is unique.
	
	Now we check existence by showing that $A$ is its own inverse.
	Indeed, $A \bigtriangleup A = (A \setminus A) \cup (A \setminus A) =
	\varnothing \cup \varnothing = \varnothing$, as desired.
	
	\item Let $A$ and $B$ be arbitrary sets.
	We first check uniqueness.
	Suppose we have two sets $C$ and $D$ such that $A \bigtriangleup C = A \bigtriangleup D = B$.
	Then, since $A \bigtriangleup C = A \bigtriangleup D$, we can take the symmetric difference with
	$A$ on both sides to conclude that $C = D$.
	Thus, since $C = D$, we conclude that if a set $C$ exists such that $A \bigtriangleup C = B$,
	then it is unique.
	
	Now we check existence by showing that $C = A \bigtriangleup B$ works.
	Indeed, $A \bigtriangleup (A \bigtriangleup B) = (A \bigtriangleup A) \bigtriangleup B
	= \varnothing \bigtriangleup B = B$, as desired.
	
	\item Let $A$ be an arbitrary set.
	We check existence by showing $B = A$ satisfies the desired property.
	Let $C \subseteq A$ be arbitrary.
	Then $B \bigtriangleup C = A \bigtriangleup C = (A \setminus C) \cup (C \setminus A) = 
	A \setminus C$, where the last equality follows from the fact that $C \subseteq A$ means
	$C \setminus A = \varnothing$.
	Since $C$ was arbitrary, we conclude that $B = A$ satisfies the desired property.
	
	Next we check uniqueness.
	Suppose we have a set $B \subseteq A$ which satisfies the desired property.
	Then in particular it satisfies the property for $C = \varnothing$, which gives
	$B \bigtriangleup \varnothing = A \setminus \varnothing = A$.
	On the other hand, we also know that $B \bigtriangleup \varnothing = B$, 
	so we conclude that $B = A$.
\end{enumerate}
\end{proof}


\begin{statement}{3.6.10}
Suppose $A$ is a set, and for every family of sets $\mathcal{F}$, if $\bigcup \mathcal{F} = A$ then $A \in \mathcal{F}$.
Prove that $A$ has exactly one element.
\end{statement}

\begin{proof}
We first prove that $A$ contains at least one element.
Suppose, towards a contradiction, that $A$ is empty.
Consider $\mathcal{F} = \varnothing$, an empty family of sets.
We then have $\bigcup \mathcal{F} = \bigcup \varnothing = \varnothing = A$.
Since if $\bigcup \mathcal{F} = A$ then $A \in \mathcal{F}$ for every family of sets $\mathcal{F}$, we conclude that $A \in \mathcal{F}$.
But this contradicts the assumption that $\mathcal{F}$ was empty.
Hence, we conclude that $A$ is nonempty, or in other words, it contains at least one element.

Next we prove that $A$ contains at most one element.
Suppose, towards a contradiction, that $A$ contains more than one element.
In other words, there are $x, y \in A$ such that $x \neq y$.
Now consider $\mathcal{F} = \{ A \setminus \{ x \}, A \setminus \{ y \} \}$.
Since $x \neq y$, we know that $x \in A \setminus \{ y \}$ and $y \in A \setminus \{ x \}$.
It then follows that $\bigcup \mathcal{F} = A$, so the property of $A$ implies that $A \in \mathcal{F}$.
However, since $A \neq A \setminus \{ x \}$ and $A \neq A \setminus \{ y \}$, this is a contradiction.
Thus, we conclude that $A$ contains at most one element.

Since we have shown that $A$ contains at least one element, and also $A$ contains at most one element, we conclude that $A$ contains exactly one element.
\end{proof}


\begin{statement}{3.6.11}
Suppose $\mathcal{F}$ is a family of sets that has the property that for every $\mathcal{G} \subseteq \mathcal{F}$, $\bigcup \mathcal{G} \in \mathcal{F}$.
Prove that there is a unique set $A$ such that $A \in \mathcal{F}$ and $\forall B \in \mathcal{F} \, (B \subseteq A)$.
\end{statement}

\begin{proof}
We first check uniqueness.
Suppose we have two sets $A_1, A_2 \in \mathcal{F}$ such that $\forall B \in \mathcal{F}$ $B \subseteq A_1$ and $B \subseteq A_2$.
Then in particular we know that $A_1 \subseteq A_2$ by using $B = A_1$ and $A_2 \subseteq A_1$ by using $B = A_2$.
In other words, $A_1 = A_2$, which proves uniqueness.

To check existence, we'll show that $A = \bigcup \mathcal{F}$ satisfies the desired property.
First we note that by the property of $\mathcal{F}$ that for every $\mathcal{G} \subseteq \mathcal{F}$, $\bigcup \mathcal{G} \in \mathcal{F}$, we know that $A \in \mathcal{F}$ by setting $\mathcal{G} = \mathcal{F}$.
Now, let $B \in \mathcal{F}$ be arbitrary.
We know that $B \subseteq \bigcup \mathcal{F} = A$ (see Exercise 3.3.8 for a proof), so since $B$ was arbitrary we conclude that $B \subseteq A$ for all $B \in \mathcal{F}$.
Thus, $A = \bigcup \mathcal{F}$ satisfies the desired property.
This completes the proof.
\end{proof}


\begin{statement}{3.6.12}
\begin{enumerate}
	\item Suppose $P(x)$ is a statement with a free variable $x$.
	Find a formula, using logical symbols we have studied, that means
	``there are exactly two values of $x$ for which $P(x)$ is true.''
	
	\item Based on your answer to Part (1), design a proof strategy for proving a statement of the form
	``there are exactly two values of $x$ for which $P(x)$ is true.''
	
	\item Prove that there are exactly two solutions to the equation $x^3 = x^2$.
\end{enumerate}
\end{statement}

\begin{proof}
\hfill
\begin{enumerate}
	\item A way to rephrase the statement 
	``there are exactly two values of $x$ for which $P(x)$ is true'' is
	``there are $x$ and $y$ such that $x \neq y$, $P(x)$ is true, $P(y)$ is true, and for all
	$z$ such that $P(z)$ is true, either $z = x$ or $z = y$.''
	Writing this out completely in logical symbols gives us
	\begin{equation*}
		(\exists x \, P(x)) \wedge (\exists y \, P(y)) \wedge (x \neq y) \wedge
		\forall z \, (P(z) \rightarrow (z = x \vee z = y)).
	\end{equation*}
	
	\item From our answer to Part (1), we can use the following proof strategy to prove
	a statement of the form
	``there are exactly two values of $x$ for which $P(x)$ is true.''
	\begin{enumerate}
		\item Demonstrate the existence of $x$ and $y$ such that $P(x)$ and $P(y)$ are true.
		
		\item Show that $x \neq y$.
		
		\item Show that for any $z$ such that $P(z)$ is true, then $z = x$ or $z =y$.
	\end{enumerate}
	
	\item We first show that $x = 0$ and $x = 1$ satisfy the equation $x^3 = x^2$.
	Plugging in $x = 0$ gives us $x^3 = 0$ and $x^2 = 0$, so $x^3 = x^2$ in this case, as desired.
	Plugging in $x = 1$ gives us $x^3 = 1$ and $x^2 = 1$, so $x^3 = x^2$ in this case, as well.
	Showing rigorously that $0 \neq 1$ is a little bit outside the scope of this book, but it can be proven
	from the Peano axioms which are used to define the natural number from a rigorous
	set-theoretic perspective.
	For our purposes, we will brush the proof under the rug and say that clearly $0 \neq 1$.
	Now suppose $z$ is a real number satisfying the equation $x^3 = x^2$.
	Plugging in $z$ and rearranging the equation gives us $0 = z^3 - z^2 = z^2(z - 1)$.
	We implicitly use the fact that the real numbers form a field to conclude that either $z^2 = 0$,
	giving us $z = 0$, or $z - 1 = 0$, giving us $z = 1$.
	In either case, we see that $z$ must be equal to one of the two solutions which we have already
	shown to exist.
	Thus, we conclude that the equation $x^3 = x^2$ has exactly two solutions, $x = 0$ and $x = 1$.
\end{enumerate}
\end{proof}


\begin{statement}{3.6.13}
\begin{enumerate}
	\item Prove that there is a unique real number $c$ such that there is a unique real number $x$
	such that $x^2 + 3x + c = 0$.
	In other words, prove that there is a unique real number $c$ such that the equation
	$x^2 + 3x + c = 0$ has exactly one solution.
	
	\item Show that it is \emph{not} the case that there is a unique real number $x$ such that there is
	a unique real number $c$ such that $x^2 + 3x + c = 0$.
	(\emph{Hint: You should be able to prove that for \emph{every} real number $x$ there is a unique
	real number $c$ such that $x^2 + 3x + c = 0$.})
\end{enumerate}
\end{statement}

\begin{proof}
\hfill
\begin{enumerate}
	\item We first show that $c = 9/4$ works.
	In other words, we show that $x^2 + 3x + 9/4 = 0$ has a unique solution.
	We'll show that this unique solution is $x = -3/2$.
	Plugging in $x = 3/2$ on the left-hand-side, we have
	\begin{equation*}
		\left( -\frac{3}{2} \right)^2 + 3 \cdot -\frac{3}{2} + \frac{9}{4}
		= \frac{9}{4} - \frac{9}{2} + \frac{9}{4}
		= 0,
	\end{equation*}
	which demonstrates existence.
	To check uniqueness, we apply the quadratic formula to see that any solution $y$
	to the equation $x^2 + 3x + 9/4 = 0$ must be
	\begin{equation*}
		y = \frac{-3 \pm \sqrt{3^2 - 4(1)(9/4)}}{2}
		= \frac{-3 \pm 0}{2} = -\frac{3}{2}.
	\end{equation*}
	Thus, we have found that $c = 9/4$ is a real number such that the equation $x^2 + 3x + c = 0$
	has a unique solution.
	
	Now we show that this value of $c$ is unique.
	Suppose $d$ is a real number such that that the equation $x^2 + 3x + d = 0$ has exactly
	one solution.
	Since $x^2 + 3x + d = 0$ has exactly one solution if and only if the discriminant
	$3^2 - 4(1)(d) = 0$, we conclude that we must have $d = 9/4 = c$.
	This proves uniqueness.
	
	\item We will follow the hint and show that for every real number $x$ there is a unique
	real number $c$ such that $x^2 + 3x + c = 0$.
	This is sufficient because it will show that such an $x$ satisfying the desired property is \emph{not}
	unique.
	Let $x$ be an arbitrary real number and take $c = -x^2 - 3x$.
	It immediately follows that $x^2 + 3x + c = x^2 + 3x + (-x^2 - 3x) = 0$, so $c = x^2 - 3x$
	satisfies the desired property.
	To check uniqueness, suppose $d$ is a real number such that $x^2 + 3x + d = 0$.
	Isolating $d$ on one side gives us $d = -x^2 - 3x = c$, which is what we wanted to show.
	Thus, since $x$ was arbitrary, we conclude that for every real number $x$ there is a unique
	real number $c$ such that $x^2 + 3x + c = 0$.
\end{enumerate}
\end{proof}


\begin{statement}{3.7.1}
Suppose $\mathcal{F}$ is a family of sets.
Prove that there is a unique set $A$ that has the following two properties:
\begin{enumerate}
	\item $\mathcal{F} \subseteq \powerset{A}$.
	
	\item $\forall B \, (\mathcal{F} \subseteq \powerset{B} \rightarrow A \subseteq B)$.
\end{enumerate}
\end{statement}

\begin{proof}
We first check uniqueness.
Suppose $A_1$ and $A_2$ are two sets that have the two desired properties.
Since $A_1$ satisfies the first property, $\mathcal{F} \subseteq \powerset{A_1}$, and $A_2$ satisfies the second property, $\forall B \, (\mathcal{F} \subseteq \powerset{B} \rightarrow A_2 \subseteq B)$, we know that $A_2 \subseteq A_1$.
Reversing the roles of $A_1$ and $A_2$ in the previous statement also shows that $A_1 \subseteq A_2$, so we conclude that $A_1 = A_2$.
In other words, if a set exists that satisfies the two desired properties, then it is unique.

Next we demonstrate existence by showing that $A = \bigcup \mathcal{F}$ satisfies the two desired properties.
First, suppose $X \in \mathcal{F}$ is arbitrary.
Then, $X \subseteq \bigcup \mathcal{F} = A$, so $X \in \powerset{A}$.
Hence, $\mathcal{F} \subseteq A$, which shows that $A$ satisfies the first property.
Next, suppose $B$ is an arbitrary set such that $\mathcal{F} \subseteq \powerset{B}$.
Then, for arbitrary $x \in A = \bigcup \mathcal{F}$ there is a set $X \in \mathcal{F}$ such that $x \in X$.
Since $X \in \mathcal{F}$ and $\mathcal{F} \subseteq \powerset{B}$, we know that $X \subseteq B$.
Thus, it follows that $x \in B$, so we conclude that $A \subseteq B$ since $x$ was arbitrary.
This shows that $A$ satisfies the second property as well.
This completes the proof of existence.
\end{proof}


\begin{statement}{3.7.2}
Prove that there is a positive real number $m$ that has the following two properties:
\begin{enumerate}
	\item For every positive real number $x$, $\frac{x}{x + 1} < m$.
	
	\item If $y$ is any positive real number with the property that for every positive real number $x$,
	$\frac{x}{x + 1} < y$, then $m \leq y$.
\end{enumerate}
\end{statement}

\begin{proof}
We first check uniqueness.
Suppose $m$ and $n$ are two positive real numbers that have the desired two properties.
Since $m$ satisfies the first property, that for every real number $x$, $\frac{x}{x + 1} < m$, and since $n$ satisfies the second property, that if $y$ is a positive real number with the property that for every positive real number $x$, $\frac{x}{x + 1} < y$ then $n \leq y$, we know that $n \leq m$.
Reversing the roles of $m$ and $n$ in the previous statement also shows that $m \leq n$, so we conclude that $m = n$.
In other words, if a positive real number exists that satisfies the two desired properties, then it is unique.

Next we demonstrate existence by showing that $m = 1$ satisfies the two desired properties.
First, let $x$ be an arbitrary positive real number.
Then, since $x < x + 1$, dividing both sides by $x + 1$ gives us $\frac{x}{x + 1} < 1 = m$.
Since $x$ was arbitrary, this inequality holds for all positive real numbers $x$.
Hence, $m = 1$ satisfies the first property.

Next, we show that $m = 1$ satisfies the second property by proving that if $y$ is a positive real number such that $m > y$, then there is a positive real number $x$ such that $\frac{x}{x + 1} \geq y$.
Suppose $y < m = 1$.
Since $0 < y < 1$, there exists a rational number $r$ such that $y \leq r < 1$.
(Note that I'm implicitly using the fact that the rational numbers are dense in the real numbers here. 
See Theorem 1.20 in Walter Rudin's \emph{Principles of Mathematical Analysis} or a similar real analysis book for a proof of this fact.)
Since $r$ is a rational number such that $0 < r < 1$, we can write $r = p/q$ for positive integers $p$ and $q$ with $p < q$.
Then, since $p < q$ and $p$ and $q$ are integers, we know that $p \leq q - 1$.
Taking $x = q - 1$ gives us
\begin{equation*}
	\frac{x}{x + 1} = \frac{q - 1}{q} \geq \frac{p}{q} = r \geq y.
\end{equation*}
In other words, we have found a positive real number $x$ such that $\frac{x}{x + 1} \geq y$.
Hence, we conclude that $m = 1$ satisfies the second property.
This completes the proof.
\end{proof}


\begin{statement}{3.7.3}
Suppose $A$ and $B$ are sets.
What can you prove about $\powerset{A \setminus B} \setminus (\powerset{A} \setminus \powerset{B})$?
\end{statement}

\begin{proof}
We will show that $\powerset{A \setminus B} \setminus (\powerset{A} \setminus \powerset{B}) = \{ \varnothing \}$.
First, since the $\varnothing \in \powerset{X}$ for any set $X$, it is clear that $\varnothing \in \powerset{A \setminus B}$ and $\varnothing \notin \powerset{A} \setminus \powerset{B}$,
so thus $\varnothing \in \powerset{A \setminus B} \setminus (\powerset{A} \setminus \powerset{B})$.
Next, let $X \in \powerset{A \setminus B} \setminus (\powerset{A} \setminus \powerset{B})$ be arbitrary.
Then $X \in \powerset{A \setminus B}$, and $X \notin \powerset{A} \setminus \powerset{B}$.
Since $X \notin \powerset{A} \setminus \powerset{B}$, either $X \notin \powerset{A}$ or $X \in \powerset{B}$.
In the case when $X \notin \powerset{A}$, there is $x \in X$ such that $x \notin A$.
But then since $X \in \powerset{A \setminus B}$, for every $x \in X$, $x \in A$ and $x \notin B$, which is a contradiction for the $x \in X$ such that $x \notin A$.
Next we consider the case when $X \in \powerset{B}$, so for every $x \in X$, $x \in B$.
But once again, since $X \in \powerset{A \setminus B}$, for every $x \in X$, $x \in A$ and $x \notin B$, we have a contradiction.
Thus, in both cases we conclude that $X$ must be empty.
In other words, since $X$ was arbitrary, this shows that $\powerset{A \setminus B} \setminus (\powerset{A} \setminus \powerset{B}) = \{ \varnothing \}$.
\end{proof}


\begin{statement}{3.7.4}
Suppose $A$, $B$, and $C$ are sets.
Prove that the following statements are equivalent:
\begin{enumerate}
	\item $(A \bigtriangleup C) \cap (B \bigtriangleup C) = \varnothing$.
	
	\item $A \cap B \subseteq C \subseteq A \cup B$.
	
	\item $A \bigtriangleup C \subseteq A \bigtriangleup B$.
\end{enumerate}
\end{statement}

\begin{proof}
We'll prove that the three statements by showing that (1) implies (2), (2) implies (3), and (3) implies (1).

First, suppose $(A \bigtriangleup C) \cap (B \bigtriangleup C) = \varnothing$.
We start by showing that $A \cap B \subseteq C$.
Let $x \in A \cap B$ be arbitrary, so $x \in A$ and $x \in B$.
Since $(A \bigtriangleup C) \cap (B \bigtriangleup C)$ is empty, we know that $x \notin A \bigtriangleup C$ or $x \notin B \bigtriangleup C$.
In the first case, if $x \notin A \bigtriangleup C = (A \setminus C) \cup (C \setminus A)$, then $x \notin A \setminus C$ and $x \notin C \setminus A$.
Then, since $x \in A$, it follows from the fact that $x \notin A \setminus C$ that $x \in C$.
Similarly, in the second case, if $x \notin B \bigtriangleup C = (B \setminus C) \cup (C \setminus B)$, the fact that $x \in B$ means that since $x \notin B \setminus C$, $x \in C$.
Thus, in both cases $x \in C$, so we conclude that $A \cap B \subseteq C$.

Next we show that $C \subseteq A \cup B$.
Let $x \in C$ be arbitrary.
Once again we use the fact that $x \notin A \bigtriangleup C$ or $x \notin B \bigtriangleup C$.
In the first case, if $x \notin A \bigtriangleup C$, the fact that $x \in C$ and $x \notin C \setminus A$ tells us that $x \in A$.
In the second case, if $x \notin B \bigtriangleup C$, the fact that $x \in C$ and $x \notin C \setminus B$ tells us that $x \in B$.
Thus, $x \in A$ or $x \in B$, or in other words, $x \in A \cup B$, so we conclude that $C \subseteq A \cup B$.
This completes the proof that (1) implies (2).

Now suppose $A \cap B \subseteq C \subseteq A \cup B$.
Let $x \in A \bigtriangleup C = (A \cup C) \setminus (A \cap C)$ be arbitrary, so $x \in A \cup C$ and $x \notin A \cap C$.
Since $x \in A \cup C$, $x \in A$ or $x \in C$.
In the first case, since $x \in A$, clearly $x \in A \cup B$.
In the second case, $x \in C$, since $C \subseteq A \cup B$, we once again have $x \in A \cup B$.
Thus, $x \in A \cup B$.
Then, since $x \notin A \cap C$, $x \notin A$ or $x \notin C$.
In the first case, since $x \notin A$, then $x \notin A \cap B$.
In the second case, $x \notin C$, it follows that $x \notin A \cap B$ since $A \cap B \subseteq C$.
Thus, $x \notin A \cap B$.
Since $x \in A \cup B$ and $x \notin A \cap B$, we conclude that $x \in A \bigtriangleup B$.
Hence, $A \bigtriangleup C \subseteq A \bigtriangleup B$.
This completes the proof that (2) implies (3).

Lastly, suppose $A \bigtriangleup C \subseteq A \bigtriangleup B$.
Suppose $x \in (A \bigtriangleup C) \cap (B \bigtriangleup C)$.
In particular, since $x \in A \bigtriangleup C \subseteq A \bigtriangleup B$, we know that $x \in A \bigtriangleup B$, so $x \in A \setminus B$ or $x \in B \setminus A$.
In the first case, since $x \in A$ and $x \in A \bigtriangleup C$, we know that $x \in A \setminus C$.
But then $x \notin B$ and $x \notin C$ means $x \notin B \bigtriangleup C$, which contradicts the assumption that $x \in (A \bigtriangleup C) \cap (B \bigtriangleup C)$.
In the second case, $x \in B \setminus A$, we get a similar contradiction.
Since $x \notin A$ and $x \in A \bigtriangleup C$, we know that $x \in C \setminus A$.
But then $x \in B$ and $x \in C$ once again means $x \notin B \bigtriangleup C$.
Since in both cases we reach a contradiction, there is no $x \in (A \bigtriangleup C) \cap (B \bigtriangleup C)$.
In other words, $(A \bigtriangleup C) \cap (B \bigtriangleup C) = \varnothing$.
This completes the proof that (3) implies (1).
\end{proof}


\begin{statement}{3.7.5}
Suppose $\{ A_i \mid i \in I \}$ is a family of sets.
Prove that if $\powerset{\bigcup_{i \in I} A_i} \subseteq \bigcup_{i \in I} \powerset{A_i}$, then there is some $i \in I$ such that $\forall j \in I \, (A_j \subseteq A_i)$.
\end{statement}

\begin{proof}
Suppose $\powerset{\bigcup_{i \in I} A_i} \subseteq \bigcup_{i \in I} \powerset{A_i}$.
This means that if $X \in \powerset{\bigcup_{i \in I} A_i}$ then there exists $i \in I$ such that $X \in \powerset{A_i}$, or in other words if $X \subseteq \bigcup_{i \in I} A_i$ then there exists $i \in I$ such that $X \subseteq A_i$.
In particular, this is true for $X = \bigcup_{i \in I} A_i$.
This means there is $i \in I$ such that $\bigcup_{i \in I} A_i \subseteq A_i$.
Now let $j \in I$ be arbitrary.
Since $A_j \subseteq \bigcup_{i \in I} A_i$ and $\bigcup_{i \in I} A_i \subseteq A_i$, we conclude that $A_j \subseteq A_i$.
Since $j$ was arbitrary, $A_j \subseteq A_i$ for all $j \in I$.
This completes the proof.
\end{proof}


\begin{statement}{3.7.6}
% Fill in
\end{statement}

\begin{proof}
% Fill in
\end{proof}


\begin{statement}{3.7.7}
% Fill in
\end{statement}

\begin{proof}
% Fill in
\end{proof}


\begin{statement}{3.7.8}
% Fill in
\end{statement}

\begin{proof}
% Fill in
\end{proof}


\begin{statement}{3.7.9}
Prove that if $\lim_{x \to c} f(x) = L$, then $\lim_{x \to c} 7f(x) = 7L$.
\end{statement}

\begin{proof}
Suppose $\lim_{x \to c} f(x) = L$.
Let $\varepsilon >0$ be arbitrary.
Since $\lim_{x \to c} f(x) = L$, there exists $\delta > 0$ such that for all real numbers $x$, if $0 < \abs{x - c} < \delta$, then $\abs{f(x) - L} < \varepsilon/7$.
Then, for this value of $\delta$, it follows that for all real numbers $x$ such that $0 < \abs{x - c} < \delta$, $\abs{7f(x) - 7L} = 7\abs{f(x) - L} < 7(\varepsilon/7) = \varepsilon$.
In other words, $\lim_{x \to c} 7f(x) = 7L$, which is what we wanted to prove.
\end{proof}


\begin{statement}{3.7.10}
Consider the following putative theorem.
\begin{theorem}
	There are irrational numbers $a$ and $b$ such that $a^b$ is rational.
\end{theorem}
Is the following proof correct?
If so, what proof strategies does it use?
If not, can it be fixed?
Is the theorem correct?
(Note: The proof uses the fact that $\sqrt{2}$ is irrational, which we'll prove in Chapter 6.)
\begin{proof}
	Either $\sqrt{2}^{\sqrt{2}}$ is rational or it's irrational.
	
	Case 1. $\sqrt{2}^{\sqrt{2}}$ is rational.
	Let $a = b = \sqrt{2}$.
	Then $a$ and $b$ are irrational, and $a^b = \sqrt{2}^{\sqrt{2}}$,
	which we are assuming in this case is rational.
	
	Case 2. $\sqrt{2}^{\sqrt{2}}$ is irrational.
	Let $a = \sqrt{2}^{\sqrt{2}}$ and $b = \sqrt{2}$.
	Then $a$ is irrational by assumption, and we know that $b$ is also irrational.
	Also
	\begin{equation*}
		a^b = \left( \sqrt{2}^{\sqrt{2}} \right)^{\sqrt{2}}
		= \left( \sqrt{2} \right)^{\sqrt{2} \cdot \sqrt{2}}
		= \left( \sqrt{2} \right)^2
		= 2,
	\end{equation*}
	which is rational.
\end{proof}
\end{statement}

\begin{proof}
The theorem and the proof are both correct.
The proof starts by considering two cases: either $\sqrt{2}^{\sqrt{2}}$ is rational or it's irrational.
In both cases, the proof demonstrates the existence of irrational numbers $a$ and $b$ such that $a^b$ is rational.
Since the two cases are exhaustive and desired values of $a$ and $b$ were found in each case, that completes the proof.
\end{proof}
\end{document}