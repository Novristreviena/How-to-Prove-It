\documentclass[12pt]{amsart}

%Below are some necessary packages for your course.
\usepackage{amsfonts,latexsym,amsthm,amssymb,amsmath,amscd,euscript,mathrsfs}
\usepackage{framed}
\usepackage{fullpage}
\usepackage{hyperref}
    \hypersetup{colorlinks=true,citecolor=blue,urlcolor =blue,linkbordercolor={1 0 0}}
\usepackage{mathtools}
\usepackage[table]{xcolor}

\newenvironment{statement}[1]{\smallskip\noindent\color[rgb]{.6627, .3529, .6314} {\bf #1.}}{}
\allowdisplaybreaks[1]

%Below are the theorem, definition, example, lemma, etc. body types.

\newtheorem{theorem}{Theorem}
\newtheorem*{proposition}{Proposition}
\newtheorem{lemma}[theorem]{Lemma}
\newtheorem{corollary}[theorem]{Corollary}
\newtheorem{conjecture}[theorem]{Conjecture}
\newtheorem{postulate}[theorem]{Postulate}
\theoremstyle{definition}
\newtheorem{defn}[theorem]{Definition}
\newtheorem{example}[theorem]{Example}

\theoremstyle{remark}
\newtheorem*{remark}{Remark}
\newtheorem*{notation}{Notation}
\newtheorem*{note}{Note}

% You can define new commands to make your life easier.
\newcommand{\BR}{\mathbb R}
\newcommand{\BC}{\mathbb C}
\newcommand{\BF}{\mathbb F}
\newcommand{\BQ}{\mathbb Q}
\newcommand{\BZ}{\mathbb Z}
\newcommand{\BN}{\mathbb N}

% We can even define a new command for \newcommand!
\newcommand{\nc}{\newcommand}

% If you want a new function, use operatorname to define that function (don't use \text)
\nc{\on}{\operatorname}
\nc{\Spec}{\on{Spec}}

\title{\emph{How to Prove It}: Chapter 2} % IMPORTANT: Change the problemset number as needed.
\date{\today}

\begin{document}

\maketitle

\vspace*{-0.25in}
\centerline{Kyle Stratton}

\begin{framed}
These are the exercises for Chapter 2 from the third edition of \emph{How to Prove It} by Daniel J. Velleman.
They are numbered (Chapter).(Section).(Exercise).
\end{framed}

\begin{statement}{2.1.1}
Analyze the logical forms of the following statements.
\begin{enumerate}
	\item Anyone who has forgiven at least one person is a saint.
	
	\item Nobody in the calculus class is smarter than everybody in the discrete math class.
	
	\item Everyone likes Mary, except Mary herself.
	
	\item Jane saw a police officer, and Roger saw one too.
	
	\item Jane saw a police officer, and Roger saw him too.
\end{enumerate}
\end{statement}

\begin{proof}
\hfill
\begin{enumerate}
	\item We can rephrase the statement as ``For all people $x$, if there exists a person $y$ who $x$ has forgiven, then $x$ is a saint.''
	Letting $S(x)$ stand for ``$x$ is a saint,'' and $F(x, y)$ for ``$x$ has forgiven $y$,'' we see that the statement has the following logical form:
	$\forall x (\exists y F(x, y) \rightarrow S(x))$.
	
	\item We can rephrase the statement as ``There does not exist a person $x$ who is in the calculus class and for all people $y$, if $y$ is in the discrete math class, then $x$ is smarter than $y$.''
	Letting $C(x)$ stand for ``$x$ is in the calculus class,'' $D(x)$ for ``$x$ is in the discrete math class,'' and $S(x, y)$ for ``$x$ is smarter than $y$,'' we see that the statement has the following logical form:
	$\neg \exists x [C(x) \wedge \forall y (D(y) \rightarrow S(x, y))]$.
	
	\item \emph{Note that I disagree with the interpretation of the statement that Velleman provides.}
	We can rephrase the statement as ``For all people $x$, if $x$ is not Mary, then $x$ likes Mary. In addition, Mary does not like herself.''
	Letting $L(x, y)$ stand for ``$x$ likes $y$,'' and $m$ for Mary, we see that the statement has the following logical form:
	$\forall x ((x \neq m) \rightarrow L(x, m)) \wedge \neg L(m, m)$.
	
	\item We can rephrase the statement as ``There is a person $x$ whom Jane saw who is also a police officer, and there is a person $y$, potentially distinct from $x$, whom Roger saw who is also a police officer.''
	Letting $S(x, y)$ stand for ``$x$ saw $y$,'' $P(x)$ for ``$x$ is a police officer'', $j$ for Jane, and $r$ for Roger, we see that the statement has the following logical form:
	$\exists x (S(j, x) \wedge P(x)) \wedge \exists y (S(r, y) \wedge P(y))$.
	
	\item We can rephrase the statement as ``There is a person $x$ whom both Jane and Roger saw who is also a police officer.''
	Letting $S(x, y)$, $P(x)$, $j$, and $r$ be as in Part (4), we see that the statement has the following logical form:
	$\exists x (S(j, x) \wedge S(r, x) \wedge P(x))$.
\end{enumerate}
\end{proof}


\begin{statement}{2.1.2}
Analyze the logical forms of the following statements.
\begin{enumerate}
	\item Anyone who has bought a Rolls Royce with cash must have a rich uncle.
	
	\item If anyone in the dorm has measles, then everyone who has a friend in the dorm will have to be quarantined.
	
	\item If nobody failed the test, then everybody who got an A will tutor someone who got a D.
	
	\item If anyone can do it, Jones can.
	
	\item If Jones can do it, anyone can.
\end{enumerate}
\end{statement}

\begin{proof}
\hfill
\begin{enumerate}
	\item We can rephrase the statement as ``For all people $x$, if $x$ bought a Rolls Royce with cash, then there exists a person $y$ who is rich and is $x$'s uncle.''
	Letting $C(x)$ stand for ``$x$ bought a Rolls Royce with cash,'' $R(x)$ for ``$x$ is rich,'' and $U(x, y)$ for ``$x$ is $y$'s uncle,'' we see that the statement has the following logical form:
	$\forall x [C(x) \rightarrow \exists y (R(y) \wedge U(y, x))]$.
	
	\item We can rephrase the statement as ``If there exists a person $x$ who lives in the dorm and has measles, then for all people $y$, if there exists a person $z$ who is a friend of $y$ and lives in the dorm, $y$ has to be quarantined.''
	Letting $D(x)$ stand for ``$x$ lives in the dorm,'' $M(x)$ stand for ``$x$ has measles,'' $F(x, y)$ for ``$x$ is a friend of $y$'', and $Q(x)$ for ``$x$ has to be quarantined, we see that the statement has the following logical form:
	$\exists x (D(x) \wedge M(x)) \rightarrow \forall y (\exists z (F(y, z) \wedge D(z)) \rightarrow Q(y))$.
	
	\item We can rephrase the statement as ``If there does not exist a person $x$ who failed the test, then for all people $y$, if $y$ got an A and there exists a person $z$ who got a D, then $y$ will tutor $z$.''
	Letting $F(x)$ stand for ``$x$ failed the test,'' $A(x)$ for ``$x$ got an A,'', $D(x)$ for ``$x$ got a D,'' and $T(x, y)$ for ``$x$ tutors $y$,'' we see that the statement has the following logical form:
	$\neg \exists x F(x) \rightarrow \forall y ((A(y) \wedge \exists z D(z)) \rightarrow T(y, z))$.
	
	\item We can rephrase the statement as ``If there exists a person $x$ who can do it, then Jones can do it.''
	Letting $C(x)$ stand for ``$x$ can do it,'' and $j$ for Jones, we see that the statement has the following logical form:
	$\exists x C(x) \rightarrow C(j)$.
	
	\item We can rephrase the statement as ``If Jones can do it, then for all people $x$, $x$ can do it.''
	Letting $C(x)$ and $j$ be as in Part (4), we see that the statement has the following logical form:
	$C(j) \rightarrow \forall x C(x)$.
\end{enumerate}
\end{proof}


\begin{statement}{2.1.3}
Analyze the logical forms of the following statements.
The universe of discourse is $\BR$.
What are the free variables in each statement?
\begin{enumerate}
	\item Every number that is larger than $x$ is larger than $y$.
	
	\item For every number $a$, the equation $ax^2 + 4x - 2 = 0$ has at least one solution if and only if $a \geq -2$.
	
	\item All solutions of the inequality $x^3 - 3x < 3$ are smaller than 10.
	
	\item If there is a number $x$ such that $x^2 + 5x = w$ and there is a number $y$ such that $4 - y^2 = w$, then $w$ is strictly between $-10$ and $10$.
\end{enumerate}
\end{statement}

\begin{proof}
\hfill
\begin{enumerate}
	\item We can rephrase the statement as ``For all $z \in \BR$, if $z > x$ then $z > y$.''
	Thus, the statement has the following logical form:
	$\forall z \in \BR (z > x \rightarrow z > y)$.
	In this statement the free variables are $x$ and $y$.
	
	\item We can rephrase the statement as ``For all $a \in \BR$, there exists a solution $x \in \BR$ to the equation $ax^2 + 4x - 2 = 0$ if and only if $a \geq -2$.
	Thus, the statement has the following logical form:
	$\forall a \in \BR (\exists x \in \BR (ax^2 + 4x - 2 = 0) \leftrightarrow a \geq -2)$.
	In this statement there are no free variables.
	
	\item We can rephrase the statement as ``For all $x \in \BR$, if $x$ is a solution to the inequality $x^3 - 3x < 3$, then $x < 10$.
	Thus, the statement has the following logical form:
	$\forall x \in \BR (x^3 - 3x < 3 \rightarrow x < 10)$.
	In this statement there are no free variables.
	
	\item We can rephrase the statement as ``If there exists $x \in \BR$ such that $x^2 + 5x = w$ and there exists $y \in \BR$ such that $4 - y^2 = w$, then $-10 < w < 10$.''
	Thus, the statement has the following logical form:
	$[\exists x \in \BR (x^2 + 5x = w)] \wedge [\exists y \in \BR (4 - y^2 = w)] \rightarrow -10 < w < 10$.
	In this statement $w$ is a free variable.
\end{enumerate}
\end{proof}


\begin{statement}{2.1.4}
Translate the following statements into idiomatic English.
\begin{enumerate}
	\item $\forall x [(H(x) \wedge \neg \exists y M(x, y)) \rightarrow U(x)]$,
	where $H(x)$ means ``$x$ is a man,'' $M(x, y)$ means ``$x$ is married to $y$,''
	and $U(x)$ means ``$x$ is unhappy.''
	
	\item $\exists z (P(z, x) \wedge S(z, y) \wedge W(y))$,
	where $P(z, x)$ means ``$z$ is a parent of $x$,'' $S(z, y)$ means ``$z$ and $y$ are siblings,''
	and $W(y)$ means ``$y$ is a woman.''
\end{enumerate}
\end{statement}

\begin{proof}
\hfill
\begin{enumerate}
	\item An initial direct translation of this statement is 
	``For all people $x$, if $x$ is male and there does not exist a person $y$ to whom $x$ is married, then $x$ is unhappy.''
	More idiomatically, this statement translates to ``All unmarried men are unhappy.''
	
	\item An initial direct translation of this statement is 
	``There exists a person $z$ who is a parent of $x$ and is siblings with the woman $y$.''
	More idiomatically, this statement translates to ``$y$ is a sister to one of $x$'s parents.''
\end{enumerate}
\end{proof}


\begin{statement}{2.1.5}
Translate the following statements into idiomatic mathematical English.
\begin{enumerate}
	\item $\forall x [(P(x) \wedge \neg (x = 2)) \rightarrow O(x)]$,
	where $P(x)$ means ``$x$ is a prime number'' and $O(x)$ means ``$x$ is odd.''
	
	\item $\exists x [P(x) \wedge \forall y (P(y) \rightarrow y \leq x)]$,
	where $P(x)$ means ``$x$ is a perfect number.''
\end{enumerate}
\end{statement}

\begin{proof}
\hfill
\begin{enumerate}
	\item An initial direct translation of this statement is 
	``For all $x$, if $x$ is a prime number not equal to 2, then $x$ is odd.''
	More idiomatically, we can rephrase this statement as ``All prime numbers aside from 2 are odd.''
	
	\item An initial direct translation of this statement is 
	``There exists $x$ such that $x$ is a perfect number and for all numbers $y$, if $y$ is a perfect number then $y$ is less than or equal to $x$.''
	More idiomatically, we can rephrase this statement as ``There exists a largest perfect number.''
\end{enumerate}
\end{proof}


\begin{statement}{2.1.6}
Translate the following statements into idiomatic mathematical English.
Are they true or false?
The universe of discourse is $\BR$.
\begin{enumerate}
	\item $\neg \exists x (x^2 + 2x + 3 = 0 \wedge x^2 + 2x - 3 = 0)$
	
	\item $\neg [\exists x (x^2 + 2x + 3 = 0) \wedge \exists x (x^2 + 2x - 3 = 0)]$
	
	\item $\neg \exists x (x^2 + 2x + 3 = 0) \wedge \neg \exists x (x^2 + 2x - 3 = 0)$
\end{enumerate}
\end{statement}

\begin{proof}
Since this exercise involves the equations $x^2 + 2x + 3 = 0$ and $x^2 + 2x - 3 = 0$, we first solve them over $\BC$ using the quadratic formula.
The solutions to $x^2 + 2x + 3 = 0$ are $x = -1 \pm \sqrt{2}i$,
while the solutions to $x^2 + 2x - 3 = 0$ are $x = -3$ and $x = 1$.
\begin{enumerate}
	\item This statement translates to ``There does not exist $x \in \BR$ which is a root of both $x^2 + 2x + 3$ and $x^2 + 2x - 3$.''
	Comparing the solutions to both equations, we conclude that this statement is true.
	
	\item This statement translates to ``The polynomials $x^2 + 2x + 3$ and $x^2 + 2x - 3$ do not both have a real root.''
	We can rephrase this as ``$x^2 + 2x + 3$ or $x^2 + 2x - 3$ does not have a real root.''
	Since $x^2 + 2x + 3$ only has complex roots $x = -1 \pm \sqrt{2}i$, we conclude that this statement is true.
	
	\item This statement translates to ``Both $x^2 + 2x + 3$ and $x^2 + 2x - 3$ do not have real roots.''
	Since $x^2 + 2x - 3$ has real roots $x = -3$ and $x = 1$, we conclude that this statement is false.
\end{enumerate}
\end{proof}


\begin{statement}{2.1.7}
Are these statements true or false?
The universe of discourse is the set of all people, and $P(x, y)$ means ``$x$ is a parent of $y$.''
\begin{enumerate}
	\item $\exists x \forall y P(x, y)$
	\item $\forall x \exists y P(x, y)$
	\item $\neg \exists x \exists y P(x, y)$
	\item $\exists x \neg \exists y P(x, y)$
	\item $\exists x \exists y \neg P(x, y)$
\end{enumerate}
\end{statement}

\begin{proof}
We translate the statements before determining whether they are true or false.
\begin{enumerate}
	\item ``There is a person $x$ who is a parent of all people $y$.'' \\
	This statement is false since there is no single person who is the parent of everyone.
	In particular, a person can't be their own parent (at least without time travel shenanigans).
	
	\item ``For all people $x$, there is a person $y$ who is $x$'s child.'' \\
	This statement is false since not everyone is a parent.
	For example prepubescent children biologically cannot be parents at that stage of their life.
	
	\item ``There does not exist a person $x$ who has a child $y$.'' \\
	This statement is false because there are people who do have children.
	
	\item ``There exists a person $x$ who does not have a child $y$.'' \\
	This statement is true, since there are people who either choose not to have children or are physically incapable of doing so.
	
	\item ``There exist people $x$ and $y$ for whom $x$ is not the parent of $y$.'' \\
	This statement is true.
	For example we can take any two people who are not related to each other.
\end{enumerate}
To summarize, Statements (1), (2), and (3) are false, while Statements (4) and (5) are true.
\end{proof}


\begin{statement}{2.1.8}
Are these statements true or false?
The universe of discourse is $\BN$.
\begin{enumerate}
	\item $\forall x \exists y (2x - y = 0)$
	\item $\exists y \forall x (2x - y = 0)$
	\item $\forall x \exists y (x - 2y = 0)$
	\item $\forall x (x < 10 \rightarrow \forall y (y < x \rightarrow y < 9))$
	\item $\exists y \exists z (y + z = 100)$
	\item $\forall x \exists y (y > x \wedge \exists z (y + z = 100))$
\end{enumerate}
\end{statement}

\begin{proof}
\hfill
\begin{enumerate}
	\item This statement is true.
	If $x$ is an arbitrary natural number, then $y = 2x$ is a natural number such that $2x - y = 0$.
	
	\item This statement is false.
	There is no single natural number $y$ such that $2x - y = 0$ for all natural numbers $x$.
	
	\item This statement is false.
	Suppose $x = 1$, then the value of $y$ such that $x - 2y = 0$ would be $y = 1/2$, but $1/2$ is not a natural number.
	
	\item This statement is true.
	The largest natural number which is strictly less than 10 is 9, so if $x < 10$ we know that $x \leq 9$.
	Then, if $y$ is a natural number $y < x$, we have the following chain of inequalities $y < x \leq 9$.
	
	\item This statement is true.
	For example, we can take $y = z = 50$.
	
	\item This statement is false.
	For example, if $x = 101$, then for any $y > 101$ there is no natural number $z$ such that $y + z = 100$.
	Similarly, for any $y$ such that there is a natural number $z$ with $y + z = 100$, then $y = 100 - z \leq 100 < 101$.
\end{enumerate}
\end{proof}


\begin{statement}{2.1.9}
Same as Exercise 2.1.8 but with $\BR$ as the universe of discourse.
\end{statement}

\begin{proof}
\hfill
\begin{enumerate}
	\item This statement is true.
	If $x$ is an arbitrary real number, then $y = 2x$ is a real number such that $2x - y = 0$.
	
	\item This statement is false.
	There is no single real number $y$ such that $2x - y = 0$ for all real numbers $x$.
	
	\item This statement is true.
	If $x$ is an arbitrary real number, then $y = x/2$ is a real number such that $x - 2y = 0$.
	
	\item This statement is false.
	For example, we can take $x = 9.9$ and $y = 9.8$.
	Then $x < 10$ and $y < x$, but $y > 9$.
	
	\item This statement is true.
	For example, we can take $y = z = 50$.
	
	\item This statement is true.
	If we let $x \in \BR$ be arbitrary, then in fact for any real number $y$ with $y > x$ we can take $z = 100 - y$ to be a real number such that $y + z = 100$.
\end{enumerate}
\end{proof}


\begin{statement}{2.1.10}
Same as Exercise 2.1.8 but with $\BZ$ as the universe of discourse.
\end{statement}

\begin{proof}
\hfill
\begin{enumerate}
	\item This statement is true.
	If $x$ is an arbitrary integer, then $y = 2x$ is an integer such that $2x - y = 0$.
	
	\item This statement is false.
	There is no single integer $y$ such that $2x - y = 0$ for all integers $x$.
	
	\item This statement is false.
	Suppose $x = 1$, then the value of $y$ such that $x - 2y = 0$ would be $y = 1/2$, but $1/2$ is not an integer.
	
	\item This statement is true.
	The largest integer which is strictly less than 10 is 9, so if $x < 10$ we know that $x \leq 9$.
	Then, if $y$ is an integer with $y < x$, we have the following chain of inequalities $y < x \leq 9$.
	
	\item This statement is true.
	For example, we can take $y = z = 50$.
	
	\item This statement is true.
	If we let $x \in \BZ$ be arbitrary, then in fact for any integer $y$ with $y > x$ we can take $z = 100 - y$ to be an integer such that $y + z = 100$.
\end{enumerate}
\end{proof}


\begin{statement}{2.2.1}
Negate these statements and then reexpress the results as equivalent positive statements.
\begin{enumerate}
	\item Everyone who is majoring in math has a friend who needs help with his or her homework.
	
	\item Everyone has a roommate who dislikes everyone.
	
	\item $A \cup B \subseteq C \setminus D$
	
	\item $\exists x \forall y [y > x \rightarrow \exists z (z^2 + 5z = y)]$
\end{enumerate}
\end{statement}

\begin{proof}
\hfill
\begin{enumerate}
	\item Let $M(x)$ stand for ``$x$ is majoring in math, $F(x, y)$ for ``$x$ is friends with $y$,''
	and $H(x)$ for ``$x$ needs help with their homework.''
	Then the original statement takes the following logical form:
	$\forall x [M(x) \rightarrow \exists y (F(x, y) \wedge H(y))]$.
	Our next step is to negate the statement.
	For this first part of the exercise we write down all of the steps in processing the negation.
	\begin{align*}
		& \neg \forall x [M(x) \rightarrow \exists y (F(x, y) \wedge H(y))] \\
		&= \neg \forall x [\neg M(x) \vee \exists y (F(x, y) \wedge H(y))] \\
		&= \exists x \neg [\neg M(x) \vee \exists y (F(x, y) \wedge H(y))] \\
		&= \exists x [M(x) \wedge \neg \exists y (F(x, y) \wedge H(y))] \\
		&= \exists x [M(x) \wedge \forall y \neg (F(x, y) \wedge H(y))] \\
		&= \exists x [M(x) \wedge \forall y (\neg F(x, y) \vee \neg H(y))] \\
		&= \exists x [M(x) \wedge \forall y (F(x, y) \rightarrow \neg H(y))]
	\end{align*}
	This last statement translates to ``There is a math major for whom all of their friends don't need homework help.''
	
	\item Let $R(x, y)$ stand for ``$x$ is $y$'s roommate,'' and $L(x, y)$ for ``$x$ likes $y$.''
	Then the original statement takes the following logical form:
	$\forall x \exists y (R(x, y) \wedge \forall z \neg L(y, z))$.
	Our next step is to negate the statement.
	From here on we will abbreviate some steps, such as passing a negation through multiple quantifiers at once.
	\begin{align*}
		& \neg \forall x \exists y (R(x, y) \wedge \forall z \neg L(y, z)) \\
		&= \exists x \forall y (\neg R(x, y) \vee \neg \forall z \neg L(y, z)) \\
		&= \exists x \forall y (\neg R(x, y) \vee \exists z L(y, z)) \\
		&= \exists x \forall y (R(x, y) \rightarrow \exists z L(y, z))
	\end{align*}
	This last statement translates to ``There is someone for whom all of their roommates like at least one person.''
	
	\item We first need to translate $A \cup B \subseteq C \setminus D$ into logical symbols.
	As in Chapter 1, we use the convention that lowercase letters stand for the statement that $x$ is in the set named with the corresponding capital letter.
	For example $a$ stands for the statement $x \in A$.
	\begin{align*}
		& \forall x (x \in A \cup B \rightarrow x \in C \setminus D) \\
		&= \forall x [(x \in A) \vee (x \in B) \rightarrow (x \in C) \wedge (x \notin D)] \\
		&= \forall x [\neg (a \vee b) \vee (c \wedge \neg d)]
	\end{align*}
	Our next step is to negate the statement.
	\begin{align*}
		& \neg \forall x [\neg (a \vee b) \vee (c \wedge \neg d)] \\
		&= \exists x [(a \vee b) \wedge (\neg c \vee d)] \\
		&= \exists x [((x \in A) \vee (x \in B)) \wedge ((x \notin C) \vee (x \in D))]
	\end{align*}
	In other words $A \cup B \nsubseteq C \setminus D$ means that there is $x \in A \cup B$
	such that $x \notin C \setminus D$, i.e. $x \notin C$ or $x \in D$.
	
	\item Since we are already given the statement in logical symbols, all we need to do is negate it.
	\begin{align*}
		& \neg \exists x \forall y [y > x \rightarrow \exists z (z^2 + 5z = y)] \\
		&= \forall x \exists y \neg [\neg(y > x) \vee \exists z (z^2 + 5z = y)] \\
		&= \forall x \exists y [y > x \wedge \forall z (z^2 + 5z \neq y)]
	\end{align*}
	In other words, the negated statement states that for all $x$, there exists $y$ such that
	$y > x$ and the equation $z^2 + 5z = y$ has no solutions in the universe of discourse.
\end{enumerate}
\end{proof}


\begin{statement}{2.2.2}
Negate these statements and then reexpress the results as equivalent positive statements.
\begin{enumerate}
	\item There is someone in the freshman class who doesn't have a roommate.
	
	\item Everyone likes someone, but no one likes everyone.
	
	\item $\forall a \in A \, \exists b \in B \, (a \in C \leftrightarrow b \in C)$
	
	\item $\forall y > 0 \, \exists x (ax^2 + bx + c = y)$
\end{enumerate}
\end{statement}

\begin{proof}
\hfill
\begin{enumerate}
	\item Let $F(x)$ stand for ``$x$ is in the freshman class,'' and $R(x, y)$ for ``$x$ is $y$'s roommate.''
	Then the original statement takes the following logical form:
	$\exists x (F(x) \wedge \neg \exists y R(x, y))$.
	Our next step is to negate the statement.
	\begin{align*}
		& \neg \exists x (F(x) \wedge \neg \exists y R(x, y)) \\
		&= \forall x (\neg F(x) \vee \exists y R(x, y)) \\
		&= \forall x (F(x) \rightarrow \exists y R(x, y))
	\end{align*}
	This last statement translates to ``Every member of the freshman class has a roommate.''
	
	\item Let $L(x, y)$ stand for ``$x$ likes $y$''.
	Then the original statement takes the following logical form:
	$(\forall x \exists y L(x, y)) \wedge \neg (\exists w \forall z L(w, z))$.
	Our next step is to negate the statement.
	\begin{align*}
		& \neg [(\forall x \exists y L(x, y)) \wedge \neg (\exists w \forall z L(w, z))] \\
		&= \neg (\forall x \exists y L(x, y)) \vee (\exists w \forall z L(w, z)) \\
		&= (\exists x \forall y \neg L(x, y)) \vee (\exists w \forall z L(w, z))
	\end{align*}
	This last statement translates to ``There is someone who likes everyone, or there is someone who dislikes everyone.''
	
	\item Since we are already given the statement in logical symbols, all we need to do is negate it.
	\begin{align*}
		& \neg [\forall a \in A \, \exists b \in B \, (a \in C \leftrightarrow b \in C)] \\
		&= \exists a \in A \, \forall b \in B \, \neg (a \in C \leftrightarrow b \in C) \\
		&= \exists a \in A \, \forall b \in B \, 
			[\neg (a \in C \rightarrow b \in C) \vee \neg (b \in C \rightarrow a \in C)] \\
		&= \exists a \in A \, \forall b \in B \,
			[(a \in C \wedge b \notin C) \vee (a \notin C \wedge b \in C)]
	\end{align*}
	
	\item Since we are already given the statement in logical symbols, all we need to do is negate it.
	\begin{equation*}
		\neg [\forall y > 0 \, \exists x (ax^2 + bx + c = y)]
		= \exists y > 0 \, \forall x (ax^2 + bx + c \neq y)
	\end{equation*}
\end{enumerate}	
\end{proof}


\begin{statement}{2.2.3}
Are these statements true or false?
The universe of discourse is $\BN$.
\begin{enumerate}
	\item $\forall x (x < 7 \rightarrow \exists a \exists b \exists c (a^2 + b^2 + c^2 = x))$
	
	\item $\exists! x (x^2 + 3 = 4x)$
	
	\item $\exists! x (x^2 = 4x + 5)$
	
	\item $\exists x \exists y (x^2 = 4x + 5 \wedge y^2 = 4y + 5)$
\end{enumerate}
\end{statement}

\begin{proof}
Recall that the convention in \emph{How to Prove It} is to include zero as a natural number.
\begin{enumerate}
	\item This statement is true.
	For each natural number $x < 7$, we can pick natural numbers $a, b, c$ such that $a^2 + b^2 + c^2 = x$.
	We list choices for each possible value of $x$.
	\begin{equation*}
		\begin{array}{c c c c}
			x = 0 & a = 0 & b = 0 & c = 0 \\
			x = 1 & a = 1 & b = 0 & c = 0 \\
			x = 2 & a = 1 & b = 1 & c = 0 \\
			x = 3 & a = 1 & b = 1 & c = 1 \\
			x = 4 & a = 2 & b = 0 & c = 0 \\
			x = 5 & a = 2 & b = 1 & c = 0 \\
			x = 6 & a = 2 & b = 1 & c = 1
		\end{array}
	\end{equation*}
	
	\item This statement is false.
	We can factor $x^2 - 4x + 3$ as $(x - 1)(x - 3)$, giving us two distinct natural number
	solutions to the equation $x^2 + 3 = 4x$.
	Thus, while solutions to the equation do exist in $\BN$, there is not a single unique solution.
	
	\item This statement is true.
	We can factor $x^2 - 4x - 5$ as $(x - 5)(x + 1)$, giving us two integer solutions
	to the equation $x^2 = 4x + 5$.
	Only one of those solutions, $x = 5$ is in $\BN$, so there exists a unique natural number
	solution to the equation.
	
	\item This statement is true.
	As discussed above, $x = y = 5$ is a natural number solution to the two equations
	$x^2 = 4x + 5$ and $y^2 = 4y + 5$, which are simply the same equation phrased in two ways.
	Since we only need $x$ and $y$ to exist, and there is no condition that they be distinct,
	it is okay to simply use $x = y = 5$.
\end{enumerate}
\end{proof}


\begin{statement}{2.2.4}
Show that the second quantifier negation law, 
which says that $\neg \forall x P(x)$ is equivalent to $\exists x \neg P(x)$,
can be derived from the first,
which says that $\neg \exists x P(x)$ is equivalent to $\forall x \neg P(x)$.
\emph{Hint: Use the double negation law.}
\end{statement}

\begin{proof}
Following the hint, we start by using the double negation law to observe that $\neg \forall x P(x)$
is equivalent to $\neg \forall x \neg \neg P(x)$.
Next, we use the first quantifier negation law to see that $\neg \forall x \neg \neg P(x)$
is equivalent to $\neg \neg \exists x \neg P(x)$.
By the double negation law once again, $\neg \neg \exists x \neg P(x)$ 
is equivalent to $\exists x \neg P(x)$.
Thus, we conclude that $\neg \forall x P(x)$ is equivalent to $\exists x \neg P(x)$.
\end{proof}


\begin{statement}{2.2.5}
Show that $\neg \exists x \in A \, P(x)$ is equivalent to $\forall x \in A \, \neg P(x)$.
\end{statement}

\begin{proof}
We first recall that $\exists x \in A \, P(x)$ is shorthand for $\exists x (x \in A \wedge P(x))$.
Similarly, $\forall x \in A \, P(x)$ is shorthand for $\forall x (x \in A \rightarrow P(x))$.
Starting from $\neg \exists x \in A \, P(x)$, expand the shorthand and use the quantifier negation laws.
\begin{align*}
	\neg \exists x \in A \, P(x)
	&= \neg \exists x (x \in A \wedge P(x)) \\
	&= \forall x \neg (x \in A \wedge P(x)) \\
	&= \forall x (x \notin A \vee \neg P(x)) \\
	&= \forall x (x \in A \rightarrow \neg P(x)) \\
	&= \forall x \in A \, \neg P(x)
\end{align*}
Thus, we see that $\neg \exists x \in A \, P(x)$ is equivalent to $\forall x \in A \, \neg P(x)$.
\end{proof}


\begin{statement}{2.2.6}
Show that the existential quantifier distributes over disjunction.
In other words show that $\exists x (P(x) \vee Q(x))$ is equivalent to $\exists x P(x) \vee \exists x Q(x)$.
\emph{Hint: Use the fact, discussed in this section, that the universal quantifier distributes over conjunction.}
\end{statement}

\begin{proof}
We use the double negation law to follow the hint and use the fact that the universal quantifier distributes over conjunction.
\begin{align*}
	\exists x (P(x) \vee Q(x))
	&= \exists x \neg \neg (P(x) \vee Q(x)) \\
	&= \neg \forall x \neg (P(x) \vee Q(x)) \\
	&= \neg \forall x (\neg P(x) \wedge \neg Q(x)) \\
	&= \neg [(\forall x \neg P(x)) \wedge (\forall x \neg Q(x))] \\
	&= \neg [\neg (\exists x P(x)) \wedge \neg (\exists Q(x))] \\
	&= \neg \neg (\exists x P(x) \vee \exists x Q(x)) \\
	&= \exists x P(x) \vee \exists x Q(x)
\end{align*}
Thus, we conclude that the existential quantifier distributes over disjunction.
In other words $\exists x (P(x) \vee Q(x))$ is equivalent to $\exists x P(x) \vee \exists x Q(x)$.
\end{proof}


\begin{statement}{2.2.7}
Show that $\exists x (P(x) \rightarrow Q(x))$ is equivalent to $\forall x P(x) \rightarrow \exists x Q(x)$.
\end{statement}

\begin{proof}
\begin{align*}
	\exists x (P(x) \rightarrow Q(x))
	&= \exists x (\neg P(x) \vee Q(x)) \\
	&= \exists x \neg P(x) \vee \exists x Q(x) \\
	&= \neg \forall x P(x) \vee \exists x Q(x) \\
	&= \forall x P(x) \rightarrow \exists x Q(x)
\end{align*}
\end{proof}


\begin{statement}{2.2.8}
Show that $(\forall x \in A \, P(x)) \wedge (\forall x \in B \, P(x))$ is equivalent to $\forall x \in (A \cup B) \, P(x)$.
\emph{Hint: Start by writing out the meanings of the bounded quantifiers in terms of unbounded quantifiers.}
\end{statement}

\begin{proof}
We follow the hint and start by writing out the meanings of the bounded quantifiers in terms of unbounded ones.
\begin{align*}
	& (\forall x \in A \, P(x)) \wedge (\forall x \in B \, P(x)) \\
	&= [\forall x (x \in A \rightarrow P(x))] \wedge [\forall x (x \in B \rightarrow P(x))] \\
	&= \forall x [(x \in A \rightarrow P(x)) \wedge (x \in B \rightarrow P(x))] \\
	&= \forall x [(x \notin A \vee P(x)) \wedge (x \notin B \vee P(x))] \\
	&= \forall x [(x \notin A \wedge x \notin B) \vee P(x)] \\
	&= \forall x [\neg (x \in A \cup B) \vee P(x)] \\
	&= \forall x (x \in A \cup B \rightarrow P(x)) \\
	&= \forall x \in (A \cup B) \, P(x)
\end{align*}
Thus, we conclude $(\forall x \in A \, P(x)) \wedge (\forall x \in B \, P(x))$ is equivalent to $\forall x \in (A \cup B) \, P(x)$.
\end{proof}


\begin{statement}{2.2.9}
Is $\forall x (P(x) \vee Q(x))$ equivalent to $\forall x P(x) \vee \forall x Q(x)$?
Explain.
\emph{Hint: Try assigning meanings to $P(x)$ and $Q(x)$.}
\end{statement}

\begin{proof}
To see why $\forall x (P(x) \vee Q(x))$ is not equivalent to $\forall x P(x) \vee \forall x Q(x)$,
consider the following example.
Let $\BZ$ be the universe of discourse, $P(x)$ stand for ``$x$ is an even number,''
and $Q(x)$ for ``$x$ is an odd number.''
In this scenario, $\forall x (P(x) \vee Q(x))$ translates to the true statement 
``Every integer is even or odd.''
On the other hand, $\forall x P(x) \vee \forall x Q(x)$ translates to the false statement
``Every integer is even, or every integer is odd.''
\end{proof}


\begin{statement}{2.2.10}
\begin{enumerate}
	\item Show that $(\exists x \in A \, P(x)) \vee (\exists x \in B \, P(x))$ is equivalent to
	$\exists x \in (A \cup B) \, P(x)$.
	
	\item Is $(\exists x \in A \, P(x)) \wedge (\exists x \in B \, P(x))$ equivalent to
	$\exists x \in (A \cap B) \, P(x)$?
	Explain.
\end{enumerate}
\end{statement}

\begin{proof}
\hfill
\begin{enumerate}
	\item We start by writing out the meanings of the bounded quantifiers.
	\begin{align*}
		&= (\exists x \in A \, P(x)) \vee (\exists x \in B \, P(x)) \\
		&= [\exists x (x \in A \wedge P(x))] \vee [\exists x (x \in B \wedge P(x))] \\
		&= \exists x [(x \in A \vee x \in B) \wedge P(x)] \\
		&= \exists x (x \in A \cup B \wedge P(x)) \\
		&= \exists x \in (A \cup B) \, P(x)
	\end{align*}
	Thus, $(\exists x \in A \, P(x)) \vee (\exists x \in B \, P(x))$ is equivalent to
	$\exists x \in (A \cup B) \, P(x)$.
	
	\item To see why $(\exists x \in A \, P(x)) \wedge (\exists x \in B \, P(x))$ is not equivalent to
	$\exists x \in (A \cap B) \, P(x)$, we consider two examples.
	The first example is when $A$ and $B$ are disjoint.
	In this case, $\exists x \in (A \cap B) \, P(x)$ is false simply because $A \cap B$ is empty.
	Even when $A$ and $B$ are not disjoint, it is not necessarily the case that the $a \in A$
	such that $P(a)$ is true be the same as the $b \in B$ such that $P(b)$ is true.
	For example, consider $A = \{ 1, 2 \}$, $B = \{ 2, 3 \}$, and $P(x)$ standing for the statement
	``$x$ is an odd number.''
	In this case, both $A$ and $B$ contain odd numbers (1 and 3, respectively),
	but $A \cap B = \{ 2 \}$ does not contain any odd numbers.
\end{enumerate}
\end{proof}


\begin{statement}{2.2.11}
Show that the statements $A \subseteq B$ and $A \setminus B = \varnothing$ are equivalent by writing each in logical symbols and then showing that the resulting formulas are equivalent.
\end{statement}

\begin{proof}
We start by writing $A \subseteq B$ in logical symbols as $\forall x (x \in A \rightarrow B)$.
\begin{align*}
	\forall x (x \in A \rightarrow B)
	&= \forall x (x \notin A \vee x \in B) \\
	&= \forall x \neg (x \in A \wedge x \notin B) \\
	&= \neg \exists x (x \in A \wedge x \notin B) \\
	&= \neg \exists x (x \in  A \setminus B)
\end{align*}
This last statement means $A \setminus B = \varnothing$.
Thus, we conclude that $A \subseteq B$ and $A \setminus B = \varnothing$ are equivalent.
\end{proof}


\begin{statement}{2.2.12}
Show that the statements $C \subseteq A \cup B$ and $C \setminus A \subseteq B$ are equivalent by writing each in logical symbols and then showing that the resulting formulas are equivalent.
\end{statement}

\begin{proof}
We start by writing $C \subseteq A \cup B$ in logical symbols as 
$\forall x (x \in C \rightarrow x \in A \cup B)$.
\begin{align*}
	\forall x (x \in C \rightarrow x \in A \cup B)
	&= \forall x (x \notin C \vee (x \in A \vee x \in B)) \\
	&= \forall x (\neg (x \in C \wedge x \notin A) \vee x \in B) \\
	&= \forall x (x \in C \setminus A \rightarrow x \in B)
\end{align*}
This last statement means that $C \setminus A \subseteq B$.
Thus, we conclude that $C \subseteq A \cup B$ and $C \setminus A \subseteq B$ are equivalent.
\end{proof}


\begin{statement}{2.2.13}
\begin{enumerate}
	\item Show that the statements $A \subseteq B$ and $A \cup B = B$ are equivalent by writing each in logical symbols and then showing that the resulting formulas are equivalent.
	\emph{Hint: You may find Exercise 1.5.11 useful.}
	
	\item Show that the statements $A \subseteq B$ and $A \cap B = A$ are equivalent.
\end{enumerate}
\end{statement}

\begin{proof}
Following the hint, we recall that in Exercise 1.5.11 we proved that
$(P \vee Q) \leftrightarrow Q$ is equivalent to $P \rightarrow Q$, and that
$(P \wedge Q) \leftrightarrow Q$ is equivalent to $Q \rightarrow P$.
\begin{enumerate}
	\item First we write out $A \cup B = B$ in logical symbols as
	$\forall x (x \in A \cup B \leftrightarrow x \in B)$.
	This is equivalent to $\forall x [(x \in A \vee x \in B) \leftrightarrow x \in B]$.
	By the first equivalence from Exercise 1.5.11, with $P(x) = x \in A$ and $Q(x) = x \in B$,
	this is equivalent to $\forall x (x \in A \rightarrow x \in B)$.
	In other words, $A \subseteq B$.
	Thus, $A \subseteq B$ and $A \cup B = B$ are equivalent.
	
	\item First we write out $A \cap B = A$ in logical symbols as
	$\forall x (x \in A \cap B \leftrightarrow x \in A)$.
	This is equivalent to $\forall x [(x \in A \wedge x \in B) \leftrightarrow x \in A]$.
	By the second equivalence from Exercise 1.5.11, with $P(x) = x \in B$ and $Q(x) = x \in A$,
	this is equivalent to $\forall x (x \in A \rightarrow x \in B)$.
	In other words, $A \subseteq B$.
	Thus, $A \subseteq B$ and $A \cap B = A$ are equivalent.
\end{enumerate}
\end{proof}


\begin{statement}{2.2.14}
Show that the statements $A \cap B = \varnothing$ and $A \setminus B = A$ are equivalent.
\end{statement}

\begin{proof}
First we write out $A \cap B = \varnothing$ in logical symbols as 
$\neg \exists x (x \in A \wedge x \in B)$.
By the quantifier negation law, this is equivalent to $\forall x (x \notin A \vee x \notin B)$.
This in turn is equivalent to $\forall x (x \in A \rightarrow x \notin B)$.
Using the fact that $(P \wedge Q) \leftrightarrow Q$ is equivalent to $Q \rightarrow P$,
with $P(x) = x \notin B$ and $Q(x) = x \in A$, we see that this is equivalent to
$\forall x [(x \in A \wedge x \notin B) \leftrightarrow x \in A]$.
In other words, $\forall x [x \in A \setminus B \leftrightarrow x \in A]$.
Thus, we conclude that $A \cap B = \varnothing$ and $A \setminus B = A$ are equivalent.
\end{proof}


\begin{statement}{2.2.15}
Let $T(x, y)$ mean ``$x$ is a teacher of $y$.''
What do the following statements mean?
Under what circumstances would each be true?
Are any of them equivalent to each other?
\begin{enumerate}
	\item $\exists! y T(x, y)$
	
	\item $\exists x \exists! y T(x, y)$
	
	\item $\exists! x \exists y T(x, y)$
	
	\item $\exists y \exists! x T(x, y)$
	
	\item $\exists! x \exists! y T(x, y)$
	
	\item $\exists x \exists y [T(x, y) \wedge \neg \exists u \exists v (T(u, v) \wedge (u \neq x \vee v \neq y))]$
\end{enumerate}
\end{statement}

\begin{proof}
\hfill
\begin{enumerate}
	\item This statement means ``Teacher $x$ has exactly one student, $y$.''
	Note that the truth of this statement will depend on who Teacher $x$ is.
	
	\item To better understand the statement, we first expand out the $\exists!$ quantifier.
	\begin{equation*}
		\exists x \exists! y T(x, y)
		= \exists x [\exists y [T(x, y) \wedge \neg \exists z (T(x, z) \wedge z \neq y)]]
	\end{equation*}
	This statement means ``There exists a teacher, $x$, with exactly one student, $y$.''
	Note that for this statement to be true, there only needs to exist at least one teacher with a single student.
	There can be multiple different teachers, each with exactly one student, and
	there can be some of those teachers who share the same single student.
	
	\item To better understand the statement, we first expand out the $\exists!$ quantifier.
	\begin{equation*}
		\exists! x \exists y T(x, y)
		= \exists x [\exists y T(x, y) \wedge \neg \exists z (\exists y T(z, y) \wedge z \neq x)]
	\end{equation*}
	This statement means ``There exists exactly one teacher, $x$, who has at least one student.''
	Note that for this statement to be true, only the teacher needs to exist and be unique;
	that unique teacher could have multiple students.
	
	\item To better understand the statement, we first expand out the $\exists!$ quantifier.
	\begin{equation*}
		\exists y \exists! x T(x, y)
		= \exists y [\exists x [T(x, y) \wedge \neg \exists z (T(z, y) \wedge z \neq x)]]
	\end{equation*}
	This statement means ``There is a student, $y$, with exactly one teacher $x$.''
	Note that for this statement to be true, there only needs to exist at least one student with a single teacher.
	There can be multiple different students, each with exactly one teacher,
	and there can be some of those students who share the same single teacher.
	
	\item To better understand the statement, we first expand out the second $\exists!$ quantifier.
	\begin{align*}
		& \exists! x \exists! y T(x, y) \\
		&= \exists! x [\exists y (T(x, y) \wedge \neg \exists z (T(x, z) \wedge z \neq y))] \\
		&= \exists! x [\exists y (T(x, y) \wedge \forall z (T(x, z) \rightarrow z = y))]
	\end{align*}
	This statement means ``There is a unique teacher, $x$, with exactly one student, $y$.''
	Note that this statement is still true if there exist other teachers who have multiple students.
	There just cannot be any teachers aside from $x$ who have exactly one student.
	
	\item To better understand the statement, we use the quantifier negation law on the
	inner negated quantifier.
	\begin{align*}
		& \exists x \exists y 
			[T(x, y) \wedge \neg \exists u \exists v (T(u, v) \wedge (u \neq x \vee v \neq y))] \\
		&= \exists x \exists y
			[T(x, y) \wedge \forall u \forall v (\neg T(u, v) \vee (u = x \wedge v = y))] \\
		&= \exists x \exists y
			[T(x, y) \wedge \forall u \forall v (T(u, v) \rightarrow (u = x \wedge v = y))]
	\end{align*}
	This statement means ``There are people, $x$ and $y$, such that $x$ is $y$'s teacher, 
	and for all other people, $u$ and $v$, if $u$ is $v$'s teacher, then $u = x$ and $v = y$.''
	In other words, $x$ and $y$ are the only people with a student-teacher relationship,
	and in particular $x$ is $y$'s teacher.
\end{enumerate}
\emph{Velleman may want us to consider the fifth and sixth statements to be equivalent to each other, but both expanding out the definition of $\exists!$ and considering the following example seem to say otherwise.}

To see why the fifth and sixth statements are not equivalent,
consider the following statement: $P(c, x)$ stands for $x^2 + 3x + c = 0$,
where the universe of discourse is $\BC$.
By the quadratic formula, the values of $c$ and $x$ which make $P(c, x)$ true satisfy
\begin{equation*}
	x = \frac{-3 \pm \sqrt{9 - 4c}}{2}.
\end{equation*}
When $9 - 4c > 0$, there are two distinct real solutions.
When $9 - 4c = 0$, or in other words when $c = 9/4$, there is exactly one real solution, $x = -3/2$.
Lastly, when $9 - 4c < 0$, there are two distinct complex solutions.
Since there is a unique value of $c$ for which there is exactly one solution to $x^2 + 3x + 2 = 0$,
the statement $\exists! c \exists! x P(c, x)$ is true.
However, since there are other values of $c$ which give different solutions to $x^2 + 3x + 2 = 0$,
the statement 
$\exists c \exists x [P(c, x) \wedge \forall u \forall v (P(u, v) \rightarrow (u = c \wedge v = x))]$
is false.
\end{proof}


\begin{statement}{2.3.1}
Analyze the logical forms of the following statements.
You may use the symbols $\in, \notin, =, \neq, \wedge, \vee, \rightarrow, \leftrightarrow,
 \forall, \text{ and } \exists$
in your answers, but not $\subseteq, \nsubseteq, \mathscr{P}, \cap, \cup, \setminus, \{, \},
\text{ or } \neg$.
(Thus, you must write out the definitions of some set theory notation, and you must use
equivalences to get rid of any occurrences of $\neg$.)
\begin{enumerate}
	\item $\mathcal{F} \subseteq \mathscr{P}(A)$
	
	\item $A \subseteq \{ 2n + 1 \mid n \in \BN \}$
	
	\item $\{ n^2 + n + 1 \mid n \in \BN \} \subseteq \{ 2n + 1 \mid n \in \BN \}$
	
	\item $\mathscr{P} \left( \bigcup_{i \in I} A_i \right) 
	\nsubseteq \bigcup_{i \in I} \mathscr{P}(A_i)$
\end{enumerate}
\end{statement}

\begin{proof}
\hfill
\begin{enumerate}
	\item We start out by expanding the logical form of being a subset.
	\begin{align*}
		\mathcal{F} \subseteq \mathscr{P}(A)
		&= \forall X (X \in \mathcal{F} \rightarrow X \in \mathscr{P}(A)) \\
		&= \forall X [X \in \mathcal{F} \rightarrow \forall x(x \in X \rightarrow x \in A)]
	\end{align*}
	
	\item We start out by expanding the logical form of being a subset.
	\begin{align*}
		A \subseteq \{ 2n + 1 \mid n \in \BN \}
		&= \forall a (a \in A \rightarrow a \in \{ 2n + 1 \mid n \in \BN \}) \\
		&= \forall a [a \in A \rightarrow \exists n \in \BN (a = 2n + 1)]
	\end{align*}
	
	\item We start out by expanding the logical form of being a subset.
	\begin{align*}
		& \{ n^2 + n + 1 \mid n \in \BN \} \subseteq \{ 2n + 1 \mid n \in \BN \} \\
		&= \forall n \in \BN (n^2 + n + 1 \in \{ 2n + 1 \mid n \in \BN \}) \\
		&= \forall n \in \BN \exists m \in \BN (n^2 + n + 1 = 2m + 1)
	\end{align*}
	
	\item First, we write out the logical form of being an element of
	$\mathscr{P} \left( \bigcup_{i \in I} A_i \right)$.
	\begin{align*}
		X \in \mathscr{P} \left( \bigcup_{i \in I} A_i \right)
		&= X \subseteq \bigcup_{i \in I} A_i \\
		&= \forall x \left( x \in X \rightarrow x \in \bigcup_{i \in I} A_i \right) \\
		&= \forall x(x \in X \rightarrow \exists i \in I (x \in A_i))
	\end{align*}
	Next, we write out the logical form of not being an element of
	$\bigcup_{i \in I} \mathscr{P}(A_i)$.
	\begin{align*}
		X \notin \bigcup_{i \in I} \mathscr{P}(A_i)
		&= \nexists i \in I (X \subseteq A_i) \\
		&= \nexists i \in I [\forall x (x \in X \rightarrow x \in A_i)] \\
		&= \forall i \in I [\exists x (x \in X \wedge x \notin A_i)]
	\end{align*}
	Since the initial logical form of $\mathscr{P} \left( \bigcup_{i \in I} A_i \right) 
	\nsubseteq \bigcup_{i \in I} \mathscr{P}(A_i)$ is
	\begin{equation*}
		\exists X 
		\left[ X \in \mathscr{P} \left( \bigcup_{i \in I} A_i \right) \wedge
		X \notin \bigcup_{i \in I} \mathscr{P}(A_i) \right],
	\end{equation*}
	we can combine the work we did analyzing the logical forms of being an element of
	$\mathscr{P} \left( \bigcup_{i \in I} A_i \right)$
	and not being an element of
	$\bigcup_{i \in I} \mathscr{P}(A_i)$
	to get the final logical form.
	\begin{equation*}
		\exists X 
		\left[ \forall x(x \in X \rightarrow \exists i \in I (x \in A_i)) \wedge
		\forall i \in I (\exists x (x \in X \wedge x \notin A_i)) \right]
	\end{equation*}
\end{enumerate}
\end{proof}


\begin{statement}{2.3.2}
Analyze the logical forms of the following statements.
You may use the symbols $\in, \notin, =, \neq, \wedge, \vee, \rightarrow, \leftrightarrow,
 \forall, \text{ and } \exists$
in your answers, but not $\subseteq, \nsubseteq, \mathscr{P}, \cap, \cup, \setminus, \{, \},
\text{ or } \neg$.
(Thus, you must write out the definitions of some set theory notation, and you must use
equivalences to get rid of any occurrences of $\neg$.)
\begin{enumerate}
	\item $x \in \bigcup \mathcal{F} \setminus \bigcup \mathcal{G}$
	
	\item $\{ x \in B \mid x \notin C \} \in \mathscr{P}(A)$
	
	\item $x \in \bigcap_{i \in I} (A_i \cup B_i)$
	
	\item $x \in \left( \bigcap_{i \in I} A_i \right) \cup \left( \bigcap_{i \in I} B_i \right)$
\end{enumerate}
\end{statement}

\begin{proof}
\hfill
\begin{enumerate}
	\item We start out by writing the logical form of being an element of the union of a family of sets.
	\begin{align*}
		x \in \bigcup \mathcal{F} \setminus \bigcup \mathcal{G}
		&= [\exists A (A \in \mathcal{F} \wedge x \in A)] \wedge
			\neg [\exists B (B  \in \mathcal{G} \wedge x \in B)] \\
		&= [\exists A (A \in \mathcal{F} \wedge x \in A)] \wedge
			[\forall B (B \notin \mathcal{G} \vee x \notin B)] \\
		&= [\exists A (A \in \mathcal{F} \wedge x \in A)] \wedge
			[\forall B (B \in \mathcal{G} \rightarrow x \notin B)]
	\end{align*}
	
	\item We start out by writing the logical form of being an element of the power set.
	\begin{align*}
		\{ x \in B \mid x \notin C \} \in \mathscr{P}(A)
		&= \{ x \in B \mid x \notin C \} \subseteq A \\
		&= \forall x (x \in B \wedge x \notin C \rightarrow x \in A)
	\end{align*}
	
	\item We start out by writing the logical form of being an element of the intersection of a
	family of sets.
	\begin{align*}
		x \in \bigcap_{i \in I} (A_i \cup B_i)
		&= \forall i \in I (x \in A_i \cup B_i) \\
		&= \forall i \in I (x \in A_i \vee x \in B_i)
	\end{align*}
	
	\item We start out by writing the logical form of being an element of the union of two sets.
	\begin{align*}
		x \in \left( \bigcap_{i \in I} A_i \right) \cup \left( \bigcap_{i \in I} B_i \right)
		&= \left( x \in \bigcap_{i \in I} A_i \right) \vee
			\left( x \in \bigcap_{i \in I} B_i \right) \\
		&= [\forall i \in I (x \in A_i)] \vee [\forall i \in I (x \in B_i)]
	\end{align*}
\end{enumerate}
\end{proof}


\begin{statement}{2.3.3}
We've seen that $\mathscr{P}(\varnothing) = \{ \varnothing \}$, and
$\{ \varnothing \} \neq \varnothing$.
What is $\mathscr{P}( \{ \varnothing \} )$?
\end{statement}

\begin{proof}
Since $\mathscr{P}(\varnothing) = \{ \varnothing \}$,
and $\{ \varnothing \} \neq \varnothing$,
$\mathscr{P}(\{ \varnothing \}) = \{ \varnothing, \{ \varnothing \} \}$.
\end{proof}


\begin{statement}{2.3.4}
Suppose $\mathcal{F} =$ \{\{red, green, blue\}, \{orange, red, blue\}, \{purple, red, green, blue\}\}.
Find $\bigcap \mathcal{F}$ and $\bigcup \mathcal{F}$.
\end{statement}

\begin{proof}
\begin{align*}
	\bigcap \mathcal{F} &= \{ x \mid \forall A \in \mathcal{F} (x \in A) \} = \{ \text{red, blue} \} \\
	\bigcup \mathcal{F} &= \{ x \mid \exists A \in \mathcal{F} (x \in A) \} =
		\{ \text{red, green, blue, orange, purple} \}
\end{align*}
\end{proof}


\begin{statement}{2.3.5}
Suppose $\mathcal{F} =$ \{\{3, 7, 12\}, \{5, 7, 16\}, \{5, 12, 23\}\}.
Find $\bigcap \mathcal{F}$ and $\bigcup \mathcal{F}$.
\end{statement}

\begin{proof}
\begin{align*}
	\bigcap \mathcal{F} &= \{ x \mid \forall A \in \mathcal{F} (x \in A) \} = \varnothing \\
	\bigcup \mathcal{F} &= \{ x \mid \exists A \in \mathcal{F} (x \in A) \} =
		\{ 3, 5, 7, 12, 16, 23 \}
\end{align*}
\end{proof}


\begin{statement}{2.3.6}
Let $I = \{ 2, 3, 4, 5 \}$, and for each $i \in I$ let $A_i = \{ i, i + 1, i - 1, 2i \}$.
\begin{enumerate}
	\item List the elements of all the sets $A_i$ for $i \in I$.
	
	\item Find $\bigcap_{i \in I} A_i$ and $\bigcup_{i \in I} A_i$.
\end{enumerate}
\end{statement}

\begin{proof}
\hfill
\begin{enumerate}
	\item For each $i \in I$, the sets $A_i$ are as follows.
	\begin{align*}
		A_2 &= \{ 2, 1, 3, 4 \} \\
		A_3 &= \{ 3, 4, 2, 6 \} \\
		A_4 &= \{ 4, 5, 3, 8 \} \\
		A_5 &= \{ 5, 6, 4, 10 \}
	\end{align*}
	
	\item We now compute $\bigcap_{i \in I} A_i$ and $\bigcup_{i \in I} A_i$
	\begin{align*}
		\bigcap_{i \in I} A_i &= \{ 4 \} \\
		\bigcup_{i \in I} A_i &= \{ 1, 2, 3, 4, 5, 6, 8, 10 \}
	\end{align*}
\end{enumerate}
\end{proof}


\begin{statement}{2.3.7}
Let $P =$ \{Johann Sebastian Bach, Napoleon Bonaparte, Johann Wolfgang von Goethe,
David Hume, Wolfgang Amadeus Mozart, Isaac Newton, George Washington\}
and let $Y =$ \{1750, 1751, 1752, $\dots$, 1759\}.
For each $y \in Y$, let $A_y =$ 
\{$p \in P \mid$ the person $p$ was alive at some time during the year $y$\}.
Find $\bigcup_{y \in Y} A_y$ and $\bigcap_{y \in Y} A_y$.
\end{statement}

\begin{proof}
We can look at Wikipedia to find the birth and death dates for all of the people in set $P$.
\begin{itemize}
	\item \href{https://en.wikipedia.org/wiki/Johann_Sebastian_Bach}{Johann Sebastian Bach}:
	31 March 1685 to 28 July 1750
	
	\item \href{https://en.wikipedia.org/wiki/Napoleon}{Napoleon Bonapart}:
	15 August 1769 to 5 May 1821
	
	\item \href{https://en.wikipedia.org/wiki/Johann_Wolfgang_von_Goethe}
	{Johann Wolfgang von Goethe}: 28 August 1749 to 22 March 1832
	
	\item \href{https://en.wikipedia.org/wiki/David_Hume}{David Hume}:
	7 May 1711 to 25 August 1776
	
	\item \href{https://en.wikipedia.org/wiki/Wolfgang_Amadeus_Mozart}
	{Wolfgang Amadeus Mozart}: 27 January 1756 to 5 December 1791
	
	\item \href{https://en.wikipedia.org/wiki/Isaac_Newton}{Isaac Newton}:
	25 December 1642 to 20 March 1726
	
	\item \href{https://en.wikipedia.org/wiki/George_Washington}{George Washington}:
	22 February 1732 to 14 December 1799
\end{itemize}
From these birth and death dates, we can write out the elements for each set $A_y$.
\begin{itemize}
	\item $A_{1750} =$ \{Bach, von Goethe, Hume, Washington\}
	
	\item $A_{y} =$ \{von Goethe, Hume, Washington\} for $y = 1751, 1752, \dots, 1755$
	
	\item $A_{y} =$ \{von Goethe, Hume, Washington, Mozart\} for $y = 1756, 1752, \dots, 1759$
\end{itemize}
Now that we have written out the sets $A_y$, we see that
$\bigcup_{y \in Y} A_y =$ \{Bach, von Goethe, Hume, Washington, Mozart\}, and
$\bigcap_{y \in Y} A_y =$ \{von Goeth, Hume, Washington\}.
\end{proof}


\begin{statement}{2.3.8}
Let $I = \{ 2, 3 \}$, and for each $i \in I$ let $A_i = \{ i, 2i \}$ and $B_i = \{ i, i + 1 \}$.
\begin{enumerate}
	\item List the elements of the sets $A_i$ and $B_i$ for $i \in I$.
	
	\item Find $\bigcap_{i \in I} (A_i \cup B_i)$
	and $\left( \bigcap_{i \in I} A_i \right) \cup \left( \bigcap_{i \in I} B_i \right)$.
	Are they the same?
	
	\item In parts (3) and (4) of Exercise 2.3.2 you analyzed the statements
	$x \in \bigcap_{i \in I} (A_i \cup B_i)$ and
	$x \in \left( \bigcap_{i \in I} A_i \right) \cup \left( \bigcap_{i \in I} B_i \right)$
	What can you conclude from your answer to Part (2) about whether or not
	these statements are equivalent?
\end{enumerate}
\end{statement}

\begin{proof}
\hfill
\begin{enumerate}
	\item $A_2 = \{ 2, 4 \}, A_3 = \{ 3, 6 \}, B_2 = \{ 2, 3 \}, B_3 = \{ 3, 4 \}$
	
	\item First we compute $\bigcap_{i \in I} (A_i \cup B_i)$.
	\begin{equation*}
		\bigcap_{i \in I} (A_i \cup B_i)
		= (A_2 \cup B_2) \cap (A_3 \cup B_3)
		= \{ 2, 3, 4 \} \cap \{ 3, 4, 6 \}
		= \{ 3, 4 \}
	\end{equation*}
	Next we compute $\left( \bigcap_{i \in I} A_i \right) \cup \left( \bigcap_{i \in I} B_i \right)$.
	\begin{equation*}
		\left( \bigcap_{i \in I} A_i \right) \cup \left( \bigcap_{i \in I} B_i \right)
		= (A_2 \cap A_3) \cup (B_2 \cap B_3)
		= \varnothing \cup \{ 3 \}
		= \{ 3 \}
	\end{equation*}
	As we can see, the two sets $\bigcap_{i \in I} (A_i \cup B_i)$ and
	$\left( \bigcap_{i \in I} A_i \right) \cup \left( \bigcap_{i \in I} B_i \right)$ are different.
	
	\item Based on the answer to Part (2), we can conclude that the statements
	$x \in \bigcap_{i \in I} (A_i \cup B_i)$ and
	$x \in \left( \bigcap_{i \in I} A_i \right) \cup \left( \bigcap_{i \in I} B_i \right)$
	are not equivalent.
\end{enumerate}
\end{proof}


\begin{statement}{2.3.9}
\begin{enumerate}
	\item Analyze the logical forms of the statements
	$x \in \bigcup_{i \in I} (A_i \setminus B_i)$,
	$x \in \left( \bigcup_{i \in I} A_i \right) \setminus \left( \bigcup_{i \in I} B_i \right)$, and
	$x \in \left( \bigcap_{i \in I} A_i \right) \setminus \left( \bigcap_{i \in I} B_i \right)$.
	Do you think that any of these statements are equivalent to each other?
	
	\item Let $I$, $A_i$, and $B_i$ be defined as in Exercise 2.3.8.
	Find $\bigcup_{i \in I} (A_i \setminus B_i)$,
	$\left( \bigcup_{i \in I} A_i \right) \setminus \left( \bigcup_{i \in I} B_i \right)$, and
	$\left( \bigcup_{i \in I} A_i \right) \setminus \left( \bigcap_{i \in I} B_i \right)$.
	Now do you think any of the statements in Part (1) are equivalent?
\end{enumerate}
\end{statement}

\begin{proof}
\hfill
\begin{enumerate}
	\item The first statement we analyze is $x \in \bigcup_{i \in I} (A_i \setminus B_i)$.
	\begin{align*}
		x \in \bigcup_{i \in I} (A_i \setminus B_i)
		&= \exists i \in I (x \in A_i \setminus B_i) \\
		&= \exists i \in I (x \in A_i \wedge x \notin B_i)
	\end{align*}
	Next, we consider 
	$x \in \left( \bigcup_{i \in I} A_i \right) \setminus \left( \bigcup_{i \in I} B_i \right)$.
	\begin{align*}
		x \in \left( \bigcup_{i \in I} A_i \right) \setminus \left( \bigcup_{i \in I} B_i \right)
		&= \left[ x \in \bigcup_{i \in I} A_i \right]
			\wedge \left[ x \notin \bigcup_{i \in I} B_i \right] \\
		&= [\exists i \in I (x \in A_i)] \wedge [\forall i \in I (x \notin B_i)]
	\end{align*}
	Finally, we consider 
	$x \in \left( \bigcup_{i \in I} A_i \right) \setminus \left( \bigcap_{i \in I} B_i \right)$.
	\begin{align*}
		x \in \left( \bigcup_{i \in I} A_i \right) \setminus \left( \bigcap_{i \in I} B_i \right)
		&= \left[ x \in \bigcup_{i \in I} A_i \right]
			\wedge \left[ x \notin \bigcap_{i \in I} B_i \right] \\
		&= [\exists i \in I (x \in A_i)] \wedge [\exists i \in I (x \notin B_i)]
	\end{align*}
	I do not think that any of the statements are equivalent to each other.
	The first states that there is some index $i \in I$ for which $x \in A_i$ and $x \notin B_i$,
	but there may be other indices for which $x \in A_i \wedge x \notin B_i$ is false.
	The second states that while there is some index $i \in I$ for which $x \in A_i$,
	$x \notin B_i$ for all indices $i \in I$.
	The third states simply that there are indices $i, j \in I$, which may be distinct,
	such that $x \in A_i$ and $x \notin B_j$.
	
	\item Recall that $A_2 = \{ 2, 4 \}, A_3 = \{ 3, 6 \}, B_2 = \{ 2, 3 \}, B_3 = \{ 3, 4 \}$,
	as defined in Exercise 2.3.8.
	We now compute each of the sets discussed in Part 1.
	\begin{equation*}
		\bigcup_{i \in I} (A_i \setminus B_i)
		= (A_2 \setminus B_2) \cup (A_3 \setminus B_3)
		= \{ 4 \} \cup \{ 6 \}
		= \{ 4, 6 \}
	\end{equation*}
	\begin{equation*}
		\left( \bigcup_{i \in I} A_i \right) \setminus \left( \bigcup_{i \in I} B_i \right)
		= \{ 2, 3, 4, 6 \} \setminus \{ 2, 3, 4 \}
		= \{ 6 \}
	\end{equation*}
	\begin{equation*}
		\left( \bigcup_{i \in I} A_i \right) \setminus \left( \bigcap_{i \in I} B_i \right)
		= \{ 2, 3, 4, 6 \} \setminus \{ 3 \}
		= \{ 2, 4, 6 \}
	\end{equation*}
	As we can see, none of the three sets are equal to each other.
	This confirms the thoughts we had in Part (1) about the non-equivalence of the three statements.
\end{enumerate}
\end{proof}


\begin{statement}{2.3.10}
Give an example of an index set $I$ and indexed families of sets
$\{ A_i \mid i \in I \}$ and $\{ B_i \mid i \in I \}$
such that
$\bigcup_{i \in I} (A_i \cap B_i) \neq
\left( \bigcup_{i \in I} A_i \right) \cap \left( \bigcup_{i \in I} B_i \right)$.
\end{statement}

\begin{proof}
Consider $I = \{ 2, 3 \}$, and $A_i = \{ i, 2i \}$ and $B_i = \{ i, i + 1 \}$ for each $i \in I$
as in Exercise 2.3.8.
Recall that $A_2 = \{ 2, 4 \}, A_3 = \{ 3, 6 \}, B_2 = \{ 2, 3 \}, B_3 = \{ 3, 4 \}$.
We can then compute $\bigcup_{i \in I} (A_i \cap B_i)$ and
$\left( \bigcup_{i \in I} A_i \right) \cap \left( \bigcup_{i \in I} B_i \right)$,
and see that the two sets are not equal.
\begin{equation*}
	\bigcup_{i \in I} (A_i \cap B_i)
	= (A_2 \cap B_2) \cup (A_3 \cap B_3)
	= \{ 2 \} \cup \{ 3 \}
	= \{ 2, 3 \}
\end{equation*}
\begin{equation*}
	\left( \bigcup_{i \in I} A_i \right) \cap \left( \bigcup_{i \in I} B_i \right)
	= (A_2 \cup A_3) \cap (B_2 \cup B_3)
	= \{ 2, 3, 4, 6 \} \cap \{ 2, 3, 4 \}
	= \{ 2, 3, 4 \}
\end{equation*}
\end{proof}


\begin{statement}{2.3.11}
Show that for any sets $A$ and $B$,
$\mathscr{P}(A \cap B) = \mathscr{P}(A) \cap \mathscr{P}(B)$
by showing that the statements $X \in \mathscr{P}(A \cap B)$ and 
$X \in \mathscr{P}(A) \cap \mathscr{P}(B)$ are equivalent.
\end{statement}

\begin{proof}
Let $A$ and $B$ be arbitrary sets.
We start by writing the logical form for $X \in \mathscr{P}(A \cap B)$.
\begin{equation*}
	\forall x [x \in X \rightarrow (x \in A \wedge x \in B)]
\end{equation*}
By the equivalence of
$P \rightarrow (Q \wedge R)$ and $(P \rightarrow Q) \wedge (P \rightarrow R)$, 
which we proved in Exercise 1.5.12,
we can rewrite the statement as
\begin{equation*}
	\forall x [(x \in X \rightarrow x \in A) \wedge (x \in X \rightarrow x \in B)].
\end{equation*}
Now we use the fact that the universal quantifier distributes over conjunction to end up with
\begin{equation*}
	\forall x (x \in X \rightarrow x \in A) \wedge \forall x (x \in X \rightarrow x \in B).
\end{equation*}
This is the logical form of the statement $X \in \mathscr{P}(A) \cap \mathscr{P}(B)$.
Therefore, since the statements $X \in \mathscr{P}(A \cap B)$ and 
$X \in \mathscr{P}(A) \cap \mathscr{P}(B)$ are equivalent,
we conclude that $\mathscr{P}(A \cap B) = \mathscr{P}(A) \cap \mathscr{P}(B)$.
\end{proof}


\begin{statement}{2.3.12}
Give examples of sets $A$ and $B$ for which
$\mathscr{P}(A \cup B) \neq \mathscr{P}(A) \cup \mathscr{P}(B)$.
\emph{Note: In my digital copy of the book this exercise has a typo and asks for examples 
of sets for $A$ and $B$ for which
$\mathscr{P}(A \cup B) = \mathscr{P}(A) \cup \mathscr{P}(B)$.}
\end{statement}

\begin{proof}
One example of sets $A$ and $B$ for which
$\mathscr{P}(A \cup B) \neq \mathscr{P}(A) \cup \mathscr{P}(B)$
is the case where $A = \{ 1, 2 \}$ and $B = \{ 2, 3 \}$.
In this situation, since $A \cup B = \{ 1, 2, 3 \}$, we have $\{ 1, 2, 3 \} \in \mathscr{P}(A \cup B)$.
However, $\{ 1, 2, 3 \} \notin \mathscr{P}(A) \cup \mathscr{P}(B)$.
Thus, for this choice of $A$ and $B$, 
$\mathscr{P}(A \cup B) \neq \mathscr{P}(A) \cup \mathscr{P}(B)$.

We can also consider two somewhat silly examples of sets $A$ and $B$ for which
$\mathscr{P}(A \cup B) = \mathscr{P}(A) \cup \mathscr{P}(B)$.
This first such example is where $A = B$.
In this case, $\mathscr{P}(A \cup B) = \mathscr{P}(A)$
and similarly $\mathscr{P}(A) \cup \mathscr{P}(B) = \mathscr{P}(A) \cup \mathscr{P}(A)
= \mathscr{P}(A)$.
The other silly example is when at least one of $A$ or $B$ is empty.
For instance, if $B =\varnothing$, then $A \cup B = A$.
This means $\mathscr{P}(A \cup B) = \mathscr{P}(A)$.
In addition, since $\mathscr{P}(\varnothing) = \{ \varnothing \}$,
we also have $\mathscr{P}(A) \cup \mathscr{P}(B) = \mathscr{P}(A) \cup \{ \varnothing \}
= \mathscr{P}(A)$, as $\varnothing \in \mathscr{P}(A)$ for any set $A$.
\end{proof}


\begin{statement}{2.3.13}
Verify the following identities by writing out (using logical symbols) what it means for an object $x$
to be an element of each set and then using logical equivalences.
\begin{enumerate}
	\item $\bigcup_{i \in I} (A_i \cup B_i)
	= \left( \bigcup_{i \in I} A_i \right) \cup \left( \bigcup_{i \in I} B_i \right)$
	
	\item $\left( \bigcap \mathcal{F} \right) \cap \left( \bigcap \mathcal{G} \right)
	= \bigcap (\mathcal{F} \cup \mathcal{G})$
	
	\item $\bigcap_{i \in I} (A_i \setminus B_i)
	= \left( \bigcap_{i \in I} A_i \right) \setminus \left( \bigcup_{i \in I} B_i \right)$
\end{enumerate}
\end{statement}

\begin{proof}
% Fill in!
\end{proof}


\begin{statement}{2.3.14}
Sometimes each set in an indexed family of sets has \emph{two} indices.
For this problem, use the following definitions:
$I = \{ 1, 2 \}, J = \{ 3, 4 \}$.
For each $i \in I$ and $j \in J$, let $A_{i, j} = \{ i, j, i + j \}$.
Thus, for example, $A_{2, 3} = \{ 2, 3, 5 \}$.
\begin{enumerate}
	\item For each $j \in J$ let $B_j = \bigcup_{i \in I} A_{i, j} = A_{1, j} \cup A_{2, j}$.
	Find $B_3$ and $B_4$.
	
	\item Find $\bigcap_{j \in J} B_j$.
	(Note that, by replacing $B_j$ with its definition, we could say that
	$\bigcap_{j \in J} B_j = \bigcap_{j \in J} \left( \bigcup_{i \in I} A_{i, j} \right)$.)
	
	\item Find $\bigcup_{i \in I} \left( \bigcap_{j \in J} A_{i, j} \right)$.
	\emph{(Hint: You may want to do this in two steps, corresponding to parts (1) and (2).)}
	Are $\bigcap_{j \in J} \left( \bigcup_{i \in I} A_{i, j} \right)$
	and $\bigcup_{i \in I} \left( \bigcap_{j \in J} A_{i, j} \right)$ equal?
	
	\item Analyze the logical forms of the statements
	$x \in \bigcap_{j \in J} \left( \bigcup_{i \in I} A_{i, j} \right)$
	and $x \in \bigcup_{i \in I} \left( \bigcap_{j \in J} A_{i, j} \right)$.
	Are the equivalent?
\end{enumerate}
\end{statement}

\begin{proof}
% Fill in!
\end{proof}


\begin{statement}{2.3.15}
\begin{enumerate}
	\item Show that if $\mathcal{F} = \varnothing$, then the statement $x \in \bigcup \mathcal{F}$
	will be false no matter what $x$ is.
	It follows that $\bigcup \varnothing = \varnothing$.
	
	\item Show that if $\mathcal{F} = \varnothing$, then the statement $x \in \bigcap \mathcal{F}$
	will be true no matter what $x$ is.
	In a context in which it is clear what the universe of discourse $U$ is,
	we might therefore want to say that $\bigcap \varnothing = U$.
	However, this has the unfortunate consequence that the notation $\bigcap \varnothing$
	will mean different things in different contexts.
	Furthermore, when working with sets whose elements are sets,
	mathematicians often do not use a universe of discourse at all.
	(For more on this, see the next exercise.)
	For these reasons, some mathematicians will consider the notation $\bigcap \varnothing$
	to be meaningless.
	We will avoid this problem in this book by using the notation $\bigcap \mathcal{F}$
	only in contexts in which we can be sure that $\mathcal{F} \neq \varnothing$.
\end{enumerate}
\end{statement}

\begin{proof}
% Fill in!
\end{proof}


\begin{statement}{2.3.16}
In Section 2.3 we saw that a set can have other sets as elements.
When discussing sets whose elements are sets, it might seem most natural to consider the universe of discourse to be the set of all sets.
However, as we will see in this problem, assuming there is such a set leads to contradictions.

Suppose $U$ were the set of all sets.
Note that in particular $U$ is a set, so we would have $U \in U$.
This is not yet a contradiction;
although most sets are not elements of themselves, perhaps some sets are elements of themselves.
But it suggests that the sets in the universe $U$ could be split into two categories:
the unusual sets that, like $U$ itself, are elements of themselves,
and the more typical sets that are not.
Let $R$ be the set of sets in the second category.
In other words, $R = \{ A \in U \mid A \notin A \}$.
This means that for any set $A$ in the universe $U$,
$A$ will be an element of $R$ if and only if $A \notin A$.
In other words, we have $\forall A \in U (A \in R \leftrightarrow A \notin A)$.
\begin{enumerate}
	\item Show that applying this last fact to the set $R$ itself
	(in other words, plugging in $R$ for $A$) leads to a contradiction.
	This contradiction was discovered by Bertrand Russell (1872--1970) in 1901,
	and is known as \emph{Russell's paradox}.
	
	\item Think some more about the paradox in part (1).
	What do you think it tells us about sets?
\end{enumerate}
\end{statement}

\begin{proof}
\hfill
\begin{enumerate}
	\item Since $R$ is a set, and therefore is an element of $U$, then we can consider the statement
	$R \in R \leftrightarrow R \notin R$.
	This statement says that if $R \in R$, then $R \notin R$, and vice-versa.
	In other words, we end up with the contradiction $R \in R \wedge R \notin R$.
	
	\item This paradox tells us that allowing for a set of all sets causes trouble,
	and that we need to be more strict and precise about the definition of a set
	beyond the naive definition of ``a collection of elements''.
	For more discussion about Russell's paradox and the ways we can look at
	\href{https://en.wikipedia.org/wiki/Russell\%27s_paradox}{Wikipedia}
	or \href{https://plato.stanford.edu/entries/russell-paradox/}
	{the Stanford encyclopedia of philosophy}.
\end{enumerate}
\end{proof}
\end{document}